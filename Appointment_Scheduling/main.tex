\documentclass{article}
\usepackage{graphicx} % Required for inserting images

\usepackage{setspace}

\usepackage{amsmath}
\usepackage{color}

\usepackage{amsthm}
\usepackage{bm}

\usepackage{indentfirst}
\setlength{\parindent}{2em}  % 用于首行缩进

\newcommand{\Z}{\mathbf{Z}}
\newcommand{\D}{\bm{\Delta}}
\newcommand{\A}{\mathbf{A}}


\newtheorem{lemma}{\hspace{2em}Lemma}
\newtheorem{proposition}{\hspace{2em}Proposition}
\newtheorem{example}{Example}

\usepackage{geometry}
\geometry{a4paper,left=1in,right=1in,top =1in, bottom = 1in}
\setstretch{1.5}   %  改变行间距

\title{Appointment Scheduling with Restricted People}

% \author{Discount}

\date{}
\begin{document}

\maketitle

\section{Preliminary Study}

The service time for customer $i$, $\xi_{i}$, stochastic with a mean of $\mu_{i}$ and a standard deviation of $\sigma_{i}$. The service times are mutually independent.
For each customer $i = 1, \ldots, n$, we use $A_{i}$ to denote the appointment time, $S_{i} = \max\{A_{i}, S_{i-1} + \xi_{i-1}\}$ denote the actual starting time of service. We assume that the customers will arrive at the appointed time. Especially, $A_{1} = S_{1} = 0$.

The waiting time for customer $i$ is $S_{i} - A_{i}$, the total waiting time is $\sum_{i=2}^n \alpha_i \left(S_i-A_i\right)$, where $\alpha_i$ is the weight for customer $i$. The overtime is $(S_{n} +\xi_{n} - T)^{+}$ and the total idle time is $\sum_{i=1}^{n-1}[S_{i+1} - (S_{i} + \xi_{i})] = S_{n} - \sum_{i=1}^{n-1} \xi_{i}$.


\textcolor{red}{In the scenario with at least 2 customers overlapping in the waiting room, we can calculate the overlapping time. Let $t_{ij}$ denote the overlapping time between two customers $i$ and $j$. Then, $t_{i,j} = (S_{i} - A_{j})^{+}$}, indicating there are at least $(j-i+1)$ customers waiting.

The duration when there are only $(j-i+1)$ people from customer $i$ to customer $j$ are waiting is $t_{i,j} - t_{i,j+1}$, $i =2, \ldots, n-1, j \geq i$.

%  customer $i$ will arrive at $A_{i}$ but start service at $S_{i}$.

\textcolor{red}{Total overlapping time: $\sum_{i=2}^{n-1} \sum_{j=i}^{n-1} \gamma_{i,j} (t_{i,j} - t_{i,j+1})$}

Problem to minimize the total time cost:

\begin{equation}
    \begin{aligned}
        \min_{\mathbf{A}} \quad & E_{\xi}\left[\left(S_n-\sum_{i=1}^{n-1} \xi_i\right)+ \sum_{i=2}^{n-1} \sum_{j=i}^{n-1} \gamma_{i,j} (t_{i,j} - t_{i,j+1}) + \beta(S_{n} +\xi_{n} - T)^{+} \right] \\
        \mbox{s.t.} \quad & S_{i} = \max\{A_{i}, S_{i-1} + \xi_{i-1}\} \\
        & S_{1} = 0
    \end{aligned}
\end{equation}

To minimize the makespan:

\begin{equation}
    \begin{aligned}
        \min_{\mathbf{A}} \quad & E_{\xi}\left(S_n + \xi_{n}\right) \\
        \mbox{s.t.} \quad & E_{\xi}\left( t_{i,j} - t_{i,j+1} \right) \leq L_{ij}, i =2, \ldots, n-1, j \geq i
    \end{aligned}
\end{equation}

$L_{ij}$ indicates constraint on the duration of $(j-i+1)$ people waiting.


\section{Model}
We redefine the scheduling problem using the following notation. Let $\D = (\Delta_1, \ldots, \Delta_n)$ denote the \textbf{appointment intervals}, where $\Delta_{i}$ is the time allocated between the start of customer $i$ and customer $i+1$. Let $\A = (A_1, \ldots, A_n)$ represent the \textbf{scheduled appointment times}, with $A_{i} = \sum_{k=1}^{i-1} \Delta_{k}$ (assuming $A_{1}=0$). Let $\Z = (Z_1, \ldots, Z_n)$ be the \textbf{random service durations}, where $Z_{i}$ is the stochastic service time for customer $i$.

The waiting time for customer $i$ is recursively defined as:
\begin{align*}
  W_{i}(\Z_{i-1}, \D_{i-1}) &= [A_{i-1} + W_{i-1}(\Z_{i-2}, \D_{i-2}) + Z_{i-1} -A_{i}]^{+} \\
    & = [W_{i-1}(\Z_{i-2}, \D_{i-2}) + Z_{i-1} - \Delta_{i-1}]^{+},
\end{align*}
where $[*]^{+} = \max(*, 0)$. $W_{1}(\Z_{0}, \D_{0}) = 0$.

Let $W_{ij}$ denote the \textbf{simultaneous waiting duration} for customers $i$ through $j$, meaning the length of time during which all customers from $i$ to $j$ are simultaneously waiting. This can be expressed recursively as: $W_{i, j}(\Z_{i-1}, \D_{j-1}) = [W_{i-1}(\Z_{i-2}, \D_{i-2}) + Z_{i-1} - \sum_{k =i-1}^{j-1} \Delta_{k}]^{+}$, where $W_{i} \equiv W_{i,i}$ is the individual waiting time for customer $i$. $W_{ij}$ captures the time window where customers $i$ to $j$ all experience waiting simultaneously due to delays from earlier customers (1 to $i-1$) and insufficient buffer times.


The finish time for customer $i$ is $T_{i}(\Z_{i}, \D_{i-1}) = A_{i} + W_{i}(\Z_{i-1}, \D_{i-1}) + Z_{i}$. We aim to minimize the total schedule span, i.e., $T_{n}$, subject to constraints on individual and group waiting times.

The formulation of the problem can be expressed as follows

\begin{equation}\label{joint_waiting}
    \begin{aligned}
        \min \quad & E \left[T_n(\Z_{n}, \D_{n-1}) \right] \\
        \mbox{s.t.} \quad & E\left[W_{i,j}(\Z_{i-1}, \D_{j-1}) \right] \leq w_{ij}, i =2, \ldots, n-1, j \geq i
    \end{aligned}
\end{equation}


In this setting, $w_{ij}$ is related with the number of customers from $i$ to $j$, i.e., $j-i +1$. We can use $w_{k}$ to indicate the upper limit on the time when there are $k$ customers waiting.

$T_{n}(\Z_{n}, \D_{n-1}) = A_{n} + W_{n}(\Z_{n-1}, \D_{n-1}) + Z_{n}$.

\begin{lemma}
For any given realization of $\Z_{n}$, $T_n(\Z_{n}, \D_{n-1})$ becomes shorter when some customer is scheduled to arrive earlier while the schedule for others remain unchanged.
\end{lemma}

The optimal schedule can be obtained by minimizing $\Delta_{i}$.

When $i =1$, the first customer doesn't need to wait, i.e., $W_{1}(\Z_{0}, \D_{0}) = 0$.

When $i =2$, only one constraint $E\left[W_{1}(\Z_{0}, \D_{0}) + Z_{1}- \Delta_{1} \right]^{+} \leq w_{1}$ is applied, then $\Delta_{1}^{*}$ can be obtained. 

When $i =3$, there are two constraints on the waiting time of the third customer.

$E\left[W_{2}(\Z_{1}, \D_{1}^{*}) + Z_{2} - \Delta_{2}\right]^{+} \leq w_{1}$.

$E\left[W_{1}(\Z_{0}, \D_{0}) + Z_{1} - \Delta_{1}^{*} - \Delta_{2}\right]^{+} \leq w_{2}$.

Then $\Delta_{2}^{*}$ can be obtained.

When $i =4$, there are two constraints on the waiting time.

$E\left[W_{3}(\Z_{2}, \D_{2}^{*}) + Z_{3} - \Delta_{3}\right]^{+} \leq w_{1}$.

$E\left[W_{2}(\Z_{1}, \D_{1}) + Z_{2} - \Delta_{2}^{*} - \Delta_{3}\right]^{+} \leq w_{2}$.

% When $i =n$, there are $(n-1)$ constraints.

\begin{proposition}
    By solving the above problems sequentially, the optimal schedule can be obtained.
\end{proposition}

Then we analyze these problems. The function on the left-hand side is decreasing in the variable $\Delta_{i}$.

When $\Delta_{1} = 0$, $E\left[W_{1}(\Z_{0}, \D_{0}) + Z_{1}- \Delta_{1} \right]^{+} = E\left[Z_{1}\right]^{+}$. If $E\left[Z_{1}\right]^{+} \leq w_{1}$, $\Delta_{1}^{*} = 0$; if $E\left[Z_{1}\right]^{+} > w_{1}$, $E\left[Z_{1}- \Delta_{1}^{*} \right]^{+} = w_1$.

If $Z_{i}$ follows from the exponential distribution with rate $\lambda$, $E\left[Z_{1}- \Delta_{1} \right]^{+} = \frac{1}{\lambda} e^{-\lambda \Delta_{1}}$, then

\begin{equation*}
	\Delta_{1}^{*} = \begin{cases}
	-\frac{\ln (\lambda w_{1})}{\lambda}, & \text{if}~ \lambda w_{1} < 1 \\
	0, & \text{if}~ \lambda w_{1} \geq 1	
	\end{cases}
\end{equation*}

% The optimal schedule for \eqref{joint_waiting} is feasible for \eqref{only_waiting} because the feasible region of \eqref{joint_waiting} is smaller than that of \eqref{only_waiting}.

% \begin{equation}\label{only_waiting}
%     \begin{aligned}
%         \min \quad & E \left[T_n(\Z_{n}, \D_{n-1}) \right] \\
%         \mbox{s.t.} \quad & E\left[W_{i,j}(\Z_{i-1}, \D_{j-1})-W_{i,j+1}(\Z_{i-1}, \D_{j}) \right] \leq w_{ij}, i =2, \ldots, n-1, j \geq i
%     \end{aligned}
% \end{equation}


% Surgical Scheduling with Constrained customer Waiting Times
% Appointment Scheduling Under a Service-Level Constraint


\newpage

\section{Literature}
% https://www.qmatic.com/blog/minimize-clinic-waiting-times-to-avoid-virus-spread

1. Possible traits: heterogeneous customers, no-show, lateness, walk-in

Different models: objective: minimize the total cost, minimize the makespan (the departure time of the last customer).

% Two possible options: the time of several people waiting, what if it is not makespan

Traditional Appointment Scheduling Model.

1. with overbooking and no-shows (partial punctuality)

- discrete n time slots.

- minimize the waiting cost, idle time and overtime costs.

- analyze three components seperately


2. Under a service-level constraint (waiting time threshold)

- makespan

- the optimal schedule can be obtained sequentially.


% \section{Deterministic Situation}
% % For the $k$th customer, arriving at $(k-1)t$, 

% % 我们考虑现有的 appointment scheduling 中 waiting time

% Suppose there are $n$ customers to be scheduled. If the hard constraint-that certain individuals cannot be in the waiting room simultaneously-cannot be satisfied, the number of customers to be scheduled must be reduced until a feasible schedule is achieved.

% Each customer $i$ has a service time $Z_{i} \in [\underline{Z}_{i}, \overline{Z}_{i}]$. The goal is to maximize the number of scheduled customers $n$. The appointment time of the last customer does not exceed a given deadline $\overline{T}$. The number of simultaneous waiting customers, $S(t)$, never exceeds the capacity $N$ at any time $t$.

% This yields the following deterministic optimization problem:

% \begin{equation}\label{deterministic_model}
%     \begin{aligned}
%         \max \quad & n \\
%         \mbox{s.t.} \quad & W_{i}(\Z_{i-1}, \D_{i-1}) = [W_{i-1}(\Z_{i-2}, \D_{i-2}) + Z_{i-1} - \Delta_{i-1}]^{+}  \\
%         & S(t) \leq N, \forall t  \in [0, \overline{T}] \\
%         & A_{n} \leq \overline{T}
%     \end{aligned}
% \end{equation}


% To obtain the number of simultaneous waiting customers, we define $j^{*}(i)$ as the largest index $j$ satisfying: 
% $$(\overline{Z}_{i-1}- \sum_{i-1}^{j-1} \Delta_{k})> 0, i =2, \ldots, n$$ 
% where $\Delta_{k}$ represents the appointment interval for customer $k$. The number of simultaneous waiting customers can be obtained by $j^{*}(i) -i +1$. Given a fixed schedule $\D$, the maximum $j^{*}$ can be computed explicitly.


% When all customers are assigned their maximum service time ($Z_{i} = \overline{Z}_{i}$), the schedule that maximizes the number of customers $n$ is obtained by solving:

% $$n^{*} = \max\left\{n\bigg|\sum_{i=1}^{n} \overline{Z}_{i} \leq \overline{T}\right\} + N +1,$$ where $\overline{T}$ is the deadline and $N$ is the waiting room capacity.

% \begin{example}
%     $Z_{i} \in [20, 40]$ for each $i$. $\overline{T} =120$. $N =2$. $n^{*} = 3 + 2 +1$. The optimal schedule is $A_{1} = 0$, $A_2 = 40$, $A_3 = 80$, $A_4 =A_5 = A_6 = 120$.
% \end{example}

% The counterpart (dual) problem aims to minimize the schedule time $A_n$ of the last customer, given a fixed number of customers $n$, while respecting the waiting area capacity constraint. The optimization problem is formulated as:

% \begin{equation}
%     \begin{aligned}
%         \min \quad & A_{n} \\
%         \mbox{s.t.} \quad & W_{i}(\Z_{i-1}, \D_{i-1}) = [W_{i-1}(\Z_{i-2}, \D_{i-2}) + Z_{i-1} - \Delta_{i-1}]^{+} \\
%         & S(t) \leq N, \forall t \in [0, A_n]
%     \end{aligned}
% \end{equation}


% \begin{example}
%     $n = 6$. $Z_{i} \in [20, 40]$ for each $i$. $N =2$. The optimal schedule is $A_{1} = 0$, $A_2 = 40$, $A_3 = 80$, $A_4 =A_5 = A_6 = 120$.
% \end{example}

% For any given realization of $\Z_n$, an optimal schedule is $A_1 = 0$, $A_i = \sum_{k=1}^{i-1} Z_k, 2 \leq i \leq n-N$, $A_{n-N+1} = \ldots = A_{n} = \overline{T}$.


% \section{Stochastic Situation}
% The soft constraint can be set as the expected number of simultaneous waiting people does not exceed certain number. 

% Or the probability of the largest number of simultaneous waiting people is less than a threshold.

% \begin{equation}\label{stochastic_model}
%     \begin{aligned}
%         \max \quad & n \\
%         \mbox{s.t.} \quad & E[S(t)] \leq N, \forall t \\
%         & A_{n} \leq \overline{T}
%     \end{aligned}
% \end{equation}

% \begin{equation}\label{stochastic}
%     \begin{aligned}
%         \min \quad & E \left[T_n(\Z_{n}, \D_{n-1}) \right] \\
%         \mbox{s.t.} \quad & E\left[S(t) \right] \leq N, \forall t
%     \end{aligned}
% \end{equation}

% The constraint $E\left[S(t) \right] \leq N, \forall t$ is equivalent to $E\left[S(A(i)) \right] \leq N, i=2,\ldots, n$.

% $S(A(i)) = \max\{j|W_{i-1} + Z_{i-1}-\sum_{i-1}^{j-1} \Delta_{k} >0\} - i + 1$.


% $T_{i}(\Z_{i}, \D_{i-1}) = \sum_{k=1}^{i-1} \Delta_{k} + W_{i}(\Z_{i-1}, \D_{i-1}) + Z_{i}$

% $W_{i}(\Z_{i-1}, \D_{i-1}) = [W_{i-1}(\Z_{i-2}, \D_{i-2}) + Z_{i-1} - \Delta_{i-1}]^{+}$

% If the constraint is hard, for each $i$, we have $\max\{j|W_{i-1} + Z_{i-1}-\sum_{i-1}^{j-1} \Delta_{k} >0\} - i + 1 \leq N$. The constraints can be converted to $Z_{i-1} + W_{i-1} \leq \sum_{i-1}^{N-2+i} \Delta_{k}, i =2 \ldots, n$.


\section{Some Instances}
We evaluate and compare three constraint scenarios: 

\begin{itemize}
    \item  Waiting Time Constraint: Limits the maximum waiting time for each customer.
    \item  Overlapping Time Constraint: Restricts two simultaneous waiting overlaps.
    \item  Combined Constraints: Enforces both waiting time and overlapping time limits.
\end{itemize}

In all cases, the schedule times of later customers do not affect the appointments of preceding customers, ensuring feasibility in sequential scheduling.

Setting the mean of simultaneous waiting time to be no larger than zero (requiring that the number of people never exceeds a certain threshold) is an overly stringent condition. In systems without a maximum service time constraint, this requirement would significantly delay the scheduled time compared to systems that only consider the single waiting time. Moreover, when extreme situations occur (the service times of the earlier customers are extremely large), the scheduled time would be postponed even further.

% direction: put the overlapping time in objective function or put it in the constraint.
An alternative approach is to set a threshold on the probability of the number of waiting customers not exceeding a certain limit.

% punctuality. congestion effect. pre-empt policy. walk-in. fairness. priority. patient preference.

The original model considers the overlapping time,

\begin{equation}\label{single_overlap}
    \begin{aligned}
        \min \quad & E[T_{n}(\Z_{n}, \D_{n-1})] \\
        \mbox{s.t.} \quad & E[W_{i,i+1}(\Z_{i-1}, \D_{i})] \leq w, i=2,\ldots, n-1
    \end{aligned}
\end{equation}

Wells-Riley model: The probability $P$ of infection in a shared space is given by:

$P=1-e^{-I \cdot q \cdot p \cdot t / Q}$

I: Number of infectious individuals
q: Quanta emission rate
p: Mask penetration factor
t: Exposure time
Q: Room ventilation rate

The overlapping constraint is not stringent/sufficient. Although setting $\Delta_1 = 0$ minimizes $T_n$, increasing $\Delta_1$ can significantly reduce Customer 2's waiting time while only slightly increasing $T_n$.

\begin{table}[ht]
    \centering
    \caption{Schedule Intervals and Times}
    \begin{tabular}{ccccccc}
    \hline
    \hline
    Waiting Model  & $\Delta_1= 0.72$ & $\Delta_2= 31.1$ & $\Delta_3=31.3$ & $\Delta_4 = 32.3$ & $\Delta_5=32.5$ & $T_{n} = 188.76$ \\
    Waiting  &  0 & 30  & 30  & 30 & 30 & 30 \\
    Overlapping & 0 &  5.2 & 7.2 & 8.2 &  8.9 & \\ 
    \hline
    Overlapping Model  & $\Delta_1= 0.1$ & $\Delta_2= 27.4$ & $\Delta_3 = 35.1$ & $\Delta_4= 34.7$ & $\Delta_5 = 34.6$ & $T_{n} = 189.53$ \\
    Waiting   & 0  & 30.6 & 34.0 & 30.3 & 30.5 & 26.8 \\
    Overlapping & 0 & 7.4 & 7.4 & 7.4 & 7.4 & \\
    \hline
    Waiting Model  & $\Delta_1= 0.72$ & $\Delta_2= 31.1$ & $\Delta_3=31.3$ & $\Delta_4 = 32.3$ & $T_{n} = 156.36$ \\
    Waiting  & 0 & 30  & 30  & 30 & 30  & total(120)\\
    Overlapping & 0 &  5.2 & 7.2 & 8.2  &  & total(20.6)\\ 
    \hline
    Overlapping Model  & $\Delta_1= 0.42$ & $\Delta_2= 27.11$ & $\Delta_3 = 35.1$ & $\Delta_4= 34.7$ & $T_{n} = 156.56$ \\
    Waiting  &  0  & 30.5 & 34.0  & 32.1 & 28.3 & total(125.0)\\
    Overlapping & 0 & 7.4 & 7.4 & 7.4  &  & total(22.2)\\
    \hline
    Overlapping (fixed $\Delta_1$)  & $\Delta_1= 5$ & $\Delta_2= 22.53$ & $\Delta_3 = 35.12$ & $\Delta_4= 34.67$ & $T_{n} = 156.58$ \\
    Waiting  &  0  & 25.8 & 34.1  & 32.2 & 28.3 & total(120.34)\\
    Overlapping & 0 & 7.4 & 7.4 & 7.4  &  & total(22.2)\\
    \hline
    Overlapping (fixed $\Delta_1$)  & $\Delta_1= 8$ & $\Delta_2= 19.53$ & $\Delta_3 = 35.13$ & $\Delta_4= 34.69$ & $T_{n} = 156.63$ \\
    Waiting  &  0  & 22.9 & 34.2  & 32.2 & 28.4 & total(117.6) \\
    Overlapping & 0 & 7.4 & 7.4 & 7.4  &  & total(22.2) \\
    \hline
    \end{tabular}
  \end{table}

% Number of customers is large.

Now we consider the waiting cost, idle cost and overtime cost.

\begin{equation}
    \label{three_term_model}
    \begin{aligned}
        \min_{\D} \quad & \left[c_{w} E_{w} + c_{i} E_{i} + c_{o} E_{o} \right] \\
        \mbox{s.t.} \quad & W_{i} = (W_{i-1} + Z_{i-1} - \Delta_{i-1})^{+} \\
        & W_{i, i+1} = (W_{i-1} + Z_{i-1} - \Delta_{i-1}- \Delta_{i})^{+} \\
        & E(W_{i,i+1}) \leq w, i = 2, \ldots, n-1
    \end{aligned}
\end{equation}

$E_{w} = \sum_{i=2}^{n} E(W_{i})$

$E_{i} = E(\sum_{i=1}^{n-1} (\Delta_{i} -Z_{i}) + W_{n})$

$E_{o} = E(\sum_{i=1}^{n-1}\Delta_{i} + W_{n}+ Z_{n} - T)^{+}$


We investigate several key questions:

When does the overlapping time constraint become active? (i.e., under what conditions does the overlapping condition take effect?) How does this constraint affect appointment scheduling? Is there an optimal approach under this constraint?


For a fixed sum $c_{i} + c_{w}$, the optimal schedule does not vary with changes in $c_{i}$ or $c_{w}$.

When $c_{w} = 0$, problem \eqref{three_term_model} is equivalent to problem \eqref{single_overlap}.

\begin{table}[ht]
    \centering
    \caption{Schedule Intervals and Times}
    \begin{tabular}{cccccccc}
    \hline
    \hline
    $c_{w}$ & $c_{i}$ & $\Delta_1$ & $\Delta_2$ & $\Delta_3$ & $\Delta_4$ & $\Delta_5$ & Largest overlap time/Threshold  \\
    \hline
    5  &  1  & 22 & 31 & 30 & 28 & 14 & 6.3462 \\
    \hline
    1  &  1  & 19 & 29 & 28 & 28 & 21 & 5.5232 \\
    \hline
    1  &  5  &  6 & 15 & 17 & 17 & 13 & 21.5018  \\
    1  &  5  &  6 & 15 & 18 & 18.61 & 17 & 18 \\
    1  &  5  & 6 & 15  & 18 & 21 & 20.23 & 15 \\
    1  &  5  & 6 & 17  & 19 & 23 & 24.10 & 12 \\
    1  &  5  & 6 & 18  & 20.83 & 26.03 & 27.97 & 9 \\
    1  &  5  & 7 & 20.25  & 27 & 28.64 & 30.15 & 6 \\
    1  &  5  & T = 125 &  &    &       &       &  Infeasible  $< 4$ \\
    \hline
    \end{tabular}
  \end{table}


For each $i$, $\Delta_{i}^{*}$ increases when the threshold of waiting/overlapping time decreases.

For the optimal schedule, the expected waiting/overlapping time ($E(W_{i})$, $E(W_{i,i+1})$) is increasing in $i$.

% If the service time is integral, then the optimal schedule interval is also integral.

For the first question, we need to find the largest overlapping time.

% \section{Overlapping Model}
% For the above model, only the seperate overlapping period is considered. From the perspective of customer $i+1$, the total overlapping time can be calculated by $E[W_{i,i+1}(\Z_{i-1}, \D_{i})] + E[W_{i+1,i+2}(\Z_{i-1}, \D_{i})]$.

% \begin{equation}\label{total_overlap}
%     \begin{aligned}
%         \min \quad & E[T_{n}(\Z_{n}, \D_{n-1})] \\
%         \mbox{s.t.} \quad & E[W_{i-1,i}(\Z_{i-1}, \D_{i}) + W_{i,i+1}(\Z_{i}, \D_{i+1})] \leq w, i=2,\ldots, N-1
%     \end{aligned}
% \end{equation}


% Several conclusions:
% For the waiting time constraint model: the average individual waiting time remains stable, but the overlapping duration grows, leading to elevated transmission risk.

% For the overlapping time model: while this model may marginally extend waiting times for a subset of customers, it effectively mitigates overall transmission risk by controlling simultaneous waiting.

% Since $E[W_{23}] \leq w$ and $E[W_{N-1, N}] \leq w$ are redundant, there are $(N-3)$ effective constraints, which gives more options for the $(N-1)$ decision variables.

% Since $T_{n}$ is increasing in $\D$. $\Delta_{1} = 0$ always makes the problem feasible, thus $\Delta_{1}^{*} =0$ is one of the optimal partial solution.

% Let the optimal second schedule interval be $\Delta_{2}^{*}$, when $\Delta_{2} > \Delta_{2}^{*}$, $T_{n} > T_{n}^{*}$; when $\Delta_{2} < \Delta_{2}^{*}$, $T_{n} > T_{n}^{*}$.

% % $T_{n}^{*}$ is convex in $\Delta_{2}$.

% Lagrangian function:
% $$L(\D; \bm{\lambda}) = T_{n} + \lambda_{1} (E[W_{23} - w]) + \lambda_{2} (E[W_{23} + W_{34} - w]) + \lambda_{3}(E[W_{34} + W_{45} - w])$$

% From the literature, we know that
% $$\frac{\partial T_n}{\partial \Delta_{i}} = 1- \prod_{k=i+1}^n 1_{\left\{W_k\left(\mathbf{Z}_{k-1}, \boldsymbol{\Delta}_{k-1}\right)>0\right\}}.$$

% Take the partial derivative with respect to $\Delta_{2}$,
% $$\frac{\partial L(\D; \bm{\lambda})}{\partial \Delta_2} = 1- \prod_{k = 3}^n 1_{\left\{W_k\left(\mathbf{Z}_{k-1}, \boldsymbol{\Delta}_{k-1}\right)>0\right\}} - (\lambda_{1}+ \lambda_{2}) 1_{\{W_{23}>0\}} - (\lambda_{2}+ \lambda_{3}) 1_{\{W_{34}>0\}}- \lambda_{3} 1_{\{W_{45}>0\}} 1_{\{W_{3}>0\}}$$ 

% When $W_{ij}=0$ or $W_{k}=0$, the second derivative does not exist. When $W_{ij} \neq 0$ and $W_{k} \neq 0$, the second derivative is 0. This indicates that the Lagrangian function (or $T_{n}$) is piecewise linear in $\Delta_2$.

% The sum of finite piecewise linear functions remains a piecewise linear function.

% The sum of finite piecewise linear functions must attain a minimum on the closed interval. 

% The minimum value will be obtained at the break points.

% $\Delta_{2}^{*}$ can be obtained by bi-section and bound?

% Then, $E[W_{23}]^{*} = E[(Z_{1}- \Delta_{2}^{*})^{+}]$ can be obtained. $E[W_{34}]^{*} = w - E[W_{23}]^{*}$, then $\Delta_{3}^{*}$ can be calculated.

The Wells-Riley model quantifies infection risk based on quanta emission, exposure time, masking, and ventilation. After an infected person leaves, aerosol risk drops significantly within minutes if ventilation is adequate.

\end{document}
