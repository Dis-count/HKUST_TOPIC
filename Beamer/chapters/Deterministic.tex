% !TeX root = ../main.tex

\section{Deterministic Model}

    \frame{\sectionpage}

    \begin{frame}{Formulation}
      When $|\Omega| =1$ in problem \eqref{sto_form}, the stochastic programming will be 
      \small
      \begin{equation}\label{one_form}
        \begin{aligned}
        \max \quad & \sum_{i=1}^{M}  \sum_{j= 1}^{N} (n_i-s) x_{ij} - \sum_{i=1}^{M} y_{i}^{+}  \\
        \text {s.t.} \quad & \sum_{j= 1}^{N} x_{ij} - y_{i}^{+}+ y_{i+1}^{+} + y_{i}^{-} = d_{i}, \quad i \in [M-1], \\
        & \sum_{j= 1}^{N} x_{ij} -y_{i}^{+} + y_{i}^{-} = d_{i}, \quad i = M, \\
        & \sum_{i=1}^{M} n_{i} x_{ij} \leq L_j, j \in [N]\\
        & y_{i}^{+}, y_{i}^{-} \in \mathbb{Z}_{+}, \quad i \in [M] \\
        & x_{ij} \in \mathbb{Z}_{+}, \quad i \in [M], j \in [N].
        \end{aligned}
      \end{equation}
    \end{frame}

    \begin{frame}{Formulation}
      \begin{equation}\label{deter_upper}
        \begin{aligned}
        \max \quad & \sum_{i=1}^{M}  \sum_{j= 1}^{N} (n_i- s) x_{ij} \\
        \text {s.t.} \quad & \sum_{j= 1}^{N} x_{ij} \leq d_{i}, \quad i \in [M], \\
        & \sum_{i=1}^{M} n_{i} x_{ij} \leq L_j, j \in [N] \\
        & x_{ij} \in \mathbb{Z}_{+}, \quad i \in [M], j \in [N].
        \end{aligned}
      \end{equation}
    \end{frame}

    \begin{frame}{Analysis}

    \end{frame}


    \begin{frame}{Branch and Price}
        \begin{itemize}
          \item Solve restricted master problem with the initial columns.
          \item Use pricing problem to generate columns.
          \item When column generation terminates, is the solution integral?

          Yes, update lower bound.

          No, update upper bound. And fathom node or branch to create two nodes.
          \item Select the next node until all nodes are explored.
        \end{itemize}
    \end{frame}

    \begin{frame}{Properties}
      \begin{itemize}
        \item When column generation gives an integer solution at the first time, then we obtain the optimal solution.
        \item $\alpha_k$ indicates the number of items for pattern $k$. $\beta_k$ indicates the left space for maximal pattern $k$. Notice that the left space is the true loss.
        \item Denote $(\alpha_k + \beta_k)$ as the loss for pattern k, $l(k)$. When $l(k)$ reaches minimum, the corresponding pattern $k$ is the optimal solution for a single line.
        \item If the group sizes are consecutive integers starting from 2, $\{2,3,\ldots,u\}$, then a greedy-based pattern is optimal, i.e., select the maximal group size,$u$, as many as possible and the left space is occupied by the group with the corresponding size. The loss is $k+1$, where $k$ is the number of times $u$ selected. Let $S = u\cdot k + r$, when $r>0$, we will have at least $\lfloor \frac{r+u}{2} \rfloor -r +1$ optimal patterns with the same loss. When $r =0$, we have only one optimal pattern.
      \end{itemize}
    \end{frame}

    \begin{frame}
      \begin{itemize}
        \item Let $I_1$ be the set of patterns with the minimal loss. And $I_i$ be the set of patterns with $i$-th minimal loss.
        \item When supply is less than demand, we must select $I_1$.
        \item When $N \leq \max_{k\in I_1} \min_{i} \{\lfloor \frac{d_i}{b_i^k}\rfloor\}$, select the corresponding pattern from $I_1$ and it is the optimal solution.
        $N$ is the number of lines, $i = 1,2,\ldots, m$, $d_m$ is the demand of the largest size, $b_m^k$ is the number of group $m$ placed in pattern $k$.

      \end{itemize}
    \end{frame}

    \begin{frame}{Integral Decomposition}
      \begin{itemize}
        \item Denote by $R_+^n$ the nonnegative orthant of Euclidean n-space. A polyhedron $P \subset R_+^n$ is called lower comprehensive if $0 \leq y \leq x \in P$ implies $y \in P$.
        \item Let $M$ be the matrix whose rows are the maximal integral points of P.
        \item An integral vector $x \in P$ is a maximal integral point of $P$ if there is no integral vector $y \in P$ with $x \neq y \geq x$.
        \item Let $kP$ be defined as $\{kx|x \in P\}$.
        The integral decomposition property holds for
        $P$ if, for each integer $k \geq 1$ and each integral vector $x \in kP$, there exist integral vectors $x^i \in P, 1 \leq i \leq k$, for which $x = \sum_{i=1}^k x^i$.
        \item The knapsack polyhedron, denoted by $P$, is the convex hull of the set $\{x \in Z_+^n|ax \leq b\}$.
      \end{itemize}
    \end{frame}

    \begin{frame}{Integer Round Up for CSP}
      \begin{itemize}
        \item Cutting stock problem has the integer round-up property if and only if P has the integral decomposition property.
        \item Any two-dimension CSP has the IRU property.
        \item The numerical experiment shows that the form of $\{2,3,\ldots,n\}$ has the IRU property.
        \item But we can check this property finitely. From S. BAUM and L.E. TROTTER.
      \end{itemize}
    \end{frame}

    \begin{frame}{Integer Round Up for CSP}
      \begin{itemize}
        \item After the column generation gives the LP solution(fractional), calculate the supply quantity. If the number is integral, keep it.
        If the supply is not integral, we can construct an integer vector which can provide the largest integral profit.
        \item Construction: Calculate the space taken by fractional supply(the value must be integer), increase the corresponding supply having the same space, delete the fractional part.
        \item Now we have an integral supply vector. If this case has IRU property, there exists an integral decomposition.
        \item Minus the integral part of the patterns from the supply. Check if the rest can be placed in several lines.
      \end{itemize}
    \end{frame}


    \begin{frame}{}
      \[\begin{split}\mbox{min}\quad & \sum y \\
      \mbox{s.t.} \quad & yM \geq w \\
      & y \geq 0 \mbox{ and integer}\end{split}\]

      \[\begin{split}\mbox{max}\quad & c w \\
      \mbox{s.t.} \quad & w \in \{Nx| x \in P\} \\
      & w \leq w_0\end{split}\]

      When P has the integral decomposition property, there exist $N$ integer vectors decomposing every $w$.
      Now, the question is how to find the maximal $w$.
    \end{frame}


    \begin{frame}{Lower Bound for Demand}
      \[\begin{split}\mbox{max}\quad & \sum_{k=1}^K(\sum_{i=1}^m (s_i-1)t_i^k) x_{k}\\
      \mbox{s.t.} \quad & \sum_{k=1}^K x_{k} \leq N \\
      & \sum_{k=1}^K t_i^k x_k \leq g_i,\quad i=1,\ldots,m\\
      & \sum_{k=1}^K t_i^k x_k \geq p_i,\quad i=1,\ldots,m\\
      & x_{k} \geq 0, \quad k = 1,\ldots,K \\
      \end{split}\]
      \begin{itemize}
        \item Use the similar method to solve.
      \end{itemize}
    \end{frame}

    \begin{frame}{Sub-problem}
      \[\begin{split}\mbox{max}\quad & \sum_{i=1}^m \left[(s_i-1) -\lambda_i - \mu_i \right] y_{i} - \tau \\
      \mbox{s.t.} \quad & \sum_{i=1}^m s_i y_i \leq S  \\
      & y_i \geq 0, \mbox{integer}\quad \mbox{for}~ i=1,\ldots,m.\\\end{split}\]
      \begin{itemize}
        \item Use column generation to generate a new pattern until all reduced costs are negative.
      \end{itemize}
    \end{frame}

    \begin{frame}{IP Formulation}
      \[\begin{split}\mbox{max}\quad & \sum_{j=1}^{m} \sum_{i=1}^n (s_i-1) x_{ij} \\
      \mbox{s.t.} \quad & \sum_{i=1}^n s_i x_{ij} \leq S, \quad j=1,\ldots,m \\
      & p_i \leq \sum_{j=1}^{m} x_{ij} \leq g_i ,\quad i=1,\ldots,n \\
      & x_{ij} \geq 0 \mbox{ and integer}, \quad i=1,\ldots,n, j=1,\ldots,m \\\end{split}\]
      \begin{itemize}
        \item $m$ indicates the number of lines.
        \item $x_{ij}$ indicates the number of group type $i$ placed in each line $j$.
      \end{itemize}
    \end{frame}

    \begin{frame}{Question}
        1. Feasibility. At first, use cutting stock model to check if the following constraints provide a feasible solution. If the minimal number of lines we need is less than the given number of lines, these constraints are feasible.
        \[\begin{split}
        & \sum_{i=1}^n s_i x_{ij} \leq S, \quad j=1,\ldots,m \\
        & p_i \leq \sum_{j=1}^{m} x_{ij},\quad i=1,\ldots,n
        \end{split}\]

    \end{frame}


    % \begin{frame}{Extension}
    %
    % \end{frame}
