% !TEX root = sum1.tex
\section{Deterministic Model}
Firstly, we consider the deterministic model under social distancing constraints. When we have precise information about the number of people, we can utilize this model to arrange seats accordingly. For instance, during a company meeting where different group members need to sit together, we can determine the venue size based on a fixed number of attendees or accept only a portion of the demand based on the venue's capacity. The objective is to maximize the number of people sitting. The solver can solve this problem quickly with the moderate problem size. However, we find that the results show that most rows will not leave any empty seats. Based on this, we introduce the concepts of the largest pattern and full pattern for each row, and we can always use the largest pattern or full pattern as the optimal solution to meet the demand.


Let $x_{ij}$ represent the number of group type $i$ planned in row $j$. The deterministic seat planning problem is formulated below, with the objective of maximizing the number of people accommodated.

\begin{equation}\label{deter_upper}
    \begin{aligned}
    \max \quad & \sum_{i=1}^{M}  \sum_{j= 1}^{N} (n_i- \delta) x_{ij} \\
    \text {s.t.} \quad & \sum_{j= 1}^{N} x_{ij} \leq d_{i}, \quad i \in \mathcal{M}, \\
    & \sum_{i=1}^{M} n_{i} x_{ij} \leq L_j, j \in \mathcal{N}, \\
    & x_{ij} \in \mathbb{Z}^{+}_{0}, \quad i \in \mathcal{M}, j \in \mathcal{N}.
    \end{aligned}
  \end{equation}
  
This seat planning problem can be regarded as a special case of the multiple knapsack problem. In this context, we define $\bm{X}$ as the aggregate solution, where $\bm{X} = (\sum_{j=1}^{N} x_{1j}, \ldots, \sum_{j=1}^{N} x_{Mj})^T$. Each element of $\bm{X}$, $\sum_{j=1}^{N} x_{ij}$, represents the available supply for group type $i$.
  
In other words, $\bm{X}$ captures the number each group type that can be allocated to the seat layout by summing up the supplies across all rows. By considering the monotone ratio between the original group sizes and the adjusted group sizes, we can determine the upper bound of supply corresponding to the optimal solution of the LP relaxation of Problem \eqref{deter_upper}, as demonstrated in Proposition \ref{sol_relax_deter}.

Although the problem size is small and the optimal solution can be easily obtained using a solver, it is still important to analyze the problem further to gain additional insights and understanding.


In many cases, the optimal solution for the seat planning problem tends to involve rows with either full patterns or the largest patterns. Distinguishing these patterns from other configurations can provide valuable insights into effective seat planning strategies that prioritize accommodating as many people as possible while adhering to social distancing guidelines.

When there is high demand for seats, it is advantageous to prioritize the largest patterns. These patterns allow for the accommodation of the largest number of individuals due to social distancing requirements. On the other hand, in scenarios with moderate demand, adopting the full pattern becomes more feasible. The full pattern maximizes seating capacity by utilizing all available seats, except those empty seats needed for social distancing measures. By considering both the largest and full patterns, we can optimize seat planning configurations to efficiently accommodate a significant number of individuals while maintaining adherence to social distancing guidelines. 
  

Although the optimal solution to the seat planning problem is complex, the LP relaxation of problem \eqref{deter_upper} has a nice property.

\begin{prop}\label{sol_relax_deter}
In the LP relaxation of problem \eqref{deter_upper}, there exists an index $\tilde{i}$ such that the optimal solutions satisfy the following conditions:

\begin{itemize}
\item For $i = 1,\ldots, \tilde{i}-1$, $x_{ij}^{*} = 0$ for all rows, indicating that no group type $i$ are assigned to any rows before index $v$.
\item For $i = \tilde{i}+1,\ldots, M$, the optimal solution assigns $\sum_{j} x_{ij}^{*} = d_{i}$ group type $i$ to meet the demand for group type $i$.
\item For $i = \tilde{i}$, the optimal solution assigns $\sum_{j} x_{ij}^{*} = \frac{L - \sum_{i = \tilde{i}+1}^{M} {d_i n_i}}{n_{\tilde{i}}}$ group type $\tilde{i}$ to the rows. This quantity is determined by the available supply, which is calculated as the remaining seats after accommodating the demands for group types $\tilde{i}+1$ to $M$, divided by the size of group type $\tilde{i}$, denoted as $n_{\tilde{i}}$.
\end{itemize}

Hence, the corresponding supply values can be summarized as follows: $X_{\tilde{i}} = \frac{L - \sum_{i = \tilde{i}+1}^{M} {d_i n_i}}{n_{\tilde{i}}}$, $X_{i} = d_{i}$ for $i = \tilde{i} +1,\ldots, M$, and $X_{i} = 0$ for $i = 1, \ldots, \tilde{i}-1$. These supply values represent the allocation of seats to each group type.
\end{prop}