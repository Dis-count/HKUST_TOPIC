% !TEX root = sum1.tex

% extension? customer selection

% use the example to explain fixed and flexible.

\section{Dynamic Situation}\label{sec_dynamic_seat}

Thirdly, we consider the dynamic model under social distancing constraints. In this scenario, we encounter two different situations. The first situation involves a fixed seat planning that is set based on the management's requirements. When a group arrives, they can choose seats from the available planning options. The predetermined seat arrangements ensure that social distancing measures are maintained, and groups can select seats that best suit their needs while adhering to the established seating arrangement. The second situation involves a flexible seat planning approach, where decisions need to be made when a group requests seats. In this case, we dynamically determine the optimal seat planning based on the group's size and the current availability of seats, taking into account social distancing requirements. By utilizing the dynamic model and considering both fixed and flexible seat planning approaches, we can effectively manage the seating arrangements while adhering to social distancing guidelines.


By considering these different models under social distancing constraints, we can effectively allocate seats and ensure a safe and comfortable environment for all attendees.

\subsection{Make the instant allocation}
In a more realistic scenario, groups arrive sequentially over time, and the seller must promptly make group assignments upon each arrival while maintaining the required spacing between groups. When a group is accepted, the seller must also determine which seats should be assigned to that group. It is essential to note that each group must be either accepted in its entirety or rejected entirely; partial acceptance is not permitted. Once the seats are confirmed and assigned to a group, they cannot be changed or reassigned to other groups.

To model this problem, we adopt a discrete-time framework. Time is divided into $T$ periods, indexed forward from $1$ to $T$. We assume that in each period, at most one group arrives and the probability of an arrival for a group of size $i$ is denoted as $p_i$, where $i$ belongs to the set $\mathcal{M}$. The probabilities satisfy the constraint $\sum_{i=1}^M p_i \leq 1$, indicating that the total probability of any group arriving in a single period does not exceed one. We introduce the probability $p_0 = 1 - \sum_{i=1}^{M} p_i$ to represent the probability of no arrival in a given period $t$. To simplify the analysis, we assume that the arrivals of different group types are independent and the arrival probabilities remain constant over time. This assumption can be extended to consider dependent arrival probabilities over time if necessary.

The state of remaining capacity in each row is represented by a vector $\mathbf{L} = (l_1, l_2, \ldots, l_N)$, where $l_j$ denotes the number of remaining seats in row $j$. Upon the arrival of a group type $i$ in period $t$, the seller needs to make a decision denoted by $u_{i,j}^{t}$, where $u_{i,j}^{t} = 1$ indicates acceptance of group type $i$ in row $j$ during period $t$, while $u_{i,j}^{t} = 0$ signifies rejection of that group type in row $j$ at that period. The feasible decision set is defined as $$U^{t}(\mathbf{L}) = \{u_{i,j}^{t} \in \{0,1\}, \forall i \in \mathcal{M}, \forall j \in \mathcal{N} | \sum_{j=1}^{N} u_{i,j}^{t} \leq 1, \forall i \in \mathcal{M}; n_{i}u_{i,j}^{t}\mathbf{e}_j \leq \mathbf{L}, \forall i \in \mathcal{M}, \forall j \in \mathcal{N}\}.$$ Here, $\mathbf{e}_j$ represents an N-dimensional unit column vector with the $j$-th element being 1, i.e., $\mathbf{e}_j = (\underbrace{0, \cdots, 0}_{j-1}, 1, \underbrace{0, \cdots, 0}_{n-j})$. In other words, the decision set $U(\mathbf{L})$ consists of all possible combinations of acceptance and rejection decisions for each group type in each row, subject to the constraints that at most one group of each type can be accepted in any row, and the number of seats occupied by each accepted group must not exceed the remaining capacity of the row.


Let $V^{t}(\mathbf{L})$ denote the maximum expected revenue earned by the best decisions regarding group seat assignments in period $t$, given remaining capacity $\mathbf{L}$. Then, the dynamic programming formula for this problem can be expressed as:

\begin{equation}\label{DP}
V^{t}(\mathbf{L}) = \max_{u_{i,j}^{t} \in U^{t}(\mathbf{L})}\left\{ \sum_{i=1}^{M} p_i ( \sum_{j=1}^{N} i u_{i,j}^{t} + V^{t+1}(\mathbf{L}- \sum_{j=1}^{N} n_i u_{i,j}^{t}\mathbf{e}_j)) + p_0 V^{t+1}(\mathbf{L})\right\}
\end{equation}
with the boundary conditions $V^{T+1}(\mathbf{L}) = 0, \forall \mathbf{L}$ which implies that the revenue at the last period is 0 under any capacity.

At the beginning of period $t$, we have the current remaining capacity vector denoted as $\mathbf{L} = (L_1, L_2, \ldots, L_N)$. Our objective is to make group assignments that maximize the total expected revenue during the horizon from period 1 to $T$ which is represented by $V^{1}(\mathbf{L})$.

Solving the dynamic programming problem described in equation \eqref{DP} can be challenging due to the curse of dimensionality, which arises when the problem involves a large number of variables or states. To mitigate this complexity, we aim to develop a heuristic method for assigning arriving groups. In our approach, we begin by generating a seat planning that consists of the largest or full patterns, as outlined in section \ref{sec_seat_planning}. This initial seat planning acts as a foundation for our heuristic method. In section \ref{sec_dynamic_seat}, building upon the generated seat planning, we further develop a dynamic seat assignment policy which guides the allocation of seats to the incoming groups sequentially. 



\subsubsection{Based on the flexible seat planning}



\subsection{Make the instant decision but late allocation}


% dynamic assignment xij 固定   基于  design  座位封死

% xij  不固定  只做决策  最后分配座位

% customer perference  选择

% 从问题的角度来思考