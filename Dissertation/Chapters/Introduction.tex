% !TEX root = sum1.tex
\section{Introduction}

\subsection{Background}
Governments worldwide have been faced with the challenge of reducing the spread of Covid-19 while minimizing the economic impact. Social distancing has been widely implemented as the most effective non-pharmaceutical treatment to reduce the health effects of the virus. This website records a timeline of Covid-19 and the relevant epidemic prevention measures \cite{Covid19Timeline}. 

For instance, in March 2020, the Hong Kong government implemented restrictive measures such as banning indoor and outdoor gatherings of more than four people, requiring restaurants to operate at half capacity. As the epidemic worsened, the government tightened measures by limiting public gatherings to two people per group in July 2020. As the epidemic subsided, the Hong Kong government gradually relaxed social distancing restrictions, allowing public group gatherings of up to four people in September 2020. In October 2020, pubs were allowed to serve up to four people per table, and restaurants could serve up to six people per table. Specifically, the Hong Kong government also implemented different measures in different venues \cite{Gov202209}.

For example, the catering businesses will have different social distancing requirements depending on their mode of operation for dine-in services. They can operate at 50\%, 75\%, or 100\% of their normal seating capacity at any one time, with a maximum of 2, 2, or 4 people per table, respectively. Bars and pubs may open with a maximum of 6 persons per table and a total number of patrons capped at 75\% of their capacity. The restrictions on the number of persons allowed in premises such as cinemas, performance venues, museums, event premises, and religious premises will remain at 85\% of their capacity.

The measures announced by the Hong Kong government mainly focus on limiting the number of people in each group and the seat occupancy rate. However, implementing these policies in operations can be challenging, especially for venues with fixed seating layouts. In our study, we will focus on addressing this challenge in commercial premises, such as cinemas and music concert venues. We aim to provide a practical tool for venues to optimize seat assignments while ensuring the safety of groups by proposing a seat assignment policy that takes into account social distancing requirements and the given seating layout. We strive to enable venues to implement social distancing measures effectively by offering a solution that provides specific seating arrangements.

\subsection{Categories}
We can categorize the seating situations into three distinct categories based on the demand and the corresponding decision-making process.

% For the deterministic situation, the full information of demand is given. We should give a seat planning to maximize the number of people accommodated. This situation corresponds to the basic problem of assigning more people with the fixed seat layout, such as in a church, a company meeting. For the stochastic situation, the demand distribution is known before the realization of true demand. We give the seat planning to maximize the expected number of people accommodated. It applies in the place where the seat has been provided in advance in case people don;t follow the rules. For the dynamic situation, we should accept or reject the group for each coming group, according to the different requirement, we may assign the seats to the group immediately or later. This situation is more common in cinema or music concert where the decison can be made for each coming group.

1. Deterministic Situation: In this scenario, we have complete and accurate information about the demand for seating. We aim to provide a seat planning that maximizes the number of people accommodated. This situation is applicable in venues like churches or company meetings, where fixed seat layouts are available, and the goal is to assign seats to accommodate as many people as possible within the given layout.

2. Stochastic Situation: In this situation, we have knowledge of the demand distribution before the actual demand is realized. We aim to generate a seat planning that maximizes the expected number of people accommodated. This approach is suitable for venues where seats have been pre-allocated to ensure compliance with social distancing rules. By considering the expected demand distribution, we can optimize the seat planning to accommodate the maximum number of people while maintaining social distancing.

3. Dynamic Situation: In this scenario, the decision to accept or reject a group is made for each incoming group based on specific requirements. Depending on the situation, we may allocate seats immediately upon arrival or at a later time. This situation is commonly encountered in venues such as cinemas or music concerts, where decisions can be made on a group-by-group basis. The goal is to make timely decisions regarding seat assignments, considering factors such as available seating capacity, social distancing requirements and the specific size of each group.

By classifying the seating situations into these three categories and tailoring the decision-making process accordingly, we can effectively manage seat assignments, optimize seating layouts, and ensure a safe and enjoyable experience for all attendees.

\subsection{Seat planning and seat assignment}
There is a requirement for the maximum number of people in one group due to the social distancing constraint. People in one group should sit together.

In our context, the seat planning means the seat partition in the planning. It includes two parts, fixed seat planning and flexible seat planning. The former one is that some seats are unavailable, it may be dismantled or sealed by plastic tape in advance. The latter one represents the current seat planning, but the planning can be altered later.

For the seat assignment, for the coming group, when accepting it, we assign the seats to the group, and the seats will not changed and be used by others.

\newpage
