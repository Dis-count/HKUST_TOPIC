% !TEX root = ./sum1.tex
\section{Stochastic Situation}\label{sec_seat_planning}

% for given arrival demand distribution but not realization-- scenarios

Secondly, we consider the stochastic model under social distancing constraints. In certain scenarios, we may have demand data for multiple days, which includes information about the number of people in each group size. Examples of such scenarios could be assembling in a church or seating groups in a cathedral. In these cases, we can utilize the stochastic model to generate a seat planning that ensures social distancing requirements are met. To maintain social distancing effectively, the venue manager needs to enforce a fixed seat planning. It is crucial for the group to adhere to the designated seating arrangement.
% By utilizing the stochastic model and implementing a fixed seat planning approach, we can optimize the seating arrangements and ensure the well-being of all attendees by minimizing the risk of virus transmission.


\subsection*{scenario-based}
In this section, we develop the scenario-based stochastic programming (SSP) to obtain the seat planning with available capacity. Due to the well-structured nature of SSP, we implement Benders decomposition to solve it efficiently. However, in some cases, solving the integer programming with Benders decomposition remains still computationally prohibitive. Thus, we can consider the LP relaxation first, then obtain a feasible seat planning by deterministic model. Based on that, we construct a seat planning composed of full or largest patterns to fully utilize all seats.

Now suppose the demand of groups is stochastic, the stochastic information can be obtained from scenarios through historical data. Use $\omega$ to index the different scenarios, each scenario $\omega \in \Omega$. Regarding the nature of the obtained information, we assume that there are $|\Omega|$ possible scenarios. A particular realization of the demand vector can be represented as $\mathbf{d}_\omega = (d_{1\omega},d_{2\omega},\ldots,d_{M,\omega})^{\intercal}$. Let $p_{\omega}$ denote the probability of any scenario $\omega$, which we assume to be positive. To maximize the expected number of people accommodated over all the scenarios, we propose a scenario-based stochastic programming to obtain a seat planning.

The seat planning can be represented by decision variables $\mathbf{x} \in \mathbb{Z}_{+}^{M \times N}$. Here, $x_{ij}$ represents the number of group type $i$ assigned to row $j$ in the seat planning. As mentioned earlier, we calculate the supply for group type $i$ as the sum of $x_{ij}$ over all rows $j$, denoted as $\sum_{j=1}^N x_{ij}$. However, considering the variability across different scenarios, it is necessary to model the potential excess or shortage of supply. To capture this characteristic, we introduce a scenario-dependent decision variable, denoted as $\mathbf{y}$. 
It includes two vectors of decisions, $\mathbf{y}^{+} \in \mathbb{Z}_{+}^{M \times |\Omega|}$ and $\mathbf{y}^{-} \in \mathbb{Z}_{+}^{M \times |\Omega|}$. Each component of $\mathbf{y}^{+}$, denoted as $y_{i\omega}^{+}$, represents the excess supply for group type $i$ for each scenario $\omega$. On the other hand, $y_{i\omega}^{-}$ represents the shortage of supply for group type $i$ for each scenario $\omega$.

Taking into account the possibility of groups occupying seats planned for larger group types when the corresponding supply is insufficient, we make the assumption that surplus seats for group type $i$ can be occupied by smaller group types $j<i$ in descending order of group size. This means that if there are excess supply available after assigning groups of type $i$ to rows, we can provide the supply to groups of type $j<i$ in a hierarchical manner based on their sizes. That is, for any $\omega$, $i \leq M-1$, 

$$y_{i \omega}^{+}=\left(\sum_{j=1}^N x_{ij}- d_{i \omega} + y_{i+1, \omega}^{+}\right)^{+}, ~y_{i \omega}^{-}=\left(d_{i \omega}- \sum_{j=1}^N x_{ij} - y_{i+1, \omega}^{+} \right)^{+},$$
where $(x)^{+}$ equals $x$ if $x>0$, $0$ otherwise. Specially, for the largest group type $M$, we have $y_{M \omega}^{+} = (\sum_{j=1}^N x_{Mj} - d_{M \omega})^{+}$, $y_{M \omega}^{-} = (d_{M \omega}- \sum_{j=1}^N x_{Mj})^{+}$. Based on the above mentioned considerations, the total supply of group type $i$ under scenario $\omega$ can be expressed as $\sum_{j= 1}^{N} x_{ij} + y_{i+1,\omega}^{+} - y_{i \omega}^{+}, i = 1, \ldots, M-1$. For the special case of group type $M$, the total supply under scenario $\omega$ is $\sum_{j= 1}^{N} x_{Mj} - y_{M \omega}^{+}$.


Then we have the formulation of SSP:
  \begin{align}
  \quad \max \quad & E_{\omega}\left[(n_{M}-\delta) (\sum_{j= 1}^{N} x_{Mj} - y_{M \omega}^{+}) + \sum_{i=1}^{M-1} (n_i-\delta) (\sum_{j= 1}^{N} x_{ij} + y_{i+1,\omega}^{+} - y_{i \omega}^{+})\right] \\
  \text {s.t.} \quad & \sum_{j= 1}^{N} x_{ij}-y_{i \omega}^{+}+
  y_{i+1, \omega}^{+} + y_{i \omega}^{-}=d_{i \omega}, \quad i = 1,\ldots, M-1, \omega \in \Omega \label{DEF_constr1} \\
  & \sum_{j= 1}^{N} x_{ij} -y_{i \omega}^{+}+y_{i \omega}^{-}=d_{i \omega}, \quad i = M, \omega \in \Omega \label{DEF_constr2}\\
  & \sum_{i=1}^{M} n_{i} x_{ij} \leq L_j, j \in \mathcal{N}  \label{DEF_constr3} \\
  & y_{i \omega}^{+}, y_{i \omega}^{-} \in \mathbb{Z}_{+}, \quad i \in \mathcal{M}, \omega \in \Omega \notag \\
  & x_{ij} \in \mathbb{Z}_{+}, \quad i \in \mathcal{M}, j \in \mathcal{N} \notag.
  \end{align}

The objective function consists of two parts. The first part represents the number of people in
the largest group type that can be accommodated, given by $(n_{M}-\delta) (\sum_{j=1}^{N} x_{Mj} - y_{M\omega}^{+})$. The second part represents the number of people in group type $i$, excluding $M$, that can be accommodated, given by $(n_i-\delta) (\sum_{j=1}^{N} x_{ij} + y_{i+1,\omega}^{+} - y_{i\omega}^{+}), i = 1, \ldots, M-1$. The overall objective function is subject to an expectation operator denoted by $E_{\omega}$, which represents the expectation with respect to the scenario set. This implies that the objective function is evaluated by considering the average values of the decision variables and constraints over the different scenarios.

By reformulating the objective function, we have
\begin{align*}
  & E_{\omega}\left[\sum_{i=1}^{M-1} (n_i-\delta) (\sum_{j= 1}^{N} x_{ij} + y_{i+1,\omega}^{+} - y_{i \omega}^{+}) + (n_M-\delta) (\sum_{j= 1}^{N} x_{Mj} - y_{M \omega}^{+})\right] \\
  =& \sum_{j =1}^{N} \sum_{i=1}^M (n_i- \delta) x_{ij} - \sum_{\omega =1}^{|\Omega|} p_{\omega} \left(\sum_{i=1}^{M}(n_i- \delta)y_{i \omega}^{+} - \sum_{i=1}^{M-1}(n_i-\delta)y_{i+1, \omega}^{+}\right) \\
  =& \sum_{j =1}^{N} \sum_{i=1}^M i \cdot x_{ij} - \sum_{\omega =1}^{|\Omega|} p_{\omega} \sum_{i = 1}^{M} y_{i \omega}^{+}
\end{align*}

In the optimal solution, at most one of $y_{i \omega}^{+}$ and $y_{i \omega}^{-}$ can be positive for any $i, \omega$. Suppose there exist $i_0$ and $\omega_0$ such that $y_{i_0 \omega_0}^{+}$ and $y_{i_0 \omega_0}^{-}$ are positive. Substracting $\min\{y_{i_0, \omega_0}^{+}, y_{i_0, \omega_0}^{-}\}$ from these two values will still satisfy constraints \eqref{DEF_constr1} and \eqref{DEF_constr2} but increase the objective value when $p_{\omega_0}$ is positive. Thus, in the optimal solution, at most one of $y_{i \omega}^{+}$ and $y_{i \omega}^{-}$ can be positive.

Considering the analysis provided earlier, we find it advantageous to obtain a seat planning that only consists of full or largest patterns. However, the seat planning associated with the optimal solution obtained by solver to SSP may not consist of the largest or full patterns. We can convert the optimal solution to another optimal solution which is composed of the largest or full patterns.

\begin{prop}\label{prop_solution}
There exists an optimal solution to the stochastic programming problem such that the patterns associated with this optimal solution are composed of the full or largest patterns under any given scenarios.
\end{prop}

Given a specific pattern, we can convert it into a largest or full pattern while ensuring that the original group type requirements are met. When multiple full patterns are possible, our objective is to generate the pattern with minimal loss. Mathematically, for any pattern $\bm{h} = (h_1, \ldots, h_N)$, we seek to find a pattern $\bm{h}{'} = (h_1{'}, \ldots, h_N{'})$ that maximizes $|\bm{h}{'}|$ while satisfying the following constraints: $h_1{'} \geq h_1$, $h_1{'} + h_2{'} \geq h_1 + h_2$, $\cdots$, $h_1{'} + \ldots + h_N{'} \geq h_1 + \ldots + h_N$. In other words, we want to find a pattern $\bm{h}'$ where each element $h_i'$ is greater than or equal to the corresponding element $h_i$ in $\bm{h}$, and the cumulative sums of the elements in $\bm{h}'$ are greater than or equal to the cumulative sums of the elements in $\bm{h}$.
By finding such a pattern $\bm{h}'$, we can ensure that the converted pattern meets or exceeds the requirements of the original group types. Among the possible full patterns that satisfy these constraints, we prioritize the one with the smallest loss.

Now, we demonstrate the specific allocation scheme.
Let $\beta= L_{j} - \sum_{i} n_{i} x_{ij}$. If row $j$ is not the largest or full, then $\beta > 0$. 
We aim to allocate the remaining unoccupied seats in row $j$ in a way that maximizes the number of planned groups that become the largest in size. Find the smallest group type in the pattern denoted as $k$. If $k = M$, it means that this row corresponds to a largest pattern. If $k \neq M$, we reduce the number of group type $k$ by one and increase the number of group type $\min \{(k+\beta), M\}$ by one, the number of unoccupied seats will be reduced correspondingly.

We continue this procedure until either all the planned groups become the largest or $\beta = 0$. If $\beta = 0$, it indicates that the pattern is full. In this case, we have assigned all the unoccupied seats to the existing groups without incurring any additional loss. Therefore, this full pattern has the minimal loss while satisfying the groups requirement. However, if all the planned groups become the largest and $\beta \neq 0$, we can repeatly follow the steps outlined below to obtain the largest pattern:

\begin{itemize}
    \item If $\beta \geq n_{M}$, we can assign $n_M$ seats to a new group type $M$.
  
    \item If $n_{1} \leq \beta < n_{M}$, we can assign $\beta$ seats to a new group type $\beta-n_{1}+1$.
    \item If $0 < \beta < n_{1}$, it means that the current pattern is already the largest possible pattern because all the planned groups in the pattern are the largest.
  \end{itemize}
  
By following these steps and always prioritizing the largest group type for seat planning, we can achieve either the largest pattern or a full pattern with minimal loss. This approach guarantees efficient seat allocation, maximizing the utilization of available seats while still accommodating the original groups' requirements.


To construct the largest or full pattern for each row, we can employ the following algorithm. Since patterns are independent of each other, we can process them row by row within a given seat planning. This enables us to optimize the seat planning by maximizing the utilization of available seats and effectively accommodating the arriving groups.

\begin{algorithm}
    \caption{Construct The Largest or Full Pattern}\label{construction}
    % \KwIn{Pattern $\bm{h}$}
    % \KwOut{Pattern}
    \While{$\beta > 0$}
      {$k \gets \min_{i}\{h_{i} \neq 0\}$\Comment*[r]{Find the smallest group type in the pattern}
      \eIf{$k \neq M$}
      {$h_{k} \gets h_{k} - 1$\; $h_{\min\{k+\beta, M\}} \gets h_{\min\{k+\beta, M\}} + 1$\;
      $\beta \gets \beta - \max\{1, M - k\}$\Comment*[r]{Change the current group type to a group type as large as possible}}
      {\eIf{$\beta \geq n_{M}$}
      {$q \gets \lfloor\frac{\beta}{n_M}\rfloor$\;
       $\beta \gets \beta - q n_M$\; $h_{M} \gets h_{M} + q$\Comment*[r]{Assign seats to as many the largest group type as possible}}
      {\eIf{$n_{1} \leq \beta < n_{M}$}
      {$h_{\beta-n_1+1} \gets h_{\beta-n_1+1} + 1$\; $\beta \gets 0$\;}
      {$\bm{h}$ is the largest\; $\beta \gets 0$\;}}
      }}
  \end{algorithm}

Let $\mathbf{n} = (n_1, \ldots, n_M)$ represent the vector of seat sizes for each group type, where $n_i$ denotes the size of seats taken by group type $i$. Let $\mathbf{L} = (L_1, \ldots, L_N)$ represent the vector of row lengths, where $L_j$ denotes the length of row $j$ as defined previously.
The constraint \eqref{DEF_constr3} can be expressed as $\mathbf{n} \mathbf{x} \leq \mathbf{L}$. This constraint ensures that the total size of seats occupied by each group type, represented by $\mathbf{n} \mathbf{x}$, does not exceed the available row lengths $\mathbf{L}$. We can use the product $\mathbf{x} \mathbf{1}$ to indicate the supply of group types, where $\mathbf{1}$ is a column vector of size $N$ with all elements equal to 1. 

The linear constraints associated with scenarios, denoted by constraints \eqref{DEF_constr1} and \eqref{DEF_constr2}, can be expressed in matrix form as:
\[\mathbf{x} \mathbf{1} + \mathbf{V} \mathbf{y}_\omega = \mathbf{d}_\omega, \omega\in \Omega,\]
where $\mathbf{V} = [\mathbf{W}, ~\mathbf{I}]$.


$$
\mathbf{W}=\left[\begin{array}{cccccc}
-1 & 1 & 0 & \ldots & \ldots & 0 \\
0 & -1 & 1 &    0   & \ldots & 0 \\
\vdots & \ddots & \ddots & \ddots & \ddots & \vdots \\
0  & \ldots   &  0  & -1 & 1 & 0 \\
0  & \ldots   &  \ldots  &  0 &  -1 & 1 \\
0 & \ldots & \ldots & \ldots & 0 & -1
\end{array}\right]_{M \times M}
$$


and $\mathbf{I}$ is the identity matrix with the dimension of $M$. For each scenario $\omega \in \Omega$,
$$
\mathbf{y}_{\omega}=\left[\begin{array}{l}
\mathbf{y}_{\omega}^{+} \\
\mathbf{y}_{\omega}^{-}
\end{array}\right], \mathbf{y}_{\omega}^{+}=\left[\begin{array}{lllll}y_{1 \omega}^{+} & y_{2 \omega}^{+} & \cdots & y_{M \omega}^{+}\end{array}\right]^{\intercal}, \mathbf{y}_{\omega}^{-}=\left[\begin{array}{llll}y_{1 \omega}^{-} & y_{2 \omega}^{-} & \cdots & y_{M \omega}^{-}\end{array}\right]^{\intercal}.
$$

As we can find, this deterministic equivalent form is a large-scale problem even if the number of possible scenarios $\Omega$ is moderate. However, the structured constraints allow us to simplify the problem by applying Benders decomposition approach. Before using this approach, we could reformulate this problem as the following form. Let $\mathbf{c}^{\intercal}\mathbf{x} = \sum_{j =1}^{N} \sum_{i=1}^M i \cdot x_{ij}$, $\mathbf{f}^{\intercal}\mathbf{y}_{\omega} = -\sum_{i=1}^{M} y_{i \omega}^{+}$. Then the SSP formulation can be expressed as below,

\begin{equation}\label{BD_master}
\begin{aligned}
\max \quad & \mathbf{c}^{\intercal} \mathbf{x}+ z(\mathbf{x}) \\
\text {s.t.} \quad & \mathbf{n} \mathbf{x} \leq \mathbf{L} \\
& \mathbf{x} \in \mathbb{Z}_{+}^{M \times N},
\end{aligned}
\end{equation}
where $z(\mathbf{x})$ is defined as

$$z(\mathbf{x}) := E(z_{\omega}(\mathbf{x})) = \sum_{\omega \in \Omega} p_{\omega} z_{\omega}(\mathbf{x}),$$ and for each scenario $\omega \in \Omega$, 

\begin{equation}\label{BD_sub}
  \begin{aligned}
    z_{\omega}(\mathbf{x}) := \max \quad & \mathbf{f}^{\intercal} \mathbf{y}_{\omega} \\
    \text {s.t.} \quad & \mathbf{V} \mathbf{y}_{\omega} = \mathbf{d}_{\omega} - \mathbf{x} \mathbf{1} \\
     & \mathbf{y}_{\omega} \geq 0.
  \end{aligned}
\end{equation}


We can solve problem \eqref{BD_master} quickly if we can efficiently solve problem \eqref{BD_sub}. Next, we will mention how to solve problem \eqref{BD_sub}.

\subsection{Solve SSP by Benders Decomposition}\label{solve_by_benders}
We reformulate problem \eqref{BD_master} into a master problem and a subproblem \eqref{BD_sub}. The iterative process of solving the master problem and subproblem is known as Benders decomposition. 
The solution obtained from the master problem provides inputs for the subproblem, and the subproblem solutions help update the master problem by adding constraints, iteratively improving the overall solution until convergence is achieved. Firstly, we generate a closed-form solution to problem \eqref{BD_sub}, then we obtain the solution to the LP relaxation of problem \eqref{BD_master} by the constraint generation.


\subsubsection{Solve The Subproblem}\label{second_stage}

Notice that the feasible region of the dual of problem \eqref{BD_sub} remains unaffected by $\mathbf{x}$. This observation provides insight into the properties of this problem. Let $\bm{\alpha}$ denote the vector of dual variables. For each $\omega$, we can form its dual problem, which is 

\begin{equation}\label{BD_sub_dual}
  \begin{aligned}
    \min \quad & \bm{\alpha}_{\omega}^{\intercal} (\mathbf{d}_{\omega}- \mathbf{x} \mathbf{1}) \\
    \text {s.t.} \quad & \bm{\alpha}_{\omega}^{\intercal} \mathbf{V} \geq \mathbf{f}^{\intercal}
  \end{aligned}
\end{equation}


\begin{lem}\label{feasible_region}
    Let $\mathbb{P} = \{\bm{\alpha} \in \mathbb{R}^{M}|\bm{\alpha}^{\intercal} \mathbf{V} \geq \mathbf{f}^{\intercal}\}$. The feasible region of problem \eqref{BD_sub_dual}, $\mathbb{P}$, is nonempty and bounded. Furthermore, all the extreme points of $\mathbb{P}$ are integral.
    \end{lem}

    
Therefore, the optimal value of the problem \eqref{BD_sub}, $z_{\omega}(\mathbf{x})$, is finite and can be achieved at extreme points of the set $\mathbb{P}$. Let $\mathcal{O}$ be the set of all extreme points of $\mathbb{P}$. That is, we have $z_{\omega}(\mathbf{x}) = \min_{\bm{\alpha}_{\omega} \in \mathcal{O}} \bm{\alpha}_{\omega}^{\intercal}(\mathbf{d}_{\omega}- \mathbf{x} \mathbf{1})$.


Alternatively, $z_{\omega}(\mathbf{x})$ is the largest number $z_{\omega}$ such that $\bm{\alpha}_{\omega}^{\intercal}(\mathbf{d}_{\omega}- \mathbf{x} \mathbf{1}) \geq z_w, \forall \bm{\alpha}_{\omega} \in \mathcal{O}$. We use this characterization of $z_w(\mathbf{x})$ in problem \eqref{BD_master} and conclude that problem \eqref{BD_master} can thus be put in the form by setting $z_w$ as the variable:

\begin{equation}\label{BD_master2}
  \begin{aligned}
    \max \quad & \mathbf{c}^{\intercal} \mathbf{x} + \sum_{\omega \in \Omega} p_{\omega} z_{\omega} \\
    \text {s.t.} \quad & \mathbf{n} \mathbf{x} \leq \mathbf{L} \\
    & \bm{\alpha}_{\omega}^{\intercal}(\mathbf{d}_{\omega}- \mathbf{x} \mathbf{1}) \geq z_{\omega}, \forall \bm{\alpha}_{\omega} \in \mathcal{O}, \forall \omega \\
     & \mathbf{x} \in \mathbb{Z}_{+}
  \end{aligned}
\end{equation}

Before applying Benders decomposition to solve problem \eqref{BD_master2}, it is important to address the efficient computation of the optimal solution to problem \eqref{BD_sub_dual}.
When we are given $\mathbf{x}^{*}$, the demand that can be satisfied by the seat planning is $\mathbf{x}^{*} \mathbf{1} = \mathbf{d}_0 = (d_{1,0},\ldots,d_{M,0})^{\intercal}$.
By plugging them in the subproblem \eqref{BD_sub}, we can obtain the value of $y_{i, \omega}$ recursively:
\begin{equation}\label{y_recursively}
\begin{aligned}
  & y_{M \omega}^{-}=\left(d_{M \omega}-d_{M 0}\right)^{+} \\
  & y_{M \omega}^{+}=\left(d_{M 0}-d_{M \omega}\right)^{+} \\
  & y_{i \omega}^{-}=\left(d_{i \omega}-d_{i 0} - y_{i+1, \omega}^{+} \right)^{+}, i =1,\ldots, M-1 \\
  & y_{i \omega}^{+}=\left(d_{i 0}- d_{i \omega} + y_{i+1, \omega}^{+}\right)^{+}, i =1,\ldots, M-1
\end{aligned}
\end{equation}

The optimal solutions to problem \eqref{BD_sub_dual} can be obtained according to the value of $\mathbf{y}_{\omega}$.

\begin{prop}\label{optimal_sol_sub_dual}
  The optimal solutions to problem \eqref{BD_sub_dual} are given by 
\begin{equation}\label{BD_sub_simplified}
  \begin{aligned}
    \alpha_{i} = 0 \quad & \text{if}~  y_{i \omega}^{-} > 0,  i =1,\ldots, M~\text{or}~ y_{i \omega}^{-} = y_{i \omega}^{+} = 0, y_{i+1, \omega}^{+}> 0, i = 1,\ldots, M-1 \\
    \alpha_{i} = \alpha_{i-1}+1 \quad & \text{if}~ y_{i \omega}^{+} > 0, i =1,\ldots, M \\
    0 \leq \alpha_{i} \leq \alpha_{i-1}+1 \quad & \text{if}~ y_{i \omega}^{-} = y_{i \omega}^{+} = 0, i = M~\text{or}~ y_{i \omega}^{-} = y_{i \omega}^{+} = 0, y_{i+1, \omega}^{+}= 0, i = 1,\ldots, M-1
  \end{aligned}
\end{equation}
\end{prop}

Instead of solving this linear programming directly, we can compute the values of $\alpha_{\omega}$ by performing a forward calculation from $\alpha_{1\omega}$ to $\alpha_{M\omega}$.

\subsubsection{Constraint Generation}\label{bender_stage}

Due to the computational infeasibility of solving problem \eqref{BD_master2} with an exponentially large number of constraints, it is a common practice to use a subset, denoted as $\mathcal{O}^t$, to replace $\mathcal{O}$ in problem \eqref{BD_master2}. This results in a modified problem known as the Restricted Benders Master Problem(RBMP). To find the optimal solution to problem \eqref{BD_master2}, we employ the technique of constraint generation. It involves iteratively solving the RBMP and incrementally adding more constraints until the optimal solution to problem \eqref{BD_master2} is obtained.

We can conclude that the RBMP will have the form:

\begin{equation}\label{BD_master3}
  \begin{aligned}
    \max \quad & \mathbf{c}^{\intercal} \mathbf{x} + \sum_{\omega \in \Omega} p_{\omega} z_{\omega} \\
    \text {s.t.} \quad & \mathbf{n} \mathbf{x} \leq \mathbf{L} \\
    & \bm{\alpha}_{\omega}^{\intercal}(\mathbf{d}_{\omega}- \mathbf{x} \mathbf{1}) \geq z_{\omega}, \bm{\alpha}_{\omega} \in \mathcal{O}^{t}, \forall \omega \\
     & \mathbf{x} \in \mathbb{Z}_{+}
  \end{aligned}
\end{equation}

To determine the initial $\mathcal{O}^{t}$, we have the following proposition.

\begin{prop}\label{one_ep_feasible}
    RBMP is always bounded with at least any one feasible constraint for each scenario.
    \end{prop}
    
Given the initial $\mathcal{O}^{t}$, we can have the solution $\mathbf{x}^{*}$ and $\mathbf{z}^{*} =(z^{*}_1,\ldots, z^{*}_{|\Omega|})$. Then $c^{\intercal} \mathbf{x}^{*} + \sum_{\omega \in \Omega} p_{\omega} z_{\omega}^{*}$ is an upper bound of problem \eqref{BD_master3}. When $\mathbf{x}^{*}$ is given, the optimal solution, $\bm{\tilde{\alpha}}_{\omega}$, to problem \eqref{BD_sub_dual} can be obtained according to Proposition \ref{optimal_sol_sub_dual}. Let $\tilde{z}_{\omega} = \bm{\tilde{\alpha}}_{\omega}(d_{\omega} - \mathbf{x}^{*} \mathbf{1})$, then $(\mathbf{x}^{*}, \mathbf{\tilde{z}})$ is a feasible solution to problem \eqref{BD_master3} because it satisfies all the constraints. Thus, $\mathbf{c}^{\intercal} \mathbf{x}^{*} + \sum_{\omega \in \Omega} p_{\omega} \tilde{z}_{\omega}$ is a lower bound of problem \eqref{BD_master2}.

If for every scenario $\omega$, the optimal value of the corresponding problem \eqref{BD_sub_dual} is larger than or equal to $z_{\omega}^{*}$, which means all contraints are satisfied, then we have an optimal solution, $(\mathbf{x}^{*}, \mathbf{z}^{*})$, to problem \eqref{BD_master2}. However, if there exists at least one scenario $\omega$ for which the optimal value of problem \eqref{BD_sub_dual} is less than $z_{\omega}^{*}$, indicating that the constraints are not fully satisfied, we need to add a new constraint $(\bm{\tilde{\alpha}}_{\omega})^{\intercal}(\mathbf{d}_{\omega} - \mathbf{x} \mathbf{1}) \geq z_{\omega}$ to RBMP.


\begin{algorithm}[h]
    \caption{Benders Decomposition}\label{cut_algo}
    \KwIn{Initial problem \eqref{BD_master3} with $\bm{\alpha}_{\omega} = 0, \forall \omega$, $LB = 0$, $UB = \infty$, $\epsilon$.}
    \KwOut{$\mathbf{x}^{*}$}
    \While{$UB - LB > \epsilon$}
      {Obtain $(\mathbf{x}^{*}, \mathbf{z}^{*})$ from problem \eqref{BD_master3}\;
      $UB \gets c^{\intercal} \mathbf{x}^{*} + \sum_{\omega \in \Omega} p_{\omega} z_{\omega}^{*}$\;
      \For{$\omega= 1, \ldots, |\Omega|$}
      {Obtain $\bm{\tilde{\alpha}}_{\omega}$ from Proposition \ref{optimal_sol_sub_dual}\; $\tilde{z}_{\omega}= (\bm{\tilde{\alpha}}_{\omega})^{\intercal}(\mathbf{d}_{\omega}- \mathbf{x}^{*} \mathbf{1})$\;
      \If{$\tilde{z}_{\omega} < z_{\omega}^{*}$}
      {Add one new constraint, $(\bm{\tilde{\alpha}}_{\omega})^{\intercal}(\mathbf{d}_{\omega}- \mathbf{x} \mathbf{1}) \geq z_{\omega}$, to problem \eqref{BD_master3}\;}
      }
      {$LB \gets c^{\intercal} \mathbf{x}^{*} + \sum_{\omega \in \Omega} p_{\omega} \tilde{z}_{\omega} $\;}
      }
\end{algorithm}

From Proposition \ref{one_ep_feasible}, we can set $\bm{\alpha}_{\omega} = \mathbf{0}$ initially. Notice that only contraints are added in each iteration, thus $UB$ is decreasing monotone over iterations. Then we can use $UB - LB < \epsilon$ to terminate the algorithm.

However, solving problem \eqref{BD_master3} even with the simplified constraints directly can be computationally challenging in some cases, so practically we first obtain the optimal solution to the LP relaxation of problem \eqref{BD_master}. Then, we generate an integral seat planning from this solution.

\subsection{Obtain The Seat Planning Composed of Full or Largest Patterns}\label{seat_assignment}
We may obtain a fractional optimal solution when we solve the LP relaxation of problem \eqref{BD_master}. This solution represents the optimal allocations of groups to seats but may involve fractional values, indicating partial assignments. Based on the fractional solution obtained, we use the deterministic model to generate a feasible seat planning. The objective of this model is to allocate groups to seats in a way that satisfies the supply requirements for each group without exceeding the corresponding supply values obtained from the fractional solution. To accommodate more groups and optimize seat utilization, we aim to construct a seat planning composed of full or largest patterns based on the feasible seat planning obtained in the last step. 


Let the optimal solution to the LP relaxation of problem \eqref{BD_master3} be $\mathbf{x}^{*}$. Aggregate $\mathbf{x}^{*}$ to the number of each group type, ${X}^{*}_{i} =\sum_{j} x^{*}_{ij}, i \in \mathbf{M}$. Replace the vector $\mathbf{d}$ with ${X}^{*}$ in the deterministic model, we have the following problem, 

\begin{equation}\label{deter_upper1}
  \left\{\max \sum_{j=1}^{N} \sum_{i=1}^{M}(n_i -\delta) x_{ij}: \sum_{i = 1}^{M} n_i x_{ij} \leq L_{j}, j \in \mathcal{N}; \sum_{j =1}^{N} x_{ij} \leq {X}^{*}_{i}, i \in \mathcal{M}; x_{ij} \in Z_{+} \right\}
\end{equation}

Then solve the resulting problem \eqref{deter_upper1} to obtain the optimal solution, $\mathbf{\tilde{x}}$, which represents a feasible seat planning. We can construct the largest or full patterns by Algorithm \ref{construction}.

\begin{algorithm}
    \caption{Seat Planning Construction}\label{seat_construction}
    % \KwIn{Scenarios set $\Omega$, Seat layout}
    % \KwOut{$\mathbf{x}, \bm{H}$}
      {Obtain the optimal solution, $\mathbf{x}^{*}$, from the LP relaxation of problem \eqref{BD_master3}\;}
      {Aggregate $\mathbf{x}^{*}$ to the number of each group type, $\tilde{X}_{i} = \sum_{j} x^{*}_{ij}, i \in \mathbf{M}$\;}
      {Obtain the optimal solution, $\tilde{\mathbf{x}}$, and the corresponding pattern, $\bm{H}$, from problem \eqref{deter_upper1} with $\tilde{X}$\;}
      {Construct the full or largest patterns by Algorithm \ref{construction} with $\tilde{\mathbf{x}}$ and $\bm{H}$\;}
  \end{algorithm}



\subsection{Based on the fixed seat planning}

To verify the performance of the seat planning obtained from stochastic programming.