% !TEX root = sum1.tex
\clearpage
\section*{Proof}

\begin{pf}[Proof of Proposition \ref{lem_pattern}]
First, we construct a feasible pattern with the size of $qM + \max\{r-\delta, 0\}$, then we prove this pattern is largest. We can utilize a greedy approach to construct a pattern, denoted as $\bm{h}_{g}$, by following the steps outlined below. This approach aims to generate a pattern that maximizes the number of people accommodated within the given constraints.

\begin{itemize}
 \item Begin by selecting the maximum group size, denoted as $n_M$, as many times as possible to fill up the available seats in the row.
 \item Allocate the remaining seats (if possible) in the row to the group with the corresponding size.
\end{itemize}

Let $L = n_M \cdot q + r$, where $q$ represents the number of times $n_M$ is selected (the quotient), and $r$ represents the remainder, indicating the number of remaining seats. It holds that $0 \leq r < n_M$. 

The number of people accommodated in the pattern $\bm{h}_{g}$ is given by $|\bm{h}_{g}| = q M + \max\{r-\delta, 0\}$. To establish the optimality of $|\bm{h}_{g}|$ as the largest number of people accommodated given the constraints of $L$, $\delta$, and $M$, we can employ a proof by contradiction.

% This expression takes into account the complete groups of size $M$ (represented by the product $q M$) and any remaining seats beyond the complete groups (represented by $\max{r-\delta, 0}$) that can be utilized to accommodate additional individuals, considering the social distancing requirement $\delta$.
% That is we need to prove $\max\{\sum_{i} (n_i - \delta)h_i| n_{i} h_{i} \leq L, h_{i} \in \mathbb{Z}\}$.

% With the capacity constraint, we have $L \geq \sum_{i} n_i h_i > \sum_{i} \delta h_i + q M + \max\{r-\delta, 0\}$. By substituting $L = n_M \cdot q + r$, we have $q(M + \delta) + r > \sum_{i} \delta h_i + q M + \max\{r-\delta, 0\}$.

Assuming the existence of a pattern $\bm{h}$ such that $|\bm{h}| > |\bm{h}_{g}|$, we can derive the following inequalities:

\begin{align*}
  & \sum_{i} (n_i - \delta) h_i > q M + \max\{r-\delta, 0\} \\
  \Rightarrow ~& L \geq \sum_{i} n_i h_i > \sum_{i} \delta h_i + q M + \max\{r-\delta, 0\} \\
  \Rightarrow ~& q(M + \delta) + r > \sum_{i} \delta h_i + q M + \max\{r-\delta, 0\} \\
  \Rightarrow ~& q \delta + r > \sum_{i} \delta h_i + \max\{r-\delta, 0\}
\end{align*}

Breaking down the above inequality into two cases:

\begin{enumerate}[(i)]
  \item When $r > \delta$, the inequality becomes $q+1 > \sum_{i} h_i$. It should be noted that $h_i$ represents the number of group type $i$ in the pattern. Since $\sum_{i} h_i \leq q$, the maximum number of people that can be accommodated is $q M < q M + r-\delta$.  
  \item When $r \leq \delta$, we have the inequality $q \delta + \delta \geq q \delta + r > \sum_{i} \delta h_i$. Similarly, we obtain $q+1 > \sum_{i} h_i$. Thus, the maximum number of people that can be accommodated is $q M$, which is not greater than $|\bm{h}_{g}|$.  
\end{enumerate}

Therefore, $\bm{h}$ cannot exist. The pattern, $\bm{h}_{g}$, is a largest pattern. The maximum number of people that can be accommodated in the largest pattern is $q M + \max\{r-\delta, 0\}$. 

% Correspondingly, the loss of the largest pattern $|\bm{h}_{g}|$ is $q \delta -\delta + \min\{r, \delta\}$.
\qed
\end{pf}

\begin{pf}[Proof of Proposition \ref{sol_relax_deter}]
  Treat the groups as the items, the rows as the knapsacks. There are $M$ types of items, the total number of which is $K = \sum_{i} d_i$, each item $k$ has a profit $p_k$ and weight $w_k$. 
  
  Then this Integer Programming is a special case of the Multiple Knapsack Problem (MKP). Consider the solution to the linear relaxation of \eqref{deter_upper}. Sort these items according to profit-to-weight ratios $\frac{p_1}{w_1} \geq \frac{p_2}{w_2} \geq \ldots \geq \frac{p_K}{w_K}$. 
  % $\delta$ is no less than 1, different types have
  Let the break item $b$ be given by $b=\min \{j: \sum_{k=1}^j w_k \geq \tilde{L}\}$, where $\tilde{L} = \sum_{j=1}^{N} L_j$ is the total size of all knapsacks. Then the Dantzig upper bound \cite{dantzig1957discrete} becomes $u_{\mathrm{MKP}}=\sum_{j=1}^{b-1} p_j+\left(\tilde{L}-\sum_{j=1}^{b-1} w_j\right) \frac{p_b}{w_b}$. The corresponding optimal solution is to accept the whole items from $1$ to $b-1$ and fractional $(\tilde{L}-\sum_{j=1}^{b-1} w_j)$ item $b$. Suppose the item $b$ belong to type $\tilde{i}$, then for $i < \tilde{i}$, $x_{ij}^{*} = 0$; for $i > \tilde{i}$, $x_{ij}^{*} = d_{i}$; for $i = \tilde{i}$, $\sum_{j} x_{ij}^{*} = (\tilde{L} - \sum_{i = \tilde{i}+1}^{M} {d_i n_i})/ n_{\tilde{i}}$. \qed
\end{pf}


% \begin{pf}[Proof of Proposition \ref{prop_onescenario}]
%   When $|\Omega| =1$ in SSP formulation, the stochastic programming will be 

%   \begin{equation}\label{one_form}
%     \begin{aligned}
%     \max \quad & \sum_{i=1}^{M}  \sum_{j= 1}^{N} (n_i-\delta) x_{ij} - \sum_{i=1}^{M} y_{i}^{+}  \\
%     \text {s.t.} \quad & \sum_{j= 1}^{N} x_{ij} - y_{i}^{+}+ y_{i+1}^{+} + y_{i}^{-} = d_{i}, \quad i = 1, \ldots, M-1, \\
%     & \sum_{j= 1}^{N} x_{ij} -y_{i}^{+} + y_{i}^{-} = d_{i}, \quad i = M, \\
%     & \sum_{i=1}^{M} n_{i} x_{ij} \leq L_j, j \in \mathcal{N}\\
%     & y_{i}^{+}, y_{i}^{-} \in \mathbb{Z}_{+}, \quad i \in \mathcal{M} \\
%     & x_{ij} \in \mathbb{Z}_{+}, \quad i \in \mathcal{M}, j \in \mathcal{N}.
%     \end{aligned}
% \end{equation}
  
%   To maximize the objective function, we can take $y_i^{+} = 0$. Notice that $y_{i}^{-} \geq 0$, thus the constraints $\sum_{j= 1}^{N} x_{ij} + y_{i}^{-} = d_{i}, i \in \mathcal{M}$ can be rewritten as $\sum_{j= 1}^{N} x_{ij} \leq d_{i}, i \in \mathcal{M}$. That is to say, problem \eqref{one_form} is equivalent to the deterministic model. \qed
% \end{pf}

\begin{pf}[Proof of Proposition \ref{prop_solution}]
  In any optimal solution where one of the corresponding patterns is not full or largest, we have the flexibility to allocate the remaining unoccupied seats. These seats can be assigned to either a new seat planning or added to an existing seat planning. Importantly, since a group can utilize the seat planning of a larger group, the allocation scheme based on the original optimal solution will not affect the optimality of the solution. 
  For each row, there are three situations to allocate the seats. First, when the rest seats can be allocated to the existing groups, then the corresponding pattern becomes a full pattern. Second, when all the existing groups are the largest groups and the rest seats cannot construct a new group, the pattern becomes the largest. Third, when all the existing groups are the largest groups and the rest seats can construct new groups, the rest seats can be used to construct the largest groups until there is no enough capacity, then the pattern becomes the largest. Finally, we can allocate the seats such that each row in the seat planning becomes either full or largest. \qed
  \end{pf}


\begin{pf}[Proof of Lemma \ref{feasible_region}]
% $W$ and $V$ are the totally unimodular matrix because 

Note that $\mathbf{f}^{\intercal} = [-\mathbf{1},~\mathbf{0}]$ and $V = [W,~I]$. Based on this, we can derive the following inequalities: $\bm{\alpha}^{\intercal}W \geq -\mathbf{1}$ and $\bm{\alpha}^{\intercal} I \geq \mathbf{0}$. These inequalities indicate that the feasible region is nonempty and bounded. Moreover, let $\alpha_0 = 0$. From this, we can deduce that $0 \leq \alpha_i \leq \alpha_{i-1} +1$ for $i \in \mathcal{M}$. Consequently, all extreme points within the feasible region are integral.
  \qed
\end{pf}

\begin{pf}[Proof of Proposition \ref{optimal_sol_sub_dual}]
  According to the complementary slackness property, we can obtain the following equations
  \begin{align*}
    & \alpha_{i} (d_{i0} - d_{i \omega} - y_{i \omega}^{+} + y_{i+1, \omega}^{+} + y_{i \omega}^{-}) = 0, i =1,\ldots, M-1 \\
    & \alpha_{i} (d_{i0} - d_{i \omega} - y_{i \omega}^{+}+ y_{i \omega}^{-}) = 0, i = M \\
    & y_{i \omega}^{+}(\alpha_{i} - \alpha_{i-1}-1) = 0, i =1,\ldots, M \\
    & y_{i \omega}^{-} \alpha_{i} = 0, i =1,\ldots, M.
  \end{align*}
  
  When $y_{i \omega}^{-} >0$, we have $\alpha_{i} =0$; when $y_{i \omega}^{+} >0$, we have $\alpha_{i} = \alpha_{i-1} +1$.
  Let $\Delta d = d_{\omega} - d_0$, then the elements of $\Delta d$ will be a negative integer, positive integer and zero.
  When $y_{i \omega}^{+} = y_{i \omega}^{-} = 0$, if $i = M$, $\Delta d_{M} =0$, the value of objective function associated with $\alpha_{M}$ is always $0$, thus we have $0 \leq \alpha_{M} \leq \alpha_{M-1}+1$; if $i < M$, we have $y_{i+1, \omega}^{+} = \Delta d_{i} \geq 0$. If $y_{i+1, \omega}^{+} > 0$, the objective function associated with $\alpha_i$ is $\alpha_{i} \Delta d_{i} = \alpha_{i} y_{i+1, \omega}^{+}$, thus to minimize the objective value, we have $\alpha_i =0$; if $y_{i+1, \omega}^{+} = 0$, we have $0 \leq \alpha_{i} \leq \alpha_{i-1} +1$.
  \qed
  \end{pf}

  \begin{pf}[Proof of Proposition \ref{one_ep_feasible}]
    Suppose we have one extreme point $\bm{\alpha}_{\omega}^{0}$ for each scenario. Then we have the following problem.
    \begin{equation}\label{lemma_eq}
      \begin{aligned}
        \max \quad & \mathbf{c}^{\intercal} \mathbf{x} + \sum_{\omega \in \Omega} p_{\omega} z_{\omega} \\
        \text {s.t.} \quad & \mathbf{n} \mathbf{x} \leq \mathbf{L} \\
        & (\bm{\alpha}_{\omega}^{0})^{\intercal}\mathbf{d}_{\omega} \geq (\bm{\alpha}_{\omega}^{0})^{\intercal} \mathbf{x} \mathbf{1} + z_{\omega}, \forall \omega \\
         & \mathbf{x} \in \mathbb{N}^{M \times N}
      \end{aligned}
    \end{equation}
    Problem \eqref{lemma_eq} reaches its maximum when $(\bm{\alpha}_{\omega}^{0})^{\intercal}\mathbf{d}_{\omega} = (\bm{\alpha}_{\omega}^{0})^{\intercal} \mathbf{x} \mathbf{1} + z_{\omega}, \forall \omega$. Substitute $z_{\omega}$ with these equations, we have 
    \begin{equation}\label{lemma_eq2}
      \begin{aligned}
        \max \quad & \mathbf{c}^{\intercal} \mathbf{x} - \sum_{\omega}p_{\omega}(\bm{\alpha}_{\omega}^{0})^{\intercal} \mathbf{x} \mathbf{1} + \sum_{\omega} p_{\omega} (\bm{\alpha}_{\omega}^{0})^{\intercal} \mathbf{d}_{\omega} \\
        \text {s.t.} \quad & \mathbf{n} \mathbf{x} \leq \mathbf{L} \\
        & \mathbf{x} \in \mathbb{N}^{M \times N}
      \end{aligned}
    \end{equation}
    Notice that $\mathbf{x}$ is bounded by $\mathbf{L}$, then the problem \eqref{lemma_eq} is bounded. Adding more constraints will not make the optimal value larger. Thus, RBMP is bounded. 
    \qed
  \end{pf}


% \begin{corollary}\label{coro_1}
%   Suppose there are two largest patterns, denoted as $h_1$ and $h_2$, with lengths $L_1 = n \cdot n_M$ and $L_2$ respectively (where $n$ is an integer), then $h_1 + h_2$ is largest with the length $L_1 + L_2$.
% \end{corollary}

% \begin{pf}[Proof of Proposition \ref{prop_construction}]
% To prove this proposition, we first prove the following lemma.

% \begin{lem}\label{generation}
%   For a feasible pattern $\bm{h}$ with $n$ groups and the length $L$, where $\sum_{i} n_{i} h_{i} \leq L \leq n \cdot n_{M}$, it is always possible to find a pattern $\bm{h}^{\prime}$ with $n$ groups which is the optimal solution to problem \eqref{gene_largest}. Additionally, $\bm{h}^{\prime}$ satisfies $|\bm{h}^{\prime}| = L - n$.
% \end{lem}
  
% \begin{pf}[Proof of Lemma \ref{generation}]
% We can demonstrate that the maximum value of $|\bm{h}^{\prime}|$ is $L - n$. For a feasible pattern $\bm{h}^{\prime}$ with $n$ groups, we have $\sum_{i}{h_i} = n$ and $\sum_{i}{n_i h_i} \leq L$. Therefore, we can deduce that $\sum_{i} {i h_i} \leq L - n$. To achieve equality, we need to transform the original pattern into a full pattern. Suppose the original pattern is not a full pattern. Let $\beta = L - \sum_{i} n_{i} h_{i}$, and identify the smallest group type in pattern $\bm{h}$ as $k$. We modify the original group of seats planned for $i$ people and allocate the unused seat(s) with the original group to accommodate $j$ people, where $j = \min\{M, \beta + i\}$. Let $\bm{h}^{1}$ represent the pattern after this allocation. As a result, we have $h^{1}_{i} = h_{i} -1$ and $h^{1}_{j} = h_{j} +1$. Thus, $\bm{h}^{1}$ will still satisfy the given constraints. This allocation process continues until $\beta$ reaches $0$. Eventually, we obtain a full pattern $\bm{h}^{\prime}$ that maximizes $|\bm{h}^{\prime}|$ while satisfying the constraints.
% \end{pf}

% Suppose there is a feasible pattern $\bm{h}$ with $n$ groups and the length $L_{j}$ for row $j$. When $L_{j} \leq n \cdot n_M$, Algorithm \ref{construction} can generate the optimal solution to problem \eqref{gene_largest} from Lemma \ref{generation}.

% When $L_{j} > n \cdot n_M$, Algorithm \ref{construction} can generate the pattern $\bm{h}^{\prime}$ which can be divided as $\bm{h}^{\prime} = \bm{h}^{\prime}_1 + \bm{h}^{\prime}_2$. The first part, $\bm{h}^{\prime}_1$, is associated with $L^1_{j} = n \cdot n_M$, the second part is associated with $L^2_{j} = L - n \cdot n_M$. According to Lemma \ref{generation}, $\bm{h}^{'}_1$ satisfies the constraints of problem \eqref{gene_largest}. Thus, $\bm{h}^{\prime}$ also satisfies these constraints. And $\bm{h}^{\prime}_2$ is a largest pattern according to Proposition \ref{lem_pattern}. Therefore, $\bm{h}^{\prime}$ is also a largest pattern.
% \end{pf}

\begin{pf}[Proof of Proposition \ref{prop_construction}]
First of all, we demonstrate the feasibility of problem \eqref{improve_seat}. Given the feasible seat planning $\bm{H}$ and $\tilde{d}_{i} = \sum_{j=1}^{N} H_{ji}$, let $\hat{x}_{ij} = H_{ji}, i \in \mathcal{M}, j \in \mathcal{N}$, then $\{\hat{x}_{ij}\}$ satisfies the first set of constraints. Because $\bm{H}$ is feasible, $\{\hat{x}_{ij}\}$ satisfies the second set of constraints and integer constraints. Thus, problem \eqref{improve_seat} always has a feasible solution. 

Suppose there exists at least one pattern $\bm{h}$ is neither full nor largest in the optimal seat planning obtained from problem \eqref{improve_seat}. Let $\beta = L - \sum_{i} n_{i} h_{i}$, and denote the smallest group type in pattern $\bm{h}$ by $k$. If $\beta \geq n_1$, we can assign at least $n_1$ seats to a new group to increase the objective value. Thus, we consider the situation when $\beta < n_1$. If $k =M$, then this pattern is largest. When $k< M$, let $h^{1}_{k} = h_{k} -1$ and $h^{1}_{j} = h_{j} +1$, where $j = \min\{M, \beta + i\}$. In this way, the constraints will still be satisfied but the objective value will increase when the pattern $\bm{h}$ changes. Therefore, by contradiction, problem \eqref{improve_seat} always generate a seat planning composed of full or largest patterns.
\end{pf}


\begin{pf}[Proof of Lemma \ref{bid-price}]
According to the Proposition \ref{sol_relax_deter}, the aggregate optimal solution to LP relaxation of problem \eqref{deter_upper} takes the form $x e_{\tilde{i}} + \sum_{i=\tilde{i}+1} ^{M} d_{i} e_{i}$, then according to the complementary slackness property, we know that $z_1, \ldots, z_{\tilde{i}} = 0$. This implies that $\beta_j \geq \frac{n_i - \delta}{n_i}$ for $i = 1,\ldots, \tilde{i}$. Since $\frac{n_i - \delta}{n_i}$ increases with $i$, we have $\beta_j \geq \frac{n_{\tilde{i}} - \delta}{n_{\tilde{i}}}$. Consequently, we obtain $z_{i} \geq n_i - \delta - n_i \frac{n_{\tilde{i}} - \delta}{n_{\tilde{i}}} = \frac{\delta(n_i-n_{\tilde{i}})}{n_{\tilde{i}}}$ for $i = h+1, \ldots, M$.
  
Given that $\mathbf{d}$ and $\mathbf{L}$ are both no less than zero, the minimum value will be attained when $\beta_j = \frac{n_{\tilde{i}} - \delta}{n_{\tilde{i}}}$ for all $j$, and $z_i = \frac{\delta(n_i-n_{\tilde{i}})}{n_{\tilde{i}}}$ for $i = \tilde{i}+1, \ldots, M$.  \qed
\end{pf}

% Problem \ref{bid-price_dual} can also be obtained by the following approximation:

% $V_{t}(\mathbf{L}) - V_{t+1}(\mathbf{L}) = \mathbb{E}_{i \sim p}\left[\max_{\substack{j \in \mathcal{N}: \\ L_j \geqslant {n}_{i}}}\left\{V_{t+1}\left(\mathbf{L}- n_{i}\mathbf{e}_j^{\intercal} \right)- V_{t+1}(\mathbf{L}) + i, 0 \right\}\right]$

% Approximation: $V_{t}(\mathbf{L}) = \theta^{t} + \sum_{j=1}^{N} L_j \beta_j$, substitute it to the above DP, we have 

% $V_{t}(\mathbf{L}) - V_{t+1}(\mathbf{L}) = \mathbb{E}_{i \sim p}\left[\max_{\substack{j \in \mathcal{N}: \\ L_j \geqslant {n}_{i}}}\left\{-n_i \beta_j + i, 0 \right\}\right] = \sum_i p_i \max_{j} \{-n_i \beta_j + i, 0\}$.

% Let $z_i = \max_{j} \{-n_i \beta_j + i, 0\}$, the constraints of dual form can be developed.

% The objective funtion: $V_{1} = \sum_{t=1}^{T} (V_{t} - V_{t+1}) = \sum_{i} d_{i} z_{i} + \sum_{j}^{N} L_{j} \beta_j$.

\newpage
