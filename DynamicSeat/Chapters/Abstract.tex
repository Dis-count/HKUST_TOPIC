% !TEX root = sum1.tex

\section*{Abstract}

Social distancing has been widely acknowledged and implemented as a non-pharmaceutical method to control the spread of infectious diseases. During a pandemic, the government may issue minimum physical distance requirements between people, which must be respected in the seating assignment. This problem is further complicated by the presence of groups that need to be seated together.

In this study, we examine the dynamic seat assignment problem with social distancing, wherein groups arrive dynamically and we must decide whether to accept or deny them. Once a group is accepted, we need to assign seats to them, while respecting the minimum physical distance requirements mandated by the government.

To address this challenge, we propose a seat assignment policy that takes into account the given seat layout. To generate a seat planning with stochastic demands of groups, we develop a scenario-based method, where we accept or deny an arriving group based on specific rules. Seat assignments can also be made after booking period based on venue requirements. We explore this relaxed setting in our paper to provide a flexible framework that can be tailored to different venue needs.

% The assignment of seats can be adjusted based on the requirements of different venues. For instance, some venues may prefer to only make the decision to accept or deny a group, without assigning seats until the booking period ends. We also consider this relaxed setting in our paper.
% For example, some venues may only make the decision to accept or deny a group without assigning seats until the booking period ends. 
% The assgn-to-seat setting can be relaxed based on the requirement of different venues that we only make the decision but not assign the seats to the groups until the booking period ends.



Our results offer valuable insights for policymakers and venue managers regarding seat utilization rates, and provide a guideline for policies related to social distancing measures. Overall, our approach provides a practical tool for venues to implement social distancing measures while optimizing seat assignments and ensuring the safety of groups.

% Firstly, we develop the scenario-based method to generate a seat planning with stochastic demands of groups. When one group arrives, we decide to accept or deny it according to some certain rules.


% Social distancing has been broadly recognized and practiced as a non-pharmaceutical way to contain the spread of infectious diseases. In this study, we consider the dynamic seat assignment problem with social distancing, where groups arrive dynamically. For each arrival, we must accept or deny the group. Once we accept the group, we should assign the seats to the group, i.e., these seats cannot be taken by others. This setting can be relaxed based on the requirement of different venues that we only make the decision but not assign the seats to the groups until the booking period ends.

% During a pandemic, the government may issue minimum physical distancing requirements between people, which must be respected in the seating assignment. This problem becomes further complicated by the existence of groups of guests who will be seated together. To address this challenge, we provide an seat assignment policy to place arriving groups with given rows of seats.


Keywords: Social Distancing, Stochastic Programming, Seat Assignment, Dynamic Arrival.
