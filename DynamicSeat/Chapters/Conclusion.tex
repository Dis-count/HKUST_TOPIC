% !TEX root = sum1.tex
\section{Conclusion}
In conclusion, this paper addresses the problem of dynamic seat assignment with social distancing in the context of a pandemic. We propose a practical algorithm that balances seat utilization rates and the associated risk of infection to obtain a final seating plan that satisfies social distancing constraints when groups arrive. Our approach provides a comprehensive solution for optimizing seat assignments while ensuring the safety of customers. Our contributions include establishing a deterministic model to analyze the effects of social distancing when demand is known, using two-stage stochastic programming and Benders decomposition methods to obtain the optimal linear solution for stochastic demand situations, and developing a feasible seating plan using scenario-based stochastic programming for the dynamic scenario problem. Our results demonstrate significant improvements over baseline strategies and provide guidance for developing attendance policies. Overall, our study highlights the importance of considering the operational significance behind social distancing and provides a new perspective for the government to adopt mechanisms for setting seat assignments to protect people in the post-pandemic era. Our study demonstrates the efficiency of obtaining the final seating plan using our proposed algorithm. The results indicate that our policy yields a seating plan that is very close to the optimal result. Moreover, our analysis provides managerial guidance on how to set the occupancy rate and largest size of one group under the background of pandemic.

% We mainly focus on how to provide a way to ...

% Our main results show that ...

% discuss the occupancy rate under different demands, 
