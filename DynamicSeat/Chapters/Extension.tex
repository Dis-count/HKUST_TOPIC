% !TEX root = sum1.tex

\section{Extension}

\subsection{Obtain Minimum Number of Rows to Cover Demand}

To find the minimum number of rows to satisfy the demand, we can formulate this problem as a cutting stock problem form and use the column generation method to obtain an approximate solution.

Similar to the concept of pattern in the CSP, let the $k$-th placing pattern of a line of seats with length $S$ into some of the $m$ group types be denoted as a vector $(t^k_1,t^k_2,\ldots,t^k_m)$. Here, $t^k_i$ represents the number of times group type $i$ is placed in the $k$-th placing pattern. For a pattern $(t^k_1,t^k_2,\ldots,t^k_m)$ to be feasible, it must satisfy: $\sum_{i=1}^m t^k_i s_i \leq S$, where $s_i$ is the size of group type $i$. Denote by $K$ the current number of placing patterns.


This problem is to decide how to place a total number of group type $i$ at least $g_i$ times, from all the available placing patterns, so that the total number of rows of seats used is minimized.

Immediately we have the master problem:

\[\begin{split}\mbox{min}\quad & \sum_{k \in K}^K x_{k}\\
 \mbox{s.t.} \quad & \sum_{k \in K}^K t_i^k x_k \geq d_i  \quad  \mbox{ for } i=1,\ldots,m \\
  & x_k \geq 0, \mbox{integer}\quad \mbox{for}~ k \in K,\ldots,K.\\
\end{split}\]

If $K$ includes all possible patterns, we can obtain the optimal solution by solving the corresponding IP. But it is clear that the patterns will be numerous, considering all possible patterns will be time-consuming.

Thus, we need to consider the linear relaxation of the master problem, and the optimal dual variable vector $\lambda$. Using $\lambda$ as the value assigned to each group type $i$, the next problem is to find a feasible pattern $(y_1,y_2,\ldots,y_m)$ that maximizes the product of $\lambda$ and $y$.


Then the corresponding sub-problem is:
\[\begin{split}\mbox{max}\quad & \sum_{i=1}^m \lambda_i y_{i}\\
        \mbox{s.t.} \quad & \sum_{i=1}^m w_i y_i \leq S  \\
        & y_i \geq 0, \mbox{integer}\quad \mbox{for}~ i=1,\ldots,m.\\
\end{split}\]

This is a knapsack problem, its solution will be used as an additional pattern in the master problem.
We should continue to add new pattern until all reduced costs are nonnegative. Then we have an optimal solution to the original linear programming problem.

But note that column generation method cannot gaurantee an optimal solution. If we want to reach the optimal solution, we should tackle with the integer formulation.

\begin{equation}
\begin{aligned}
\min \sum_{k \in K}^{K} y_{k} & \\
\sum_{k=1}^{K} x_{i k} & \geq d_{i} \quad i=1, \ldots, n \\
\sum_{i=1}^{n} a_{i} x_{i k} & \leq S y_{k} \quad k=1, \ldots, K \\
y_{k} & \in\{0,1\} \quad k=1, \ldots, K \\
x_{i k} & \geq 0 \text { and integer } i=1, \ldots, n ; k=1, \ldots, K
\end{aligned}
\end{equation}

$y_k =1$ if line $k$ is used and 0 otherwise, $x_{ik}$ is the number of times group $i$ is placed in row $k$, and $K$ is the upper bound on the number of the rows needed.


\subsection{Online booking} 

The groups can select the seats arbitrarily but with some constraints.

When the customers book the tickets, they will be asked how many seats they are going to book at first. Then we give the possible row numbers for their selection. Finally we give the seat number for their choice.

The second step is based on the other choices of reserved groups, we need to check which rows in the seat assignment include the corresponding-size seats.

In third step, we need to check whether the chosen number destroys the pattern of the row. Use subset sum problem to check every position in the row.

\subsubsection{Mapping Sequences to Scenarios by Continuous Time}

When the time is continuous, we only need to count the number of different group types, the method will be the same.

For the nonhomogeneous Poisson process, $N_{i}(T) \sim Poisson (\int_0^{T} p_i(s)\lambda(s) \mathrm{d} s)$. If the Poisson process is homogeneous, $N_{i} \sim Poisson(\lambda p_i T)$ during the interval $[0, T]$. Then for a realization of the Poisson process, we can still apply the method in Section \ref{MappingSeq}.


\subsection{How to give a balanced seat assignment}

Notice we only give the solution of how to assign seats for each row, but the order is not fixed.

In order to obtain a balanced seat assignment, we use a greedy way to place the seats.

Sort each row by the number of people. Then place the smallest one in row 1, place the largest one in row 2, the second smallest one in row 3 and so on. 

For each row, sort the groups in an ascending/descending order. In a similar way.


\subsection{Property}
Although the solver can solve this problem easily, the analyses on the property of the solution to this problem can help us generate a method for the dynamic situation. 

At first, we consider the types of pattern, which refers to the seat assignment for each row. For each pattern $k$, we use $\alpha_k, \beta_k$ to indicate the number of groups and the left seat, respectively. Denote by $l(k) = \alpha_k + \beta_k -1$ the loss for pattern $k$. The loss represents the number of people lost compared to the situation without social distancing.

Let $I_1$ be the set of patterns with the minimal loss. Then we call the patterns from $I_1$ are largest. Similarly, the patterns from $I_2$ are the second largest, so forth and so on. The patterns with zero left seat are called full patterns. We use the descending form $(t_1, t_2, \ldots, t_k)$ to represent a pattern, where $t_i$ is the new group size. 

\begin{example}\label{ex_largest}
  The length of one row is $S = 21$ and the new size of groups be $L = [2, 3, 4, 5]$. Then these patterns, $(5, 5, 5, 5, 1),(5, 4, 4, 4, 4),(5, 5, 5, 3, 3)$, belong to $I_1$. The demand is $[10, 12, 9, 8]_d$.
\end{example}

To represent a pattern with a fixed length of form, we can use a $(m+1)-$dimensional vector with $m$ group types. The aggregated form can be expressed as $[n_0, n_1, \ldots, n_m]$, where $n_i$ is the number of $i$-th group type, $i=1,\ldots,m$. 
$n_0$ is the number of left seat, its value can only be $0, 1$ because two or more left seats will be assigned to groups. Thus the pattern, $[1, 0, 0, 0, 4]$, is not full because there is one left seat.

Suppose $u$ is the size of the largest group allowed, all possible seats can be assigned are the consecutive integers from 2 to $u$, i.e., $[2,3,\ldots,u]$.
Then we can use the following greedy way to generate the largest pattern. Select the maximal group size,$u$, as many as possible and the left space is assigned to the group with the corresponding size. Let $S = u\cdot q + r$, where $q$ is the number of times $u$ selected. When $r>0$, there are $d[0][u-r][q+1]$ largest patterns with the same loss of $q$. When $r =0$, there is only one possible largest pattern.

Use dynamic programming to solve. $d[k][i][j]$ indicates the number of assignment of using $i$ capacity to allocate $j$ units, $k$ is the number of capacity allocated on the last unit. In our case, $u-r$ is the capacity need to be allocated, $q+1$ is the number of units which corresponds to the groups. Notice that we only consider the number of combinations, so we fix the allocation in ascending order, which means the allocation in current unit should be no less than the last unit.  

The number of largest patterns equals the number of different schemes that allocate $u-r$ on $q+1$ units,i.e., $d[0][u-r][q+1]$.

The recurrence relation is $d[k][i][j] = \sum_{t=k}^{i-k} d[t][i-k][j-1]$. 
When $i < k$, $d[k][i][j] =0$; when $i \geq k$, $d[k][i][1] =1$.

\begin{lem}
% If all patterns associated with an integral feasible solution belong to $I_1$, then this solution is optimal.
The seat assignment made up of the largest patterns is optimal.
\end{lem}

This lemma holds because we cannot find a better solution occupying more seats.

When the demand is so large that the largest patterns can be generated in all rows, an optimal seat assignment can be obtained.

\begin{prop}\label{prop_I_1}
  Let $k^{*} = \arg \max_{k\in I_1} \min_{i} \{\lfloor \frac{d_i}{b_i^k}\rfloor\}$. 
  When $N \leq \max_{k\in I_1} \min_{i} \{\lfloor \frac{d_i}{b_i^k}\rfloor\}$, the optimal seat assignment can be constructed by repeating pattern $k^*$ $N$ times.
  $N$ is the number of rows, $d_i$ is the number of $i$-th group type, $i = 1,2,\ldots, m$, $b_i^k$ is the number of group type $i$ placed in pattern $k$.
\end{prop}

Use the above example \ref{ex_largest} to explain. Take $(5,5,5,5), (5,4,4,4,4), (5,5,4,4,3)$ as the alternative patterns, denoted by pattern $1, 2, 3$ respectively. When $k = 1,2,3$, $\min_{i} \{\lfloor \frac{d_i}{b_i^k}\rfloor\}$ will be $2,3,5$, thus $k^{*}= 3$. So when $N \leq 5$, we can select the pattern $(5,5,4,4,3)$ five times to construct the optimal seat assignment.

\begin{prop}\label{prop_I_2}
  We can construct a seat assignment in the following way. Every time we can select one pattern from $I_1$, then minus the corresponding number of group types from demand and update demand. Repeat this procedure until we cannot generate a largest pattern. If the number of generated patterns is no less than the number of rows, this assignment is optimal.
\end{prop}

% This lead

\newpage
