% !TEX root = sum1.tex
\section{Introduction}

Reducing the spread of the virus while minimizing the economic impact is a significant challenge for governments worldwide. During the epidemic, many companies and countries have implemented restrictive measures, including social distancing, as it is believed to be the most effective non-pharmaceutical treatment to reduce the health effects of Covid-19. Additional measures such as hand washing and the use of masks have also been strongly recommended or enforced by companies and governments.

While social distancing is a recognized and practiced method for containing the spread of infectious diseases, operational guidance for its implementation is still lacking, particularly in social distance measures that involve operational details. An example of such a measure is ensuring social distancing in seating plans.

In this paper, we address the dynamic seating assignment problem with a given set of seats in the context of a pandemic. The government may issue minimum physical distance requirements between people, which must be implemented in the seating plan. The problem becomes further complicated by the existence of groups of guests who can sit together. To address this challenge, we develop a mechanism for seat planning, which includes a model to assess the riskiness of a seating plan and a solution approach to balance seat utilization rates and the associated risk of infection.
Our proposed algorithm has the potential to help companies and governments optimize seat assignments while maintaining social distance measures and ensuring the safety of groups.


% Note that the original global schedule could be changed as the price changes. Meanwhile, ...
% To promote ..., it is of great importance to ...

Our main contributions in this paper are summarized as follows:

First, this study presents the first attempt to consider the arrangement of seat assignments with social distancing under dynamic arrivals. While many studies in the literature highlight the importance of social distance in controlling the spread of the virus, they often focus too much on the model and do not provide much insight into the operational significance behind social distance \cite{barry2021optimal, fischetti2021safe}. Recent studies have explored the effects of social distance on health and economics, mainly in the context of aircraft \cite{salari2020social, ghorbani2020model, salari2022social}. Our study provides a new perspective to help the government adopt a mechanism for setting seat assignments to protect people in the post-pandemic era.


Second, we establish a deterministic model to analyze the effects of social distancing when the demand is known. Due to the medium size of the problem, we can solve the IP model directly. We then consider the stochastic demand situation where the demands of different group types are random. By using two-stage stochastic programming and Benders decomposition methods, we obtain the optimal linear solution.

Third, to address the dynamic scenario problem, we first obtain a feasible seating plan using scenario-based stochastic planning. We then make a decision for each incoming group based on a nested policy, either accepting or rejecting the group. Our results demonstrate a significant improvement over a first-come first-served baseline strategy and provide guidance on how to develop attendance policies.

This study focuses on dynamic seat assignment with social distancing, where a one-seat distance must be maintained between different groups. Our goal is to obtain the final seating plan that satisfies social distancing constraints and implement the seat assignment when groups arrive. Overall, our proposed approach provides a comprehensive solution for dynamic seat assignment with social distancing, taking into account both deterministic and stochastic demand scenarios.

% Our results demonstrate the effectiveness of our approach in balancing social distancing requirements with revenue generation, providing valuable insights for policymakers and venue managers. Specifically, our proposed approach can help cinemas, concert venues, and other public spaces optimize seat assignments while ensuring the safety of patrons. It provides a practical tool for venues to implement social distancing measures in a flexible and efficient manner, adapting to changes in demand and maximizing revenue generation while maintaining social distancing measures.

% With this new .., we illustrate how to assign the seats by the govenment/stakeholder to balance health and economic issues. In addition, we also provide managerial guidance for the government on how to publish the related policy to make the tradeoff between economic maintenance and risk management.


The rest of this paper is structured as follows. The following section reviews relevant literature. We describe the motivating problem in Section 3. In Section 4, we establish the model and analyze its properties. Section 5 demonstrates the dynamic form and its property. Section 6 gives the results. The conclusions are shown in Section 7.

\newpage
