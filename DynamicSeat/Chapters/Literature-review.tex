% !TEX root = sum1.tex
\section{Literature Review}

\subsection{Dynamic seat assignment}
Dynamic seat assignment is a process of assigning seats to passengers on a transportation vehicle, such as an airplane, train, or bus, in a way that maximizes the efficiency and convenience of the seating arrangements \cite{hamdouch2011schedule, berge1993demand}. This approach is also used in cinemas and concert venues to optimize the seating arrangements for the audience, with the goal of providing the best possible experience for each audience member while maximizing ticket sales and revenue for the venue.

However, implementing dynamic seat assignment with social distance in cinemas and concert venues presents several challenges. The primary challenge is to balance the need for safety with the desire to maximize revenue for the venue. This requires careful consideration of various factors, such as the layout of the venue, the number of available seats, and the preferences of the audience. To implement dynamic seat assignment with social distance, venues may need to adjust their seating plans to accommodate the recommended distance between audience members. This may involve reducing the overall number of seats or reconfiguring the seating arrangements.

Implementing dynamic seat assignment with social distance can be done manually by the staff or through automated systems that use algorithms to optimize the seat assignments based on various factors, such as ticket sales, seat availability, and customer preferences. However, the implementation of social distance measures poses unique challenges that require careful planning and consideration of various factors to balance safety with revenue generation.

Despite these challenges, dynamic seat assignment with social distance is a promising approach to help ensure the safety of audience members in cinemas, concert venues, and other public spaces during the COVID-19 pandemic.


\subsection{Seat assignemnt with social distance}

Almost all existing work focuses on the static version of seat assignment with social distance.





% Dynamic seat assignment with social distance can be implemented manually by the venue staff, or through automated systems that use algorithms to optimize the seating arrangements based on various factors, such as the number of available seats, customer preferences, and the recommended distance between audience members.


\subsection{Scenario generation}

It is challenging to consider all the possible realizations; thus, it is practicable to use discrete distributions with a finite number of scenarios to approximate the random demands. This procedure is often called scenario generation.

Some papers consider obtaining a set of scenarios that realistically represents the distributions of the random parameters but is not too large. \cite{feng2013scenario} \cite{casey2005scenario}
\cite{henrion2018problem}

Another process to reduce the calculation is called scenario reduction. It tries to approximate the original scenario set with a smaller subset that retains essential features.



% Every time we can regenerate the scenario based on the realized demands. (Use the conditional distribution or the truncated distribution)


% Suppose that the groups arrive from small to large according to their size. Once a larger group comes, the smaller one will never appear again.

% When a new group arrives (suppose we have accepted $n$ groups with the same size), we accept or reject it according to the supply (when $n+1 < \text{supply}$, we accept it). 
% then update the scenario set according to the truncated distribution. We can obtain a new supply with the new probability and scenario set.



\newpage
