% !TEX root = sum1.tex
\section{Literature Review}

\subsection{Seat assignment with social distance}

While social distancing is a recognized and practiced method for containing the spread of infectious diseases, operational guidance for its implementation is still lacking, particularly in social distance measures that involve operational details. An example of such a measure is ensuring social distancing in seating plans.

Most of the current research on seat assignment with social distancing focuses on the static version of the problem. This typically involves creating an IP model with social distancing constraints, which is then solved either heuristically or directly. However, the social distancing constraint in their works- which requires individuals to maintain a certain distance from one another - means that a fixed seating layout cannot fully utilize available space and may not be practical for commercial use.

As we transition into the post-pandemic era, many people may still be concerned about the risk of contracting the virus. This may lead to hesitancy in attending social events or gatherings that require close proximity to others.

Maintaining social distancing can be challenging in crowded public spaces, which can cause further anxiety for individuals.

Meanwhile, social distancing measures can have a significant economic impact, especially on businesses that rely on large crowds or close contact between individuals.

However, if organizers can develop a scheme that balances the need for social distancing with economic considerations, it could help to alleviate people's concerns and protect their health. Such a scheme would go a long way in dispelling people's doubts and fears. Although social distance has reduced some seat income, people have less potential concerns and are more willing to buy tickets.




% Almost all existing work focuses on the static version of seat assignment with social distance. They commonly build a IP model with social distance constraint, then solve it heuristically or directly.
% The social distance constraint is that people should keep a certain distance. However, for a fixed seating layout, the setting cannot fully utilize seating space and cannot be applied in commerce.


% Dynamic seat assignment with social distance can be implemented manually by the venue staff, or through automated systems that use algorithms to optimize the seating arrangements based on various factors, such as the number of available seats, customer preferences, and the recommended distance between audience members.



\subsection{Dynamic seat assignment}
Dynamic seat assignment is a process of assigning seats to passengers on a transportation vehicle, such as an airplane, train, or bus, in a way that maximizes the efficiency and convenience of the seating arrangements \cite{hamdouch2011schedule, berge1993demand}. This approach is also used in cinemas and concert venues to optimize the seating arrangements for the audience, with the goal of providing the best possible experience for each audience member while maximizing ticket sales and revenue for the venue.

However, implementing dynamic seat assignment with social distance in cinemas and concert venues presents several challenges. The primary challenge is to balance the need for safety with the desire to maximize revenue for the venue. This requires careful consideration of various factors, such as the layout of the venue, the number of available seats, and the preferences of the audience. To implement dynamic seat assignment with social distance, venues may need to adjust their seating plans to accommodate the recommended distance between audience members. This may involve reducing the overall number of seats or reconfiguring the seating arrangements.

Implementing dynamic seat assignment with social distance can be done manually by the staff or through automated systems that use algorithms to optimize the seat assignments based on various factors, such as ticket sales, seat availability, and customer preferences. However, the implementation of social distance measures poses unique challenges that require careful planning and consideration of various factors to balance safety with revenue generation.

Despite these challenges, dynamic seat assignment with social distance is a promising approach to help ensure the safety of audience members in cinemas, concert venues, and other public spaces during the COVID-19 pandemic.

Dynamic Knapsack Problem?


% \subsection{Scenario generation}

% It is challenging to consider all the possible realizations; thus, it is practicable to use discrete distributions with a finite number of scenarios to approximate the random demands. This procedure is often called scenario generation.

% Some papers consider obtaining a set of scenarios that realistically represents the distributions of the random parameters but is not too large. \cite{feng2013scenario} \cite{casey2005scenario}
% \cite{henrion2018problem}

% Another process to reduce the calculation is called scenario reduction. It tries to approximate the original scenario set with a smaller subset that retains essential features.



% Every time we can regenerate the scenario based on the realized demands. (Use the conditional distribution or the truncated distribution)


% Suppose that the groups arrive from small to large according to their size. Once a larger group comes, the smaller one will never appear again.

% When a new group arrives (suppose we have accepted $n$ groups with the same size), we accept or reject it according to the supply (when $n+1 < \text{supply}$, we accept it). 
% then update the scenario set according to the truncated distribution. We can obtain a new supply with the new probability and scenario set.


\newpage
