% !TEX root = sum1.tex
\section{Literature Review}

\cite{gilmore1961linear} gives the well-known initial compact formulation.

\cite{scheithauer1995modified} carries out the computational experiments with a lot of randomly generated instances of the one-dimensional cutting stock problem. 


% Firstly, multiple knapsack problem.
%
% Secondly, stochastic programming/ benders decomposition?
% 
%  Dynamic

It is challenging to consider all the possible realizations; thus, it is practicable to use discrete distributions with a finite number of scenarios to approximate the random demands. This procedure is often called scenario generation.

Some papers consider obtaining a set of scenarios that realistically represents the distributions of the random parameters but is not too large. \cite{feng2013scenario} \cite{casey2005scenario}
\cite{henrion2018problem}

Another process to reduce the calculation is called scenario reduction. It tries to approximate the original scenario set with a smaller subset that retains essential features.


% For the stochastic situation, we assume the group sizes are discretized from independent random variables following some distribution.(non-negative)

Every time we can regenerate the scenario based on the realized demands. (Use the conditional distribution or the truncated distribution)


Suppose that the groups arrive from small to large according to their size. Once a larger group comes, the smaller one will never appear again.

When a new group arrives (suppose we have accepted $n$ groups with the same size), we accept or reject it according to the supply (when $n+1 < \text{supply}$, we accept it), then update the scenario set according to the truncated distribution. We can obtain a new supply with the new probability and scenario set.

With the conclusion of section , we know how to reject a request. Once we reject one group, we will reject all groups of the same size. 


\newpage
