% !TEX root = sum1.tex
\section{Literature Review}

The present study is closely connected to the following research areas -- seat assignment with social distancing and dynamic seat assignment. The subsequent sections review literature pertaining to each perspective and highlight significant differences between the present study and previous research.

Define 
seat planning:

seat assignment:

\subsection{Seat assignment with social distancing}
Since the outbreak of covid-19, social distancing is a well-recognized and practiced method for containing the spread of infectious diseases \cite{moosa2020effectiveness}. An example of operational guidance is ensuring social distancing in seating plans.
% particularly in social distancing measures that involve operational details. 

Social distancing in seat planning has attacted considerable attention from the research area. The applications include the allocation of seats on airplanes \cite{ghorbani2020model}, classroom layout planning \cite{bortolete2022support}, seat assignment in long-distancing trains \cite{haque2022optimization}. The social distancing can be implemented in various forms, such as fixed distances or seat lengths. Fischetti et al.\cite{fischetti2021safe} used the Euclidean distancing between positions in restaurants and beach umbrellas. However, different scenarios may require different forms of social distancing; for instance, on an airplane, the distancing between seat assignments and the aisle must be considered \cite{salari2022social}, while in a classroom, maximizing social distancing between students is a priority \cite{bortolete2022support}.


% and family-group seat selection in a theater. \cite{fischetti2021safe}

% 根据社交距离,可以分为固定社交距离((classroom),和固定
% fixed social distancing, 
% distancing between seat assignment and the aisle

% 根据场景可以分为飞机,火车,教室。

% 根据座位是否固定,分为固定座位和非固定座位。

% \cite{bortolete2022support} presents a tool for classroom layout planning that considers social distancing during the COVID-19 pandemic. The tool addresses both fixed and non-fixed position seat allocation problems, utilizing integer optimization models and circle packing techniques.

% Ghorbani et al. \cite{ghorbani2020model} proposes a model to minimize the economic impact of COVID-19 on businesses that have implemented social distancing measures, while also minimizing the health impact on customers and employees. The study applies the model to the scenario of aircraft seat layouts using the Annealing Monte Carlo technique, demonstrating the necessity of using the model to find optimal arrangements of passengers.

% The COVID-19 pandemic has led to strict regulations that have affected major economies, prompting policymakers and planners to devise strategies that balance virus control with economic activities.

% Haque and Hamid \cite{haque2022optimization} discusses the potential implications of seat inventory management in long-distancing passenger trains and proposes a novel seat assignment policy that aims to reduce virus diffusion risk among passengers by minimizing interaction. A mixed-integer linear programming problem is formulated to simultaneously maximize operator revenue and minimize virus diffusion, and tested using real-life data from Indian Railways. 


These researchs focus on the static version of the problem. This typically involves creating an IP model with social distancing constraints(\cite{bortolete2022support, ghorbani2020model, haque2022optimization}), which is then solved either heuristically or directly. The seat allocation of the static form is useful for fixed people, for example, the students in one class. 
But it is not be practical for the dynamic arrivals in commercial events.


% As we transition into the post-pandemic era, many people may still be concerned about the risk of contracting the virus. This may lead to hesitancy in attending social events or gatherings that require close proximity to others.


However, if organizers can develop a scheme that balances the need for social distancing with economic considerations, it could help to alleviate people's concerns and protect their health. Such a scheme would go a long way in dispelling people's doubts and fears. Although social distancing has reduced some seat income, people have less potential concerns and are more willing to buy tickets.

% Almost all existing work focuses on the static version of seat assignment with social distancing. They commonly build a IP model with social distancing constraint, then solve it heuristically or directly.
% The social distancing constraint is that people should keep a certain distancing. However, for a fixed seating layout, the setting cannot fully utilize seating space and cannot be applied in commerce.


% Dynamic seat assignment with social distancing can be implemented manually by the venue staff, or through automated systems that use algorithms to optimize the seating arrangements based on various factors, such as the number of available seats, customer preferences, and the recommended distancing between audience members.

% \subsection{Group seat reservation}
In our setting, the groups must be accepted on an all-or-none basis, i.e., members of the same family or group are seated together. 
This group seat reservation policy has various applications in industries such as hotels \cite{li2013modeling}, working spaces\cite{fischetti2021safe}, public transport\cite{deplano2019offline}, sports arenas\cite{kwag2022optimal}, and large-scale events \cite{lewis2016creating}. This policy has significant impacts on passenger satisfaction and revenue, with the study \cite{yuen2002group} showing that passenger groups increase revenue by filling seats that would otherwise be empty. Traditional works \cite{clausen2010off, deplano2019offline}in transportation focus on maximizing capacity utilization or reducing total capacity needed for passenger rail, typically modeling these problems as knapsack or binpacking problems.

In contrast, there is a lack of research on group seat reservations for booking tickets for cinemas, where available seats are typically displayed for customers to choose from for low-demand movie tickets. For concerts with high demand, it is usually not possible to choose seats independently, and the organizer will inform seat information after confirming the order. 
For movies of the same time period, the ticket prices are the same, while for the same concert, although there are different ticket prices, for the same region, the ticket prices are the same. Therefore, we can consider different ticket prices separately. 
In the absence of an epidemic, all requests for tickets can be considered one by one.However, the COVID-19 pandemic has shed new light on the potential benefits of group reservations, as they can improve revenue without increasing the risk of infection. Moore et al. \cite{moore2021seat} suggest that relaxing social distancing norms within a family may increase occupancy without an significant increasing in virus transmission risk.

% However, for booking tickets for cinemas or concerts, there are almost no articles discussing group seat reservations. For movie tickets with low demand, the system will display a table of available seats for customers to choose from. For concerts with high demand, it is usually not possible to choose seats independently. The organizer will inform the seat information after confirming the order.
% The recent pandemic has offered new insights on group reservations as they exhibit improvement in revenue without an increase in the risk of getting infected. 

In our study, the group seat reservation policy arouses a bigger problem when we try to find the seat assignment policy. Some related literature mentioned the seat assignment under pandemic for groups are represented below.

Fischetti et al. \cite{fischetti2021safe} proposed a seating assignment for known groups of customers in amphitheaters. Haque and Hamid \cite{haque2022optimization} considers grouping passengers with the same origin-destination pair of travel and assigning seats in long-distance passenger trains. Salari et al. \cite{salari2022social} performed group seat assignment in airplanes during the pandemic and found that increasing passenger groups can yield greater social distancing than single passengers. Haque and Hamid \cite{haque2023social} aim to optimize seating assignments on trains by minimizing the risk of virus spread while maximizing revenue. 

The specific number of groups in their models is known in advance. But in our study, we only know the arrival probabilities of different groups.

% discusses the potential implications of seat inventory management in long-distance passenger trains 


% Dundar and Karakose \cite{dundar2021seat} proposed a two-stage algorithm for classroom seat assignment during the pandemic, with the first phase maximizing total allocations and the second phase maximizing the minimum interpersonal distance between students. 

% Seat Assignments With Physical Distancing in Single-Destination Public Transit Settings



% spiral seat numbering

% For the sake of convenience, we list some of the intricate characteristics of the related studies and our study in Table 1.


\subsection{Dynamic seat assignment}
Dynamic seat assignment is a process of assigning seats to passengers on a transportation vehicle, such as an airplane, train, or bus, in a way that maximizes the efficiency and convenience of the seating arrangements \cite{hamdouch2011schedule, berge1993demand}. 

Jiang et al. \cite{jiang2015dynamic} proposes a revenue management approach for high-speed rail (HSR) passenger ticket assignment with dynamic adjustments. The approach integrates short-term demand forecasting, ticket assignment, and dynamic ticket adjustment mechanisms to allocate passenger tickets during presale periods and avoid situations where tickets are insufficient at some stations while seats remain empty.

% Dynamic capacity management in airlines involves reassigning aircraft to match the supply and demand for seats, resulting in increased passenger capacity and revenue. \cite{de2004impact} proposes an EMSRd derivative of the EMSRb booking limit calculation method that considers the impact of future capacity changes. The EMSRd approach efficiently solves the fleet assignment problem and shows potential revenue improvements of over 1\%.

Zhu et al. \cite{zhu2023assign} addresses a revenue management problem arising from the selling of high-speed train tickets in China, where each request must be assigned to a single seat throughout the entire journey. The authors propose a modified network revenue management model and introduce a bid-price control policy based on a novel maximal sequence principle. They also propose a "re-solving a dynamic primal" policy that achieves uniformly bounded revenue loss. The study reveals connections between this problem and traditional network revenue management problems and shows that the impact of the assign-to-seat restriction can be limited with the proposed methods.

Dynamic seat assignment is also used in cinemas and concert venues to optimize the seating arrangements for the audience, with the goal of providing the best possible experience for each audience member while maximizing ticket sales and revenue for the venue. However, implementing dynamic seat assignment with social distancing in cinemas and concert venues presents several challenges. The primary challenge is to balance the need for safety with the desire to maximize revenue for the venue. This requires careful consideration of various factors, such as the layout of the venue, the number of available seats, and the preferences of the audience. To implement dynamic seat assignment with social distancing, venues may need to adjust their seating plans to accommodate the recommended distancing between audience members. This may involve reducing the overall number of seats or reconfiguring the seating arrangements.

% Implementing dynamic seat assignment with social distancing can be done manually by the staff or through automated systems that use algorithms to optimize the seat assignments based on various factors, such as ticket sales, seat availability, and customer preferences. However, the implementation of social distancing measures poses unique challenges that require careful planning and consideration of various factors to balance safety with revenue generation.

Despite these challenges, dynamic seat assignment with social distancing is a promising approach to help ensure the safety of audience members in cinemas, concert venues, and other public spaces during the COVID-19 pandemic.

% Dynamic Knapsack Problem? Revenue management?



% \subsection{Scenario generation}

% It is challenging to consider all the possible realizations; thus, it is practicable to use discrete distributions with a finite number of scenarios to approximate the random demands. This procedure is often called scenario generation.

% Some papers consider obtaining a set of scenarios that realistically represents the distributions of the random parameters but is not too large. \cite{feng2013scenario} \cite{casey2005scenario}
% \cite{henrion2018problem}

% Another process to reduce the calculation is called scenario reduction. It tries to approximate the original scenario set with a smaller subset that retains essential features.



% Every time we can regenerate the scenario based on the realized demands. (Use the conditional distribution or the truncated distribution)


% Suppose that the groups arrive from small to large according to their size. Once a larger group comes, the smaller one will never appear again.

% When a new group arrives (suppose we have accepted $n$ groups with the same size), we accept or reject it according to the supply (when $n+1 < \text{supply}$, we accept it). 
% then update the scenario set according to the truncated distribution. We can obtain a new supply with the new probability and scenario set.


\newpage
