% !TEX root = sum1.tex
\section{Introduction}
Governments worldwide have been faced with the challenge of reducing the spread of Covid-19 while minimizing the economic impact. Social distancing has been widely implemented as the most effective non-pharmaceutical treatment to reduce the health effects of the virus. 
This website records a timeline of Covid-19 and the relevant epidemic prevention measures\cite{Covid19Timeline}. For instance, in March 2020, the Hong Kong government implemented restrictive measures such as banning indoor and outdoor gatherings of more than four people, requiring restaurants to operate at half capacity. As the epidemic worsened, the government tightened measures by limiting public gatherings to two people per group in July 2020. As the epidemic subsided, the Hong Kong government gradually relaxed social distancing restrictions, allowing public group gatherings of up to four people in September 2020. In October 2020, pubs were allowed to serve up to four people per table, and restaurants could serve up to six people per table. 

Specifically, the Hong Kong government also implemented different measures in different venues \cite{Gov202209}. For example, the catering businesses will have different social distancing requirements depending on their mode of operation for dine-in services. They can operate at 50\%, 75\%, or 100\% of their normal seating capacity at any one time, with a maximum of 2, 2, or 4 people per table, respectively. Bars and pubs may open with a maximum of 6 persons per table and a total number of patrons capped at 75\% of their capacity. The restrictions on the number of persons allowed in premises such as cinemas, performance venues, museums, event premises, and religious premises will remain at 85\% of their capacity.

The measures announced by the Hong Kong government mainly focus on limiting the number of people in each group and the seat occupancy rate. However, implementing these policies in operations can be challenging, especially for venues with fixed seating layouts. In our study, we will focus on addressing this challenge in commercial premises, such as cinemas and music concert venues.

We aim to provide a practical tool for venues to optimize seat assignments while ensuring the safety of groups by proposing a seat assignment policy that takes into account social distancing requirements and the given seating layout. Additionally, we will offer guidance on setting appropriate occupancy rates and group sizes. We strive to enable venues to implement social distancing measures effectively by providing a tailored solution that accommodates their specific seating arrangements and operational constraints.

Some papers use seat assignment to express seat planning, to avoid confusion about seat planning and seat assignment, we clarify the difference between them.

For deterministic demand, seat planning equals seat assignment.

Before we introduce the ways of traditional booking tickets, we need to define the seat planning and seat assignment. Seat planning involves determining the best layout and arrangement of seats in a venue or space based on factors such as the size of the room, the number of attendees, and the type of event. This can include deciding on the number of rows and columns, the spacing between seats, and any special requirements such as seats partition with social distancing. 
Seat assignment, on the other hand, involves assigning specific seats to attendees based on factors such as ticket type and availability. This is typically done closer to the event date and can involve a variety of methods such as manual seat selection by the attendee or an automated system that assigns seats based on predetermined criteria.

% In summary, seat planning is about designing the overall seating arrangement, while seat assignment is about assigning specific seats to individual attendees.

When purchasing tickets for movies or concerts, there are generally two approaches to seat assignment: seat assignment after the booking period and seat assignment during the booking period.

After all groups arrived./ for each group arrival.

The seat assignment after booking period approach involves delaying seat assignment until after the reservation deadline has passed. This means that the organizer does not need to immediately allocate seats to customers, but has to make the decision to accept or deny, so implementing social distancing restrictions will not affect the booking process. After the reservation deadline, the seller will inform customers of the seat layout information before admission. For instance, in venues such as singing concert halls, where there is high ticket demand and numerous seats available, organizers usually do not determine the seats during booking. Instead, they will inform customers of the seat information after the overall demands are determined. This approach allows for more flexibility in seat assignments and can accommodate changes in group sizes or preferences. However, in other venues where seating options are limited, it may be necessary to assign seats immediately upon accepting a group.

On the other hand, the seat assignment during the booking period approach typically involves the cinema releasing the seating charts online, which show the available and unavailable seats, when there are no social distancing requirements. Customers can then choose their desired seats and reserve them by paying for their tickets. After successful payment, the seats are allocated to the customers. However, due to social distancing requirements, this approach needs to be modified. Seat assignments are still arranged when groups book their tickets, but the seller will provide the seat information directly. For example, in movie theaters with relatively few seats, the demands for tickets are usually low enough to allow for free selection of seats directly online. Early seat planning can satisfy the requirement of social distancing and save costs without changing seat allocation. The seat allocation could remain for one day because the same film genre will likely attract similar groups with similar seating preferences.


Our study mainly focuses on the latter situation where customers come dynamically, and the seat assignment needs to be made immediately without knowing the number and composition of future customers. In Section 6, we also consider the situation where the seat assignment can be made after the booking period. 

% Dynamic seat assignment is also used in cinemas and concert venues to optimize the seating arrangements for the audience, with the goal of providing the best possible experience for each audience member while maximizing ticket sales and revenue for the venue. However, implementing dynamic seat assignment with social distancing in cinemas and concert venues presents several challenges. The primary challenge is to balance the need for safety with the desire to maximize revenue for the venue. This requires careful consideration of various factors, such as the layout of the venue, the number of available seats, and the preferences of the audience. To implement dynamic seat assignment with social distancing, venues may need to adjust their seating plans to accommodate the recommended distancing between audience members. This may involve reducing the overall number of seats or reconfiguring the seating arrangements.

% Despite these challenges, dynamic seat assignment with social distancing is a promising approach to help ensure the safety of audience members in cinemas, concert venues, and other public spaces during the COVID-19 pandemic.

This paper focuses on addressing the dynamic seating assignment problem with a given set of seats in the context of a pandemic. The government issues a maximum number of people allowed in each group and a maximum capacity percentage, which must be implemented in the seat planning. The problem becomes further complicated by the existence of groups of guests who can sit together.

To address this challenge, we have developed a mechanism for seat planning. Our proposed algorithm includes a solution approach to balance seat utilization rates and the associated risk of infection. Our goal is to obtain the final seating plan that satisfies social distancing constraints and implement the seat assignment when groups arrive.

Our approach provides a practical tool for venues to optimize seat assignments while ensuring the safety of their customers. The proposed algorithm has the potential to help companies and governments optimize seat assignments while maintaining social distancing measures and ensuring the safety of groups. Overall, our study offers a comprehensive solution for dynamic seat assignment with social distancing in the context of a pandemic.


Our main contributions in this paper are summarized as follows:

First, this study presents the first attempt to consider the arrangement of seat assignments with social distancing under dynamic arrivals. While many studies in the literature highlight the importance of social distancing in controlling the spread of the virus, they often focus too much on the model and do not provide much insight into the operational significance behind social distancing \cite{barry2021optimal, fischetti2021safe}. Recent studies have explored the effects of social distancing on health and economics, mainly in the context of aircraft \cite{salari2020social, ghorbani2020model, salari2022social}. Our study provides a new perspective to help the government adopt a mechanism for setting seat assignments to protect people in the post-pandemic era.

Second, we establish a deterministic model to analyze the effects of social distancing when the demand is known. Due to the medium size of the problem, we can solve the IP model directly. We then consider the stochastic demand situation where the demands of different group types are random. By using two-stage stochastic programming and Benders decomposition methods, we obtain the optimal linear solution.

Third, to address the dynamic scenario problem, we first obtain a feasible seating plan using scenario-based stochastic programming. We then make a decision for each incoming group based on a nested policy, either accepting or rejecting the group. Our results demonstrate a significant improvement over a first-come first-served baseline strategy and provide guidance on how to develop attendance policies.



% Our results demonstrate the effectiveness of our approach in balancing social distancing requirements with revenue generation, providing valuable insights for policymakers and venue managers. Specifically, our proposed approach can help cinemas, concert venues, and other public spaces optimize seat assignments while ensuring the safety of patrons. It provides a practical tool for venues to implement social distancing measures in a flexible and efficient manner, adapting to changes in demand and maximizing revenue generation while maintaining social distancing measures.

% With this new .., we illustrate how to assign the seats by the govenment/stakeholder to balance health and economic issues. In addition, we also provide managerial guidance for the government on how to publish the related policy to make the tradeoff between economic maintenance and risk management.


The rest of this paper is structured as follows. The following section reviews relevant literature. We describe the motivating problem in Section 3. In Section 4, we establish the stochastic model,analyze its properties and give the seating planning. Section 5 demonstrates the dynamic seat assignment during booking period and after booking period. Section 6 gives the results. The conclusions are shown in Section 7.

\newpage
