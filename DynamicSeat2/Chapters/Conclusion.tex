% !TEX root = sum1.tex
\section{Conclusion}
In conclusion, this paper addresses the problem of dynamic seat assignment with social distancing in the context of a pandemic. The problem is different from the seat planning problem studied in the literature owing to consideration of real-time seat assignment. After analyzing the static and the dunamic version of this problem, we propose an efficient allocation policy. Specifically, we develop a stochastic programming to obtain the target seat planning, then introduce stochastic planning policy to assign seats for arriving groups, which includes how to assign the group when there is no corresponding planning in the target seat planning. We also consider the classical bid-price control, booking limit control policies which are inferior to our proposed approach. Numerical experiments show the effectiveness of our proposed approach.



% Our contributions include establishing a deterministic model to analyze the effects of social distancing when demand is known, using Benders decomposition methods to obtain the optimal solution for scenario-based stochastic programming, and developing a seat assignment policy for the dynamic situation. Our results demonstrate significant improvements over baseline strategies and provide guidance for developing attendance policies. Overall, our study highlights the importance of considering the operational significance behind social distancing and provides a new perspective for the government to adopt mechanisms for setting seat assignments to protect people in the post-pandemic era. Our study demonstrates the efficiency of obtaining the final seat planning using our proposed algorithm. 

% Moreover, our analysis provides managerial guidance on how to set the occupancy rate and largest size of one group under the background of pandemic.


% \begin{table}[H]
%     \centering
%     \caption{xxxx}
%     \begin{tabular}{cccc}
%  \hline
%  a & aaaaaaaaa & aaaaaaaaaaaaaa & aaaaaaaaaaaaaaaaaaaa \\
%  \hline
%  a & \makebox[5ex][r]{123} & \makebox[6ex][r]{123456} & \makebox[6ex][r]{1}\\
%  a & \makebox[5ex][r]{12345} & \makebox[6ex][r]{123} & \makebox[6ex][r]{123} \\
%  a & \makebox[5ex][r]{1} & \makebox[6ex][r]{1234} & \makebox[6ex][r]{123456} \\
%  \hline
%  \end{tabular}
%  \end{table}