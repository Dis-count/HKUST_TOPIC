% !TEX root = sum1.tex
\section{Literature Review}

The present study is closely connected to the following research areas -- seat planning with social distancing and dynamic seat assignment. The subsequent sections review literature pertaining to each perspective and highlight significant differences between the present study and previous research.


\subsection{Seat Planning with Social Distancing}
Since the outbreak of covid-19, social distancing is a well-recognized and practiced method for containing the spread of infectious diseases \cite{moosa2020effectiveness}. An example of operational guidance is ensuring social distancing in seating plans.
% particularly in social distancing measures that involve operational details. 

Social distancing in seat planning has attacted considerable attention from the research area. The applications include the allocation of seats on airplanes \cite{ghorbani2020model}, classroom layout planning \cite{bortolete2022support}, seat planning in long-distancing trains \cite{haque2022optimization}. The social distancing can be implemented in various forms, such as fixed distances or seat lengths. Fischetti et al.\cite{fischetti2021safe} consider how to plant positions with social distancing in restaurants and beach umbrellas. Different venues may require different forms of social distancing; for instance, on an airplane, the distancing between seats and the aisle must be considered \cite{salari2022social}, while in a classroom, maximizing social distancing between students is a priority \cite{bortolete2022support}.

These researchs focus on the static version of the problem. This typically involves creating an IP model with social distancing constraints(\cite{bortolete2022support, ghorbani2020model, haque2022optimization}), which is then solved either heuristically or directly. The seat allocation of the static form is useful for fixed people, for example, the students in one class. But it is not be practical for the dynamic arrivals in commercial events.


% and family-group seat selection in a theater. \cite{fischetti2021safe}


% Dynamic seat assignment with social distancing can be implemented manually by the venue staff, or through automated systems that use algorithms to optimize the seating arrangements based on various factors, such as the number of available seats, customer preferences, and the recommended distancing between audience members.

% \subsection{Group seat reservation}
The recent pandemic has shed light on the benefits of group reservations, as they have been shown to increase revenue without increasing the risk of infection \cite{moore2021seat}. In our specific setting, we require that groups be accepted on an all-or-none basis, meaning that members of the same family or group must be seated together. However, the group seat reservation policy poses a significant challenge when it comes to determining the seat assignment policy.


This group seat reservation policy has various applications in industries such as hotels \cite{li2013modeling}, working spaces\cite{fischetti2021safe}, public transport\cite{deplano2019offline}, sports arenas\cite{kwag2022optimal}, and large-scale events \cite{lewis2016creating}. This policy has significant impacts on passenger satisfaction and revenue, with the study \cite{yuen2002group} showing that passenger groups increase revenue by filling seats that would otherwise be empty. Traditional works \cite{clausen2010off, deplano2019offline}in transportation focus on maximizing capacity utilization or reducing total capacity needed for passenger rail, typically modeling these problems as knapsack or binpacking problems.

Some related literature mentioned the seat planning under pandemic for groups are represented below.
Fischetti et al. \cite{fischetti2021safe} proposed a seating planning for known groups of customers in amphitheaters. Haque and Hamid \cite{haque2022optimization} considers grouping passengers with the same origin-destination pair of travel and assigning seats in long-distance passenger trains. Salari et al. \cite{salari2022social} performed group seat assignment in airplanes during the pandemic and found that increasing passenger groups can yield greater social distancing than single passengers. Haque and Hamid \cite{haque2023social} aim to optimize seating assignments on trains by minimizing the risk of virus spread while maximizing revenue. The specific number of groups in their models is known in advance. But in our study, we only know the arrival probabilities of different groups.

This paper \cite{blom2022filling} discusses strategies for filling a theater by considering the social distancing and group arrivals, which is similar to ours. However, unlike our project, it only focuses on a specific location layout and it is still based on a static situation by giving the proportion of different groups.

% In contrast, there is a lack of research on group seat reservations for booking tickets for cinemas, where available seats are typically displayed for customers to choose from for low-demand movie tickets. For concerts with high demand, it is usually not possible to choose seats independently, and the organizer will inform seat information after confirming the order. 
% For movies of the same time period, the ticket prices are the same, while for the same concert, although there are different ticket prices, for the same region, the ticket prices are the same. Therefore, we can consider different ticket prices separately. 
% In the absence of an epidemic, all requests for tickets can be considered one by one. However, the COVID-19 pandemic has shed new light on the potential benefits of group reservations, as they can improve revenue without increasing the risk of infection. 


% Dundar and Karakose \cite{dundar2021seat} proposed a two-stage algorithm for classroom seat assignment during the pandemic, with the first phase maximizing total allocations and the second phase maximizing the minimum interpersonal distance between students. 


\subsection{Dynamic Seat Assignment}
Our model in its static form can be viewed as a specific instance of the multiple knapsack problem \cite{pisinger1999exact}, where we aim to assign a subset of groups to some distinct rows. In our dynamic form, the decision to accept or reject groups is made at each stage as they arrive. The related problem can be dynamic knapsack problem \cite{kleywegt1998dynamic}, where there is one knapsack.

% and dynamic bin-packing problem \cite{coffman1983dynamic, berndt2020fully} where items arrive and depart online and all knapsacks are the same. 

% However, solving this problem is strongly NP-hard, which means that finding an optimal solution for large instances of the problem is computationally challenging \cite{pisinger1999exact}.

Dynamic seat assignment is a process of assigning seats to passengers on a transportation vehicle, such as an airplane, train, or bus, in a way that maximizes the efficiency and convenience of the seating arrangements \cite{hamdouch2011schedule, berge1993demand, zhu2023assign}. 

Our problem is closely related to the network revenue management (RM) problem \cite{williamson1992airline}, which is typically formulated as a dynamic programming (DP) problem. However, for large-scale problems, the exponential growth of the state space and decision set makes the DP approach computationally intractable. To address this challenge, we propose using scenario-based programming \cite{feng2013scenario, casey2005scenario, henrion2018problem} to determine the seat planning. In this approach, the aggregated supply can be considered as a protection level for each group type. Notably, in our model, the supply of larger groups can also be utilized by smaller groups. This is because our approach focuses on group arrival rather than individual unit, which sets it apart from traditional partitioned and nested approaches \cite{curry1990optimal, van2008simulation}.


Traditional revenue management focuses on decision-making issues, namely accepting or rejecting a request \cite{gallego1997multiproduct}.However, our paper not only addresses decision-making, but also emphasizes the significance of assignment, particularly in the context of seat assignment. This sets it apart from traditional revenue management methods and makes the problem more challenging.

Similarly, the assign-to-seat approach introduced by Zhu et al. \cite{zhu2023assign} also highlights the importance of seat assignment in revenue management. This approach addresses the challenge of selling high-speed train tickets in China, where each request must be assigned to a single seat for the entire journey and takes into account seat reuse. This further emphasizes the significance of seat assignment and sets it apart from traditional revenue management methods.



% The authors propose a modified network revenue management model and introduce a bid-price control policy based on a novel maximal sequence principle. They also propose a "re-solving a dynamic primal" policy that achieves uniformly bounded revenue loss. The study reveals connections between this problem and traditional network revenue management problems and shows that the impact of the assign-to-seat restriction can be limited with the proposed methods.


% Jiang et al. \cite{jiang2015dynamic} proposes a revenue management approach for high-speed rail (HSR) passenger ticket assignment with dynamic adjustments. The approach integrates short-term demand forecasting, ticket assignment, and dynamic ticket adjustment mechanisms to allocate passenger tickets during presale periods and avoid situations where tickets are insufficient at some stations while seats remain empty.



% Implementing dynamic seat assignment with social distancing can be done manually by the staff or through automated systems that use algorithms to optimize the seat assignments based on various factors, such as ticket sales, seat availability, and customer preferences. However, the implementation of social distancing measures poses unique challenges that require careful planning and consideration of various factors to balance safety with revenue generation.


% \subsection{Scenario generation}

% It is challenging to consider all the possible realizations; thus, it is practicable to use discrete distributions with a finite number of scenarios to approximate the random demands. This procedure is often called scenario generation.

% Some papers consider obtaining a set of scenarios that realistically represents the distributions of the random parameters but is not too large. \cite{feng2013scenario} \cite{casey2005scenario}
% \cite{henrion2018problem}

% Another process to reduce the calculation is called scenario reduction. It tries to approximate the original scenario set with a smaller subset that retains essential features.



% Every time we can regenerate the scenario based on the realized demands. (Use the conditional distribution or the truncated distribution)


% Suppose that the groups arrive from small to large according to their size. Once a larger group comes, the smaller one will never appear again.

% When a new group arrives (suppose we have accepted $n$ groups with the same size), we accept or reject it according to the supply (when $n+1 < \text{supply}$, we accept it). 
% then update the scenario set according to the truncated distribution. We can obtain a new supply with the new probability and scenario set.


\newpage
