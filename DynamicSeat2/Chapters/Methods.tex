% !TEX root = sum1.tex

\section{Dynamic Seat Assignment(DSA) for Each Group Arrival}
In this section, we discuss policies for assigning seats dynamically. We need to make decisions not only on whether to accept or reject an arrival but also on how to assign seats to each group if we decide to accept it. We first present the dynamic seat assignment policy for each group arrival, which incorporates group-type control to optimize resource utilization. We also compare this approach with the bid-price policy. Finally, we establish the FCFS policy as the benchmark.

We can estimate the arrival rate from the historical data, $p_i = \frac{N_{i}}{N_{0}}, i \in \mathcal{M}$, where $N_{0}$ is the number of total groups, $N_{i}$ is the number of group type $i$. Recall that we assume there are $T$ independent periods, with one group arriving in each period. There are $M$ different group types. Let $\mathbf{y}$ be a discrete random variable indicating the number of people in the group, and let $\mathbf{p}$ be a vector probability, where $p(y = i) = p_i$, $i \in \mathcal{M}$ and $\sum_{i} p_{i} =1$.

% Next, we discuss the dynamic seat assignment policy for all group arrivals together.

\subsection{Group-type(Supply) Control}\label{nested_policy}
Feasible seat planning represents the supply for each group type. We can use supply control to determine whether to accept a group. Specifically, if there is a supply available for an arriving group, we will accept the group. However, if there is no corresponding supply for the arriving group, we need to decide whether to use a larger group supply to meet the group's needs. When a group is accepted to occupy larger-size seats, the remaining empty seat(s) can be reserved for future demand.

In the following section, we will demonstrate how to decide whether to accept the current group to occupy larger-size seats when there is no corresponding supply available.


When the number of remaining periods is $T_r$, for any $j>i$, we can use one supply of group type $j$ to accept a group of $i$. In that case, when $i+\delta \leq j$, $(j-i-\delta)$ seats can be provided for one group of $j-i-\delta$ with $\delta$ seats of social distancing. Let $D_j$ be the random variable indicates the number of group type $j$ in $T_r$ periods. The expected number of accepted people is $i + (j-i-\delta)P(D_{j-i-\delta} \geq x_{j-i-\delta}+1; T_r)$, where $P(D_i \geq x_i; T_r)$ is the probability of that the demand of group type $i$ in $T_r$ periods is no less than $x_i$, the remaining supply of group type $i$. Thus, the term, $P(D_{j-i-\delta} \geq x_{j-i-\delta}+1; T_r)$, indicates the probability that the demand of group type $(j-i-\delta)$ in $T_r$ periods is no less than its current remaining supply plus 1. When $i <j < i+\delta$, the expected number of accepted people is $i$.

Similarly, when we retain the supply of group type $j$ by rejecting a group of $i$, the expected number of accepted people is $j P(D_{j} \geq x_{j}; T_r)$. $P(D_{j} \geq x_{j}; T_r)$ indicates the probability that the demand of group type $j$ in $T_r$ periods is no less than its current remaining supply.

Let $d(i,j)$ be the difference of expected number of accepted people between acceptance and rejection on group $i$ occupying $(j+\delta)$-size seats. If $j \geq i+\delta$, $d(i,j)$ equals $i + (j-i-\delta)P(D_{j-i-\delta} \geq x_{j-i-\delta}+1) - j P(D_{j} \geq x_{j})$, otherwise, $d(i,j)$ equals $i - j P(D_{j} \geq x_{j})$.One intuitive decision is to choose the largest difference. For all $j >i$, find the largest $d(i,j)$, denoted as $d(i,j^{*})$. If $d(i,j^{*}) >0$, we will place the group of $i$ in $(j^{*}+\delta)$-size seats. Otherwise, reject the group.


\begin{remark}
  This control is based on the current feasible seat planning. We can combine other methods when a new seat planning can be regenerated.
\end{remark}

\subsection{Stochastic Planning Policy}
Stochastic planning policy involves using the objective value of accepting or rejecting an arrival to make a decision. To determine this objective value, we need to consider the potential outcomes that could result from accepting the current arrival (i.e., the Value of Acceptance), as well as the potential outcomes that could result from rejecting it (i.e., the Value of Rejection).

The Value of Acceptance considers the scenarios that could arise if we accept the current arrival, while the Value of Rejection considers the same scenarios if we reject it. By comparing the Value of Acceptance and the Value of Rejection, we can make an informed decision about whether to accept or reject the arrival based on which option has the higher objective value. This approach takes into account the uncertain nature of the decision-making environment and allows for a more optimal decision to be made.

\subsubsection{Algorithm Based on Stocahstic Planning Policy}
The feasible seat planning can be obtained from Algorithm \ref{feasible_seat} before the group arrivals. In accordance with the group-type control discussed in the previous section, we determine whether to accept or reject group arrivals and which row to assign the group in.

The algorithm is shown below:

\begin{algorithm}[H]
  \caption{Dynamic seat assignment algorithm}\label{algo_dynamic_policy}
  \begin{description}
    \item[Step 1.] Obtain the set of patterns, $\mathbf{P} = \{P_1,\ldots,P_{N}\}$, from the feasible seat planning algorithm. The corresponding aggregate supply is $\X = [x_{1}, \ldots, x_{M}]$.
    \item[Step 2.] For the arrival group type $i$ at period $T{'}$, find the first $k \in \mathcal{N}$ such that $i \in P_k$. Accept the group, update $P_{k} = P_{k}/(i)$ and $x_{i} = x_{i} -1$. Go to step 4.
    \item[Step 3.] If $i \notin P_k, \forall k \in \mathcal{N}$, find $d(i,j^{*})$. If $d(i,j^{*})>0$, find the first $k \in \mathcal{N}$ such that $j^{*} \in P_k$. If value of acceptance is larger than value of rejection, accept group type $i$ and update $P_{k} = P_{k}/(j^{*})$, $x_{j^{*}} = x_{j^{*}} -1$. Then update $x_{j-i-\delta} = x_{j-i-\delta} + 1$ and $P_{k}= P_{k} \cup (j^{*}-i-\delta)$ when $j^{*}-i-\delta > 0$. If $d(i,j^{*}) \leq 0$, reject group type $i$.
    \item[Step 4.] If $T{'} \leq T$, move to next period, set $T{'} = T{'}+1$, go to step 2. Otherwise, terminate this algorithm.
  \end{description}
\end{algorithm}

\subsection{Bid-price}
The dual problem of linear relaxation of problem \eqref{deter_upper} is:

\begin{equation}\label{bid-price_dual}
  \begin{aligned}
  \min \quad & \sum_{i=1}^{M} d_i z_i + \sum_{j= 1}^{N} L_j \beta_{j} \\
  \text {s.t.} \quad & z_{i} + \beta_j n_i \geq (n_i-\delta), \quad i \in \mathcal{M}, j \in \mathcal{N} \\
  & z_{i} \geq 0, i \in \mathcal{M}, \beta_{j} \geq 0, j \in \mathcal{N}.
  \end{aligned}
\end{equation}

% $\beta_{j}$ can be interpreted as the 
% $z_{i}$

The bid-price is a term used to describe the marginal value or opportunity cost of a seat. When a group type $i$ arrives, we can calculate $i -\beta_{j} n_i$ for all $j$ and choose $\arg \max_{j} \{i -\beta_{j} n_i\}$ as the row to allocate that group.

\begin{lem}
According to Lemma \ref{sol_relax_deter}, there exists $h$ such that the aggregate optimal solution to relaxation of problem \eqref{deter_upper} takes the form $x e_{h} + \sum_{i=h+1} ^{M} d_{i} e_{i}$, $x = (L- \sum_{i = h+1}^{M} {d_i n_i})/ n_h$. Then the optimal solution to probelm \eqref{bid-price_dual} is given by $z_1 ,\ldots, z_h =0$, $z_{i} = \frac{\delta(n_i-n_h)}{n_h}$ for $i = h+1, \ldots, M$ and $\beta_j = \frac{n_h - \delta}{n_h}$ for all $j$. The bid-price policy will accept group type $i$, $i \geq h$, when the capacity allows.
\end{lem}

However, we can find that $\beta_{j}$ does not vary with $j$, which means bid-price cannot decide which row to assign the group to.

\begin{pf}
According to the Lemma \ref{sol_relax_deter}, the aggregate optimal solution to relaxation of problem \eqref{deter_upper} takes the form $x e_{h} + \sum_{i=h+1} ^{M} d_{i} e_{i}$, then according to the complementary slackness property, we know that $z_1, \ldots, z_h = 0$. This implies that $\beta_j \geq \frac{n_i - \delta}{n_i}$ for $i = 1,\ldots, h$. Since $\frac{n_i - \delta}{n_i}$ increases with $i$, we have $\beta_j \geq \frac{n_h - \delta}{n_h}$. Consequently, we obtain $z_{i} \geq n_i - \delta - n_i \frac{n_h - \delta}{n_h} = \frac{\delta(n_i-n_h)}{n_h}$ for $i = h+1, \ldots, M$.

Given that $d$ and $L$ are both no less than zero, the minimum value will be attained when $\beta_j = \frac{n_h - \delta}{n_h}$ for all $j$, and $z_i = \frac{\delta(n_i-n_h)}{n_h}$ for $i = h+1, \ldots, M$. Then the bid-price decision $i -\beta_{j} n_i = i - \frac{n_h -\delta}{n_h} n_i = \frac{\delta (i-h)}{n_h}$. When $i < h$, $i -\beta_{j} n_i <0$. When $i \geq h$, $i -\beta_{j} n_i \geq 0$. \qed
\end{pf}

The bid-price control policy based on the static model is stated below.

\begin{algorithm}[H]
  \caption{Bid-price algorithm}
  \begin{description}
    \item[Step 1.] Observe the arrival group type $i$ at period $t = 1, \ldots, T$.
    \item[Step 2.] Solve \eqref{bid-price_dual} with $d_i^{t} = (T-t) \cdot p_i$ and $\mathbf{L}^{t}$, obtain an optimal solution $\beta_{j}^{t}$.
    \item[Step 3.] Set $k = \arg \max_{j} \{i -\beta_{j}^{t} n_i | n_i \mathbf{e}_{k}^{\top} \leq \mathbf{L}\}$.
    \item[Step 4.] If $i -\beta_{k}^{t} n_i \geq 0$, then assign the group to row $k$, update $\mathbf{L}^{t+1} = \mathbf{L}^{t} - n_i \mathbf{e}_{k}^{\top}$; otherwise, reject the group, let $\mathbf{L}^{t+1} = \mathbf{L}^{t}$. 
    \item[Step 5.] If $t \leq T$, move to next period, set $t = t+1$, go to step 2. Otherwise, terminate this algorithm.
  \end{description}
\end{algorithm}


\subsection{Dynamic Programming Base-heuristic}
Since the original dynamic programming problem is too complex to solve directly, we can instead consider a simplified version of the problem, known as the relaxation problem. By solving the relaxation problem, we can make decisions for each group arrival based on the dynamic programming approach.

Relax all rows to one row with the same capacity by $L = \sum_{j=1}^{N} L_j$. The deterministic problem is

\begin{equation}\label{relax_deter}
  \begin{aligned}
  \max \quad & \sum_{i=1}^{M} (n_i- \delta) x_{i} \\
  \text {s.t.} \quad & x_{i} \leq d_{i}, \quad i \in \mathcal{M}, \\
  & \sum_{i=1}^{M} n_{i} x_{i} \leq L \\
  & x_{i} \in \mathbb{Z}_{+}, \quad i \in \mathcal{M}.
  \end{aligned}
\end{equation}

\begin{lem}
 Let $\mathbf{X}$ be the solution of linear relaxation of problem \eqref{relax_deter}. $\mathbf{X}$ is the same as the aggregate
 solution of linear relaxation of problem \eqref{deter_upper}.
\end{lem}

% Let $v_{r}^{*}$ and $v^{*}$ denote the optimal values of problem \eqref{relax_deter} and \eqref{deter_upper}, respectively. After relaxation, assume that each row has a capacity of $L$ seats, which can be filled. For each seperate row, the maximum number of empty seats in each row is $s$. Then, the total number of empty seats in $N$ rows is given by $Ns$. Therefore, the biggest difference between $v_{r}^{*}$ and $v^{*}$ is the number of people accommodated in the $Ns$ empty seats.
% The difference between them is zero only when the groups corresponding to the optimal solution of problem \eqref{relax_deter} can be accommodated in $N$ rows. The numerical results indicate that this is the case for most scenarios, which suggests that a DP-based approach can be used to solve the dynamic seat assignment problem after all groups have arrived.

Let $u$ denote the decision, where $u(t) = 1$ if we accept a request in period $t$, $u(t) =0$ otherwise. Similar to the DP in section \ref{sec_dynamic}, the DP with one row can be expressed as:

$$V_{t}(L) = \mathbb{E}_{i \sim p} [\max_{u \in \{0,1\}} \{ {[V_{t+1}(L-n_i u)+ i u]}\}], L \geq 0, V_{T+1}(L) =0, \forall L$$

After accepting one group, assign it in some row arbitrarily when the capacity of the row allows.

% \subsection{Break tie for Bid-price and DP base-heuristic}
% To determine which row to place accepted groups in when there are multiple options, follow these steps:

% 1. Check if the remaining capacity of the current row is greater than the maximum group size or equal to the current group size. If it is, accept the current arrival and place the group in that row.
% 2. Otherwise, consider the next row.
% 3. Repeat steps 3 and 4 until a row is found that can accommodate the current group size.


\subsection{Break Tie for Stochastic Planning}
A tie occurs when a small group is accepted by a larger planned group. To accept the smaller group, check if the current row contains at least two planned groups, including one larger group. If so, accept the smaller group in that row. If not, move on to the next row and repeat the check. If no available row is found after checking all rows, place the smaller group in the first row that contains the larger group.

By following this approach, the number of unused seats in each row can be reduced, leading to better capacity utilization.



% \subsection{Seat Planning Charts Online}
% We are able to provide an online seat planning solution by using our method. For a feasible seating arrangement, we provide a pattern for each row. The sequence of groups within each pattern can be arranged arbitrarily, allowing for a flexible seat planning that can accommodate realistic operational constraints. Therefore, any fixed sequence of groups within each pattern can be used to construct a seating plan that meets practical needs.

% We need to assign seats to the group for each arrival. In each period, the group can select the row they want to sit when the capacity is enough. FCFS will be more appropriate. But M1 and M3 can also be used.


% update the scenario and the probability, add constraints when re-calculating stochastic programming.
% we can update the supply whenever some demand exceeds the supply.


% Then use Algorithm \ref{algo_nested_policy} to make the decision.


% \subsubsection{Ticket Reservation with Row Selection}
% There are two methods to achieve this goal.

% First, we can generate patterns planning according to stochastic information. Calculate the maximal supply from the mean demand. The supply gives the number of patterns with different losses; then we prepare the corresponding pattern planning for each row. Every group will be assigned in each period according to the designated row as long as the capacity allows. 

% The number of combinations is enormous.

% The second one is named the seat row selection method based on the stochastic seat assignment method. The initial supply is obtained from the stochastic model, then update the supply from the deterministic model after the first period. After accepting one group, we update the accepted demand and remaining seats. The policy follows section \ref{nested_policy}.


% Set the mean demand as the initial supply, update the supply from deterministic model by setting accpected demand as the lower bound. / can be used in scenario 1,2.


% Partially dynamic: at the beginning stage, the capacity is sufficient, thus we will accept all arrivals. 
% Multiple planning approach 


\subsection{Benchmark}
 
% The benchmarks for the above two situations are described below.

\subsubsection{FCFS}\label{largest_pattern}
For dynamic seat assignment for each group arrival, the intuitive but trivial method will be on a first-come-first-served basis. Each accepted request will be assigned seats row by row. If the capacity of a row is insufficient to accommodate a request, we will allocate it to the next available row. If a subsequent request can fit exactly into the remaining capacity of a partially filled row, we will assign it to that row immediately. Then continue to process requests in this manner until the capacity of all rows is fully utilized.


% we can continue to use the first-come, first-served approach for seat assignment across all rows. 

\subsubsection{FCFS-based}\label{FCFS-based}
For dynamic seat assignment after all group arrivals, we can continue to use the first-come, first-served approach for seat assignment. Relax all rows to one row with the total number of seats. For each arrival, we need to check the feasibility of constructing a seat assignment in $N$ rows. If the seat assignment is feasible, we accept the request; otherwise, we reject it. The threshold capacity is $(L -u +1)$.

% find the target arrival when the number of seats taken by the preceding arrivals does not exceed the capacity.Then we obtain a new sub-sequence, including the arrivals from the first to the target and a possible arrival. 

% And use the nested policy to accept or reject one group in the remaining arrivals. 


% For the convenience of calculation, we check the feasibility of constructing a seat assignment from the end of the sub-sequence. When it is not feasible for the seat assignment, we should delete the group one by one from this sub-sequence until a feasible seat assignment is found. In reality, we need to check the feasibility one group by one.



% Each request will be assigned row by row. When the capacity of one row is not enough for the request, we arrange it in the next row. If the following request can take up the remaining capacity of some row exactly, we place it in that row immediately. We check each request until the capacity is used up. 



% There is a reservation stage, we only decide to accept or reject.
% After certain periods, there is a seat selection stage.

% Or use partial static information to estimate the probabilities, then generate new plannings.

% Multiple scenario approach:
% Suppose that we know the probabilities, we can use the sampling demands to estimate which patterns we should use. For example, three rows with $I_1$, seven rows with $I_2$. For each arrival, if there exists one scenario containing this group, we accept it; otherwise we use nested policy to accept or reject it.

% \subsubsection{Largest Patterns Planning}\label{largest_pattern}
% For each row, we choose the patterns from $I_1$. Accept the group such that the largest pattern can be maintained. When the arrival cannot be assigned in the planning patterns from $I_1$, we can change the largest pattern to a second largest pattern according to the coming arrival.

% \begin{algorithm}[H]\label{algo_largest}
%   \caption{Method by using the largest patterns}
%   \begin{description}
%     \item[Step 1.] Generate the largest pattern by the greedy way for each row.
%     \item[Step 2.] Denote the minimal and maximal size of group in the pattern of row $i$ as $\min_i$ and $\max_i$. 
%     \item[Step 3.] For the arrival with the size of $a$ in period $t$, if there exists $i$ such that $\min_i + \max_i >= a$ and $a > \min_i$, accept this arrival at row $i$, go to step 5; otherwise, go to step 4.
%     \item[Step 4.] Find a maximal group of seats to accept this arrival, otherwise, reject this arrival.
%     % Use the information of realized arrivals to make the decision.
%     \item[Step 5.] Move to the next period. Repeat step 3. 
%   \end{description}
% \end{algorithm}

% Step 2: the pattern will have the same loss by these procedures. ($\min_i$ can be 0.)

% \begin{lem}
%   Any largest patterns can be generated by the largest pattern constructed from the greedy method.
% \end{lem}

% This method can be used without stochastic information. The performance will improve when the total demand can construct the largest patterns for all rows.


% Once: Obtain the supply from the stochastic model by benders decomposition. Use the deterministic model to obtain a heuristi supply. Then use the multi-class rule to decide whether to accept the group at each period.

% Several: Initially, set the mean demand for all periods as the upper bound of demand. Then obtain the supply from the deterministic model. Set the accepted demand as the lower bound of demand, the upper bound of demand will be the sum of accepted demand and mean demand for the remaining periods. Update the lower bound and upper bound when some supply runs out.

% One counterexample: [15,21,13,3] /[15,21,10,3]  reject 4

% Different types of movies will have different probabilities, consider the preference for policy when demand = supply.


% The numbers in the `performance compared to the optimal' represent M1, M2, M3, M4, M5, M6 respectively in order.

% The maximal number of people served can be obtained by \eqref{deter_upper} with a realized sequence of arrival.

% Some important information:
% The government will give the restriction: 入座人数50% /相连座位 4个
