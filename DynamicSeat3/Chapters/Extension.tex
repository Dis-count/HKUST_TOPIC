% !TEX root = sum1.tex

\section{Other Results}

% comparing the running time of decomposition and integer programming, comparing the number of people served using the seat planning and integer programming methods,

\subsection{Running times of solving SSP and relaxation of SSP with Benders Decomposition}\label{Bender_IP}
% At first, we compare the running time of these two methods. 

The running times of solving SSP directly and solving the LP relaxation of SSP with Benders decomposition are shown in Table \ref{tab_1}.

\begin{table}[ht]
  \centering
  \scriptsize
  \caption{Running times of solving SSP and relaxation of SSP with Benders Decomposition}\label{tab_1}
  \begin{tabular}{|l|l|l|l|l|l|l|}
  \hline
  \# of scenarios & Demands & \# of rows & \# of groups & \# of seats & Running time of IP(s) & Benders (s) \\
  \hline
  1000  & (150, 350) & 30 & 8 & (21, 50) & 5.1  & 0.13 \\
  5000  &            & 30 & 8 &         & 28.73 & 0.47  \\
  10000 &            & 30 & 8 &         & 66.81  & 0.91 \\
  50000 &            & 30 & 8 &         & 925.17 & 4.3 \\
  \hline
  1000  & (1000, 2000) & 200 & 8 & (21, 50) & 5.88 & 0.29 \\
  5000  &              & 200 & 8 &          & 30.0 & 0.62 \\
  10000 &              & 200 & 8 &          & 64.41 & 1.09 \\
  50000 &              & 200 & 8 &          & 365.57 & 4.56\\
  \hline
  1000  & (150, 250) & 30 & 16 & (41, 60) & 17.15  & 0.18 \\
  5000  &            & 30 & 16 &          & 105.2  & 0.67 \\
  10000 &            & 30 & 16 &          & 260.88 & 1.28 \\
  50000 &            & 30 & 16 &          & 3873.16 & 6.18 \\
  \hline
  \end{tabular}
\end{table}

The parameters in the columns of the table are the number of scenarios, the range of demands, running time of SSP, running time of Benders decomposition method, the number of rows, the number of group types and the number of seats for each row, respectively. Take the first experiment as an example, the scenarios of demands are generated from (150, 350) randomly, the number of seats for each row is generated from (21, 50) randomly.

It is evident that the utilization of Benders decomposition can enhance computational efficiency.

% about 1000 seats.

% The second one:
% The number of seats for each row L is generated from (21, 50) randomly, about 7000 seats.
% The scenarios of demands are generated from (1000, 2000) randomly.

% The third one:
% The number of seats for each row L is generated from 41-60 randomly, about 1500 seats.
% The scenarios of demands are generated from (150, 250) randomly.

\subsection{Seat Planning From SSP Relaxation versus Seat Planning From SSP}
In this section, we compare the accpected people of seat plannings from SSP relaxation and SSP under group-type control policy.

An arrival sequence can be expressed as $\{y_{1}, y_{2}, \ldots, y_{T}\}$. Let $N_{i} = \sum_{t} I(y_t = i)$, i.e., the number of times group type $i$ arrives during $T$ periods. Then the scenarios, $(N_1, \ldots, N_{M})$, follow a multinomial distribution, $$p\left(N_1, \ldots, N_{M} \mid \mathbf{p}\right)=\frac{T !}{N_{1}!, \ldots, N_{M}!} \prod_{i=1}^{M} p_{i}^{N_i}, T = \sum_{i=1}^{M} N_{i}.$$

It is clear that the number of different sequences is $M^{T}$. The number of different scenarios is $O(T^{M-1})$ which can be obtained by the following DP.

Use $D(T, M) $ to denote the number of scenarios, which equals the number of different solutions to $x_{1}+\ldots + x_{M} = T, \mathbf{x} \geq 0$. Then, we know the recurrence relation $D(T, M) = \sum_{i= 0}^{T} D(i, M-1)$ and boundary condition, $D(i,1) = 1$. So we have $D(T,2) = T+1$, $D(T,3) = \frac{(T+2)(T+1)}{2}, D(T,M) = O(T^{M-1})$. The number of scenarios is too large to enumerate all possible cases. Thus, we choose to sample some sequences from the multinomial distribution.

Then, we will show these two seat plannings have a close performance when considering group-type control policy.

\begin{table}[ht]
    \centering
    \small
    \caption{Seat planning from SSP relaxation versus seat planning from SSP}
    \begin{tabular}{|l|l|l|l|l|l|l|}
    \hline
    \# samples & T & probabilities & \# rows & people served by SSP relaxation & people served by SSP \\
    \hline
    1000  & 45  & [0.4,0.4,0.1,0.1] & 8 & 85.30 & 85.3 \\
    1000  & 50  & [0.4,0.4,0.1,0.1] & 8 & 97.32 & 97.32 \\
    1000  & 55  & [0.4,0.4,0.1,0.1] & 8 & 102.40 & 102.40  \\ % slow
    1000  & 60  & [0.4,0.4,0.1,0.1] & 8 & 106.70 & NA  \\
    1000  & 65  & [0.4,0.4,0.1,0.1] & 8 & 108.84 & 108.84 \\
    \hline
    1000  & 35  & [0.25,0.25,0.25,0.25] & 8 & 87.16 & 87.08 \\
    1000  & 40  & [0.25,0.25,0.25,0.25] & 8 & 101.32 & 101.24 \\
    1000  & 45  & [0.25,0.25,0.25,0.25] & 8 & 110.62 & 110.52 \\
    1000  & 50  & [0.25,0.25,0.25,0.25] & 8 & 115.46 & NA \\
    1000  & 55  & [0.25,0.25,0.25,0.25] & 8 & 117.06 & 117.26 \\
    \hline
    5000  & 300  & [0.25,0.25,0.25,0.25] & 30 & 749.76 & 749.76 \\
    5000  & 350  & [0.25,0.25,0.25,0.25] & 30 & 866.02 & 866.42 \\
    5000  & 400  & [0.25,0.25,0.25,0.25] & 30 & 889.02 & 889.44 \\
    5000  & 450  & [0.25,0.25,0.25,0.25] & 30 & 916.16 & 916.66 \\
    \hline
    \end{tabular}
\end{table}

Each entry of people served is the average of 50 instances.
IP will spend more than 2 hours in some instances, as `NA' showed in the table.
The number of seats is 20 when the number of rows is 8, the number of seats is 40 when the number of rows is 30.

% We can find that the people served by Benders decomposition and IP under nested policy are close. But obtaining the near-optimal seat assignment will be faster.

% Thus, we can use the near-optimal seat assignment from the decomposition approach.

% --------------------------------------------------

% \subsection{Seat Planning Charts Online}
% We are able to provide an online seat planning solution using our method. For a feasible seating arrangement, we provide a pattern for each row. The sequence of groups within each pattern can be arranged arbitrarily, allowing for a flexible seat planning that can accommodate realistic operational constraints. Therefore, any fixed sequence of groups within each pattern can be used to construct a seating plan that meets practical needs.

% \begin{algorithm}[H]
%   \caption{Seat Planning}\label{seat_online}
%   \begin{description}
%     \item[Step 1.] Construct a seat planning from the feasible seat planning algorithm, the aggregated supply is $\X^{0} = [x_1,\ldots,x_{m}]$.
%     \item[Step 2.] For the arrival group type $i$ at period $T{'}$, if $x_{i} > 0$, accept it. Let $x_{i} = x_{i} -1$. Go to step 4.
%     \item[Step 3.] If $x_{i} = 0$, find $d(i,j^{*})$. If $d(i,j^{*})>0$, accpect group type $i$. Set $x_{j^{*}} = x_{j^{*}} -1$. Let $x_{j-i-1} = x_{j-i-1} + 1$ when $j-i-1>0$. If $d(i,j^{*}) \leq 0$, reject group type $i$.
%     \item[Step 4.] If $T{'} \leq T$, move to next period, set $T{'} = T{'}+1$, go to step 2. Otherwise, terminate this algorithm.
%   \end{description}
% \end{algorithm}


% \subsection{Online booking} 

% The groups can select the seats arbitrarily but with some constraints.

% When the customers book the tickets, they will be asked how many seats they are going to book at first. Then we give the possible row numbers for their selection. Finally we give the seat number for their choice.

% The second step is based on the other choices of reserved groups, we need to check which rows in the seat assignment include the corresponding-size seats.

% In third step, we need to check whether the chosen number destroys the pattern of the row. Use subset sum problem to check every position in the row.


% \subsection{How to give a balanced seat assignment}

% Notice we only give the solution of how to assign seats for each row, but the order is not fixed.

% In order to obtain a balanced seat assignment, we use a greedy way to place the seats.

% Sort each row by the number of people. Then place the smallest one in row 1, place the largest one in row 2, the second smallest one in row 3 and so on. 

% For each row, sort the groups in an ascending/descending order. In a similar way.


% \subsection{Property}
% Although the solver can solve this problem easily, the analyses on the property of the solution to this problem can help us generate a method for the dynamic situation. 

% At first, we consider the types of pattern, which refers to the seat assignment for each row. For each pattern $k$, we use $\alpha_k, \beta_k$ to indicate the number of groups and the left seat, respectively. Denote by $l(k) = \alpha_k + \beta_k -1$ the loss for pattern $k$. The loss represents the number of people lost compared to the situation without social distancing.

% Let $I_1$ be the set of patterns with the minimal loss. Then we call the patterns from $I_1$ are largest. Similarly, the patterns from $I_2$ are the second largest, so forth and so on. The patterns with zero left seat are called full patterns. We use the descending form $(t_1, t_2, \ldots, t_k)$ to represent a pattern, where $t_i$ is the new group size. 

% \begin{example}\label{ex_largest}
%   The length of one row is $S = 21$ and the new size of groups be $L = [2, 3, 4, 5]$. Then these patterns, $(5, 5, 5, 5, 1),(5, 4, 4, 4, 4),(5, 5, 5, 3, 3)$, belong to $I_1$. The demand is $[10, 12, 9, 8]_d$.
% \end{example}

% To represent a pattern with a fixed length of form, we can use a $(m+1)-$dimensional vector with $m$ group types. The aggregated form can be expressed as $[n_0, n_1, \ldots, n_m]$, where $n_i$ is the number of $i$-th group type, $i=1,\ldots,m$. 
% $n_0$ is the number of left seat, its value can only be $0, 1$ because two or more left seats will be assigned to groups. Thus the pattern, $[1, 0, 0, 0, 4]$, is not full because there is one left seat.

% Suppose $u$ is the size of the largest group allowed, all possible seats can be assigned are the consecutive integers from 2 to $u$, i.e., $[2,3,\ldots,u]$.
% Then we can use the following greedy way to generate the largest pattern. Select the maximal group size,$u$, as many as possible and the left space is assigned to the group with the corresponding size. Let $S = u\cdot q + r$, where $q$ is the number of times $u$ selected. When $r>0$, there are $d[0][u-r][q+1]$ largest patterns with the same loss of $q$. When $r =0$, there is only one possible largest pattern.

% Use dynamic programming to solve. $d[k][i][j]$ indicates the number of assignment of using $i$ capacity to allocate $j$ units, $k$ is the number of capacity allocated on the last unit. In our case, $u-r$ is the capacity need to be allocated, $q+1$ is the number of units which corresponds to the groups. Notice that we only consider the number of combinations, so we fix the allocation in ascending order, which means the allocation in current unit should be no less than the last unit.  

% The number of largest patterns equals the number of different schemes that allocate $u-r$ on $q+1$ units,i.e., $d[0][u-r][q+1]$.

% The recurrence relation is $d[k][i][j] = \sum_{t=k}^{i-k} d[t][i-k][j-1]$. 
% When $i < k$, $d[k][i][j] =0$; when $i \geq k$, $d[k][i][1] =1$.

% \begin{lem}
% % If all patterns associated with an integral feasible solution belong to $I_1$, then this solution is optimal.
% The seat assignment made up of the largest patterns is optimal.
% \end{lem}

% This lemma holds because we cannot find a better solution occupying more seats.

% When the demand is so large that the largest patterns can be generated in all rows, an optimal seat assignment can be obtained.

% \begin{prop}\label{prop_I_1}
%   Let $k^{*} = \arg \max_{k\in I_1} \min_{i} \{\lfloor \frac{d_i}{b_i^k}\rfloor\}$. 
%   When $N \leq \max_{k\in I_1} \min_{i} \{\lfloor \frac{d_i}{b_i^k}\rfloor\}$, the optimal seat assignment can be constructed by repeating pattern $k^*$ $N$ times.
%   $N$ is the number of rows, $d_i$ is the number of $i$-th group type, $i = 1,2,\ldots, m$, $b_i^k$ is the number of group type $i$ placed in pattern $k$.
% \end{prop}

% Use the above example \ref{ex_largest} to explain. Take $(5,5,5,5), (5,4,4,4,4), (5,5,4,4,3)$ as the alternative patterns, denoted by pattern $1, 2, 3$ respectively. When $k = 1,2,3$, $\min_{i} \{\lfloor \frac{d_i}{b_i^k}\rfloor\}$ will be $2,3,5$, thus $k^{*}= 3$. So when $N \leq 5$, we can select the pattern $(5,5,4,4,3)$ five times to construct the optimal seat assignment.

% \begin{prop}\label{prop_I_2}
%   We can construct a seat assignment in the following way. Every time we can select one pattern from $I_1$, then minus the corresponding number of group types from demand and update demand. Repeat this procedure until we cannot generate a largest pattern. If the number of generated patterns is no less than the number of rows, this assignment is optimal.
% \end{prop}

% % This lead

% \newpage
