% !TEX root = sum1.tex
\section{Seat Assignment with Dynamic Demand}\label{sec_dynamic}
In this section, we discuss how to assign the arriving groups in the dynamic situation. We consider different situations, seat assignment under fixed seat planning and flexible seat planning. The premises for dynamic demand is that we adopt the discrete time, there is at most one group arrival at each period, $t = 1, \ldots, T$. The probability of an arrival of group type $i$ is $p_i$.


\subsection{Seat Assignment under Fixed Seat Planning}
This situation involves a fixed seat planning that is set based on the management's requirements. When a group arrives, they can choose seats from the available planning options. The predetermined seat arrangements ensure that social distancing measures are maintained, and groups can select seats that best suit their needs while adhering to the established seating arrangement.

The seats, which were arranged for social distancing purposes, need to be dismantled
before people arrive to prevent them from occupying those seats. When each group
arrives, we make decisions regarding whether to accept or reject them based on the
predetermined seat planning. We use the group-type control. Obtain the seat planning from stochastic programming. Suppose the corresponding supply is $[X_1, \ldots, X_M]$. For the arrival of group type i, if $X_i > 0$, accept it directly, assign it the seats planned for group type i; if $X_i = 0$, determine which group type to accept it.

\subsubsection{Determine The Group Type}\label{nested_policy}
% One intuitive approach is to utilize stochastic programming to make decisions by comparing the values obtained when accepting or rejecting the currently arriving group. Stochastic programming aids in generating seat planning, and when there are available seats planned for the group, we readily accept and allocate it to the corresponding position according to the seat planning. When there are no suitable supply available for the current group, we need to perform calculations of stochastic programming by considering the placement of the group in each possible row, then compare these values to make a decision. However, it is important to note that integer stochastic programming can be computationally expensive and will be unsolvable in some instances. Additionally, when there is no supply in the seat planning available for the current group, evaluating the values associated with placing the group in each possible row can require a significant amount of computation.

% Therefore, in order to mitigate the computational challenges, we utilize the LP relaxation of stochastic programming as an approximation to compare the values when deciding whether to accept or reject a group. However, one challenge arises from the fact that the LP relaxation results in the same objective values for the acceptance group in any possible row. This poses the question of determining which row to place the group in when we accept it. To address this challenge, we developed the group-type control policy which narrows down the row options based on the seat planning.

The group-type control aims to find the group type to assign the arriving group, that helps us narrow down the option of rows for seat assignment. Seat planning serves as a representation of the supply available for each group type. Based on the supply, we can determine whether to accept an incoming group. When a group arrives, if there is sufficient supply available for an arriving group, we will accept the group and choose the group type accordingly. However, if there is no corresponding supply available for the arriving group, we need to decide whether to use a larger group's supply to meet the need of the arriving group. When a group is accepted and assigned to larger-size seats, the remaining empty seat(s) can be reserved for future demand without affecting the rest of the seat planning. To determine whether to use larger seats to accommodate the incoming group, we compare the expected values of accepting the group in the larger seats and rejecting the group based on the current seat planning. Then we identify the possible rows where the incoming group can be assigned based on the group types and seat availability.


Specifically, suppose the supply is $(x_1, \ldots, x_M)$ at period $t$, the number of remaining periods is $(T-t)$. For the coming group type $i$, if $x_i > 0$, then accept it, let $x_i = x_i -1$.
If $x_i = 0$, in the following part, we will demonstrate how to decide whether to accept the group to occupy larger-size seats when there is no corresponding supply available. For any $j=i+1, \ldots, M$, we can use one supply of group type $j$ to accept a group type $i$. In that case, when $j = i+1, \ldots, i+\delta$, the expected number of accepted people is $i$ and the remaining seats beyond the accepted group, which is $j-i$, will be wasted. When $j = i+\delta+1, \ldots, M$, the rest $(j-i-\delta)$ seats can be provided for one group type $j-i-\delta$ with $\delta$ seats of social distancing. Let $D_j^{t}$ be the random variable indicates the number of group type $j$ in $t$ periods. The expected number of accepted people is $i + (j-i-\delta)P(D_{j-i-\delta}^{T-t} \geq x_{j-i-\delta}+1)$, where $P(D_i^{T-t} \geq x_i)$ is the probability that the demand of group type $i$ in $(T-t)$ periods is no less than $x_i$, the remaining supply of group type $i$. Thus, the term, $P(D_{j-i-\delta}^{T-t} \geq x_{j-i-\delta}+1)$, indicates the probability that the demand of group type $(j-i-\delta)$ in $(T-t)$ periods is no less than its current remaining supply plus 1. 

Similarly, when we retain the supply of group type $j$ by rejecting a group of $i$, the expected number of accepted people is $j P(D_{j}^{T-t} \geq x_{j})$. The term, $P(D_{j}^{T-t} \geq x_{j})$, indicates the probability that the demand of group type $j$ in $(T-t)$ periods is no less than its current remaining supply.

Let $d^{t}(i,j)$ be the difference of expected number of accepted people between acceptance and rejection on group $i$ occupying $(j+\delta)$-size seats at period $t$. Then we have
\begin{equation*}
	d^{t}(i,j) = \begin{cases}
    i + (j-i-\delta)P(D_{j-i-\delta}^{T-t} \geq x_{j-i-\delta}+1) - j P(D_{j}^{T-t} \geq x_{j}), &\text{if}~ j = i+\delta+1, \ldots, M \\
    i - j P(D_{j}^{T-t} \geq x_{j}), &\text{if}~ j = i+1, \ldots, i+\delta.
		\end{cases}
\end{equation*}
% If $j \geq i+\delta$, $d(i,j)$ equals $i + (j-i-\delta)P(D_{j-i-\delta} \geq x_{j-i-\delta}+1; T_r) - j P(D_{j} \geq x_{j}; T_r)$, otherwise, $d(i,j)$ equals $i - j P(D_{j} \geq x_{j})$. 

One intuitive decision is to choose $j$ with the largest difference. For all $j = i+1, \ldots, M$, find the largest $d^{t}(i,j)$, denoted as $d^{t}(i,j^{*})$. If $d^{t}(i,j^{*}) >0$, we will plan to assign the group type $i$ in $(j^{*}+\delta)$-size seats. Otherwise, reject the group.

Group-type control policy can only tell us which group type's seats are planned to provide for the smaller group based on the current planning, we still need to further compare the values of the stochastic programming problem when accepting or rejecting a group on the specific row. 


\subsubsection{Performances}
The assignment is based on the fixed seat planning and we use the group-type control to make the
decision.

\begin{table}[ht]
  \centering
  \begin{tabular}{|l|l|l|l|l|}
  \hline
   T & Probabilities & \# of rows & SSP (\%) & Expected demand (\%) \\
  \hline
  70  & [0.25, 0.25, 0.25, 0.25]  & 10 & 94.97 & 94.71  \\
  80  &   &  & 96.48 & 96.16  \\
  90  &   &  & 97.94 & 97.36  \\
  100  &   &  & 98.91 & 96.27  \\
  \hline
  70  & [0.25, 0.35, 0.05, 0.35]  & 10 & 95.90 & 95.60 \\
  80  &   &  & 97.06 & 96.69 \\
  90  &   &  & 98.58 & 98.58 \\
  100  &   &  & 99.47 & 95.97 \\
  \hline
  70  & [0.15, 0.25, 0.55, 0.05]  & 10 & 97.41 & 96.70 \\
  80  &   &  & 98.85 & 96.06 \\
  90  &   &  & 98.73 & 97.63 \\
  100  &   &  & 98.46 & 98.19 \\
  \hline
  140  & [0.25, 0.25, 0.25, 0.25]  & 20 & 95.83 & 95.78 \\
  160  &   &  & 97.46 & 96.89 \\
  180  &   &  & 99.05 & 96.42 \\
  200  &   &  & 99.74 & 97.57 \\
  \hline
  \end{tabular}
\end{table}

\subsection{Seat Assignment under Flexible Seat Planning}\label{sec_dynamic_seat}
This situation involves a flexible seat planning approach, where decisions need to be made when a group requests seats. In this case, we dynamically determine the optimal seat planning based on the group's size and the current availability of seats, taking into account social distancing requirements. In a more realistic scenario, groups arrive sequentially over time, and the seller must promptly make group assignments upon each arrival while maintaining the required spacing between groups. When a group is accepted, the seller must also determine which seats should be assigned to that group. It is essential to note that each group must be either accepted in its entirety or rejected entirely; partial acceptance is not permitted. Once the seats are confirmed and assigned to a group, they cannot be changed or reassigned to other groups.


To model this problem, we adopt a discrete-time framework. Time is divided into $T$ periods, indexed forward from $1$ to $T$. We assume that in each period, at most one group arrives and the probability of an arrival for a group of size $i$ is denoted as $p_i$, where $i$ belongs to the set $\mathcal{M}$. The probabilities satisfy the constraint $\sum_{i=1}^M p_i \leq 1$, indicating that the total probability of any group arriving in a single period does not exceed one. We introduce the probability $p_0 = 1 - \sum_{i=1}^{M} p_i$ to represent the probability of no arrival in a given period $t$. To simplify the analysis, we assume that the arrivals of different group types are independent and the arrival probabilities remain constant over time. This assumption can be extended to consider dependent arrival probabilities over time if necessary.

The state of remaining capacity in each row is represented by a vector $\mathbf{L} = (l_1, l_2, \ldots, l_N)$, where $l_j$ denotes the number of remaining seats in row $j$. Upon the arrival of a group type $i$ in period $t$, the seller needs to make a decision denoted by $u_{i,j}^{t}$, where $u_{i,j}^{t} = 1$ indicates acceptance of group type $i$ in row $j$ during period $t$, while $u_{i,j}^{t} = 0$ signifies rejection of that group type in row $j$ at that period. The feasible decision set is defined as $$U^{t}(\mathbf{L}) = \{u_{i,j}^{t} \in \{0,1\}, \forall i \in \mathcal{M}, \forall j \in \mathcal{N} | \sum_{j=1}^{N} u_{i,j}^{t} \leq 1, \forall i \in \mathcal{M}; n_{i}u_{i,j}^{t}\mathbf{e}_j \leq \mathbf{L}, \forall i \in \mathcal{M}, \forall j \in \mathcal{N}\}.$$ Here, $\mathbf{e}_j$ represents an N-dimensional unit column vector with the $j$-th element being 1, i.e., $\mathbf{e}_j = (\underbrace{0, \cdots, 0}_{j-1}, 1, \underbrace{0, \cdots, 0}_{n-j})$. In other words, the decision set $U(\mathbf{L})$ consists of all possible combinations of acceptance and rejection decisions for each group type in each row, subject to the constraints that at most one group of each type can be accepted in any row, and the number of seats occupied by each accepted group must not exceed the remaining capacity of the row.


Let $V^{t}(\mathbf{L})$ denote the maximum expected revenue earned by the best decisions regarding group seat assignments in period $t$, given remaining capacity $\mathbf{L}$. Then, the dynamic programming formula for this problem can be expressed as:

\begin{equation}\label{DP}
V^{t}(\mathbf{L}) = \max_{u_{i,j}^{t} \in U^{t}(\mathbf{L})}\left\{ \sum_{i=1}^{M} p_i ( \sum_{j=1}^{N} i u_{i,j}^{t} + V^{t+1}(\mathbf{L}- \sum_{j=1}^{N} n_i u_{i,j}^{t}\mathbf{e}_j)) + p_0 V^{t+1}(\mathbf{L})\right\}
\end{equation}
with the boundary conditions $V^{T+1}(\mathbf{L}) = 0, \forall \mathbf{L}$ which implies that the revenue at the last period is 0 under any capacity.

At the beginning of period $t$, we have the current remaining capacity vector denoted as $\mathbf{L} = (L_1, L_2, \ldots, L_N)$. Our objective is to make group assignments that maximize the total expected revenue during the horizon from period 1 to $T$ which is represented by $V^{1}(\mathbf{L})$.

Solving the dynamic programming problem described in equation \eqref{DP} can be challenging due to the curse of dimensionality, which arises when the problem involves a large number of variables or states. To mitigate this complexity, we aim to develop a heuristic method for assigning arriving groups. In our approach, we begin by generating a seat planning that consists of the largest or full patterns, as outlined in section \ref{sec_seat_planning}. This initial seat planning acts as a foundation for our heuristic method. When the supply for group type $i$ is not sufficient, i.e., $X_i=0$, the seat assignment process involves two main steps. First, we determine the appropriate group type used to accommodated the arriving group. The second step is to choose a specific row according to the group type and make the final decision by evaluating values of stochastic programming. For the whole periods, we regenerate the seat planning under certain conditions to optimize computational efficiency.

% In section \ref{sec_dynamic_seat}, building upon the generated seat planning, we further develop a dynamic seat assignment policy which guides the allocation of seats to the incoming groups sequentially.

% Our policy involves making decisions regarding seat allocations for each arriving group based on a seat planning which can be obtained from Section \ref{sec_seat_planning}. 

\subsubsection{Decision on Assigning The Group to A Specific Row}
To make the final decision, first, we determine a specific row by the tie-breaking rule, then we assign the group based on the values of relaxed stochastic programming. To determine the appropriate row for seat assignment, we can apply a tie-breaking rule among the possible options obtained by the group-type control. This rule helps us decide on a particular row when there are multiple choices available. 

\subsubsection*{Break Tie for Determining A Specific Row}
A tie occurs when there are serveral rows to accommodate the group. As mentioned in section 4, $\beta$ represents the remaining capacity in a row after considering the seat allocation for other groups.
When the supply is sufficient for the particular group type, we assign the group to the row with the smallest $\beta$ value. That allows us to fill in the row according to the full pattern. When the supply is insufficient for the group type and we plan to assign the group to seats designated for larger groups, we follow a similar approach. In this case, we assign the group to a row that contains the larger group and has the largest $\beta$ value. That helps to reconstruct the pattern with smaller $\beta$ value. When there are multiple rows with the same $\beta$, we can choose randomly. By considering the row with the $\beta$ value in both scenarios, we prioritize filling seats and aim to minimize the fragmentation of available seating capacity. This approach helps optimize the utilization of seats and leads to better capacity management.

As an example to illustrate group-type control and the tie-breaking rule, consider a scenario where we have four rows with available seats as follows: row 1 has 3 seats, row 2 has 4 seats, row 3 has 5 seats and row 4 has 6 seats. The corresponding patterns for each row are $(0,1,0,0)$, $(0,0,1,0)$, $(0,0,0,1)$ and $(0,0,0,1)$, respectively. There are $M =4$ groups, the social distance is $\delta =1$. Now, a group of one arrives,  and the group-type control indicates the possible rows where the group can be assigned. We assume this group can be assigned to the seats of the largest group according to the group-type control, then we have two choices: row 3 or row 4. To determine which row to select, we can apply the breaking tie rule. The $\beta$ value of the rows will be used as the criterion, we would choose row 4 because its $\beta$ value is larger. Because when we assign it in row 4, there will be two seats reserved for future group of one, but when we assign it in row 3, there will be one seat remaining unused.

In the above example, the group of one can be assigned to any row with the available seats. The group-type control can help us find the larger group type that can be used to place the arriving group while maximizing the expected values. Maybe there are multiple rows containing the larger group type. Then we can choose the row containing the larger group type according to the breaking tie rule. 
Finally, we use the stochastic programming to calculate the VoA and VoR. Comparing these values allows us to make the decision on whether to accept or reject the group.


\subsubsection{Decision on Assigning The Group to A Specific Row}
To make the final decision, first, we determine a specific row by the tie-breaking rule, then we assign the group based on the values of relaxed stochastic programming. To determine the appropriate row for seat assignment, we can apply a tie-breaking rule among the possible options obtained by the group-type control. This rule helps us decide on a particular row when there are multiple choices available. 

\subsubsection*{Break Tie for Determining A Specific Row}
A tie occurs when there are serveral rows to accommodate the group. As mentioned in section 4, $\beta$ represents the remaining capacity in a row after considering the seat allocation for other groups.
When the supply is sufficient for the particular group type, we assign the group to the row with the smallest $\beta$ value. That allows us to fill in the row according to the full pattern. When the supply is insufficient for the group type and we plan to assign the group to seats designated for larger groups, we follow a similar approach. In this case, we assign the group to a row that contains the larger group and has the largest $\beta$ value. That helps to reconstruct the pattern with smaller $\beta$ value. When there are multiple rows with the same $\beta$, we can choose randomly. By considering the row with the $\beta$ value in both scenarios, we prioritize filling seats and aim to minimize the fragmentation of available seating capacity. This approach helps optimize the utilization of seats and leads to better capacity management.

As an example to illustrate group-type control and the tie-breaking rule, consider a scenario where we have four rows with available seats as follows: row 1 has 3 seats, row 2 has 4 seats, row 3 has 5 seats and row 4 has 6 seats. The corresponding patterns for each row are $(0,1,0,0)$, $(0,0,1,0)$, $(0,0,0,1)$ and $(0,0,0,1)$, respectively. There are $M =4$ groups, the social distance is $\delta =1$. Now, a group of one arrives,  and the group-type control indicates the possible rows where the group can be assigned. We assume this group can be assigned to the seats of the largest group according to the group-type control, then we have two choices: row 3 or row 4. To determine which row to select, we can apply the breaking tie rule. The $\beta$ value of the rows will be used as the criterion, we would choose row 4 because its $\beta$ value is larger. Because when we assign it in row 4, there will be two seats reserved for future group of one, but when we assign it in row 3, there will be one seat remaining unused.

In the above example, the group of one can be assigned to any row with the available seats. The group-type control can help us find the larger group type that can be used to place the arriving group while maximizing the expected values. Maybe there are multiple rows containing the larger group type. Then we can choose the row containing the larger group type according to the breaking tie rule. 
Finally, we use the stochastic programming to calculate the VoA and VoR. Comparing these values allows us to make the decision on whether to accept or reject the group.

% When the available supply is sufficient to accommodate an arriving group, we can directly assign the group seats and update the remaining supply accordingly.

% Notice that the relaxed stochastic programming will have the same value.

\subsubsection*{Decision on Assigning The Group}
Then, we compare the values of the relaxed stochastic programming when accepting the group at the chosen row versus rejecting it. This evaluation allows us to assess the potential revenues and make the final decision. Simultaneously, after this calculation, we can generate a new seat planning according to Algorithm 3. For the situation where the supply is enough in the first step, we can skip the final step because we already accept the group. Specifically, after we plan to assign the arriving group in a specific row, we determine whether to place the arriving group in the row based on the values of the stochastic programming problem. For the objective values of the relaxed stochastic programming, we consider the potential revenues that could result from accepting the current arrival, i.e., the Value of Acceptance (VoA), as well as the potential outcomes that could result from rejecting it, i.e., the Value of Rejection (VoR). 

Let us consider the set of scenarios, denoted as $\Omega^{t}$, at period $t$. The VoA is the value of 
relaxed stochastic programming by considering the scenario set $\Omega^{t}$ when we accept the arriving group at period $t$. On the other hand, the VoR is calculated when we reject the arriving arrival. Suppose the available supply is $\mathbf{L}^{t} = (L_1^{t}, \ldots, L_N^{t})$ before making the decision at period $t$. We calculate the relaxed stochastic programming with $\mathbf{L}^{t}= (L_1^{t}, \ldots, L_N^{t})$ when we reject group type $i$ as the VoR. If we plan to accept group type $i$ in row $j$, we need to subtract the size of group type $i$ from $L_j^{t}$. Let $L_j^{t} = L_j^{t} - n_{i}$, then we calculate the relaxed stochastic programming with $\mathbf{L}^{t}= (L_1^{t}, \ldots, L_N^{t})$ when we accept group type $i$ in row $j$ as the VoA.

In each period, we can calculate the relaxed stochastic programming values only twice: once for the acceptance option (VoA) and once for the rejection option (VoR). By comparing the values of VoA and VoR, we can determine whether to accept or reject the group arrival. The decision will be based on selecting the option with the higher expected value, i.e., if the VoA is larger than the VoR, we accept the arrival; if the VoA is less than the VoR, we will reject the incoming group.

% This adjustment ensures that the stochastic programming model captures the impact of rejecting group type $i$ on the remaining seats.

% In such cases, we refer to the corresponding planning group row in the group-type control, where we determine which group to break in order to accommodate the incoming group. 

By combining the group-type control strategy with the evaluation of relaxed stochastic programming values, we obtain a comprehensive decision-making process within a single period. This integrated approach enables us to make informed decisions regarding the acceptance or rejection of incoming groups, as well as determine the appropriate row for the assignment when acceptance is made. By considering both computation time savings and potential revenues, we can optimize the overall performance of the seat assignment process.

% In essence, this decision-making approach weighs the potential benefits associated with accepting or rejecting an arrival, allowing us to select the option that maximizes the objective value and aligns with our overall goals.

\subsubsection{Regenerate The Seat Planning}
To optimize computational efficiency, it is not necessary to regenerate the seat planning for every period. Instead, we can employ a more streamlined approach. Considering that largest group type can meet the needs of all smaller group types, thus, if the supply for the largest group type diminishes from one to zero, it becomes necessary to regenerate the seat planning. This avoids rejecting the largest group due to infrequent regenerations. Another situation that requires seat planning regeneration is when we determine whether to assign the arriving group to a larger group. In such case, we can obtain the corresponding seat planning after solving the relaxed stochastic programmings. By regenerating the seat planning in such situations, we ensure that we have an accurate supply and can give the allocation of seats based on the group-type control and the comparisons of VoA and VoR.


The decision-making algorithm is shown below.


\begin{algorithm}[H]
  \caption{Dynamic Seat Assignment}\label{algo_dynamic_policy}
  % \KwIn{Seat planning $\bm{H}$, Supply $\bm{X}$, $\mathbf{L}$}
  % \KwOut{Decision}
  \For{$t =1, \ldots, T$}
  { Observe group type $i$\;
    \eIf{$X_{i} > 0$}
    {Find row $k$ such that $H_{ki} >0$ according to tie-breaking rule\; Accept group type $i$ in row $k$, $L_{k} \gets L_{k} -n_{i}$\; $H_{ki} \gets H_{ki} -1$, $X_{i} \gets X_{i} -1$\Comment*[r]{Accept group type $i$ when the supply is sufficient}
    \If{$i = M$ and $X_{M} =0$}
    {Regenerate $\bm{H}$ from Algorithm 3\; Update the corresponding $\bm{X}$\Comment*[r]{Regenerate the seat planning when the supply of the largest group type is 0}}
    }
    {Calculate $d^{t}(i, j^{*})$\;
    \eIf{$d^{t}(i, j^{*}) >0 $}
    {Find row $k$ such that $H_{kj^{*}} > 0$ according to tie-breaking rule\; 
    Calculate the VoA under scenario $\Omega^{t}_{A}$ and the VoR under scenario $\Omega^{t}_{R}$\;
    \eIf{VoA $>$ VoR}
    {Accept group type $i$, $L_{k} \gets L_{k} - n_{i}$\; Regenerate $\bm{H}$ from Algorithm 3\; Update the corresponding $\bm{X}$\;}
    {Reject group type $i$\; Regenerate $\bm{H}$ from Algorithm 3\; Update the corresponding $\bm{X}$\;}}
    {Reject group type $i$\;}
    }}
\end{algorithm}


