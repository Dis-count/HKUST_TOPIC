% !TEX root = sum1.tex
\section{Results}
We carried out several experiments, including analyzing the performances of different policies, evaluating the impact of implementing social distancing.
In the experiment, we set the following parameters. The seat layout consists of 10 rows, each containing 20 physical seats. To account for social distancing measures, one seat is designated as one dummy seat. The group sizes considered range from 1 to 4 people. In our experiments, we simulate the arrival of exactly one group in each period, i.e., $p_0 = 0$. The average number of people per period, denoted as $\gamma$, can be expressed as $\gamma = p_1 \cdot 1 + p_2 \cdot 2 + p_3 \cdot 3 + p_4 \cdot 4$, where $p_1$, $p_2$, $p_3$, and $p_4$ represent the probabilities of groups with one, two, three, and four people, respectively. We assume that $p_4$ always has a positive value. 


\subsection{Performances of Different Policies}
In this section, we compare the performance of five assignment policies to the optimal one, which can be obtained by solving the deterministic model after observing all arrivals. The policies under examination are the dynamic seat assignment policy, DP Base-heuristic, bid-price control, booking limit control and FCFS policy. 

For the DP Base-heuristic, we consider a simplified dynamic programming by relaxing all rows to a single row with the same total capacity, denoted as $L = \sum_{j=1}^{N} L_j$. With this simplification, we can make decisions for each group arrival based on the relaxed dynamic programming. Bid-price control is a classical approach discussed extensively in the literature on network revenue management. It involves setting bid prices for different group types, which determine the eligibility of groups to take the seats. Specifically, we estimate the bid price of a seat by the shadow price of the capacity constraint in the LP relaxation of problem \eqref{deter_upper}. Then we assign the seats by comparing the revenues of accepting or denying the groups. Booking limit control policy involves setting a maximum number of reservations for each group type. In this policy, we solve problem \eqref{deter_upper} with the expected demand during the time period. Then for every type of requests, we only allocate a fixed amount according to the static solution and reject all other exceeding requests.

In order to assess the effectiveness of different policies across different demand levels, we conducted experiments spanning a range of 60 to 100 periods. For our analysis, we considered three probability distributions: [0.25, 0.25, 0.25, 0.25], [0.25, 0.35, 0.05, 0.35], and [0.15, 0.25, 0.55, 0.05]. These distributions have a common mean $\gamma$, ensuring that the expected number of people for each period remains consistent.

The table presents the performance results of five different policies: DSA, DP1, Bid-price, Booking, and FCFS, which stand for dynamic seat assignment, dynamic programming based heuristic, bid-price, booking-limit, and first come first served, respectively. The procedures for the last four policies are detailed in the appendix \ref{policies}. Each entry in the table represents the average performance across 100 instances. Performance is evaluated by comparing the ratio of the number of accepted people under each policy to the number of accepted people under the optimal policy, which assumes complete knowledge of all incoming groups before making seat assignments.


\begin{table}[ht]
  \centering
  \caption{Performances of Different Policies}
  \begin{tabular}{|l|l|l|l|l|l|l|}
  \hline
  Probabilities &  T &  DSA (\%) & DP1 (\%) & Bid-price (\%) & Booking (\%) & FCFS (\%) \\
  \hline
  [0.25,0.25,0.25,0.25] & 60  &  99.12 & 98.42 & 98.38 & 96.74 & 98.17 \\
  & 70                             & 98.34 & 96.87 & 96.24 & 97.18 & 94.75 \\
  & 80                             & 98.61 & 95.69 & 96.02 & 98.00 & 93.18 \\
  & 90                             & 99.10 & 96.05 & 96.41 & 98.31 & 92.48 \\
  & 100                            & 99.58 & 95.09 & 96.88 & 98.70 & 92.54 \\
   \hline
  [0.25, 0.35, 0.05, 0.35] & 60  &  98.94 & 98.26 & 98.25 & 96.74 & 98.62 \\
  & 70                           & 98.05 & 96.62 & 96.06 & 96.90 & 93.96 \\
  & 80                           & 98.37 & 96.01 & 95.89 & 97.75 & 92.88 \\
  & 90                           & 99.01 & 96.77 & 96.62 & 98.42 & 92.46 \\
  & 100                          & 99.23 & 97.04 & 97.14 & 98.67 & 92.00 \\
  \hline
  [0.15, 0.25, 0.55, 0.05] & 60  &  99.14 & 98.72 & 98.74 & 96.61 & 98.07 \\
  & 70                             & 99.30 & 96.38 & 96.90 & 97.88 & 96.25 \\
  & 80                             & 99.59 & 97.75 & 97.87 & 98.55 & 95.81 \\
  & 90                             & 99.53 & 98.45 & 98.69 & 98.81 & 95.50 \\
  & 100                            & 99.47 & 98.62 & 98.94 & 98.90 & 95.25 \\
  \hline
  \end{tabular}
\end{table}

We can find that DSA is better than DP Base-heuristic, bid-price policy and booking limit policy consistently, and FCFS policy works worst. The main reason is that DP Base-heuristic and bid-price policy can only make the decision to accept or deny, cannot decide which row to assign the group to. Booking limit policy does not consider using larger seats to meet the group demand. FCFS accepts groups in sequential order until the capacity cannot accommodate more.

% For the DSA, DP Base-heuristic, bid-price policies, their performance tends to initially drop and then increase as the number of periods increases. When the number of periods is small, the demand for capacity is relatively low, and the policies can achieve relatively optimal performance. However, as the number of periods increases, the policies may struggle to always obtain a perfect allocation plan, leading to a decrease in performance. Nevertheless, when the number of periods continue to become larger, these policies tend to accept larger groups, and as a result, narrow the gap with the optimal value, leading to an increase in performance. As the number of periods increases, the performance for the booking limit policy is getting better because the numebr of seats planned for the largest groups that are not occupied will drop.

The performance of DSA, DP Base-heuristic, and bid-price policies follows a pattern where it initially decreases and then gradually improves as $T$ increases. When $T$ is small, the demand for capacity is generally low, allowing these policies to achieve relatively optimal performance. However, as $T$ increases, it becomes more challenging for these policies to consistently achieve a perfect allocation plan, resulting in a decrease in performance. Nevertheless, as $T$ continues to grow, these policies tend to accept larger groups, thereby narrowing the gap between their performance and the optimal value. Consequently, their performance improves. In contrast, the booking limit policy shows improved performance as $T$ increases because it reduces the number of unoccupied seats reserved for the largest groups.

The performance of the policies can vary based on different probabilities. For different probability distributions listed, DSA performs more stably and consistently for the same demand under different probabilities, while for DP and bid price, their performance fluctuates more.

\subsection{Impact of Implementing Social Distancing in DSA}
In this section, our focus is to analyze the influence of social distancing on the number of accepted individuals. Intuitively, when demand is small, we will accept all arrivals, thus there is no difference whether we implement the social distancing. What is interesting for us is when the difference occurs. Our primary objective is to determine the first time period at which, on average, the number of people accepted without social distancing is not less than the number accepted with social distancing plus one. This critical point, referred to as the gap point, denoted by $\tilde{T}$, is of interest to us. Additionally, we will examine the corresponding occupancy rate, $\beta(\tilde{T})$, at this gap point. It should be noted that the difference at a specific time period may vary depending on the total number of periods considered. Therefore, when evaluating the difference at a particular time period, we assume that there are a total of such periods under consideration.

It is evident that as the demand increases, the effect of social distancing becomes more pronounced. We aim to determine the specific time period where the absence of social distancing results in a higher number of accepted individuals compared to when social distancing measures are in place. Additionally, we will calculate the corresponding occupancy rate during this period.

By analyzing and comparing the data, we can gain insights into the relation between demand, social distancing, the number of accepted individuals, and occupancy rates. This information is valuable for understanding the impact of social distancing policies on overall capacity utilization and making informed decisions regarding resource allocation and operational strategies.

\subsubsection{Estimation of Gap Point}
To find such a first period, we aim to find the maximum period such that we could assign all the groups during these periods into the seats, i.e., for each group type $i$, we have $\sum_{j} x_{ij} \geq d_i$, where $x_{ij}$ is the number of group type $i$ in row $j$. Meanwhile, we have the capacity constraint $\sum_{i} n_{i} x_{ij} \leq L_j$, thus, $\sum_{i} n_i d_i \leq \sum_{i} n_i \sum_{j} x_{ij} \leq \sum_{j} L_{j}$. Notice that $E(d_i) = p_i T$, we have $\sum_{i} n_i p_i T \leq \sum_{j} L_{j}$ by taking the expectation. Let $L = \sum_{j} L_{j}$, representing the total number of seats, $\gamma = \sum_{i} i p_i$, representing the average number of people who arrive in each period, we can obtain $T \leq \frac{L}{\gamma + \delta}$, then the upper bound of the expected maximum period is $T' = \frac{L}{\gamma + \delta}$.


Assuming that we accept all incoming groups within $T'$ periods, filling all the available seats, the corresponding occupancy rate at this period can be calculated as $\frac{\gamma T'}{(\gamma+ \delta)T' - N \delta} = \frac{\gamma}{\gamma +\delta} \frac{L}{L-N \delta}$. However, it is important to note that the actual maximum period will be smaller than $T{'}$ because it is impossible to accept groups to fill all seats exactly. To estimate the gap point when applying DSA, we can use $y_1 = c_1 \frac{L}{\gamma + \delta}$, where $c_1$ is a discount rate compared to the ideal assumption. Similarly, we can estimate the corresponding occupancy rate as $y_2 = c_2 \frac{\gamma}{\gamma +\delta} \frac{L}{L-N \delta}$, where $c_2$ is a discount rate for the occupancy rate compared to the ideal scenario.


To analyze the relation between the increment of $\gamma$ and the gap point, we define each combination $(p_1, p_2, p_3, p_4)$ satisfying $p_1 + p_2 + p_3 + p_4 = 1$ as a probability combination. We conducted an analysis using a sample of 200 probability combinations. The figure below illustrates the gap point as a function of the increment of $\gamma$, along with the corresponding estimations. For each probability combination, we considered 100 instances and plotted the gap point as blue points. Additionally, the occupancy rate at the gap point is represented by red points.

To provide estimations, we utilize the equations $y_1 = \frac{c_1 L}{\gamma + \delta}$(blue line in the figure) and $y_2 = c_2 \frac{\gamma}{\gamma + \delta} \frac{L}{L-N \delta}$(orange line in the figure), which are fitted to the data. These equations capture the relation between the gap point and the increment of $\gamma$, allowing us to approximate the values. By examining the relation between the gap point and the increment of $\gamma$, we can find that $\gamma$ can be used to estimate gap point.

\begin{figure}[ht]
  \centering
    \includegraphics[width=0.8\textwidth]{./Figures/re2.pdf}
  \caption{Gap points and their estimation under 200 probabilities}
\end{figure}


The fitting values of $c_1$ and $c_2$ can be affected by different seat layouts. To investigate this impact, we conduct several experiments using different seat layouts, specifically with the number of rows $\times$ the number of seats configurations set as 10 $\times$ 16, 10 $\times$ 21, 10 $\times$ 26 and 10 $\times$ 31. Similarly, we perform an analysis using a sample of 100 probability combinations, each with a mean equal to $\gamma$. The values of $\gamma$ range from 1.5 to 3.4. We employed an Ordinary Least Squares (OLS) model to fit the data and derive the parameter values. The goodness of fit is assessed using the R-square values, which are found to be 1.000 for all models, indicating a perfect fit between the data and the models.

The results of the estimation of $c_1$ and $c_2$ are presented in the table below:

\begin{table}[ht]
  \centering
  \caption{Fitting values of $c_1$ and $c_2$}
  \begin{tabular}{|c|c|c|}
  \hline
   Seat layout(\# of rows $\times$ \# of seats) & Fitting Values of $c_1$ & Fitting Values of $c_2$  \\
  \hline
   10 $\times$ 11 & 0.909 $\pm$ 0.013  & 89.89 $\pm$ 1.436 \\
   10 $\times$ 16 & 0.948 $\pm$ 0.008  & 94.69 $\pm$ 0.802 \\
   10 $\times$ 21 & 0.955 $\pm$ 0.004 & 95.44 $\pm$ 0.571 \\
   10 $\times$ 26 & 0.966 $\pm$ 0.004 & 96.23 $\pm$ 0.386 \\
   10 $\times$ 31 & 0.965 $\pm$ 0.003 & 96.67 $\pm$ 0.434 \\
   10 $\times$ 36 & 0.968 $\pm$ 0.003 & 97.04 $\pm$ 0.289 \\
   \hline
  \end{tabular}
\end{table}

As the number of seats in each row increases, the fitting values of $c_1$ and $c_2$ will gradually increase.


\subsubsection{Impact of Social Distancing as The Demand Increases}
Now, we consider the impact of social distance as demands increase by changing $T$. Specifically, we consider two situations: $\gamma = 1.9$ and $\gamma = 2.5$. Here, $T$ varies from 30 to 120, the step size is 1. Other parameters are set the same as before.

The figure below displays the outcomes of groups who were accepted under two different conditions: with social distancing measures and without social distancing measures. For the former case, we employ DSA to obtain the results. In this case, we consider the constraints of social distancing and optimize the seat allocation accordingly. For the latter case, we adopt a different approach. We simply accept all incoming groups as long as the capacity allows, without considering the constraints of groups needing to sit together. This means that we prioritize filling the available seats without enforcing any specific seating arrangements or social distancing requirements. We conduct an analysis using a sample of 100 probability combinations associated with the same $\gamma$. The occupancy rate at different demands is calculated as the mean of these 100 samples. The figures depicting the results are presented below.

\begin{figure}[h]
  \centering
  \subfigure[When $\gamma =1.9$]{
    \label{Fig.sub.1}
    \includegraphics[width=0.48\textwidth]{./Figures/p2.pdf}}
  \subfigure[When $\gamma =2.5$]{
    \label{Fig.sub.2}
    \includegraphics[width=0.48\textwidth]{./Figures/p1.pdf}}
  \caption{The occupancy rate over the number of arriving people}
  \label{Fig.lable}
\end{figure}

The analysis consists of three stages. 
In the first stage, when the capacity is sufficient, the outcome remains unaffected by the implementation of social distancing measures. In the second stage, as the value of $T$ increases, the difference between the outcomes with and without social distancing measures becomes more pronounced. Finally, in the third stage, as $T$ continues to increase, both scenarios reach their maximum capacity acceptance. At this point, the gap between the outcomes with and without social distancing measures begins to converge. For the social distancing situation, according to Proposition \ref{lem_pattern}, when the largest pattern is assigned to each row, the resulting occupancy rate is $\frac{16}{20} = 80\%$, which is the upper bound of occupancy rate. The below table presents the occupancy rate differences for different demand levels (130, 150, 170, 190, 210).


\begin{table}[ht]
  \centering
  \caption{Gap points and occupancy rate differences under different demands of different gammas}
\begin{tabular}{llllllll}
  \hline
  \multicolumn{1}{|l|}{\multirow{2}{*}{$\gamma$}} & \multicolumn{1}{l|}{\multirow{2}{*}{$\tilde{T}$}} & \multicolumn{1}{l|}{\multirow{2}{*}{$\beta(\tilde{T})$}} & \multicolumn{5}{l|}{$\Delta \beta(T)$ under different demands}   \\ 
  \cline{4-8} 
  \multicolumn{1}{|l|}{}  & \multicolumn{1}{l|}{} & \multicolumn{1}{l|}{} & \multicolumn{1}{l|}{130} & \multicolumn{1}{l|}{150} & \multicolumn{1}{l|}{170} & \multicolumn{1}{l|}{190} & \multicolumn{1}{l|}{210} \\ 
  \hline
  \multicolumn{1}{|l|}{1.9}  & \multicolumn{1}{l|}{69} & \multicolumn{1}{l|}{65.52}  & \multicolumn{1}{l|}{0.25}  & \multicolumn{1}{l|}{5.82}  & \multicolumn{1}{l|}{12.82} & \multicolumn{1}{l|}{20.38} & \multicolumn{1}{l|}{24.56} \\
  \hline                                    
  \multicolumn{1}{|l|}{2.1}  & \multicolumn{1}{l|}{64} & \multicolumn{1}{l|}{67.74} & \multicolumn{1}{l|}{0.05}  & \multicolumn{1}{l|}{4.11}  & \multicolumn{1}{l|}{11.51} & \multicolumn{1}{l|}{18.77} & \multicolumn{1}{l|}{21.87} \\ 
  \hline           
  \multicolumn{1}{|l|}{2.3}  & \multicolumn{1}{l|}{61} & \multicolumn{1}{l|}{69.79}  & \multicolumn{1}{l|}{0}  & \multicolumn{1}{l|}{2.29}  & \multicolumn{1}{l|}{10.21} & \multicolumn{1}{l|}{17.36} & \multicolumn{1}{l|}{21.16} \\ 
  \hline           
  \multicolumn{1}{|l|}{2.5}  & \multicolumn{1}{l|}{57} & \multicolumn{1}{l|}{70.89} & \multicolumn{1}{l|}{0}  & \multicolumn{1}{l|}{1.45}  & \multicolumn{1}{l|}{9.30} & \multicolumn{1}{l|}{15.78} & \multicolumn{1}{l|}{19.80} \\ 
  \hline          
  \multicolumn{1}{|l|}{2.7}  & \multicolumn{1}{l|}{53} & \multicolumn{1}{l|}{71.28}  & \multicolumn{1}{l|}{0}  & \multicolumn{1}{l|}{1.38}  & \multicolumn{1}{l|}{7.39} & \multicolumn{1}{l|}{14.91} & \multicolumn{1}{l|}{19.14} \\ 
  \hline            
\end{tabular}
\end{table}


As the value of $\gamma$ increases, the period of gap point will decrease and the corresponding occupancy rate will increase. This suggests that larger groups are more likely to be accepted, resulting in a smaller number of accepted groups overall. Consequently, the occupancy rate increases due to the allocation of seats to larger groups. The percentage difference is negligible when the demand is small, but it becomes more significant as the demand increases.


We examine the impact of implementing social distancing on the occupancy rate and explore strategies to minimize revenue loss. Consider the situation where the gap point is $\tilde{T}$ and is determined by the parameters $\delta$, $\gamma$, $M$, and $L$. The corresponding occupancy rate is $\beta(\tilde{T})$. When the actual number of people (demand) is less than $L \cdot \beta(\tilde{T})$, implementing social distancing does not affect the revenue. However, if the actual number of people exceeds $L \cdot \beta(\tilde{T})$, enforcing social distancing measures will lead to a reduction in revenue. The extent of this loss can be assessed through simulations by using the specified parameters. To mitigate the potential loss, the seller can increase the value of $\gamma$. This can be achieved by implementing certain measures, such as setting a limit on the number of single-person groups or allowing for larger group sizes. The government can set a requirement for a higher occupancy rate limit, for example, for family movies in cinemas, $\gamma$ will be relatively large, so a higher occupancy rate limit can be set. By doing so, the objective is to minimize the negative impact of social distancing while maximizing revenue within the constraints imposed by the occupancy rate.