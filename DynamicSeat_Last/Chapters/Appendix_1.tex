% !TEX root = sum1.tex
\newpage

\appendix
\section{Policies for Dynamic Seat Assignment}\label{policies}

\subsection{Relaxed Dynamic Programming Heuristic (RDPH)}

According to the RDP formulation in \eqref{DP_relaxed}, we can determine whether to accept or reject each request. For accepted requests, we must then decide on specific seat assignments. However, without a predefined seat plan, this assignment process lacks clear guidelines. To resolve this, we implement a modified Best Fit rule \citep{johnson1974fast}, assigning each group to the row with the minimum remaining capacity that can still accommodate it. An important prerequisite for this assignment is verifying seat availability. Specifically, if the group size exceeds the maximum remaining capacity across all rows, the request must be rejected.

This policy is stated in the following algorithm.

\begin{algorithm}[H]
  \caption{RDP Heuristic}\label{algo_dp_heuris}
  Calculate $V^{t}(l)$ by \eqref{DP_relaxed}, $\forall t =2, \ldots, T; \forall l = 1, 2, \ldots, l^{1}=\tilde{L}$\;
  \For{$t =1, \ldots, T$}{
    {Observe a request of group type $i$\;}
    \eIf{$\max_{j \in \mathcal{N}} L_{j}^{t} \geq n_i$ and $V^{t+1}(l^{t}) \leq V^{t+1}(l^{t}-n_i) + i$}
    {Set $k = \arg \min_{j \in \mathcal{N}}\{L_j^{t}|L_j^{t} \geq n_i\} $ and break ties\;
    Assign the group to row $k$, let $L_{k}^{t+1} \gets L_{k}^{t} - n_{i}$, $l^{t+1} \gets l^{t}- n_{i}$\;}
    {Reject the group and let $L_{k}^{t+1} \gets L_{k}^{t}$, $l^{t+1} \gets l^{t}$\;}}
\end{algorithm}

\subsection{Bid-Price Control (BPC) Policy}
Bid-price control is a classical approach discussed extensively in the literature on network revenue management. It involves setting bid prices for different group types, which determine the eligibility of groups to take the seats. Bid-prices refer to the opportunity costs of taking one seat. As usual, we estimate the bid price of a seat by the shadow price of the capacity constraint corresponding to some row. In this section, we will demonstrate the implementation of the bid-price control policy. 

The dual of LP relaxation of the SPDR problem is:

\begin{equation}\label{bid-price_dual}
  \begin{aligned}
  \min \quad & \sum_{i=1}^{M} d_i z_i + \sum_{j= 1}^{N} L_j \beta_{j} \\
  \text {s.t.} \quad & z_{i} + \beta_j n_i \geq (n_i-\delta), \quad i \in \mathcal{M}, j \in \mathcal{N} \\
  & z_{i} \geq 0, i \in \mathcal{M}, \beta_{j} \geq 0, j \in \mathcal{N}.
  \end{aligned}
\end{equation}

In \eqref{bid-price_dual}, $\beta_{j}$ can be interpreted as the bid-price for a seat in row $j$. A request is only accepted if the revenue it generates is no less than the sum of the bid prices of the seats it uses. Thus, if $i -\beta_{j} n_i \geq 0$, we will accept the group type $i$. And choose $j^{*} = \arg \max_{j} \{i -\beta_{j} n_i\}$ as the row to allocate that group.


\begin{lem}\label{bid-price}
 The optimal solution to problem \eqref{bid-price_dual} is given by $z_1 = \ldots = z_{\tilde{i}} =0$, $z_{i} = \frac{\delta(n_i-n_{\tilde{i}})}{n_{\tilde{i}}}$ for $i = \tilde{i}+1, \ldots, M$ and $\beta_j = \frac{n_{\tilde{i}} - \delta}{n_{\tilde{i}}}$ for all $j$.
\end{lem}

The bid-price decision can be expressed as $i - \beta_j n_i = i - \frac{n_{\tilde{i}} - \delta}{n_{\tilde{i}}} n_i = \frac{\delta (i - \tilde{i})}{n_{\tilde{i}}}$. When $i < \tilde{i}$, $i - \beta_j n_i < 0$. When $i \geq \tilde{i}$, $i - \beta_j n_i \geq 0$. This implies that group type $i$ greater than or equal to $\tilde{i}$ will be accepted if the capacity allows. However, it should be noted that $\beta_j$ does not vary with $j$, which means the bid-price control cannot determine the specific row to assign the group to. We maintain the same tie-breaking rule as in RDPH, assigning each group to the row with the minimum residual capacity while still satisfying the accommodation requirement.

% In practice, groups are often assigned arbitrarily based on availability when the capacity allows, which may result in the empty seats.

The bid-price control policy based on the static model is stated below.

\begin{algorithm}[H]
  \caption{Bid-Price Control}\label{algo_bid}
  \For{$t =1, \ldots, T$}{
    {Observe a request of group type $i$\;}
    {Solve the LP relaxation of the SPDR problem with $\bm{d}^{t} = (T-t) \cdot \bm{p}$ and $\mathbf{L}^{t}$\;
    Obtain $\tilde{i}$ such that the aggregate optimal solution is $x e_{\tilde{i}} + \sum_{i=\tilde{i}+1} ^{M} d_{i}^{t} e_{i}$\;}
    \eIf{$i \geq \tilde{i}$ and $\max_{j \in \mathcal{N}}{L_j^{t}} \geq n_i$}
    {Set $k = \arg \min_{j \in \mathcal{N}}\{L_j^{t}|L_j^{t} \geq n_i\} $ and break ties\;
    Assign the group to row $k$, let $L_{k}^{t+1} \gets L_{k}^{t} - n_{i}$ \;}
    {Reject the group\;}}
\end{algorithm}


\subsection{Booking-Limit Control (BLC) Policy}
The booking-limit control policy involves setting a maximum number of reservations that can be accepted for each request. By controlling the booking-limits, revenue managers can effectively manage demand and allocate inventory to maximize revenue. 
Under this booking limit policy, we first solve the SPDR problem using the expected demand. We establish a fixed allocation quota for each group type based on this static solution. For the incoming request, we accept it if the corresponding supply is sufficient, otherwise reject it. Then the request will be assigned to an arbitrary row containing the planned seats for the corresponding group type.



% In this policy, we solve the SPDR problem with the expected demand. Then for every type of requests, we allocate a fixed amount according to the static solution and reject all other exceeding requests.
% When we solve the LP relaxation of SPDRP, the aggregate optimal solution is the limits for each group type. Interestingly, the bid-price control policy is found to be equivalent to the booking limit control policy.

% we can develop the booking limit control policy.

\begin{algorithm}[H]
  \caption{Booking-Limit Control}\label{algo_booking}
  \For{$t =1, \ldots, T$}{
    {Observe a request of group type $i$\;}
    {Solve the SPDR problem with $\bm{d}^{t} = (T-t) \cdot \bm{p}$ and $\mathbf{L}^{t}$\;
    Obtain the seat plan $\bm{H}^{t}$ and the supply $\bm{X}^{t}$\;}
    \eIf{$X_i^{t} > 0$}
    {Set $k = \arg \min_{j \in \mathcal{N}} \{L_j^{t} - \sum_{i}n_i H_{ij}^{t}|H_{ij}^{t} >0\}$\;
    Break ties arbitrarily\;
    Assign the group to row $k$, let $L_{k}^{t+1} \gets L_{k}^{t} - n_{i}$, $H_{ik}^{t+1} \gets H_{ik}^{t}-1$\;}
    {Reject the group\;}}
\end{algorithm}


% \subsubsection*{First-Come-First-Served (FCFS) Policy}
% In the seat assignment for each group arrival, the intuitive but trivial method will be on a first-come-first-served basis. Each accepted request will be assigned seats row by row. If the capacity of a row is insufficient to accommodate a request, we will allocate it to the next available row. If a subsequent request can fit exactly into the remaining capacity of some row, we will assign it to that row immediately. Then continue to process requests in this manner until all rows cannot accommodate any groups.

% \begin{algorithm}[H]
%   \caption{FCFS Policy Algorithm}\label{algo_fcfs}
%   \For{$t =1, \ldots, T$}{
%     {Observe a request of group type $i$\;}
%     \eIf{$\max_{j \in \mathcal{N}}{L_j^{t}} \geq n_i$}
%     {Set $k = \arg \min_{j \in \mathcal{N}}\{L_j^{t}|L_j^{t} \geq n_i\}$\;
%     Break ties arbitrarily\;
%     Assign the group to row $k$, let $L_{k}^{t+1} \gets L_{k}^{t} - n_{i}$\;}
%     {Reject the group and let $L_{k}^{t+1} \gets L_{k}^{t}$\;}}
% \end{algorithm}

\newpage
