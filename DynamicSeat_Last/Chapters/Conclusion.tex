% !TEX root = sum1.tex
\section{Conclusion}\label{sec_conclusion}
We study the seat planning and seat assignment problem under social distancing requirement. 
Specifically, we first consider seat planning with deterministic requests. To utilize all seats, we introduce the full and largest patterns. Subsequently, we investigate seat planning with stochastic requests. To tackle this problem, we propose a scenario-based stochastic programming model.
Then, we utilize the Benders decomposition method to efficiently obtain seat plan, which serves as a reference for dynamic seat assignment. Last but not least, we introduce an approach to address the problem of dynamic seat assignment by integrating relaxed dynamic programming and a group-type control policy. 

We conduct several numerical experiments to investigate various aspects of our approach. First, we analyze different policies for dynamic seat assignment. In terms of dynamic seat assignment policies, we consider the classical bid-price control, booking limit control in revenue management, dynamic programming-based heuristics, and the first-come-first-served policy. Comparatively, our proposed policy exhibited superior and consistent performance.

Building upon our policies, we further evaluate the impact of implementing social distancing. By defining the gap point to characterize the situations under which social distancing begins to cause loss to an event, the experiments show that the gap point depends mainly on the mean of the group size.
This lead us to estimate the gap point by the mean of the group size.

Our models and analysis are developed for the social distancing requirement on the physical distance and group size, where we can determine an expected occupancy rate for any given event in a venue, and a maximum achievable occupancy rate for all events. Sometimes the government also imposes a maximum allowed occupancy rate to tighten the social distancing requirement. This maximum allowed rate is effective for an event if it is lower than the expected occupancy rate of the event. Furthermore,
the maximum allowed rate will be redundant if it is higher than the maximum achievable rate for all
events. The above qualitative insights are stable with respect to different parameters in the model, such as the minimum physical distances, the maximum group sizes, and the layout of the venue.



% While different probability distributions represent varying group profiles, our policy demonstrates consistent performance across different scenarios.


% Overall, our study emphasizes the operational significance of social distancing in the seat allocation and offers a fresh perspective for sellers and governments to implement seat assignment mechanisms that effectively promote social distancing requirement.


Future research can be pursued in several ways. First, when the seating requests are established, a scattered seat assignment can be examined to maximize the distance between adjacent groups when sufficient seating is available. Second, more flexible scenarios where individuals can select the seats by their preference may be considered. Third, research could also examine the problem where 
individuals can arrive and leave at different times in the shared areas.

