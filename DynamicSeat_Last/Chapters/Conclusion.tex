% !TEX root = sum1.tex
\section{Conclusion}\label{sec_conclusion}
We study the seating management problem under social distancing requirement. Specifically, we first consider the seat planning with deterministic requests problem. To utilize all seats, we introduce the full and largest patterns. Subsequently, we investigate the seat planning with stochastic requests problem. To tackle this problem, we propose a scenario-based stochastic programming model. Then, we utilize the Benders decomposition method to efficiently obtain a seat plan, which serves as a reference for dynamic seat assignment. Last but not least, we introduce an approach to address the problem of dynamic seat assignment by integrating the relaxed dynamic programming and the group-type control allocation. 

We conduct several numerical experiments to investigate various aspects of our approach. First, we analyze different policies for dynamic seat assignment. In terms of dynamic seat assignment policies, we consider the bid-price control, booking-limit control, relaxed dynamic programming heuristic policies. Comparatively, our proposed policy exhibited superior and consistent performance.

Building upon our policies, we further evaluate the impact of implementing social distancing. By introducing the concept of the threshold of requests to characterize situations under which social distancing begins to cause loss to an event, our experiments show that the threshold of requests depends mainly on the mean of the group size. This lead us to estimate the threshold of requests by the mean of the group size.


Our models and analyses are developed for the social distancing requirement on the physical distance and group size, where we can determine an threshold of occupancy rate for any given event in a venue, and a maximum achievable occupancy rate for all events. Sometimes the government also imposes a maximum allowable occupancy rate to tighten the social distancing requirement. This maximum allowable rate is effective for an event if it is lower than the threshold of occupancy rate of the event. Furthermore, the maximum allowable rate will be redundant if it is higher than the maximum achievable rate for all events. The above qualitative insights are stable concerning the tightness of the policy as well as the specific characteristics of various venues.


% While different probability distributions represent varying group profiles, our policy demonstrates consistent performance across different scenarios.


% Overall, our study emphasizes the operational significance of social distancing in the seat allocation and offers a fresh perspective for sellers and governments to implement seat assignment mechanisms that effectively promote social distancing requirement.


Future research can be pursued in several directions. First, when seating requests are predetermined, a scattered seat assignment approach can be explored to maximize the distance between adjacent groups when sufficient seating is available. Second, more flexible scenarios could be considered, such as allowing individuals to select seats based on their preferences. Third, research could also investigate scenarios where individuals arrive and leave at different times, adding another layer of complexity to the problem.


