% !TEX root = sum1.tex
\section{Introduction}
%After the pandemic, it is crucial to sum up the lessons learned and evaluate the policies implemented during this time, including social distancing, nucleic acid testing, and quarantine measures. Our focus will be on analyzing the requirements for social distancing. By examining relevant policies and their specific implementations, we aim to gain valuable insights that can assist the government in providing tools and making informed decisions to mitigate the negative impacts of social distancing policies. Our work aims to provide the government with targeted decision-making support so that they can adopt the optimal social distancing measures based on actual circumstances. This approach will help protect public health while minimizing damage to the socioeconomic system. Furthermore, it will enhance the overall effectiveness of pandemic control and better prepare us for future public health crises.

Social distancing is a proven concept for containing the spread of an infectious disease. It has been widely adopted worldwide, for example, during the most recent Covid 19 pandemic. As a general principle, social distancing measures can be specified from different dimensions. The basic requirement of social distancing is the specification of a minimum physical distance between people in public areas. For example, the World Health Organization (WHO) suggests social distancing as to ``keep physical distance of at least 1 meter from others'' \cite{AdviceforPublic}. In the US, the Center for Disease and Control (CDC) refers to social distancing as ``keeping a safe space between yourself and other people who are not from your household'' \cite{CDC}. 
Note that under such a requirement, social distancing is actually applied with respect to groups of people. Similarly in Hong Kong, the government has adopted social distancing measures, in the recent Covid 19 pandemic, by limiting the size of groups in public gathering to two, four, and six people per group over time. Moreover, the Hong Kong government has also adopted an upper limit of the total number of people in a venue; for example, restaurants can operate at 50\% or 75\% of their normal seating capacity. 

It is unrealistic to keep locked down all the time. The government hopes to implement some policies to reopen, but the requirement of social distancing is not applicable to all places. Strict physical distance requirements make some cinemas with small row spacing and seat spacing only able to accommodate fewer customers, making the implementation of this policy impractical. Therefore, we hope to develop a policy that is easy to implement and satisfies both the government and businesses to meet requirements, and evaluate the impact of these strategies, in order to provide insights for policy implementation and business operations.

The implementation of social distancing measures has an extended impact beyond disease control.  In particular, social distancing may disrupt the usual operations in certain sectors. For example, a restaurant needs to change or redesign the layout of its tables in order to fulfill the requirement of social distancing. Such change implies smaller capacity, fewer customers and less revenue. In such a context, an effected firm faces a new operational problem of optimizing its operations flow under given social distancing policies.

The impact of enforcing social distancing measures on the economy is also an important factor for governmental decision making. Facing an outbreak of an infectious disease, a government shall declare a social distancing policy based on a holistic analysis, considering not only the severity of the outbreak, but also the potential impact on all stakeholders. What is particularly important is the level of business loss suffered by the industries that are directly affected.
 

% The measures implemented by the Hong Kong government primarily concentrate on restricting social distancing, group sizes and occupancy rates. However, implementing these policies in practice can pose challenges, particularly for fixed seating layouts with dynamic arrivals of people. In the original commercial case without social distancing requirements, customers did not need to sit together, so the focus was solely on total capacity. Under social distancing constraints, placing groups in row runs the risk of being unable to find matching demand, potentially leaving empty seats.

% The distancing measures discussed can also be applied to other domains, such as the arrangement of animal cages and the placement of GPUs. During customs quarantine, different types of animals from various regions need to maintain a certain distance to meet epidemic prevention requirements. Implementing a reasonable distance-based placement can ensure safety while not occupying an excessive amount of space. Similarly, with the rapid development of big data and artificial intelligence, a large number of GPUs are required as computational tools. GPU clustering can lead to overheating, which is detrimental to heat dissipation and affects operational efficiency. Arranging the placement of GPUs with appropriate distances can help address this issue and improve the overall system performance.



We will address the above issues of social distancing in the context of seating management. Consider  a venue, such as a cinema or a conference hall, which is to be used by an event. The venue is equipped with seats of multiple rows. In the event, requests for seats are in groups where each group contains a limited number of people who will sit consecutively in one row. Each row can accommodate multiple groups as long as any two adjacent groups in the same row are separated by one or multiple empty seats, as the requirement of the social distancing measures. The objective is to accept the  number of people as many as possible.

%[examples

%https://www.info.gov.hk/gia/general/202204/15/P2022041500474.htm

% https://www.nme.com/features/how-will-social-distancing-work-in-cinemas-2686328

% https://www.afterdigital.co.uk/insights-and-updates/blog/theatre-industry-coronavirus-future


We will consider three approaches in managing the seats, referred to as seat planning with deterministic requests, seat planning with stochastic requests, and seat assignment, respectively. Each of the above problems is a standalone problem with suitable situations. Together, these problems are inherently connected to each other, jointly forming a complete suite of solutions schemes for seating management under the social distancing constraints.


In seat assignment, seating requests arrive dynamically one by one, the problem is to assign seats, or reject the request, upon the arrival of each request. In other words, seat planning is an offline optimization problem, and seat assignment is an online optimization problem.

In seat planning with deterministic requests, we are given the complete information about seating requests in groups, and the problem is to find a partition of the layout into small segments, called a seating plan, to match the seating requests. Such a problem is applicable for cases of which participants and their groups are known, such as people from the same family in a church gathering, and staff from the same office in a company meeting.
 
In seat planning with stochastic requests, we  need to find a seating plan facing the requests in terms of a probabilistic distribution. This problem is for situations where a new layout needs to be made for serving multiple events with different seating requests. For example, there are theaters physically removing some seats during the Covid-19 outbreak \cite{Berlin_theater}.

Seat assignment can be used for those  commercial applications where requests arrive as a stochastic process, for example, tickets selling in movie theaters.

Seat planning and assignment Second, seat planning 

before rel;;;;;
where fixed seat layouts are available, and the goal is to assign seats to accommodate as many people as possible within the given layout. The seat planning obtained shows the utilization of all seats as many as possible. Thus, we introduce the concept of full or largest pattern to indicate the seat partition of each row. For the seat planning that does not utilize all available seats, we propose to improve the seat planning by incorporating full or largest patterns.


For the stochastic situation, we know that the demand distribution before the actual demand is realized. We aim to generate a seat planning that maximizes the expected number of people accommodated. This approach is suitable for venues where seats have been pre-allocated to ensure compliance with social distancing rules. With the given demand scenarios, we develop the scenario-based stochastic programming to obtain the seat planning. To solve this problem efficiently, we apply the Benders decomposition technique. However, in some cases, solving the integer programming with Benders decomposition remains still computationally prohibitive. Thus, we can consider the LP relaxation then obtain a feasible seat planning by deterministic model. Based on that, we construct a seat planning composed of full or largest patterns to utilize all seats fully.


% In order to adhere to social distancing guidelines, it is important to understand the process of generating seat planning based on known groups and how to assign seats to incoming groups. Additionally, it is of interest to explore how the social distancing constraints impact the sellers and the specific policies formulated by the government to address social distancing concerns.

We also address the dynamic seat assignment problem with a given set of seats in the context of social distancing. In the seat assignment, for the coming group, when accepting it, we assign the seats to the group, and the seats will not be used by others in the future. We intend to shed light on the problem described and propose a practical dynamic seat assignment policy. In particular, we investigate the following questions.

How to use the property of seat planning problem to design the dynamic seat assignment policy? How good is the performance of this policy compared with other policies?

What kind of insights regarding the social distancing and occupancy rates can we obtain when implementing the dynamic seat assignment policy?


% We also address the dynamic seat assignment problem with a given set of seats in the context of social distancing. Solving the problem by dynamic programming can be prohibitive due to the curse of dimensionality, which arises when the problem involves a large number of variables or states.


% We introduce the gap point to refer to the first period at which, on average, the number of people accepted without social distancing is not less than that accepted with social distancing plus one. By sampling many probability combinations, the results show that $\tilde{T}$ and the corresponding occupancy rate, $\beta(\tilde{T})$, can be estimated with $\gamma$, the expected number of people per period. Different $\gamma$ corresponds to different $\tilde{T}$. When the total number of periods, $T$, is less than $\tilde{T}$, we tend to accept all incoming groups. In this case, whether to implement the social distancing restriction is no difference. When $T$ is larger, there will be more groups rejected when implementing the social distancing requirement. The government can consider the potential losses when making policies regarding group size and occupancy rate. Similarly, the seller can implement corresponding measures to adhere to these requirements.

We have the following insights:

1.  With the requirement of social distancing, there is no need for the government to set a separate attendance rate. We set the gap point to represent the size of the demand. The gap point is specifically the number of periods, and the demand is the product of the gap point and the probability distribution of customer arrival. It can be estimated by the average number of customers at each period. And it has little to do with different probability distributions.

2. When the demand is smaller than the demand corresponding to the gap point, social distancing will not have an impact. The occupancy rate corresponding to the gap point is the maximum occupancy rate when there is no impact.

3.  For the choice of group size that allows sitting together, when the demand is high, we tend to allow more people to sit together, and when the demand is low, we can limit the number of groups that sit together to be smaller.

4. For the choice of social distancing, when the demand is low, two seats can be chosen as the social distance.

5.  Regarding the different layouts, we found that under the requirement of a single seat in social distancing, the occupancy rate of small venues is actually higher than that of large venues. When the total number of seats is the same, layouts with more rows have higher occupancy rates.

% 1. 有了social distancing 的要求,政府没必要单独设定上座率。我们设定了gap point 来表示需求的大小,当需求小于gap point 对应的需求时,social distancing 不会有影响。gap point 对应的occupancy rate 是无影响时的最大上座率。
% 2. gap point 具体来说是阶段数,需求是gap point 和顾客到来的概率分布的乘积。可以通过每个阶段的顾客数均值来进行估计。而跟不同的概率分布的关系不大。
% 3. 对于允许坐在一起的 group 大小的选择,当需求大时,我们倾向于允许更多的人坐在一起,当需求小时,可以限制坐在一起的group 更小。
% 4. 对于 social distancing 的选择,当需求较小时,可以选择两个座位作为社交距离。
% 5. 关于 不同的layout, 我们发现在social distancing 单个座位的要求下,小的场所上座率反而会比大的场所要高。在总座位数一样的情况下,排数较多的layout 上座率较高。

Our main contributions in this paper are summarized as follows. First, we establish a deterministic model to analyze the effects of social distancing when the demand is known. Due to the medium size of the problem, we can solve the IP model directly. We then develop the scenario-based stochastic programming by considering the stochastic demands of different group types. By using Benders decomposition methods, we can obtain the seat planning quickly. 

Second, we consider the seat planning problem with the stochastic demand which can guide to develop dynamic seat assignment policy to address the problem in the dynamic situation. The results demonstrate a significant improvement over the traditional control policies and provide insights on the implementation of social distancing.

Third, we present the first attempt to consider seat assignments with social distancing under dynamic arrivals. Our study offers a new perspective to assist the government in implementing a mechanism for seat assignments that promotes social distancing during a pandemic. When demand is low, simpler policies, such as first-come-first-served or bid-price strategies, can be effectively applied.
Our policy, DSA, demonstrates stable and consistent performance across different probability distributions. We also investigate the critical point where the social distancing constraint significantly impacts seat assignments. When demand exceeds the corresponding occupancy rate at this critical point, the differences become pronounced. Furthermore, we estimate both the critical point and the occupancy rate based on the average number of people arriving in each period.

% Insights?



1. 

The rest of this paper is structured as follows. The following section reviews relevant literature. We describe the motivating problem in Section 3. In Section 4, we establish the stochastic model, analyze its properties and obtain the seat planning. Section 5 introduces the dynamic seat assignment problem.
Section 6 demonstrates the dynamic seat assignment policy to assign the seats for incoming groups. Section 7 gives the numerical results and the insights of implementing social distancing. The conclusions are shown in Section 8.
\newpage
