% !TEX root = sum1.tex

{\bf Terminologies to use}

We use {\em seating management} to refer to the general problem which includes {\em seat planning with deterministic requests}, 
{\em seat planning with stochastic requests}, and {\em Seat Assignment.}

Each problem is defined for an {\em event} which has multiple {\em seating  requests}, where each request has a {\em group} of people to be seated.

\section{Introduction}
Social distancing is a proven concept for containing the spread of an infectious disease. It has been widely adopted worldwide, for example, during the most recent Covid 19 pandemic. As a general principle, social distancing measures can be specified from different dimensions. The basic requirement of social distancing is the specification of a minimum physical distance between people in public areas. For example, the World Health Organization (WHO) suggests social distancing as to ``keep physical distance of at least 1 meter from others'' \cite{AdviceforPublic}. In the US, the Center for Disease and Control (CDC) refers to social distancing as ``keeping a safe space between yourself and other people who are not from your household'' \cite{CDC}. 
Note that under such a requirement, social distancing is actually applied with respect to groups of people. Similarly in Hong Kong, the government has adopted social distancing measures, in the recent Covid 19 pandemic, by limiting the size of groups in public gatherings to two, four, and six people per group over time. Moreover, the Hong Kong government has also adopted an upper limit on the total number of people in a venue; for example, restaurants can operate at 50\% or 75\% of their normal seating capacity. 

The implementation of social distancing measures has an extended impact beyond disease control. In particular, social distancing may disrupt the usual operations in certain sectors. For example, a restaurant needs to change or redesign the layout of its tables in order to fulfill the requirement of social distancing. Such change implies smaller capacity, fewer customers and less revenue. In such a context, an effected firm faces a new operational problem of optimizing its operations flow under given social distancing policies.

The impact of enforcing social distancing measures on economic activities is also an important factor for governmental decision making. Facing an outbreak of an infectious disease, a government shall declare a social distancing policy based on a holistic analysis, considering not only the severity of the outbreak, but also the potential impact on all stakeholders. What is particularly important is the level of business loss suffered by the industries that are directly affected.  

However, the requirement of social distancing is not applicable to all places. Strict physical distance requirements make some cinemas with small row spacing and seat spacing only able to accommodate fewer customers, making the implementation of this policy impractical. Therefore, we hope to develop a policy that is easy to implement and satisfies both the government and businesses to meet requirements, and evaluate the impact of these strategies, in order to provide insights for policy implementation and business operations.


We will address the above issues of social distancing in the context of seating management. Consider a venue, such as a cinema or a conference hall, which is to be used in an event. The venue is equipped with seats of multiple rows. In the event, requests for seats are in groups where each group contains a limited number of people. Any group can be accepted or rejected, and the people in an accepted group  will sit consecutively in one row. Each row can accommodate multiple groups as long as any two adjacent groups in the same row are separated by one or multiple empty seats, as the requirement of the social distancing measures. The objective is to accept the number of individuals as many as possible.

We will consider three models for managing the seats, referred to as seat planning with deterministic requests, seat planning with stochastic requests, and seat assignment, respectively. As we elaborate below, each of these models defines a standalone problem with suitable situations. Together, they are inherently connected to each other, jointly forming a suite of solution schemes for seating management under the social distancing constraints.

In seat planning with deterministic requests, we are given the complete information about seating requests in groups, and the problem is to find a seating plan which specifies a partition of the layout into small segments to match the seating requests. Such a problem is applicable for cases of which participants and their groups are known, such as people from the same family in a church gathering, and staff from the same office in a company meeting. We formulate the problem by Integer Programming and discuss some characteristics of the optimal plan.
 
In seat planning with stochastic requests, we need to find a seating plan facing the requests in terms of a probabilistic distribution. This problem may find its applications in situations where a new layout needs to be made for serving multiple events with different seating requests. For example  \cite{Berlin_theater}, there are theaters physically removing some seats during the Covid-19 outbreak, where the remaining seats essentially form a seating plan with stochastic requests. We formulate the problem by scenario-based optimization and develop an algorithm by Benders decomposition.

In seat assignment, groups of seating requests arrive dynamically. The problem is to decide, upon the arrival of each group of request, whether to accept or reject the group, and assign seats for each accepted groups. Seat assignment can be used for those commercial applications where requests arrive as a stochastic process, for example, tickets selling in movie theaters.

The above three problems are closely related to each other with respect to problem solving methods and managerial insights. For example, in seat planning with deterministic requests, we identify some useful concepts such as the full patterns and largest patterns, which are important in the solution development for the other two problems. In addition, the duality analysis in the seat planning with deterministic requests facilitates the subproblem solving in the Benders decomposition algorithm for set planning with stochastic requests. Also, the solution of seat planning with stochastic requests can be used  as a reference seating plan in seat assignment.


Besides developing models and solution schemes for operational solutions satisfying  social distancing requirements, we are also interested in understanding the impact of social  distancing realized over  particular events. Note that although the seating capacity  is reduced by social distancing, this does not necessarily mean the same reduction of the number of people to be held for an event, especially when the event  needs a small number of seats. For example, consider a seating plan with 70 seats available in a venue of 100 seats, i.e., a 30\% reduction of the seating capacity. If an event held in the venue needs less than 70 seats, then it is possible that there will be a small number of people to be rejected, which implies that the loss caused by the social distancing is much less than 30\%. It is important for a government to include such an effect in policy making.

We address the above issue from the following aspects.

1. We introduce the concept of gap point to characterize the situations in which social distancing begins to cause loss to an event. Roughly speaking, given a distribution of the group size of each request, the gap point can be specified as an upper bound of the number of requests in an event such that if an event has fewer requests than the gap point, then the event will virtually not be affected by social distancing. Our computational experiments show that the gap point depends mainly on the mean of the group size, and relatively insensitive to its exact distribution. This offers an easy way to estimate the gap point and  the impact of social distancing.

% (different policies have different gap points, and our policy performs more stably.)

2. Our models and analysis are developed for the social distancing requirement on the physical distance and group size, where we can determine a target occupancy rate for any given event in a venue, and a maximum achievable occupancy rate for all events. Sometimes the government also imposes a maximum allowable occupancy rate to tighten the social distancing requirement. This maximum allowable rate is effective for an event if it is lower than the target occupancy rate of the event. Furthermore, the maximum allowable rate will be redundant if it is higher than the maximum achievable rate for all events.

3. The above qualitative insights are stable with respect to different parameters in the model, such as the layout of the venue, the maximum group sizes and the minimum physical distances. 

The rest of this paper is structured as follows. We review the relevant literature in Section 2. Then we introduce the major issues brought by social distancing and define the seating planning with deterministic requests in Section 3. In Section 4, we establish the stochastic model, analyze its properties and obtain the seat planning. Section 5 introduces the dynamic seat assignment problem.
Section 6 demonstrates the dynamic seat assignment policy to assign the seats for incoming groups. Section 7 gives the numerical results and the insights of implementing social distancing. The conclusions are shown in Section 8.
\newpage
