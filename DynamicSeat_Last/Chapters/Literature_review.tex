% !TEX root = sum1.tex
\section{Literature Review}\label{literature}

%\subsection{Seating Management with Social Distancing}
Seating management is a practical problem that presents unique challenges in various applications, each with its own complexities, particularly when accommodating group-based seating requests. For instance, in passenger rail services, groups differ not only in size but also in their departure and arrival destinations, requiring them to be assigned consecutive seats \citep{clausen2010off, deplano2019offline}. In social gatherings such as weddings or dinner galas, individuals often prefer to sit together at the same table while maintaining distance from other groups they may dislike \citep{lewis2016creating}. In parliamentary seating assignments, members of the same party are typically grouped in clusters to facilitate intra-party communication as much as possible \citep{vangerven2022parliament}. In e-sports gaming centers, customers arrive to play games in groups and require seating arrangements that allow them to sit together \citep{kwag2022optimal}.

Incorporating social distancing into seating management has introduced an additional layer of complexity, sparking a new stream of research. Some works focus on the layout design and determine seating positions to maximize physical distance between individuals, such as students in classrooms \citep{bortolete2022support} or customers in restaurants and beach umbrella arrangements \citep{fischetti2023safe}. Other works assume the seating layout is fixed, and assign seats to individuals while adhering to social distancing guidelines. For example, \citet{bortolete2022support} also consider the fixed seat setting for students in the classroom, \citet{salari2020social} consider the seat assignment in the airplanes. These studies consider the seating management with social distancing for the individual requests.


Our work relates to seating management with social distancing for group-based requests, which has found its applications in various areas, including single-destination public transits \citep{moore2021seat}, airplanes \citep{ghorbani2020model, salari2022social}, trains \citep{haque2022optimization, haque2023social}, and theaters \citep{blom2022filling}. Due to the diversity of applications, there are different issues to handle. For example, \citet{salari2022social} take the distance between different groups into account, leading to the development of a seating assignment strategy that outperforms the simplistic airline policy of blocking all middle seats. In \citet{haque2023social}, when designing seat allocation for groups with social distancing, was not only the transmission risk inside the train considered, but also the transmission risk between different cities where the stops were located. \citet{blom2022filling} address group-based seating problem in theaters. While they primarily focus on scenarios with known groups - referred to as seat planning with deterministic requests in our work - we consider a broader range of demand patterns. Specifically, we also examine group-based seat planning with stochastic requests. We also consider dynamic seat assignment, assuming that groups arrive sequentially according to a stochastic process.


Technically speaking, when all requests are known, SPDRP belongs to the category of the multiple knapsack problem (MKP) \citep{martello1990knapsack}. Most existing work focus on deriving bounds or competitive ratios for algorithms designed to solve the general multiple knapsack problem \citep{khuri1994zero, ferreira1996solving, pisinger1999exact, chekuri2005polynomial}. However, in our problem, the sizes of item weights, profits and knapsack capacities are all integers. Additionally, there are many identical groups of the same size, making the aggregation form particularly useful for determining the seat plan. Our work emphasizes the structure and properties of the solution to this specific problem, offering valuable insights for subsequent research in the dynamic situation.


% To address stochastic demand, we propose a scenario-based programming approach \cite{feng2013scenario, casey2005scenario, henrion2018problem} to determine the seat plan. Unlike existing literature, which typically considers the total supply for different demands or treats each type of supply as a protection level for corresponding demand types, we introduce a hierarchical consumption mechanism for the aggregated supply. Specifically, we adapt the constraints to reflect the hierarchical utilization of seats, where seats planned for larger groups can be repurposed for smaller groups. Additionally, the seat plan derived under demand uncertainty can serve as a reference for dynamic seat assignment, enhancing flexibility and effectiveness in real-time decision-making.


SADRP falls under the category of the dynamic multiple knapsack problem. A related problem is the dynamic stochastic knapsack problem, which has been extensively studied in the literature \citep{kleywegt1998dynamic, kleywegt2001dynamic, papastavrou1996dynamic}. In the dynamic stochastic knapsack problem, requests arrive sequentially, and their resource requirements and rewards are unknown prior to arrival but revealed upon arrival. In SADRP, requests also arrive sequentially; however, the sizes of resource requirements and rewards depend on the type of request and are revealed upon arrival. Additionally, SADRP involves multiple knapsacks, adding another layer of complexity.


There is limited research on the dynamic or stochastic multiple knapsack problem. \citep{perry2009approximate} employs multiple knapsacks to model multiple time periods for solving a multiperiod, single-resource capacity reservation problem. Essentially, it remains a dynamic knapsack problem but with time-varying capacity. \citep{tonissen2017column} considers a two-stage stochastic multiple knapsack problem with a set of scenarios where the capacity of the knapsacks can be subject to disturbances. This problem is similar to SPSRP in our work, where the number of items is stochastic.


Generally speaking, SADRP relates to the \textit{network revenue management} (NRM) problem, which focuses on deciding whether to accept or reject a request \citep{gallego1997multiproduct}. The NRM problem can be fully characterized by a dynamic programming (DP) formulation. However, a significant challenge arises because the number of states grows exponentially with the problem size, rendering direct solutions computationally infeasible. To address this, various approaches have been proposed, such as deriving bid-price or booking-limit controls from static formulations or approximating the value function using simplified structures. \citet{talluri1998analysis} was the first to propose bid-price control policies. Since then, a significant body of literature has focused on deriving refined bid prices and tighter bounds on value functions. In our work, we also consider a bid-price control policy, similar to the certainty equivalence control policy proposed by \citet{bertsimas2003revenue}, which directly uses the optimal value from a static model to approximate the initial value function. The seminal contribution to booking-limit control is from \citet{gallego1997multiproduct}, which studied a static model and introduced make-to-stock and make-to-order policies. However, these policies lack flexibility in handling stochastic demand.


Most of the work focuses on individual requests and booking decisions. The decisions in our problem must be made on an all-or-none basis for each request, which introduces additional complexity in handling group arrivals \citep{talluri2006theory}. 

The introduction of group-based characteristics complicates seat management. 

Notably, in our model, the supply planned for larger groups can also be utilized by smaller groups. This flexibility stems from our approach, which focuses on group arrivals rather than individual units, distinguishing it from traditional partitioned and nested approaches \citep{curry1990optimal, van2008simulation}.


Another key characteristic of our study is the importance of seat assignment, which distinguishes it from traditional revenue management. 





The assign-to-seat feature introduced by \citet{zhu2023assign} further emphasizes the significance of seat assignment. This approach tackles the challenge of selling high-speed train tickets, where each request must be assigned to a specific seat for the entire journey. However, this paper focuses on individual passengers rather than groups, which sets it apart from our research.


Similar group-based characteristics are also observed in multi-day stays in hotel revenue management \citep{aydin2018decomposition, bitran1995application}, 

% group-booking? still based on the quantity, don't consider the specific assignment



\newpage
