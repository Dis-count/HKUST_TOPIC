% !TEX root = sum1.tex
\section{Literature Review}\label{literature}
The present study is closely connected to the following research areas -- seat planning with social distancing and dynamic seat assignment. The subsequent sections review literature about each perspective and highlight significant differences between the present study and previous research.


\subsection{Seat Planning with Social Distancing}
Seating management is a practical problem that exists in many applications with different issues to handle, especially in the context of accommodating group-based seating requests.  For example, in  passenger rail services, work has been done on problems for maximizing capacity utilization or reducing the total capacity required \cite{clausen2010off, deplano2019offline}. In another example, such as weddings or dinner events \cite{lewis2016creating}, the main focus is on satisfying customer preferences and enhancing the overall experience. 
% Add more to specify the feature.

Including social distancing in seating management has added another dimension of consideration, forming a new stream of research. In some cases, it involves layout design, specifically, to determine the seating location within a given venue, for example, with the aim of maximizing the physical distancing between students in a classroom \cite{bortolete2022support}, and positioning tables in restaurants and beach umbrellas \cite{fischetti2023safe}. In other situations with predetermined seating layout, individuals are assigned seats while adhering to social distancing guidelines, for instance, problems in the air travel \cite{ghorbani2020model} and long-distance train travel \cite{haque2022optimization}. Such work highlights the relevance and importance in seating management with the consideration of social distancing.


Our work belongs to seating management with social ditancing for group-based requests, which has found its applications across various areas, including airplanes \cite{salari2022social}, trains \cite{haque2023social}, sports arenas \cite{kwag2022optimal}, and theaters \cite{blom2022filling}. Because of the diversity in applications, there are different issues to handle, for example, in \cite{salari2022social}, ... 

% 
% {\bf{what to emphasize?}}

% {\bf{why not mention group seat planning without social distancing}}
% Generally speaking, group seat reservation had two main applications in terms of revenue optimization and customer preference satisfaction. In the transportation domain, such as for passenger rail services, the primary focus of these studies was on maximizing capacity utilization or reducing the total capacity required \cite{clausen2010off, deplano2019offline}. Similar optimization techniques have also been applied in the context of seat planning for large events, such as weddings or dinner events \cite{lewis2016creating}. In these cases, the focus was on satisfying customer preferences and enhancing the overall experience.


Our work in \cite{blom2022filling} is the most related one to our research, both addressing group-based seating problem in theaters. In \cite{blom2022filling}, the primarily focuses on the cases with known groups, which is referred to as seat planning with deterministic requests in this paper, we have a broader scope. We also consider group-based seat planning with stochastic requests. Additionally, we incorporate dynamic seat assignment, assuming that groups arrive with a certain probability, to provide a comprehensive solution pattern.


% \cite{moore2021seat} 

% Some related literature mentioned the seat planning under pandemic for groups are represented below.
% Fischetti et al. \cite{fischetti2021safe} propose a seating planning for known groups of customers in amphitheaters. Haque and Hamid \cite{haque2022optimization} consider assigning seats to passengers in long-distance passenger trains. Salari et al. \cite{salari2022social} perform group seat assignment in airplanes during the pandemic and found that increasing passenger groups can yield greater social distancing than single passengers. Haque and Hamid \cite{haque2023social} aim to optimize seating assignments on trains by minimizing the risk of virus spread while maximizing revenue. The specific number of groups in their models is known in advance. Blom et al. \cite{blom2022filling} discuss strategies for filling a theater by considering the social distancing and group arrivals, which is similar to ours. 

% However, unlike our project, it only focuses on a specific location layout and it is still based on a static situation by giving the proportion of different groups.


% Dundar and Karakose \cite{dundar2021seat} proposed a two-stage algorithm for classroom seat assignment during the pandemic, with the first phase maximizing total allocations and the second phase maximizing the minimum interpersonal distance between students.

\subsection{Dynamic Seat Assignment}
%  我们的工作可以看成是 dynamic 
%  There little study, only one mildly related to our work, he mentioned something, 简略的说 到底哪里不一样?

%  Dynamic 与我们的哪里不一样?这些文献是怎么来的? 这里没有接受拒绝的问题。

%  我们的工作与group RM 有关  difffence from RM in two parts: 典型的RM decision on acceptance and rejection, 我们的 accept need to assign, 

%  在某些 RM 里面也研究了 Group , 但跟我们的概念不一样

%  Assignment: 方面 有火车 怎么不一样?

In dynamic seat assignment, the decision to either reject or accept-and-assign groups is made at each stage upon their arrival. This problem can be regarded as a special case of the dynamic multiple knapsack problem. When there is one row, the related problem is dynamic knapsack problem \cite{kleywegt1998dynamic}. Our model in its static form, deterministic request, can be viewed as a specific instance of the multiple knapsack problem \cite{pisinger1999exact}. There is little study, only one mildly related to the stochastic and dynamic multiple knapsack problem. It mentioned that {\bf{to be added}}


Dynamic seat assignment has applications in the transportation industry, including airplanes, trains, and buses \cite{hamdouch2011schedule, berge1993demand}. This process involves assigning seats to passengers in a manner that maximizes the efficiency and convenience of seating arrangements, without considering the acceptance or rejection of requests.

% We have two distinct features: group-based and assignment.

Our work is closely related to the group-based network revenue management (RM) problem \cite{williamson1992airline}, which focuses on accepting or rejecting a request \cite{gallego1997multiproduct}. One of the characteristics we are studying is that the decision should be made on an all-or-none basis for each group, which is the real complication in group arrivals \cite{talluri2006theory}. 

In hotel revenue management, group characteristics can also be observed in multi-day stays \cite{aydin2018decomposition, bitran1995application}, which differs from the concept of a group in our problem.

Another key characteristic of our study is the importance of seat assignment, which distinguishes it from traditional revenue management. The assign-to-seat feature introduced by Zhu et al. \cite{zhu2023assign} further emphasizes the significance of seat assignment. This approach tackles the challenge of selling high-speed train tickets, where each request must be assigned to a specific seat for the entire journey. However, this paper focuses on individual passengers rather than groups, which sets it apart from our research.


% Our work not only focuses on how to make the group-based seat assignment with non-reusable seats, also contribute the revenue insights with social distancing.


% This further emphasizes the significance of seat assignment and sets it apart from traditional revenue management methods.


% To address this challenge, we propose using scenario-based programming \cite{feng2013scenario, casey2005scenario, henrion2018problem} to determine the seat planning. In this approach, the aggregated supply can be considered as a protection level for each group type. Notably, in our model, the supply of larger groups can also be utilized by smaller groups. This is because our approach focuses on group arrival rather than individual unit, which sets it apart from traditional partitioned and nested approaches \cite{curry1990optimal, van2008simulation}.


% The authors propose a modified network revenue management model and introduce a bid-price control policy based on a novel maximal sequence principle. They also propose a "re-solving a dynamic primal" policy that achieves uniformly bounded revenue loss. The study reveals connections between this problem and traditional network revenue management problems and shows that the impact of the assign-to-seat restriction can be limited with the proposed methods.


% Jiang et al. \cite{jiang2015dynamic} proposes a revenue management approach for high-speed rail (HSR) passenger ticket assignment with dynamic adjustments. The approach integrates short-term demand forecasting, ticket assignment, and dynamic ticket adjustment mechanisms to allocate passenger tickets during presale periods and avoid situations where tickets are insufficient at some stations while seats remain empty.



% Implementing dynamic seat assignment with social distancing can be done manually by the staff or through automated systems that use algorithms to optimize the seat assignments based on various factors, such as ticket sales, seat availability, and customer preferences. However, the implementation of social distancing measures poses unique challenges that require careful planning and consideration of various factors to balance safety with revenue generation.


\newpage
