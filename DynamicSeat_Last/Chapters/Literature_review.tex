% !TEX root = sum1.tex
\section{Literature Review}\label{literature}
The present study is closely connected to the following research areas: seat management with social distancing, multiple knapsack problem and network revenue management. The subsequent sections review the literature on each perspective and highlight significant differences between the present study and previous research.


\subsection{Seating Management with Social Distancing}
Seating management is a practical problem that presents unique challenges in various applications, each with its own complexities, particularly when accommodating group-based seating requests. For instance, in passenger rail services, groups differ not only in size but also in their departure and arrival destinations, requiring them to be assigned consecutive seats \cite{clausen2010off, deplano2019offline}. In social gatherings such as weddings or dinner galas, individuals often prefer to sit together at the same table while maintaining distance from other groups they may dislike \cite{lewis2016creating}. In parliamentary seating assignments, members of the same party are typically grouped in clusters to facilitate intra-party communication as much as possible \cite{vangerven2022parliament}. In e-sports gaming centers, customers arrive to play games in groups and require seating arrangements that allow them to sit together \cite{kwag2022optimal}.

Incorporating social distancing into seating management has introduced an additional layer of complexity, sparking a new stream of research. In some studies, addressing social distancing involves optimizing the layout design itself. The focus is on determining seating positions within a given venue to maximize physical distance between individuals, such as students in classrooms \cite{bortolete2022support} or customers in restaurants and beach umbrella arrangements \cite{fischetti2023safe}. In other cases, where the seating layout is fixed, individuals are assigned seats while adhering to social distancing guidelines. Examples include problems in air travel \cite{ghorbani2020model} and long-distance train travel \cite{haque2022optimization}. These studies underscore the growing relevance and importance of seating management in the context of social distancing.


Our work relates to seating management with social distancing for group-based requests, which has found its applications in various areas, including airplanes \cite{salari2022social}, trains \cite{haque2023social}, and theaters \cite{blom2022filling}. Due to the diversity of applications, there are different issues to handle. For example, in \cite{salari2022social}, the distance between different groups is taken into account, leading to the development of a seating assignment strategy that outperforms the simplistic airline policy of blocking all middle seats. In \cite{haque2023social}, when designing seat allocation for groups with social distancing, not only was the transmission risk inside the train considered, but also the transmission risk between different cities where the stops were located.


Our work in this paper is most closely related to \cite{blom2022filling}, in that both addressing group-based seating problem in theaters. In \cite{blom2022filling}, they primarily focus on the cases with known groups, which is referred to as seat planning with deterministic requests in this paper, we have a broader scope. We also consider group-based seat planning with stochastic requests. Additionally, we incorporate dynamic seat assignment, assuming that groups arrive with a certain probability, to provide a comprehensive solution pattern.


\subsection{Multiple Knapsack Problem}
When all requests are known in SPDR, this problem belongs to the \textit{multiple knapsack problem} (MKP) \cite{ferreira1996solving, pisinger1999exact}, which has been studied with efficient algorithm. However, our problem focuses on the analysis of properties. The knapsacks can have different capacities and there are many identical groups of the same size, thus, the aggregation form would be useful to obtain the seat plan. Furthermore, seat planning to utilize all seats is a crucial aspect of our analysis, which shows that our proposed approach will be different from those in the literature.

% To address stochastic demand, we propose a scenario-based programming approach \cite{feng2013scenario, casey2005scenario, henrion2018problem} to determine the seat plan. Unlike existing literature, which typically considers the total supply for different demands or treats each type of supply as a protection level for corresponding demand types, we introduce a hierarchical consumption mechanism for the aggregated supply. Specifically, we adapt the constraints to reflect the hierarchical utilization of seats, where seats planned for larger groups can be repurposed for smaller groups. Additionally, the seat plan derived under demand uncertainty can serve as a reference for dynamic seat assignment, enhancing flexibility and effectiveness in real-time decision-making.


SADDP falls under the category of the dynamic multiple knapsack problem. When there is only one row, the problem reduces to the dynamic knapsack problem \cite{kleywegt1998dynamic}. However, there is limited research on the dynamic multiple knapsack problem, with only one mildly related study \cite{perry2009approximate}. This study models a multiperiod, single-resource capacity reservation problem by using multiple knapsacks to represent multiple time periods, so that this paper does not apply to ours.

% differs from our focus on the realistic multiple knapsack problem in seat assignment.

\subsection{Network Revenue Management}
Our work is also related to the group-based \textit{network revenue management} (NRM) problem, which focuses on deciding whether to accept or reject a request \cite{gallego1997multiproduct}. The NRM problem can be fully characterized by a dynamic programming (DP) formulation. However, a significant challenge arises because the number of states grows exponentially with the problem size, rendering direct solutions computationally infeasible. To address this, various approaches have been proposed, such as deriving bid-price or booking-limit controls from static formulations or approximating the value function using simplified structures.

\cite{talluri1998analysis} was the first to propose the bid-price control policy. Since then, a significant body of literature has focused on deriving refined bid prices and tighter bounds on value functions. In our work, we also consider a bid-price control policy, similar to the certainty equivalence control policy proposed by \cite{bertsimas2003revenue}, which directly uses the optimal value from a static model to approximate the initial value function. The seminal contribution to booking-limit control is from \cite{gallego1997multiproduct}, which studied a static model and introduced make-to-stock and make-to-order policies. However, these policies lack flexibility in handling stochastic demand and do not perform well for the group-based problem.


One of the key characteristics of our problem is that decisions must be made on an all-or-none basis for each group, which introduces additional complexity in handling group arrivals \cite{talluri2006theory}. Similar group-based characteristics are also observed in multi-day stays in hotel revenue management \cite{aydin2018decomposition, bitran1995application}, though the concept of a ``group'' in those contexts differs from our problem.


% This flexibility arises because our approach focuses on group arrivals rather than individual units, distinguishing it from traditional partitioned and nested allocation methods \cite{curry1990optimal, van2008simulation}.

The introduction of group-based characteristics complicates seat management when using traditional bid-price and booking-limit control policies. Notably, in our model, the supply planned for larger groups can also be utilized by smaller groups. This flexibility stems from our approach, which focuses on group arrivals rather than individual units, distinguishing it from traditional partitioned and nested approaches \cite{curry1990optimal, van2008simulation}.


% This is because our approach focuses on group arrival rather than individual unit, which sets it apart from traditional partitioned and nested approaches \cite{curry1990optimal, van2008simulation}.


Another key characteristic of our study is the importance of seat assignment, which distinguishes it from traditional revenue management. The assign-to-seat feature introduced by Zhu et al. \cite{zhu2023assign} further emphasizes the significance of seat assignment. This approach tackles the challenge of selling high-speed train tickets, where each request must be assigned to a specific seat for the entire journey. However, this paper focuses on individual passengers rather than groups, which sets it apart from our research.


% Our work not only focuses on how to make the group-based seat assignment with non-reusable seats, also contribute the revenue insights with social distancing.

% This further emphasizes the significance of seat assignment and sets it apart from traditional revenue management methods.


% The authors propose a modified network revenue management model and introduce a bid-price control policy based on a novel maximal sequence principle. They also propose a "re-solving a dynamic primal" policy that achieves uniformly bounded revenue loss. The study reveals connections between this problem and traditional network revenue management problems and shows that the impact of the assign-to-seat restriction can be limited with the proposed methods.


% Jiang et al. \cite{jiang2015dynamic} proposes a revenue management approach for high-speed rail (HSR) passenger ticket assignment with dynamic adjustments. The approach integrates short-term demand forecasting, ticket assignment, and dynamic ticket adjustment mechanisms to allocate passenger tickets during presale periods and avoid situations where tickets are insufficient at some stations while seats remain empty.

% Implementing dynamic seat assignment with social distancing can be done manually by the staff or through automated systems that use algorithms to optimize the seat assignments based on various factors, such as ticket sales, seat availability, and customer preferences. However, the implementation of social distancing measures poses unique challenges that require careful planning and consideration of various factors to balance safety with revenue generation.


\newpage
