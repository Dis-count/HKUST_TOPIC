% !TEX root = sum1.tex
\section{Literature Review}

The present study is closely connected to the following research areas -- seat planning with social distancing and dynamic seat assignment. The subsequent sections review literature about each perspective and highlight significant differences between the present study and previous research.


\subsection{Seat Planning with Social Distancing}
Social distancing in seat planning has attracted considerable attention from the research area. The applications include the allocation of seats on airplanes \cite{ghorbani2020model}, classroom layout planning \cite{bortolete2022support}, seat planning in long-distancing trains \cite{haque2022optimization}. There are two basic forms: fixed seat layout and determination of the layout. Fischetti et al.\cite{fischetti2021safe} consider how to plant positions with social distancing in restaurants and beach umbrellas. On an airplane, the distancing between seats and the aisle must be considered \cite{salari2022social}, while in a classroom, maximizing social distancing between students is a priority \cite{bortolete2022support}. 


The seat allocation of the static form is useful for fixed people, for example, the students in one class. In our specific setting, we not only consider the static problem, but analyze the seat assignment in the dynamic situation.


The recent pandemic has shed light on the benefits of group reservations, as they have been shown to increase revenue without increasing the risk of infection \cite{moore2021seat}. The group seat reservation has various applications in industries such as working spaces\cite{fischetti2021safe}, public transport\cite{deplano2019offline}, sports arenas\cite{kwag2022optimal}, and large-scale events \cite{lewis2016creating}. Traditional works \cite{clausen2010off, deplano2019offline}in transportation focus on maximizing capacity utilization or reducing total capacity needed for passenger rail, typically modeling these problems as knapsack or bin-packing problems.


% However, the group seat reservation policy poses a significant challenge when it comes to determining the seat assignment policy.

% This policy has significant impacts on passenger satisfaction and revenue, with the study \cite{yuen2002group} showing that passenger groups increase revenue by filling seats that would otherwise be empty. 


Some related literature mentioned the seat planning under pandemic for groups are represented below.
Fischetti et al. \cite{fischetti2021safe} propose a seating planning for known groups of customers in amphitheaters. Haque and Hamid \cite{haque2022optimization} consider assigning seats to passengers in long-distance passenger trains. Salari et al. \cite{salari2022social} perform group seat assignment in airplanes during the pandemic and found that increasing passenger groups can yield greater social distancing than single passengers. Haque and Hamid \cite{haque2023social} aim to optimize seating assignments on trains by minimizing the risk of virus spread while maximizing revenue. The specific number of groups in their models is known in advance. Blom et al. \cite{blom2022filling} discuss strategies for filling a theater by considering the social distancing and group arrivals, which is similar to ours. 

The major difference in our study is that we require that groups be accepted on an all-or-none basis, meaning that members of the same family or group must be seated together. We give the seat assignment assuming the groups arrive in a certain probability.

% However, unlike our project, it only focuses on a specific location layout and it is still based on a static situation by giving the proportion of different groups.



% In contrast, there is a lack of research on group seat reservations for booking tickets for cinemas, where available seats are typically displayed for customers to choose from for low-demand movie tickets. For concerts with high demand, it is usually not possible to choose seats independently, and the organizer will inform seat information after confirming the order. 
% For movies of the same time period, the ticket prices are the same, while for the same concert, although there are different ticket prices, for the same region, the ticket prices are the same. Therefore, we can consider different ticket prices separately. 
% In the absence of an epidemic, all requests for tickets can be considered one by one. However, the COVID-19 pandemic has shed new light on the potential benefits of group reservations, as they can improve revenue without increasing the risk of infection. 


% Dundar and Karakose \cite{dundar2021seat} proposed a two-stage algorithm for classroom seat assignment during the pandemic, with the first phase maximizing total allocations and the second phase maximizing the minimum interpersonal distance between students. 


%  where we aim to assign a subset of groups to some distinct rows.

\subsection{Dynamic Seat Assignment}
In dynamic seat assignment, the decision to accept or reject groups is made at each stage as they arrive. When there is one row, the related problem is dynamic knapsack problem \cite{kleywegt1998dynamic}. Our model in its static form can be viewed as a specific instance of the multiple knapsack problem \cite{pisinger1999exact}. To the best of our knowledge, there are no existing studies that have specifically addressed the dynamic multiple knapsack problem. Dynamic seat assignment is a process of assigning seats to passengers on a transportation vehicle, such as an airplane, train, or bus, in a way that maximizes the efficiency and convenience of the seating arrangements \cite{hamdouch2011schedule, berge1993demand, zhu2023assign}. 

% and dynamic bin-packing problem \cite{coffman1983dynamic, berndt2020fully} where items arrive and depart online and all knapsacks are the same. 

% However, solving this problem is strongly NP-hard, which means that finding an optimal solution for large instances of the problem is computationally challenging \cite{pisinger1999exact}.

Our problem is closely related to the network revenue management (RM) problem \cite{williamson1992airline}, which is typically formulated as a dynamic programming (DP) problem. However, for large-scale problems, the exponential growth of the state space and decision set makes the DP approach computationally intractable. 


One of the characteristics we are studying is that the decision should be made on an all-or-none basis for each group, which is the real complication in group arrivals \cite{talluri2006theory}. The group trait reflected in the RM is the hotel revenue management considering multiple day stays \cite{aydin2018decomposition, bitran1995application}. 


% We have two distinct features: group-based and assignment.

Another characteristic in our study is the significance of seat assignment. This sets it apart from traditional revenue management focusing on decision-making issues, namely accepting or rejecting a request \cite{gallego1997multiproduct}. Similarly, the assign-to-seat feature introduced by Zhu et al. \cite{zhu2023assign} also highlights the importance of seat assignment in revenue management. This trait addresses the challenge of selling high-speed train tickets in China, where each request must be assigned to a single seat for the entire journey and takes into account seat reuse. This work focuses on the individual seat assignment while our work relies on the group-based setting.


% which makes the problem more challenging.


% This further emphasizes the significance of seat assignment and sets it apart from traditional revenue management methods.


% To address this challenge, we propose using scenario-based programming \cite{feng2013scenario, casey2005scenario, henrion2018problem} to determine the seat planning. In this approach, the aggregated supply can be considered as a protection level for each group type. Notably, in our model, the supply of larger groups can also be utilized by smaller groups. This is because our approach focuses on group arrival rather than individual unit, which sets it apart from traditional partitioned and nested approaches \cite{curry1990optimal, van2008simulation}.


% The authors propose a modified network revenue management model and introduce a bid-price control policy based on a novel maximal sequence principle. They also propose a "re-solving a dynamic primal" policy that achieves uniformly bounded revenue loss. The study reveals connections between this problem and traditional network revenue management problems and shows that the impact of the assign-to-seat restriction can be limited with the proposed methods.


% Jiang et al. \cite{jiang2015dynamic} proposes a revenue management approach for high-speed rail (HSR) passenger ticket assignment with dynamic adjustments. The approach integrates short-term demand forecasting, ticket assignment, and dynamic ticket adjustment mechanisms to allocate passenger tickets during presale periods and avoid situations where tickets are insufficient at some stations while seats remain empty.



% Implementing dynamic seat assignment with social distancing can be done manually by the staff or through automated systems that use algorithms to optimize the seat assignments based on various factors, such as ticket sales, seat availability, and customer preferences. However, the implementation of social distancing measures poses unique challenges that require careful planning and consideration of various factors to balance safety with revenue generation.


% \subsection{Scenario generation}

% It is challenging to consider all the possible realizations; thus, it is practicable to use discrete distributions with a finite number of scenarios to approximate the random demands. This procedure is often called scenario generation.

% Some papers consider obtaining a set of scenarios that realistically represents the distributions of the random parameters but is not too large. \cite{feng2013scenario} \cite{casey2005scenario}
% \cite{henrion2018problem}

% Another process to reduce the calculation is called scenario reduction. It tries to approximate the original scenario set with a smaller subset that retains essential features.



% Every time we can regenerate the scenario based on the realized demands. (Use the conditional distribution or the truncated distribution)


% Suppose that the groups arrive from small to large according to their size. Once a larger group comes, the smaller one will never appear again.

% When a new group arrives (suppose we have accepted $n$ groups with the same size), we accept or reject it according to the supply (when $n+1 < \text{supply}$, we accept it). 
% then update the scenario set according to the truncated distribution. We can obtain a new supply with the new probability and scenario set.


\newpage
