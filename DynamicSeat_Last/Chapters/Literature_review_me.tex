% !TEX root = sum1.tex
\section{Literature Review}
The present study is closely connected to the following research areas -- seat planning with social distancing and dynamic seat assignment. The subsequent sections review literature about each perspective and highlight significant differences between the present study and previous research.

% Seating management is practical problem and has many different application, 例如
%  火车上,怎么样增加 capacity, 尤其是 group-based, 大的活动上,

% social distancing 增加了另外一个维度上的考虑 使其更加复杂, 

%  不固定 layout design 

%  给定layout  怎么给座位  没有考虑 group 要求   个人的

%  group-based seat planning, 飞机 火车, 

% 特别是,电影院,social distance group 来体现,静态的palnning 问题,是我们研究的一小部分,我们研究了 stochastic planning 和 动态的  给出了更完整的解决方案。

\subsection{Seat Planning with Social Distancing}
Seating management is a practical problem and has many different applications, for instance, increasing capacity in the trains, group-based seating arrangements for the special events. Social distancing introduced has added another dimension of consideration, making it more complicated. It involves various applications to address the challenges posed by the need for physical distancing in different settings.


In the context of layout design, specifically determining the seating location within a given venue, there are various applications. For example, maximizing the social distancing between students in a classroom is discussed in \cite{bortolete2022support}. Fischetti et al.\cite{fischetti2023safe} consider how to plant positions with social distancing and apply these principles in restaurants and beach umbrellas. In other scenarios where the layout is predetermined, individuals are assigned seats while adhering to social distancing guidelines. For instance, seat planning within a fixed layout is addressed in the literature related to air travel \cite{ghorbani2020model, salari2022social}. Additionally, the optimization of seating arrangements has been explored in the context of long-distance train travel \cite{haque2022optimization}.

Generally speaking, group seat reservation had two main applications in terms of revenue optimization and customer preference satisfaction. In the transportation domain, such as for passenger rail services, the primary focus of these studies was on maximizing capacity utilization or reducing the total capacity required \cite{clausen2010off, deplano2019offline}. Similar optimization techniques have also been applied in the context of seat planning for large events, such as weddings or dinner events \cite{lewis2016creating}. In these cases, the focus was on satisfying customer preferences and enhancing the overall experience.

The pandemic has heightened the emphasis on group-based seat planning, particularly in light of social distancing mandates. It has highlighted the potential benefits of group reservations in mitigating the spread of infections, as they can be designed to enhance revenue without increasing transmission risks. The challenge of group seat reservations has become more complex due to social distancing constraints, leading to diverse applications across various industries, including airplanes \cite{salari2022social}, trains \cite{haque2023social}, sports arenas \cite{kwag2022optimal}, and theaters \cite{blom2022filling}.

It is important to note that the scenario presented in \cite{blom2022filling} is similar to the problem we are addressing, which aims to maximize the number of individuals accommodated. However, it primarily focuses on a known fixed proportion of groups for seat planning, which is just one aspect of our work. We also consider group-based seat planning with stochastic requests. Additionally, we incorporate dynamic seat assignment, assuming that groups arrive with a certain probability, to provide a comprehensive solution pattern.


% The recent pandemic has shed light on the benefits of group reservations, it brings more thought to the group seat reservation, it becomes more important and more complex with the social distancing constraint. As they have been shown to increase revenue without increasing the risk of infection, The group seat reservation with social distancing has various applications in industries such as airplanes \cite{salari2022social}, trains \cite{salari2022social, haque2023social}, sports arenas\cite{kwag2022optimal}, theaters \cite{blom2022filling}.

% \cite{moore2021seat} 

% Some related literature mentioned the seat planning under pandemic for groups are represented below.
% Fischetti et al. \cite{fischetti2021safe} propose a seating planning for known groups of customers in amphitheaters. Haque and Hamid \cite{haque2022optimization} consider assigning seats to passengers in long-distance passenger trains. Salari et al. \cite{salari2022social} perform group seat assignment in airplanes during the pandemic and found that increasing passenger groups can yield greater social distancing than single passengers. Haque and Hamid \cite{haque2023social} aim to optimize seating assignments on trains by minimizing the risk of virus spread while maximizing revenue. The specific number of groups in their models is known in advance. Blom et al. \cite{blom2022filling} discuss strategies for filling a theater by considering the social distancing and group arrivals, which is similar to ours. 


% However, unlike our project, it only focuses on a specific location layout and it is still based on a static situation by giving the proportion of different groups.

% In contrast, there is a lack of research on group seat reservations for booking tickets for cinemas, where available seats are typically displayed for customers to choose from for low-demand movie tickets. For concerts with high demand, it is usually not possible to choose seats independently, and the organizer will inform seat information after confirming the order. 
% For movies of the same time period, the ticket prices are the same, while for the same concert, although there are different ticket prices, for the same region, the ticket prices are the same. Therefore, we can consider different ticket prices separately. 
% In the absence of an epidemic, all requests for tickets can be considered one by one. However, the COVID-19 pandemic has shed new light on the potential benefits of group reservations, as they can improve revenue without increasing the risk of infection. 


% Dundar and Karakose \cite{dundar2021seat} proposed a two-stage algorithm for classroom seat assignment during the pandemic, with the first phase maximizing total allocations and the second phase maximizing the minimum interpersonal distance between students.


\subsection{Dynamic Seat Assignment}
%  我们的工作可以看成是 dynamic 
%  There little study, only one mildly related to our work, he mentioned something, 简略的说 到底哪里不一样?

%  Dynamic 与我们的哪里不一样?这些文献是怎么来的? 这里没有接受拒绝的问题。

%  我们的工作与group RM 有关  difffence from RM in two parts: 典型的RM decision on acceptance and rejection, 我们的 accept need to assign, 

%  在某些 RM 里面也研究了 Group , 但跟我们的概念不一样

%  Assignment: 方面 有火车 怎么不一样?

In dynamic seat assignment, the decision to either reject or accept-and-assign groups is made at each stage upon their arrival. This problem can be regarded as a special case of the dynamic multiple knapsack problem. When there is one row, the related problem is dynamic knapsack problem \cite{kleywegt1998dynamic}. Our model in its static form, deterministic request, can be viewed as a specific instance of the multiple knapsack problem \cite{pisinger1999exact}. There is little study, only one mildly related to the stochastic and dynamic multiple knapsack problem. It mentioned that {\bf{to be added}}


Dynamic seat assignment has applications in the transportation industry, including airplanes, trains, and buses \cite{hamdouch2011schedule, berge1993demand}. This process involves assigning seats to passengers in a manner that maximizes the efficiency and convenience of seating arrangements, without considering the acceptance or rejection of requests.

% We have two distinct features: group-based and assignment.

Our work is closely related to the group-based network revenue management (RM) problem \cite{williamson1992airline}, which focuses on accepting or rejecting a request \cite{gallego1997multiproduct}. One of the characteristics we are studying is that the decision should be made on an all-or-none basis for each group, which is the real complication in group arrivals \cite{talluri2006theory}. 

In hotel revenue management, group characteristics can also be observed in multi-day stays \cite{aydin2018decomposition, bitran1995application}, which differs from the concept of a group in our problem.

Another key characteristic of our study is the importance of seat assignment, which distinguishes it from traditional revenue management. The assign-to-seat feature introduced by Zhu et al. \cite{zhu2023assign} further emphasizes the significance of seat assignment. This approach tackles the challenge of selling high-speed train tickets, where each request must be assigned to a specific seat for the entire journey. However, this paper focuses on individual passengers rather than groups, which sets it apart from our research.


% Our work not only focuses on how to make the group-based seat assignment with non-reusable seats, also contribute the revenue insights with social distancing.

% This further emphasizes the significance of seat assignment and sets it apart from traditional revenue management methods.


% To address this challenge, we propose using scenario-based programming \cite{feng2013scenario, casey2005scenario, henrion2018problem} to determine the seat planning. In this approach, the aggregated supply can be considered as a protection level for each group type. Notably, in our model, the supply of larger groups can also be utilized by smaller groups. This is because our approach focuses on group arrival rather than individual unit, which sets it apart from traditional partitioned and nested approaches \cite{curry1990optimal, van2008simulation}.


% The authors propose a modified network revenue management model and introduce a bid-price control policy based on a novel maximal sequence principle. They also propose a "re-solving a dynamic primal" policy that achieves uniformly bounded revenue loss. The study reveals connections between this problem and traditional network revenue management problems and shows that the impact of the assign-to-seat restriction can be limited with the proposed methods.


% Jiang et al. \cite{jiang2015dynamic} proposes a revenue management approach for high-speed rail (HSR) passenger ticket assignment with dynamic adjustments. The approach integrates short-term demand forecasting, ticket assignment, and dynamic ticket adjustment mechanisms to allocate passenger tickets during presale periods and avoid situations where tickets are insufficient at some stations while seats remain empty.



% Implementing dynamic seat assignment with social distancing can be done manually by the staff or through automated systems that use algorithms to optimize the seat assignments based on various factors, such as ticket sales, seat availability, and customer preferences. However, the implementation of social distancing measures poses unique challenges that require careful planning and consideration of various factors to balance safety with revenue generation.

\newpage
