% !TEX root = sum1.tex
\clearpage
\section{Proofs}

\begin{pf}{Proof of Proposition \ref{sol_relax_deter}}
  We model the problem as a special case of the multiple knapsack problem, then we consider the LP relaxation of this problem. In the model, groups are categorized into $M$ distinct types. Each type $i$ is characterized by a fixed group size $n_i$, which serves as the weight, and an associated profit equal to $i$. For every type $i$, there are $d_i$ groups. Altogether, the total number of groups is given by $K = \sum_{i=1}^{M} d_i$. Each individual group $k$ inherits its profit and weight from its type; specifically, if group $k$ belongs to type $i$, then its profit $p_k$ is $i$, and its weight $w_k$ is $n_i$. To apply the greedy approach for the LP relaxation of \eqref{deter_upper}, sort these groups in non-increasing order of their profit-to-weight ratios: $\frac{p_1}{w_1} \geq \frac{p_2}{w_2} \geq \ldots \geq \frac{p_K}{w_K}$. The break group $b$ is the smallest index such that the cumulative weight of group 1 to group $b$ meets or exceeds the total capacity $\tilde{L}$: $b=\min \{j: \sum_{k=1}^j w_k \geq \tilde{L}\}$, where $\tilde{L} = \sum_{j=1}^{N} L_j$ is the total size of all rows. For the LP relaxation of \eqref{deter_upper}, the Dantzig upper bound \citep{dantzig1957discrete} is given by $u_{\mathrm{MKP}}=\sum_{j=1}^{b-1} p_j+\left(\tilde{L}-\sum_{j=1}^{b-1} w_j\right) \frac{p_b}{w_b}$. The corresponding optimal solution is to accept the whole groups from $1$ to $b-1$ and fractional $(\tilde{L}-\sum_{j=1}^{b-1} w_j)$ group $b$. Suppose the group $b$ belong to type $\tilde{i}$, then for $i < \tilde{i}$, $x_{ij}^{*} = 0$; for $i > \tilde{i}$, $x_{ij}^{*} = d_{i}$; for $i = \tilde{i}$, $\sum_{j=1}^{N} x_{ij}^{*} = (\tilde{L} - \sum_{i = \tilde{i}+1}^{M} {d_i n_i})/ n_{\tilde{i}}$.

  % There are $M$ group types. Each type $i$ has a fixed size $n_i$ (weight of the group), a profit $i$. Each type $i$ has $d_i$ groups. The total number of groups is $K = \sum_{i=1}^{M} d_i$, each group $k$ has a profit $p_k$ and weight $w_k$. If group $k$ belongs to type $i$, then $p_k = i$, $w_k = n_i$.

  % Treat the groups as the items, the rows as the knapsacks. There are $M$ types of items, the total number of which is $K = \sum_{i=1}^{M} d_i$, each item $k$ has a profit $p_k$ and weight $w_k$. 
  % Sort these items according to profit-to-weight ratios $\frac{p_1}{w_1} \geq \frac{p_2}{w_2} \geq \ldots \geq \frac{p_K}{w_K}$. Let the break item $b$ be given by $b=\min \{j: \sum_{k=1}^j w_k \geq \tilde{L}\}$, where $\tilde{L} = \sum_{j=1}^{N} L_j$ is the total size of all knapsacks. 


\end{pf}


\begin{pf}{Proof of Proposition \ref{lem_pattern}}
First, we construct a feasible pattern with the size of $qM + \max\{r-\delta, 0\}$, then we prove this pattern is largest. Let $L = n_M \cdot q + r$, where $q$ represents the number of times $n_M$ is selected (the quotient), and $r$ represents the remainder, indicating the number of remaining seats. It holds that $0 \leq r < n_M$. The number of individuals accommodated in the pattern $\bm{h}_{g}$ is given by $|\bm{h}_{g}| = q M + \max\{r-\delta, 0\}$. To establish the optimality of $|\bm{h}_{g}|$ as the largest number of individuals accommodated given the constraints of $L$, $\delta$, and $M$, we can employ a proof by contradiction.

Assuming the existence of a pattern $\bm{h}$ such that $|\bm{h}| > |\bm{h}_{g}|$, we can derive the following inequalities:

\begin{align*}
  & \sum_{i=1}^{M} (n_i - \delta) h_i > q M + \max\{r-\delta, 0\} \\
  \Rightarrow ~& L \geq \sum_{i=1}^{M} n_i h_i > \sum_{i=1}^{M} \delta h_i + q M + \max\{r-\delta, 0\} \\
  \Rightarrow ~& q(M + \delta) + r > \sum_{i=1}^{M} \delta h_i + q M + \max\{r-\delta, 0\} \\
  \Rightarrow ~& q \delta + r > \sum_{i=1}^{M} \delta h_i + \max\{r-\delta, 0\}
\end{align*}

\begin{enumerate}[(i)]
  \item When $r > \delta$, the inequality becomes $q+1 > \sum_{i=1}^{M} h_i$. It should be noted that $h_i$ represents the number of group type $i$ in the pattern. Since $\sum_{i=1}^{M} h_i \leq q$, the maximum number of individuals that can be accommodated is $q M < q M + r-\delta$.  
  \item When $r \leq \delta$, we have the inequality $q \delta + \delta \geq q \delta + r > \sum_{i=1}^{M} \delta h_i$. Similarly, we obtain $q+1 > \sum_{i=1}^{M} h_i$. Thus, the maximum number of individuals that can be accommodated is $q M$, which is not greater than $|\bm{h}_{g}|$.  
\end{enumerate}

Therefore, $\bm{h}$ cannot exist. The maximum number of individuals that can be accommodated in the largest pattern is $q M + \max\{r-\delta, 0\}$.

Additionally, any largest pattern $\bm{h}$ under $M$ remains feasible for $(M+1)$, implying that $\phi(M, L^{0}, \delta) \leq \phi(M+1, L^{0}, \delta)$. Similarly, any largest pattern $\bm{h}$ under $L$ is a feasible pattern under $L+1$, thus, $\phi(M, L^{0}, \delta) \leq \phi(M, L^{0}+1, \delta)$. Any largest pattern $\bm{h}$ under $\delta+1$ is a feasible pattern under $\delta$, thus, $\phi(M, L^{0}, \delta+1) \leq \phi(M, L^{0}, \delta)$.
So, $\phi(M, L^{0}, \delta)$ is non-decreasing with $M$ and $L^{0}$, and non-increasing with $\delta$, respectively.

\end{pf}

\begin{pf}{Proof of Proposition \ref{prop_construction}}
  First of all, we demonstrate the feasibility of problem \eqref{improve_seat}. Given the feasible seat plan $\bm{H}$ and $\tilde{d}_{i} = \sum_{j=1}^{N} H_{ij}$, let $\hat{x}_{ij} = H_{ij}, i \in \mathcal{M}, j \in \mathcal{N}$, then $\{\hat{x}_{ij}\}$ satisfies the first set of constraints. Because $\bm{H}$ is feasible, $\{\hat{x}_{ij}\}$ satisfies the second set of constraints and integer constraints. Thus, problem \eqref{improve_seat} always has a feasible solution. 
  
  Suppose there exists at least one pattern $\bm{h}$ is neither full nor largest in the optimal seat plan obtained from problem \eqref{improve_seat}. Let $s = L - \sum_{i=1}^{M} n_{i} h_{i}$, and denote the smallest group type in pattern $\bm{h}$ by $k$. If $s \geq n_1$, we can assign at least $n_1$ seats to a new group to increase the objective value. Thus, we consider the situation when $s < n_1$. If $k =M$, then this pattern is largest. When $k< M$, let $h^{1}_{k} = h_{k} -1$ and $h^{1}_{j} = h_{j} +1$, where $j = \min\{M, s + k\}$. In this way, the constraints will still be satisfied but the objective value will increase when the pattern $\bm{h}$ changes. Therefore, by contradiction, problem \eqref{improve_seat} always generate a seat plan composed of full or largest patterns.
\end{pf}


\begin{pf}{Proof of Proposition \ref{prop_solution}}
Suppose that $\bm{H}$ is the seat plan associated with the optimal solution to SBSP, but there exists a pattern that is neither full nor the largest. The corresponding excess of supply is $\mathbf{y}^{+}$. According to Proposition \ref{prop_construction}, $\bm{H}{'}$ can be obtained from $\bm{H}$. The seat plan, $\bm{H}{'}$, is composed of full or largest patterns and satisfies all constraints of SBSP. The corresponding excess of supply is $\mathbf{y}^{'+}$.

Then we will demonstrate that for each scenario $\omega$, the objective function of SBSP, given by $\sum_{j =1}^{N} \sum_{i=1}^M i \cdot x_{ij} - \sum_{i = 1}^{M} y_{i \omega}^{+}$, does not decrease when transitioning from $\bm{H}$ to $\bm{H}{'}$.

Let $\Delta y_{M \omega}^{+} = y_{M \omega}^{'+} - y_{M \omega}^{+}$, $\Delta \sum_{j=1}^{N} x_{Mj} = \sum_{j=1}^{N} x_{Mj}^{'} - \sum_{j=1}^{N} x_{Mj}$. According to \eqref{y_recursively}, when $i$ changes from $M$ to 1, we obtain the following inequalities.

\begin{align*}
  \Delta y_{M \omega}^{+} \geq &~ \Delta \sum_{j=1}^{N} x_{Mj} \\
  \Delta y_{M-1, \omega}^{+} \geq &~ \Delta y_{M \omega}^{+} + \Delta \sum_{j=1}^{N} x_{M-1,j} \geq \Delta \sum_{j=1}^{N} (x_{Mj} + x_{M-1,j}) \\
  \vdots ~\cdots \geq & ~\cdots ~\vdots \\
  \Delta y_{1, \omega}^{+} \geq &~ \Delta \sum_{j=1}^{N} \sum_{i=1}^{M} x_{i,j}
\end{align*}

Since the objective function does not decrease, $\bm{H}{'}$ represents the optimal solution to SBSP and is composed of full or largest patterns.
\end{pf}

\begin{pf}{Proof of Lemma \ref{feasible_region}}
Note that $\mathbf{f}^{\intercal} = [-\mathbf{1},~\mathbf{0}]$ and $\mathbf{V} = [\mathbf{W},~\mathbf{I}]$. Based on this, we can derive the following inequalities: $\bm{\alpha}^{\intercal} \mathbf{W} \geq -\mathbf{1}$ and $\bm{\alpha}^{\intercal} \mathbf{I} \geq \mathbf{0}$. According to the expression of $\mathbf{W}$ and $\mathbf{I}$, we can deduce that $0 \leq \alpha_i \leq \alpha_{i-1} +1$ for $i \in \mathcal{M}$ by letting $\alpha_0 = 0$. These inequalities indicate that the feasible region is nonempty and bounded. For $i \in \mathcal{M}$, $\alpha_{i}$ is only bounded by $\alpha_{i-1}+1$ and $0$, thus, all extreme points within the feasible region are integral.
\end{pf}

\begin{pf}{Proof of Proposition \ref{optimal_sol_sub_dual}}
  According to the complementary slackness property, we can obtain the following equations
  \begin{align*}
    & \alpha_{i} (d_{i0} - d_{i \omega} - y_{i \omega}^{+} + y_{i+1, \omega}^{+} + y_{i \omega}^{-}) = 0, i =1,\ldots, M-1 \\
    & \alpha_{i} (d_{i0} - d_{i \omega} - y_{i \omega}^{+}+ y_{i \omega}^{-}) = 0, i = M \\
    & y_{i \omega}^{+}(\alpha_{i} - \alpha_{i-1}-1) = 0, i =1,\ldots, M \\
    & y_{i \omega}^{-} \alpha_{i} = 0, i =1,\ldots, M.
  \end{align*}
  

    When $y_{i \omega}^{-} >0$, we have $\alpha_{i} =0$. When $y_{i \omega}^{+} >0$, we have $\alpha_{i} = \alpha_{i-1} +1$. When $y_{i \omega}^{+} = y_{i \omega}^{-} = 0$, let $\Delta d = d_{\omega} - d_0$,
    \begin{itemize}
      \item if $i = M$, $\Delta d_{M} =0$, the value of objective function associated with $\alpha_{M}$ is always $0$, thus we have $0 \leq \alpha_{M} \leq \alpha_{M-1}+1$;
      \item if $i < M$, we have $y_{i+1, \omega}^{+} = \Delta d_{i} \geq 0$.
      \begin{itemize}
        \item If $y_{i+1, \omega}^{+} > 0$, the objective function associated with $\alpha_i$ is $\alpha_{i} \Delta d_{i} = \alpha_{i} y_{i+1, \omega}^{+}$, thus to minimize the objective value, we have $\alpha_i =0$.
        \item If $y_{i+1, \omega}^{+} = 0$, we have $0 \leq \alpha_{i} \leq \alpha_{i-1} +1$.
      \end{itemize}
    \end{itemize}
  \end{pf}

  % \begin{pf}{Proof of Proposition \ref{one_ep_feasible}}
  %   Suppose we have one extreme point $\bm{\alpha}_{\omega}^{0}$ for each scenario. Then we have the following problem.
  %   \begin{equation}\label{lemma_eq}
  %     \begin{aligned}
  %       \max \quad & \mathbf{c}^{\intercal} \mathbf{x} + \sum_{\omega \in \Omega} p_{\omega} z_{\omega} \\
  %       \text {s.t.} \quad & \mathbf{n} \mathbf{x} \leq \mathbf{L} \\
  %       & (\bm{\alpha}_{\omega}^{0})^{\intercal}\mathbf{d}_{\omega} \geq (\bm{\alpha}_{\omega}^{0})^{\intercal} \mathbf{x} \mathbf{1} + z_{\omega}, \forall \omega \\
  %        & \mathbf{x} \in \mathbb{N}^{M \times N}
  %     \end{aligned}
  %   \end{equation}
  %   Problem \eqref{lemma_eq} reaches its maximum when $(\bm{\alpha}_{\omega}^{0})^{\intercal}\mathbf{d}_{\omega} = (\bm{\alpha}_{\omega}^{0})^{\intercal} \mathbf{x} \mathbf{1} + z_{\omega}, \forall \omega$. Substitute $z_{\omega}$ with these equations, we have 
  %   \begin{equation}\label{lemma_eq2}
  %     \begin{aligned}
  %       \max \quad & \mathbf{c}^{\intercal} \mathbf{x} - \sum_{\omega}p_{\omega}(\bm{\alpha}_{\omega}^{0})^{\intercal} \mathbf{x} \mathbf{1} + \sum_{\omega} p_{\omega} (\bm{\alpha}_{\omega}^{0})^{\intercal} \mathbf{d}_{\omega} \\
  %       \text {s.t.} \quad & \mathbf{n} \mathbf{x} \leq \mathbf{L} \\
  %       & \mathbf{x} \in \mathbb{N}^{M \times N}
  %     \end{aligned}
  %   \end{equation}
  %   Notice that $\mathbf{x}$ is bounded by $\mathbf{L}$, then the problem \eqref{lemma_eq} is bounded. Adding more constraints will not make the optimal value larger. Thus, RBMP is bounded. 
  % \end{pf}


\begin{pf}{Proof of Lemma \ref{bid-price}}
According to the Proposition \ref{sol_relax_deter}, the aggregate optimal solution to LP relaxation of problem \eqref{deter_upper} takes the form $x e_{\tilde{i}} + \sum_{i=\tilde{i}+1} ^{M} d_{i} e_{i}$, then according to the complementary slackness property, we know that $z_1= \ldots= z_{\tilde{i}} = 0$. This implies that $\beta_j \geq \frac{n_i - \delta}{n_i}$ for $i = 1,\ldots, \tilde{i}$. Since $\frac{n_i - \delta}{n_i}$ increases with $i$, we have $\beta_j \geq \frac{n_{\tilde{i}} - \delta}{n_{\tilde{i}}}$. Consequently, we obtain $z_{i} \geq n_i - \delta - n_i \frac{n_{\tilde{i}} - \delta}{n_{\tilde{i}}} = \frac{\delta(n_i-n_{\tilde{i}})}{n_{\tilde{i}}}$ for $i = h+1, \ldots, M$.

Given that $\mathbf{d}$ and $\mathbf{L}$ are both no less than zero, the minimum value will be attained when $\beta_j = \frac{n_{\tilde{i}} - \delta}{n_{\tilde{i}}}$ for all $j$, and $z_i = \frac{\delta(n_i-n_{\tilde{i}})}{n_{\tilde{i}}}$ for $i = \tilde{i}+1, \ldots, M$.  
\end{pf}

\begin{pf}{Proof of Corollary \ref{maximum_phi}}
According to Proposition \ref{lem_pattern}, $\phi(M, L^{0}, \delta)$ is non-decreasing with $M$ and  $L^{0}$, and non-increasing with $\delta$, respectively. Consequently, when $M$ increases while $\bm{L}$ remains unchanged, the maximum achievable occupancy rate does not decrease; when $\delta$ increases while $M$ and $\bm{L}^{0}$ remain unchanged, the maximum achievable occupancy rate does not increase; when $\bm{L}^{0}$ increases while $M$ and $\delta$ remain unchanged, the maximum achievable occupancy rate may either increase or decrease.

\end{pf}

\newpage
