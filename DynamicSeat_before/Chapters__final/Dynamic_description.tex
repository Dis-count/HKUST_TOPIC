\section{Dynamic Seat Assignment with Social Distancing}\label{sec_dynamic_seat}
In many commercial situations, groups arrive sequentially over time, and the seller must promptly make group assignments upon each arrival while maintaining the required spacing between groups. When a group is accepted, the seller must also determine which seats should be assigned to that group. It is essential to note that each group must be either accepted in its entirety or rejected entirely. Once the seats are confirmed and assigned to a group, they cannot be changed or reassigned to other groups.

To model this problem, we adopt a discrete-time framework. Time is divided into $T$ periods, indexed forward from $1$ to $T$. We assume that in each period, at most one group arrives and the probability of an arrival for a group of size $i$ is denoted as $p_i$, where $i$ belongs to the set $\mathcal{M}$. The probabilities satisfy the constraint $\sum_{i=1}^M p_i \leq 1$, indicating that the total probability of any group arriving in a single period does not exceed one. We introduce the probability $p_0 = 1 - \sum_{i=1}^{M} p_i$ to represent the probability of no arrival in a given period $t$. To simplify the analysis, we assume that the arrivals of different group types are independent and the arrival probabilities remain constant over time. This assumption can be extended to consider dependent arrival probabilities over time if necessary.

At time $t$, the state of remaining capacity in each row is represented by a vector $\mathbf{L}^{t} = (l_1^{t}, l_2^{t}, \ldots, l_N^{t})$, where $l_j^{t}$ denotes the number of remaining seats in row $j$ at time $t$. Upon the arrival of a group type $i$ at time $t$, the seller needs to make a decision denoted by $u_{i,j}^{t}$, where $u_{i,j}^{t} = 1$ indicates acceptance of group type $i$ in row $j$ during period $t$, while $u_{i,j}^{t} = 0$ signifies rejection of that group type in row $j$. The feasible decision set is defined as $$U^{t}(\mathbf{L}^{t}) = \left\{u_{i,j}^{t} \in \{0,1\}, \forall i \in \mathcal{M}, \forall j \in \mathcal{N} \big| \sum_{j=1}^{N} u_{i,j}^{t} \leq 1, \forall i \in \mathcal{M}; n_{i}u_{i,j}^{t}\mathbf{e}_j \leq \mathbf{L}^{t}, \forall i \in \mathcal{M}, \forall j \in \mathcal{N}\right\}.$$ 
Here, $\mathbf{e}_j$ represents an N-dimensional unit column vector with the $j$-th element being 1, i.e., $\mathbf{e}_j = (\underbrace{0, \cdots, 0}_{j-1}, 1, \underbrace{0, \cdots, 0}_{n-j})$. In other words, the decision set $U(\mathbf{L}^{t})$ consists of all possible combinations of acceptance and rejection decisions for each group type in each row, subject to the constraints that at most one group of each type can be accepted in any row, and the number of seats occupied by each accepted group must not exceed the remaining capacity of the row.

Let $V^{t}(\mathbf{L}^{t})$ denote the maximum expected revenue earned by the best decisions regarding group seat assignments in period $t$, given remaining capacity $\mathbf{L}^{t}$. Then, the dynamic programming formula for this problem can be expressed as:

\begin{equation}\label{DP}
V^{t}(\mathbf{L}^{t}) = \max_{u_{i,j}^{t} \in U^{t}(\mathbf{L})}\left\{ \sum_{i=1}^{M} p_i \bigl( \sum_{j=1}^{N} i u_{i,j}^{t} + V^{t+1}(\mathbf{L}^{t}- \sum_{j=1}^{N} n_i u_{i,j}^{t}\mathbf{e}_j)\bigr) + p_0 V^{t+1}(\mathbf{L}^{t})\right\}
\end{equation}
with the boundary conditions $V^{T+1}(\mathbf{L}) = 0, \forall \mathbf{L}$ which implies that the revenue at the last period is 0 under any capacity.

At the beginning of period $t$, we have the current remaining capacity vector denoted as $\mathbf{L} = (L_1, L_2, \ldots, L_N)$. Our objective is to make group assignments that maximize the total expected revenue during the horizon from period 1 to $T$ which is represented by $V^{1}(\mathbf{L})$.

Solving the dynamic programming problem described in equation \eqref{DP} can be challenging due to the curse of dimensionality, which arises when the problem involves a large number of variables or states. To mitigate this complexity, we aim to develop a heuristic method for assigning arriving groups. In our approach, we begin by generating a seat planning, as outlined in section \ref{sec_seat_planning}. This seat planning acts as a foundation for the seat assignment. In section \ref{sec_dynamic}, building upon the generated seat planning, we further develop a dynamic seat assignment policy which guides the allocation of seats to the incoming groups sequentially. 


