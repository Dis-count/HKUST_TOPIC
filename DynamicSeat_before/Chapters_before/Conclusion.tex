% !TEX root = sum1.tex
\section{Conclusion}

% In conclusion, this paper addresses the problem of dynamic seat assignment with social distancing in the context of a pandemic. The problem is different from the seat planning problem studied in the literature owing to consideration of real-time seat assignment. After analyzing the static and the dynamic version of this problem, we propose an efficient allocation policy. Specifically, we develop a stochastic programming to obtain the target seat planning, then introduce stochastic planning policy to assign seats for arriving groups, which includes how to assign the group when there is no corresponding planning in the target seat planning. We also consider the classical bid-price control, booking limit control policies which are inferior to our proposed approach. Numerical experiments show the effectiveness of our proposed approach.



% Our contributions include establishing a deterministic model to analyze the effects of social distancing when demand is known, using Benders decomposition methods to obtain the optimal solution for scenario-based stochastic programming, and developing a seat assignment policy for the dynamic situation. Our results demonstrate significant improvements over baseline strategies and provide guidance for developing attendance policies. Overall, our study highlights the importance of considering the operational significance behind social distancing and provides a new perspective for the government to adopt mechanisms for setting seat assignments to protect people in the post-pandemic era. Our study demonstrates the efficiency of obtaining the final seat planning using our proposed algorithm. 

% Moreover, our analysis provides managerial guidance on how to set the occupancy rate and largest size of one group under the background of pandemic.


% \begin{table}[H]
%     \centering
%     \caption{xxxx}
%     \begin{tabular}{cccc}
%  \hline
%  a & aaaaaaaaa & aaaaaaaaaaaaaa & aaaaaaaaaaaaaaaaaaaa \\
%  \hline
%  a & \makebox[5ex][r]{123} & \makebox[6ex][r]{123456} & \makebox[6ex][r]{1}\\
%  a & \makebox[5ex][r]{12345} & \makebox[6ex][r]{123} & \makebox[6ex][r]{123} \\
%  a & \makebox[5ex][r]{1} & \makebox[6ex][r]{1234} & \makebox[6ex][r]{123456} \\
%  \hline
%  \end{tabular}
%  \end{table}

Since the outbreak of the pandemic, social distancing has been widely recognized as a crucial measure for containing the spread of the virus. It has been implemented in seating areas to ensure safety. While static seating arrangements can be addressed through integer programming by defining specific social distancing constraints, dealing with the dynamic situations is challenging.

% Since the outbreak of the pandemic, it has been widely recognized that social distancing is a crucial measure to contain the spread of the virus and It has been used in seating areas. The static situation can be solved by integer programming with defining different social distancing constraints, however, the dynamic situation is difficult to deal with. Our paper addresses the problem of dynamic seat assignment with social distancing in the context of a pandemic.

Our paper focuses on the problem of dynamic seat assignment with social distancing in the context of a pandemic. To tackle this problem, we propose a scenario-based stochastic programming approach to obtain a seat planning that adheres to social distancing constraints. We utilize the benders decomposition method to solve this model efficiently, leveraging its well-structured property. However, solving the integer programming formulation directly can be computationally prohibitive in some cases. Therefore, in practice, we consider the linear programming relaxation of the problem and devise an approach to obtain the seat planning, which consists of full or largest patterns. In our approach, seat planning can be seen as the supply for each group type. We assign groups to seats when the supply is sufficient. However, when the supply is insufficient, we employ a stochastic planning policy to make decisions on whether to accept or reject group requests. 

% This policy takes into account various factors and uncertainties to optimize the allocation of seats while maximizing the number of accommodated individuals.


We conducted several experiments to investigate various aspects of our approach. These experiments included comparing the running time of the benders decomposition method and integer programming, analyzing different policies for dynamic seat assignment, and evaluating the impact of implementing social distancing. The results of our experiments demonstrated that the benders decomposition method efficiently solves our model. In terms of dynamic seat assignment policies, we considered the classical bid-price control, booking limit control in revenue management, dynamic programming-based heuristics, and the first-come-first-served policy. Comparatively, our proposed policy exhibited superior performance.

Building upon our policies, we further evaluated the impact of implementing social distancing. By defining the gap point as the period at which the difference between applying and not applying social distancing becomes evident, we established a relationship between the gap point and the expected number of people in each period. We observed that as the expected number of people in each period increased, the gap point occurred earlier, resulting in a higher occupancy rate at the gap point.

Overall, our study highlights the importance of considering the operational significance behind social distancing and provides a new perspective for the government to adopt mechanisms for setting seat assignments to protect people during the pandemic. 

% Moreover, our analysis provides managerial guidance on how to set the occupancy rate and largest size of one group under the background of pandemic.