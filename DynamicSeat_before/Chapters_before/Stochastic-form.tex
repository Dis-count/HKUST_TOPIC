% !TEX root = ./sum1.tex
% \section{Stochastic Demands Situation}

\section{Seat Planning Composed of Full or Largest Patterns}
In this section, we develop the scenario-based stochastic programming(SSP) to obtain the seat planning with available capacity. Due to the well-structured nature of SSP, we implement Benders decomposition to solve it efficiently. However, in some cases, solving the integer programming with Benders decomposition remains still computationally prohibitive. Thus, we can consider the LP relaxation first, then obtain a feasible seat planning by deterministic model. Based on that, we construct a seat planning composed of full or largest patterns to fully utilize all seats.


% Consider the seller who give the seat planning based on the scenarios then assign the groups to seats according to the realized true demand.

\subsection{Scenario-based Stochastic Programming (SSP) Formulation}
Now suppose the demand of groups is stochastic, the stochastic information can be obtained from scenarios through historical data. Use $\omega$ to index the different scenarios, each scenario $\omega \in \Omega$. Regarding the nature of the obtained information, we assume that there are $|\Omega|$ possible scenarios. A particular realization of the demand vector can be represented as $\mathbf{d}_\omega = (d_{1\omega},d_{2\omega},\ldots,d_{M,\omega})^{\intercal}$. Let $p_{\omega}$ denote the probability of any scenario $\omega$, which we assume to be positive. To maximize the expected number of people accommodated over all the scenarios, we propose a scenario-based stochastic programming to obtain a seat planning.

The seat planning can be represented by decision variables $\mathbf{x} \in \mathbb{Z}{+}^{M \times N}$. Here, $x_{ij}$ denotes the number of group type $i$ assigned to row $j$. The total supply for group type $i$ can be calculated as the sum of $x_{ij}$ over all rows $j$, i.e., $\sum_{j=1}^N x_{ij}$. There is a scenario-dependent decision variable, $\mathbf{y}$, to be chosen. It includes two vectors of decisions, $\mathbf{y}^{+} \in \mathbb{Z}_{+}^{M \times |\Omega|}$ and $\mathbf{y}^{-} \in \mathbb{Z}_{+}^{M \times |\Omega|}$. Each component of $\mathbf{y}^{+}$, denoted as $y_{i\omega}^{+}$, represents the surplus supply for group type $i$ for each scenario $\omega$. On the other hand, $y_{i\omega}^{-}$ represents the shortage of supply for group type $i$ for each scenario $\omega$.

Considering that the group can occupy the seats planned for the larger group type when the corresponding supply is not enough, we assume that the surplus seats for group type $i$ can be occupied by smaller group type $j<i$ in the descending order of the group size. That is, for any $\omega$, $i \leq M-1$, 

$$y_{i \omega}^{+}=\left(\sum_{j=1}^N x_{ij}- d_{i \omega} + y_{i+1, \omega}^{+}\right)^{+}, y_{i \omega}^{-}=\left(d_{i \omega}- \sum_{j=1}^N x_{ij} - y_{i+1, \omega}^{+} \right)^{+}.$$

where $(x)^{+}$ equals $x$ if $x>0$, $0$ otherwise. Specially, for the largest group type $M$, we have $y_{M \omega}^{+} = (\sum_{j=1}^N x_{Mj} - d_{M \omega})^{+}$, $y_{M \omega}^{-} = (d_{M \omega}- \sum_{j=1}^N x_{Mj})^{+}$.


% Because the demand is unknown when the seat assignment is planned, there is no way to expect that the supply in the first stage can meet the demand exactly. Fortunately, we can find some remedies in practice, for example, seats of 5 can be assigned to a group of 4 with one empty seat as social distancing. However, the decision maker will confront seats shortage or excess without these measures. Therefore, to deal with possible demands, the wait-and-see measures (called recourses) should be considered in planning seat assignment.

% which is positive when the supply is larger than the actual demand, zero otherwise.
% which is positive when the supply is less than the actual demand and zero otherwise.

% which include the number of holding groups, $y_{i \omega}^{+}$, positive when the supply overestimates the actual demand and the number of short groups, $y_{i \omega}^{-}$, positive when the supply understimates the actual demand for group type $i$ in scenario $\omega$.

% The assignment will be determined before the realization of the random demand, here-and-now policy.

Then we have the formulation of SSP:
    \begin{align}
    \quad \max \quad & E_{\omega}\left[(n_{M}-\delta) (\sum_{j= 1}^{N} x_{Mj} - y_{M \omega}^{+}) + \sum_{i=1}^{M-1} (n_i-\delta) (\sum_{j= 1}^{N} x_{ij} + y_{i+1,\omega}^{+} - y_{i \omega}^{+})\right] \\
    \text {s.t.} \quad & \sum_{j= 1}^{N} x_{ij}-y_{i \omega}^{+}+
    y_{i+1, \omega}^{+} + y_{i \omega}^{-}=d_{i \omega}, \quad i = 1,\ldots, M-1, \omega \in \Omega \label{DEF_constr1} \\
    & \sum_{j= 1}^{N} x_{ij} -y_{i \omega}^{+}+y_{i \omega}^{-}=d_{i \omega}, \quad i = M, \omega \in \Omega \label{DEF_constr2}\\
    & \sum_{i=1}^{M} n_{i} x_{ij} \leq L_j, j \in \mathcal{N}  \label{DEF_constr3} \\
    & y_{i \omega}^{+}, y_{i \omega}^{-} \in \mathbb{Z}_{+}, \quad i \in \mathcal{M}, \omega \in \Omega \notag \\
    & x_{ij} \in \mathbb{Z}_{+}, \quad i \in \mathcal{M}, j \in \mathcal{N} \notag.
    \end{align}

The objective function consists of two parts. The first part represents the number of the largest group type that can be accommodated, given by $\sum_{j=1}^{N} x_{Mj} - y_{M\omega}^{+}$. Here, $\sum_{j=1}^{N} x_{Mj}$ represents the total supply for group type $M$ and $y_{M\omega}^{+}$ represents the number of surplus supply for group type $M$ in scenario $\omega$. The second part represents the number of group type $i$, excluding $M$, that can be accommodated. It is given by $\sum_{j=1}^{N} x_{ij} + y_{i+1,\omega}^{+} - y_{i\omega}^{+}$. Similarly, $\sum_{j=1}^{N} x_{ij}$ represents the total supply for group type $i$, $y_{i\omega}^{+}$ represents the number of surplus supply for group type $i$ in scenario $\omega$. The overall objective function is subject to an expectation operator denoted by $E_{\omega}$, which represents the expectation with respect to the scenario set. This implies that the objective function is evaluated by considering the average values of the decision variables and constraints over the different scenarios in the set.


By reformulating the objective function, we have

\begin{align*}
  & E_{\omega}\left[\sum_{i=1}^{M-1} (n_i-\delta) (\sum_{j= 1}^{N} x_{ij} + y_{i+1,\omega}^{+} - y_{i \omega}^{+}) + (n_M-\delta) (\sum_{j= 1}^{N} x_{Mj} - y_{M \omega}^{+})\right] \\
  =& \sum_{j =1}^{N} \sum_{i=1}^M (n_i- \delta) x_{ij} - \sum_{\omega =1}^{|\Omega|} p_{\omega} \left(\sum_{i=1}^{M}(n_i- \delta)y_{i \omega}^{+} - \sum_{i=1}^{M-1}(n_i-\delta)y_{i+1, \omega}^{+}\right) \\
  =& \sum_{j =1}^{N} \sum_{i=1}^M i \cdot x_{ij} - \sum_{\omega =1}^{|\Omega|} p_{\omega} \sum_{i = 1}^{M} y_{i \omega}^{+}
\end{align*}

% Plug in $s_i = i+1$, the objective function is $\sum_{j =1}^{N} \sum_{i=1}^m i x_{ij} - \sum_{\omega =1}^{S} p_{\omega} \sum_{i=1}^{m} y_{i \omega}^{+}$.
% The last equality holds because of $n_i- \delta = i, i \in \mathcal{M}$. 

The stochastic programming has the following properties.

For any $i, \omega$, at most one of $y_{i \omega}^{+}$ and $y_{i \omega}^{-}$ can be positive of the optimal solution. Suppose there exist $i_0$ and $\omega_0$ such that $y_{i_0 \omega_0}^{+}$ and $y_{i_0 \omega_0}^{-}$ are positive. Substracting $\min\{y_{i_0, \omega_0}^{+}, y_{i_0, \omega_0}^{-}\}$ from these two values will still satisfy constraints \eqref{DEF_constr1} and \eqref{DEF_constr2} but increase the objective value when $p_{\omega_0}$ is positive. Thus, in the optimal solution, at most one of $y_{i \omega}^{+}$ and $y_{i \omega}^{-}$ can be positive.

% \begin{prop}\label{prop_onescenario}
%   The deterministic problem \eqref{deter_upper} is a special case of stochastic programming when the number of scenarios $|\Omega|$ is equal to 1.
% \end{prop}

% That is to say, the optimal solution to the deterministic problem is one of the optimal solutions to the stochastic programming problem, while the optimal values remain unchanged.

\begin{prop}\label{prop_solution}
There exists an optimal solution to the stochastic programming problem such that the patterns associated with this optimal solution are composed of the full or largest patterns under any given scenarios.
\end{prop}

\begin{pf}[Proof of Proposition \ref{prop_solution}]
In any optimal solution where one of the corresponding patterns is not full or largest, we have the flexibility to allocate the remaining unoccupied seats. These seats can be assigned to either a new seat planning or added to an existing seat planning. Importantly, since a group can utilize the seat planning of a larger group, the allocation scheme based on the original optimal solution will not affect the optimality of the overall solution. Next, we explain the allocation scheme that ensures each row becomes full or largest. Let $\beta= L_{j} - \sum_{i} n_{i} x_{ij}$. If row $j$ is not full or largest, then $\beta > 0$. 
To allocate the remaining unoccupied seats in row $j$, we can follow the following steps:

\begin{itemize}
  \item If $\beta \geq n_{M}$, we can assign $n_M$ seats to a new group type $M$. We repeat this procedure until $\beta < n_{M}$. 
  \item If $n_{1} \leq \beta < n_{M}$, we can assign $\beta$ seats to a new group type $\beta-n_{1}+1$.   
  \item If $0 < \beta < n_{1}$, we need to assign the seats to the existing seat planning. Find the smallest group type denoted as $i^0$. If $i^0 = M$, it means that this row represents a largest pattern, we already change this row to a largest pattern. If $i^0 \neq M$, we reduce the number of group type $i^0$ by one and increase the number of group type $\min {(i^0+\beta), M}$ by one. We repeat this procedure until $\beta = 0$ or until this pattern is changed to a largest one.
\end{itemize}

By repeating these steps for each row, we can ensure that each row in the seat planning becomes either full or largest. This allocation scheme maintains the optimality of the solution while maximizing the utilization of available seats. \qed
\end{pf}

% 对于任何最优解,如果相应的pattern 不是full 或者 最大的,则我们可以通过将剩下未被占据的座位重新划分。可以分配给一个新的group,或者分配给原有的group. 由于小组可以利用大组的座位,保留原有最优解对应的分配方案,将剩下未被占据的座位重新划分不会影响最优性。现在我们需要说明存在一种分配方案使得每排都是full 或者是最大的。


Suppose there exists an optimal solution where at least one corresponding pattern is not full or largest. In such a case, we can utilize the following algorithm to construct another optimal solution in which all patterns are both full and the largest.

% \begin{algorithm}[H]
%   \caption{Construct The Full or Largest Pattern Algorithm}\label{construction}
%     \begin{description}
%     \item[Step 1] Given the solution $\{x_{ij}\}$. Let $X$ represent the supply of the given solution. Check if every pattern is full or largest. Start from $j =1$. 
%     \item[Step 2] Let $\beta = L_{j} - \sum_{i} n_{i} x_{ij}$. 
%     \item[Step 2.1] If $\beta = 0$, let $j =j + 1$. Continue step 2.
%     \item[Step 2.2] If $\beta \geq n_{M}$, let $\beta = \beta - n_{M}$, $X_{M} = X_{M} + 1$. Continue this step until $\beta < n_{M}$. Go to step 2.3 or 2.4.
%     \item[Step 2.3] If $n_{1} \leq \beta < n_{M}$, let $\beta = 0$, $X_{\beta-n_{1}+1} = X_{\beta-n_{1}+1} + 1$, $j = j+1$. Go to step 2. 
%     \item[Step 2.4] If $n_{1} > \beta > 0$, find the smallest group type denoted as $i^0$ in row $j$.
%     \item[Step 3] If $i^0 = M$, then the row represents a largest pattern and let $j = j +1$, go to Step 2. If $i^0 \neq M$, reduce the number of group type $i^0$ by one and increase the number of group type $\min \{(i^0+\beta), M\}$ by one. Go to Step 2.
%    \end{description}
% \end{algorithm}

Notice that this procedure does not impact the optimality of the solution. Instead, it maximizes the utilization of available seats.


Let $\mathbf{n} = (n_1, \ldots, n_M)$, $\mathbf{L} = (L_1, \ldots, L_N)$ where $n_i$ is the size of seats taken by group type $i$ and $L_j$ is the length of row $j$ as we defined above. Then the constraint \eqref{DEF_constr3} can be expressed as $\mathbf{n} \mathbf{x} \leq \mathbf{L}$.

The linear constraints associated with scenarios, i.e., constraints \eqref{DEF_constr1} and \eqref{DEF_constr2}, can be written in a matrix form as
\[\mathbf{x} \mathbf{1} + \mathbf{V} \mathbf{y}_\omega = \mathbf{d}_\omega, \omega\in \Omega,\]

where $\mathbf{1}$ is a column vector of size $N$ with all 1s, $\mathbf{V} = [\mathbf{W}, ~\mathbf{I}]$.

$$
\mathbf{W}=\left[\begin{array}{cccccc}
-1 & 1 & 0 & \ldots & \ldots & 0 \\
0 & -1 & 1 &    0   & \ldots & 0 \\
\vdots & \ddots & \ddots & \ddots & \ddots & \vdots \\
0  & \ldots   &  0  & -1 & 1 & 0 \\
0  & \ldots   &  \ldots  &  0 &  -1 & 1 \\
0 & \ldots & \ldots & \ldots & 0 & -1
\end{array}\right]_{M \times M}
$$

and $\mathbf{I}$ is the identity matrix with the dimension of $M$. For each scenario $\omega \in \Omega$,
$$
\mathbf{y}_{\omega}=\left[\begin{array}{l}
\mathbf{y}_{\omega}^{+} \\
\mathbf{y}_{\omega}^{-}
\end{array}\right], \mathbf{y}_{\omega}^{+}=\left[\begin{array}{lllll}y_{1 \omega}^{+} & y_{2 \omega}^{+} & \cdots & y_{M \omega}^{+}\end{array}\right]^{\intercal}, \mathbf{y}_{\omega}^{-}=\left[\begin{array}{llll}y_{1 \omega}^{-} & y_{2 \omega}^{-} & \cdots & y_{M \omega}^{-}\end{array}\right]^{\intercal}.
$$

As we can find, this deterministic equivalent form is a large-scale problem even if the number of possible scenarios $\Omega$ is moderate. However, the structured constraints allow us to simplify the problem by applying Benders decomposition approach. Before using this approach, we could reformulate this problem as the following form. Let $\mathbf{c}^{\intercal}\mathbf{x} = \sum_{j =1}^{N} \sum_{i=1}^M i \cdot x_{ij}$, $\mathbf{f}^{\intercal}\mathbf{y}_{\omega} = -\sum_{i=1}^{M} y_{i \omega}^{+}$. Then the SSP formulation can be expressed as below,

\begin{equation}\label{BD_master}
\begin{aligned}
\max \quad & \mathbf{c}^{\intercal} \mathbf{x}+ z(\mathbf{x}) \\
\text {s.t.} \quad & \mathbf{n} \mathbf{x} \leq \mathbf{L} \\
& \mathbf{x} \in \mathbb{Z}_{+}^{M \times N},
\end{aligned}
\end{equation}

where $z(\mathbf{x})$ is defined as

$$z(\mathbf{x}) := E(z_{\omega}(\mathbf{x})) = \sum_{\omega \in \Omega} p_{\omega} z_{\omega}(\mathbf{x}),$$ and for each scenario $\omega \in \Omega$, 

\begin{equation}\label{BD_sub}
  \begin{aligned}
    z_{\omega}(\mathbf{x}) := \max \quad & \mathbf{f}^{\intercal} \mathbf{y}_{\omega} \\
    \text {s.t.} \quad & \mathbf{V} \mathbf{y}_{\omega} = \mathbf{d}_{\omega} - \mathbf{x} \mathbf{1} \\
     & \mathbf{y}_{\omega} \geq 0.
  \end{aligned}
\end{equation}

% The objective function of problem \eqref{sto_form} can be expressed as $c{'}\mathbf{x} + \sum_{\omega} p_{\omega}f{'}y_{\omega}$. 

Problem \eqref{BD_sub} maintains the same mathematical form across different scenarios. Therefore, if we can efficiently solve problem \eqref{BD_sub}, it implies that we can also solve problem \eqref{BD_master} quickly.

\subsection{Solve SSP by Benders Decomposition}\label{solve_by_benders}
We reformulate problem \eqref{BD_master} into a master problem and a subproblem \eqref{BD_sub}. The iterative process of solving the master problem and subproblem is known as Benders decomposition. 
The solution obtained from the master problem provides inputs for the subproblem, and the subproblem solutions help update the master problem by adding constraints, iteratively improving the overall solution until convergence is achieved. Firstly, we generate a closed-form solution to problem \eqref{BD_sub}, then we obtain the solution to the LP relaxation of problem \eqref{BD_master} by the constraint generation.

\subsubsection{Solve The Subproblem}\label{second_stage}
% Consider a $\mathbf{x}$ such that $\mathbf{n x} \leq \mathbf{L}$ and $\mathbf{x} \geq 0$ and suppose that this represents the seat planning. 

Notice that the feasible region of the dual of problem \eqref{BD_sub} remains unaffected by $\mathbf{x}$. This observation provides insight into the properties of this problem. Let $\bm{\alpha}$ denote the vector of dual variables. For each $\omega$, we can form its dual problem, which is 

\begin{equation}\label{BD_sub_dual}
  \begin{aligned}
    \min \quad & \bm{\alpha}_{\omega}^{\intercal} (\mathbf{d}_{\omega}- \mathbf{x} \mathbf{1}) \\
    \text {s.t.} \quad & \bm{\alpha}_{\omega}^{\intercal} \mathbf{V} \geq \mathbf{f}^{\intercal}
  \end{aligned}
\end{equation}

\begin{prop}\label{feasible_region}
Let $\mathbb{P} = \{\bm{\alpha} \in \mathbb{R}^{M}|\bm{\alpha}^{\intercal} \mathbf{V} \geq \mathbf{f}^{\intercal}\}$. The feasible region of problem \eqref{BD_sub_dual}, $\mathbb{P}$, is nonempty and bounded. Furthermore, all the extreme points of $\mathbb{P}$ are integral.
\end{prop}

Therefore, the optimal value of the problem \eqref{BD_sub}, $z_{\omega}(\mathbf{x})$, is finite and can be achieved at extreme points of the set $\mathbb{P}$. Let $\mathcal{O}$ be the set of all extreme points of $\mathbb{P}$. That is, we have $z_{\omega}(\mathbf{x}) = \min_{\bm{\alpha}_{\omega} \in \mathcal{O}} \bm{\alpha}_{\omega}^{\intercal}(\mathbf{d}_{\omega}- \mathbf{x} \mathbf{1})$.

% We assume that $\mathbb{P}$ is nonempty and has at least one extreme point.  Then, either the dual problem \eqref{BD_sub_dual} has an optimal solution and $z_{\omega}(\mathbf{x})$ is finite, or the primal problem \eqref{BD_sub} is infeasible and $z_{\omega}(\mathbf{x}) = \infty$.  

% The dual problem \eqref{BD_sub_dual} has an optimal solution and $z_{\omega}(\mathbf{x})$ is finite.

% Let $z_{\omega}$ be the lower bound of $z_{\omega}(x)$. 

Alternatively, $z_{\omega}(\mathbf{x})$ is the largest number $z_{\omega}$ such that $\bm{\alpha}^{\intercal}(\mathbf{d}_{\omega}- \mathbf{x} \mathbf{1}) \geq z_w, \forall \bm{\alpha} \in \mathcal{O}$. We use this characterization of $z_w(\mathbf{x})$ in problem \eqref{BD_master} and conclude that problem \eqref{BD_master} can thus be put in the form:

\begin{equation}\label{BD_master2}
  \begin{aligned}
    \max \quad & \mathbf{c}^{\intercal} \mathbf{x} + \sum_{\omega \in \Omega} p_{\omega} z_{\omega} \\
    \text {s.t.} \quad & \mathbf{n} \mathbf{x} \leq \mathbf{L} \\
    & \bm{\alpha}^{\intercal}(\mathbf{d}_{\omega}- \mathbf{x} \mathbf{1}) \geq z_{\omega}, \bm{\alpha} \in \mathcal{O}, \forall \omega \\
     & \mathbf{x} \in \mathbb{Z}_{+}, z_{\omega} ~\text{is free}
  \end{aligned}
\end{equation}

% Then $z_{\omega}(\mathbf{x}) > -\infty$ if and only if $\bm{\alpha}^{\intercal}(\mathbf{d}_{\omega}- \mathbf{x} \mathbf{1}) \geq 0, \bm{\alpha} \in \mathcal{F}$, which stands for the feasibility cut.

% Because the feasible region is bounded, then feasibility cuts are not needed. Let $z_{\omega}$ be the lower bound of $z_{\omega}(x)$ such that $\bm{\alpha}^{\intercal}(\mathbf{d}_{\omega}- \mathbf{x} \mathbf{1}) \geq z_{\omega}, \bm{\alpha} \in \mathcal{O}$, which is the optimality cut.

% \begin{corollary}\label{coro_1}
%   Only the optimality cuts, $\alpha^{\intercal}(\mathbf{d}_{\omega}- \mathbf{x} \mathbf{1}) \geq z_{\omega}$, will be included in the decomposition approach.
% \end{corollary}


% When $\mathbf{x}$ is fixed, we can give the optimal solution to problem \eqref{BD_sub_dual} rather than solving this linear programming. 


Before applying Benders decomposition to solve problem \eqref{BD_master2}, it is important to address the efficient computation of the optimal solution to problem \eqref{BD_sub_dual}.
When we are given $\mathbf{x}^{*}$, the demand that can be satisfied by the seat planning is $\mathbf{x}^{*} \mathbf{1} = \mathbf{d}_0 = (d_{1,0},\ldots,d_{M,0})^{\intercal}$.
By plugging them in the subproblem \eqref{BD_sub}, we can obtain the value of $y_{i \omega}$ recursively:
\begin{equation}\label{y_recursively}
\begin{aligned}
  & y_{M \omega}^{-}=\left(d_{M \omega}-d_{M 0}\right)^{+} \\
  & y_{M \omega}^{+}=\left(d_{M 0}-d_{M \omega}\right)^{+} \\
  & y_{i \omega}^{-}=\left(d_{i \omega}-d_{i 0} - y_{i+1, \omega}^{+} \right)^{+}, i =1,\ldots, M-1 \\
  & y_{i \omega}^{+}=\left(d_{i 0}- d_{i \omega} + y_{i+1, \omega}^{+}\right)^{+}, i =1,\ldots, M-1
\end{aligned}
\end{equation}

For scenario $\omega$, the optimal value is $\mathbf{f}^{\intercal} y_{\omega}$. The dual optimal solution can be obtained by the following proposition.

\begin{prop}\label{optimal_sol_sub_dual}
  The optimal solutions to problem \eqref{BD_sub_dual} are given by 
\begin{equation}\label{BD_sub_simplified}
  \begin{aligned}
    & \alpha_{i} = 0 \quad \text{if}~  y_{i \omega}^{-} > 0,  y_{i \omega}^{+} = 0, i =1,\ldots, M \\
    & \alpha_{i} = \alpha_{i-1}+1 \quad \text{if}~ y_{i \omega}^{+} > 0,  y_{i \omega}^{-} = 0, i =1,\ldots, M \\
    & \alpha_{i} = 0 \quad \text{if}~ y_{i \omega}^{-} = y_{i \omega}^{+} = 0, y_{i+1, \omega}^{+}> 0, i = 1,\ldots, M-1 \\
    & 0 \leq \alpha_{i} \leq \alpha_{i-1}+1 \quad \text{if}~ y_{i \omega}^{-} = y_{i \omega}^{+} = 0, y_{i+1, \omega}^{+}= 0, i = 1,\ldots, M-1 \\
    & 0 \leq \alpha_{i} \leq \alpha_{i-1}+1 \quad \text{if}~ y_{i \omega}^{-} = y_{i \omega}^{+} = 0, i = M
  \end{aligned}
\end{equation}
\end{prop}

% \begin{remark}
% During the calculation, we choose the optimal solution $\alpha_{i} = \alpha_{i-1} +1$ when $0 \leq \alpha_{i} \leq \alpha_{i-1} +1$.
% \end{remark}

Instead of solving this linear programming directly, we can compute the values of $\alpha_{\omega}$ by performing a forward calculation from $\alpha_{1\omega}$ to $\alpha_{M\omega}$.

\subsubsection{Constraint Generation}\label{bender_stage}
% Benders decomposition works with only a subset of those exponentially many constraints and adds more constraints iteratively until the optimal solution of Benders Master Problem(BMP) is attained. This procedure is known as delayed constraint generation.

Due to the computational infeasibility of solving problem \eqref{BD_master2} with an exponentially large number of constraints, it is a common practice to use a subset, denoted as $\mathcal{O}^t$, to replace $\mathcal{O}$ in problem \eqref{BD_master2}. This results in a modified problem known as the Restricted Benders Master Problem(RBMP). To find the optimal solution of problem \eqref{BD_master2}, we employ a technique called delayed constraint generation. It involves iteratively solving the RBMP and incrementally adding more constraints until the optimal solution to problem \eqref{BD_master2} is obtained.


We can conclude that the RBMP will have the form:

\begin{equation}\label{BD_master3}
  \begin{aligned}
    \max \quad & \mathbf{c}^{\intercal} \mathbf{x} + \sum_{\omega \in \Omega} p_{\omega} z_{\omega} \\
    \text {s.t.} \quad & \mathbf{n} \mathbf{x} \leq \mathbf{L} \\
    & \bm{\alpha}^{\intercal}(\mathbf{d}_{\omega}- \mathbf{x} \mathbf{1}) \geq z_{\omega}, \bm{\alpha} \in \mathcal{O}^{t}, \forall \omega \\
     & \mathbf{x} \in \mathbb{Z}_{+}, z_{\omega} ~\text{is free}
  \end{aligned}
\end{equation}

To determine the initial $\mathcal{O}^{t}$, we have the following proposition.

\begin{prop}\label{one_ep_feasible}
RBMP is always bounded with at least any one feasible constraint for each scenario.
\end{prop}

Given the initial $\mathcal{O}^{t}$, we can have the solution $\mathbf{x}_{0}$ and $\mathbf{z}^{0} =(z^{0}_1,\ldots, z^{0}_S)$. Then $c^{\intercal} \mathbf{x}_0 + \sum_{\omega \in \Omega} p_{\omega} z_{\omega}^{0}$ is an upper bound of problem \eqref{BD_master3}. When $\mathbf{x}_0$ is given, the optimal solution, $\bm{\alpha}_{\omega}^{1}$, to problem \eqref{BD_sub_dual} can be obtained according to Proposition \ref{optimal_sol_sub_dual}. Let $z_{\omega}^{(0)} = \bm{\alpha}_{\omega}^{1}(d_{\omega} - \mathbf{x}_0 \mathbf{1})$ and $(\mathbf{x}_0, \mathbf{z}^{(0)})$ is a feasible solution to problem \eqref{BD_master3} because it satisfies all the constraints. Thus, $\mathbf{c}^{\intercal} \mathbf{x}_0 + \sum_{\omega \in \Omega} p_{\omega} z_{\omega}^{(0)}$ is a lower bound of problem \eqref{BD_master2}.

If for every scenario $\omega$, the optimal value of the corresponding problem \eqref{BD_sub_dual} is larger than or equal to $z_{\omega}^{0}$, which means all contraints are satisfied, then we have an optimal solution, $(\mathbf{x}_{0}, \mathbf{z}_{\omega}^{0})$, to the BMP. 

However, if there exists at least one scenario $\omega$ for which the optimal value of problem \eqref{BD_sub_dual} is less than $z_{\omega}^{0}$, indicating that the constraints are not fully satisfied, we need to add a new constraint $(\bm{\alpha}_{\omega}^{1})^{\intercal}(\mathbf{d}_{\omega} - \mathbf{x} \mathbf{1}) \geq z_{\omega}$ to RBMP.

% $z_{\omega}^{(0)} = \alpha_{\omega}^{*}(d_{\omega} - \mathbf{x}_0 \mathbf{1})$ will give a minimal upper bound of $z_{\omega}$, thus all the left constraints associated with other extreme points are redundant.when the extreme points are $\alpha_{\omega}$.

% The problem \eqref{lemma_eq} associated with $\alpha_{\omega}$ will give an optimal solution $(x_1, z_{\omega}^{1})$. (Upper bound)


The steps of the algorithm are described as below,

% \begin{algorithm}[H]\label{cut_algo}
%   \caption{The benders decomposition algorithm}
%     \begin{description}
%     \item[Step 1.] Solve problem \eqref{BD_master3} with all $\alpha_{\omega}^0 = \mathbf{0}$ for each scenario. Then, obtain the solution $(\mathbf{x}_0, \mathbf{z}^{0})$.
%     \item[Step 2.] Set the upper bound $UB = c^{\intercal} \mathbf{x}_0 + \sum_{\omega \in \Omega} p_{\omega} z_{\omega}^{0}$.
%     \item[Step 3.] For $x_0$, we can obtain $\alpha_{\omega}^{1}$ and $z_{\omega}^{(0)}$ for each scenario, set the lower bound $LB = c^{\intercal} x_0 + \sum_{\omega \in \Omega} p_{\omega} z_{\omega}^{(0)}$.
%     \item[Step 4.] For each $\omega$, if $(\alpha_{\omega}^{1})^{\intercal}(\mathbf{d}_{\omega}- \mathbf{x}_0 \mathbf{1}) < z_{\omega}^{0}$, add one new constraint, $(\alpha_{\omega}^{1})^{\intercal}(\mathbf{d}_{\omega}- \mathbf{x} \mathbf{1}) \geq z_{\omega}$, to RBMP.
%     \item[Step 5.] Solve the updated RBMP, obtain a new solution $(x_1, z^{1})$ and update UB.
%     \item[Step 6.] Repeat step 3 until $UB - LB < \epsilon$.
%     % (In our case, UB converges.)
%    \end{description}
%   \end{algorithm}

From the Lemma \ref{one_ep_feasible}, we can set $\bm{\alpha}_{\omega}^0 = \mathbf{0}$ initially in {\bf Step 1}. Notice that only contraints are added in each iteration, thus $LB$ and $UB$ are both monotone. Then we can use $UB - LB < \epsilon$ to terminate the algorithm in {\bf Step 6}.


After the algorithm terminates, we obtain the optimal $\mathbf{x}^{*}$. The demand that can be satisfied by the arrangement is $\mathbf{x}^{*} \mathbf{1} = \mathbf{d}_0 = (d_{1,0},\ldots,d_{M,0})$. Solving problem \eqref{BD_master3} directly can be computationally challenging in some cases, so practically we first obtain the optimal solution to the LP relaxation of problem \eqref{BD_master}. Then, we generate an integral seat planning from this solution.

\subsection{Obtain The Seat Planning Composed of Full or Largest Patterns}\label{seat_assignment}
We may obtain a fractional solution using the decomposition method. This solution represents the optimal allocations of groups to seats but may involve fractional values, indicating partial assignments. Based on the fractional solution obtained, we use the deterministic model to generate a feasible seat planning. The objective of this model is to allocate groups to seats in a way that satisfies the supply requirements for each group without exceeding the corresponding supply values obtained from the fractional solution. To accommodate more groups and optimize seat utilization, we aim to construct a seat planning composed of full or largest patterns based on the feasible seat planning obtained in the previous step. 


% This involves adjusting the allocation of seats within the seat planning to maximize the utilization of available seats. The goal is to create patterns where each row is either completely filled or consists of the largest possible group.

% these constraints while potentially resulting in patterns that are not necessarily full or largest.



% The decomposition method only gives a fractional solution and the stochastic model does not provide an appropriate seat planning when the number of people in scenario demands is way smaller than the number of the seats.
% The objective is to obtain the maximal number of people placed according to the demand scenarios. It will not provide an appropriate seat assignment when the number of people associated with scenario demands is way less than the number of available seats because there are multiple optimal solutions and the solution given by solver probably does not utilize all the empty seats.

% Suppose that each row has the same length. Then the optimal integrated solution 
% $(0,\ldots, x,d_{h+1}, \ldots, d_{m})$ has the same objective value with an integer solution. Deciding if these items can fit into a specified number of rows is the decision problem of the bin-packing problem. If the items associated with the integer solution can be put in the given number of rows, this solution is optimal; otherwise, it is not optimal. Since the bin-packing problem is NP-hard, problem \eqref{deter_upper} is also NP-hard.

% its objective is to obtain the maximal number of people served, not the optimal seat assignment. It will not provide an appropriate solution when the number of arriving people in the scenarios is way less than the number of total seats because it does not utilize all the empty seats.

Let the optimal solution to the relaxation of SSP be $\mathbf{x}^{*}$. Aggregate $\mathbf{x}^{*}$ to the number of each group type, ${X}_{i}^{0} =\sum_{j} x^{*}_{ij}, i \in \mathbf{M}$. Replace the vector $\mathbf{d}$ with $\mathbf{X}^{0}$ in the deterministic model, we have the following problem, 

\begin{equation}\label{deter_upper1}
  \{\max \sum_{j=1}^{N} \sum_{i=1}^{M}(n_i -\delta) x_{ij}: \sum_{i = 1}^{M} n_i x_{ij} \leq L_{j}, j \in \mathcal{N}; \sum_{j =1}^{N} x_{ij} \leq X_{i}^{0}, i \in \mathcal{M}; x_{ij} \in Z^{+} \}
\end{equation}

Then solve the resulting problem \eqref{deter_upper1} to obtain the optimal solution, $\mathbf{x}^{1}$, which represents a feasible seat planning.

To maximize the utilization of seats, we should assign full or largest patterns to each row. We will check if every row is full. When row $j$ is not full, i.e., $\sum_{i} n_{i} x_{ij} < L_{j}$, let $\beta = L_{j} - \sum_{i} n_{i} x_{ij}$. If $\beta$ is no less than $n_M$, then let $\beta = \beta - n_M$, $s_M = s_M +1$ until $\beta$ is less than $n_M$. If row $j$ is still not full, then find the smallest group size in row $j$ and mark it as $i^0$. If the smallest group is exactly the largest, then the row represents a largest pattern and check next row. Otherwise, reduce the number of group type $i^0$ by one and increase the number of group type $\min \{(i^0+\beta), M\}$ by one. Continue this procedure until this row is full.


This procedure can be described in {\bf Step 4} of the following algorithm.

% \begin{algorithm}[H]
%   \caption{Seat Planning Construction Algorithm}\label{seat_construction}
%     \begin{description}
%     \item[Step 1.] Obtain the solution, $\mathbf{x}^{*}$, from stochatic linear programming by benders decomposition. Aggregate $\mathbf{x}^{*}$ to the number of each group type, ${s}_{i}^{0} =\sum_{j} x^{*}_{ij}, i \in \mathbf{M}$.
%     \item[Step 2.] Solve problem \eqref{deter_upper1} to obtain the optimal solution, $\mathbf{x}^{1}$. 
%     \item[Step 3.] Construct the full or largest patterns mentioned in algorithm \ref{construction}.
%    \end{description}
%   \end{algorithm}

For Algorithm \ref{seat_construction}, {\bf Step 2} can give a feasible seat planning, {\bf Step 4} can give the full or largest patterns for each row. The sequence of groups within each pattern can be arranged arbitrarily, allowing for a flexible seat planning that can accommodate realistic operational constraints. 

% Therefore, any fixed sequence of groups within each pattern can be used to construct a seat planning that meets practical needs.

% Thus, we can obtain a feasible seat planning by solving stochastic programming once and deterministic programming twice.

% 1. Why should not we use the subset sum problem to decompose the whole problem, it will destroy global optimality. But notice that when we arrange row by row, it may also affect the optimality.

% Many symmetry structure/ Every step we need to solve a multiple knapsack problem(difficult).

% We are able to provide an online seat planning by using our method.

