% !TEX root = sum1.tex
\section{Seat Assignment with Dynamic Demand}\label{sec_dynamic}

In this section, we address the assignment of arriving groups in the dynamic situation. Our policy involves making seat allocation decisions for each arriving group based on a seat planning strategy outlined in Section \ref{sec_seat_planning}. If the supply for a group arrival is adequate, we accept the group directly. However, when the supply is insufficient, we have two methods to make the decision. The first method is based on an adjusted SSP, while the second method is based on the LP relaxation of SSP. Typically, we prefer the former method if time permits. However, to ensure consistent computation time, we opt for the latter method.


Specifically, we obtain the seat planning, $x_{ij}, \forall i,j$, by SSP. The seat planning can be obtained by either solving the LP relaxation of SSP or the original SSP, depending on the computational time. Let the corresponding supply be denoted as $[X_1, \ldots, X_M]$, where $X_i = \sum_{j} x_{ij}$ for all $i$. When a group type $i$ arrives, if $X_i > 0$, we accept the group directly and assign them the seats originally planned for group type $i$, adhering to the break-tie rules. If $X_i = 0$, we make the decision based on the adjusted SSP or the LP relaxation of SSP.

\subsection{Assignment Based on The Adjusted SSP}

For the first method, to determine the appropriate row assignment for the current group type $k$, we introduce the decision variables $a_j, j \in \mathcal{N}$ indicating whether we accept it in row $j$.

\begin{equation}\label{adjusted_SSP}
  \begin{aligned}
  \max \quad & \sum_{j} k a_j + E_{\omega}\left[\sum_{i=1}^{M-1} (n_i-\delta) (\sum_{j= 1}^{N} x_{ij} + y_{i+1,\omega}^{+} - y_{i \omega}^{+}) + (n_{M}-\delta) (\sum_{j= 1}^{N} x_{Mj} - y_{M \omega}^{+})\right] \\
  \text {s.t.} \quad & \sum_{j= 1}^{N} x_{ij}-y_{i \omega}^{+}+
  y_{i+1, \omega}^{+} + y_{i \omega}^{-}=d_{i \omega}, \quad i = 1,\ldots,M-1, \omega \in \Omega \\
  & \sum_{j= 1}^{N} x_{ij} -y_{i \omega}^{+}+y_{i \omega}^{-}=d_{i \omega}, \quad i = M, \omega \in \Omega \\
  & \sum_{i=1}^{M} n_{i} x_{ij} \leq L_j - n_k a_j, j \in \mathcal{N} \\
  & \sum_{j=1}^{N} a_j \leq 1 \\
  & x_{ij} \in \mathbb{N}, \quad i \in \mathcal{M}, j \in \mathcal{N}, y_{i \omega}^{+}, y_{i \omega}^{-} \in \mathbb{N}, \quad i \in \mathcal{M}, \omega \in \Omega,  a_j \in \{0,1\}, j \in \mathcal{N}.
  \end{aligned}
\end{equation}

When $X_i = 0$, we can solve the aforementioned problem to make the decision and update the seat planning accordingly. Once the decision is made, we proceed to assign seats to the next incoming group based on the updated seat planning. This iterative process ensures that the seat assignments are continuously adjusted and optimized as new groups arrive.


\subsection{Assignment Based on The LP Relaxation of SSP}
When $X_i = 0$, we have three steps to assign the group arrival. First, we develop the group-type control policy to determine the group type that can accommodate the arriving group. Second, we choose a specific row based on the group type and the break-tie rule. In the third step, we make the decision by comparing the values of the relaxed SSPs, considering both accepting and rejecting the arriving group.

To optimize the computational efficiency, we consider certain conditions to determine when to regenerate the seat planning, ensuring that it is not regenerated unnecessarily in every period. This iterative process continues for the next incoming group, enabling real-time seat assignments based on the current seat planning, while minimizing the frequency of seat planning regeneration. In the following part, we will refer to this policy as Dynamic Seat Assignment (DSA).


\subsubsection{Group-type Control}\label{nested_policy}
% One intuitive approach is to utilize the relaxed SSP to make decisions by comparing the values obtained when accepting or rejecting the currently arriving group. The SSP aids in generating seat planning, and when there are available seats planned for the group, we readily accept and allocate it to the corresponding position according to the seat planning. 

When $X_i = 0$ for the arriving group type $i$, we need to perform calculations of the SSP by considering the placement of the group in each possible row, then compare these values to make a decision. However, it is important to note that integer stochastic programming can be computationally expensive and will be unsolvable in some instances. Additionally, evaluating the values associated with placing the group in each possible row can require a significant amount of computation. Therefore, in order to mitigate the computational challenges, we utilize the LP relaxation of SSP as an approximation to compare the values when deciding whether to accept or reject a group. However, one challenge arises from the fact that the LP relaxation results in the same objective values for the acceptance group in any possible row. This poses the question of determining which row to place the group in when we accept it. To address this challenge, we developed the group-type control policy.
% which narrows down the row options based on the seat planning.

The group-type control aims to find the group type to assign the arriving group, that helps us narrow down the option of rows for seat assignment. The policy considers whether to use a larger group's supply to meet the need of the arriving group when given a seat planning. When a group is accepted and assigned to larger-size seats, the remaining empty seat(s) can be reserved for future demand without affecting the rest of the seat planning. To determine whether to use larger seats to accommodate the incoming group, we compare the expected values of accepting the group in the larger seats and rejecting the group based on the current seat planning. Then we identify the possible rows where the incoming group can be assigned based on the group types and seat availability.


% Based on our previous considerations, we know that groups can be assigned to the seats which are planned for a larger group type if the supply for their own group type is insufficient. 

Specifically, suppose the supply is $(X_1, \ldots, X_M)$ at period $t$, the number of remaining periods is $(T-t)$. For the arriving group type $i$ when $X_i = 0$, we demonstrate how to decide whether to accept the group to occupy larger-size seats. For any $\hat{i}=i+1, \ldots, M$, we can use one supply of group type $\hat{i}$ to accept a group type $i$. In that case, when $\hat{i} = i+1, \ldots, i+\delta$, the expected number of accepted people is $i$ and the remaining seats beyond the accepted group, which is $\hat{i}-i$, will be wasted. When $\hat{i} = i+\delta+1, \ldots, M$, the rest $(\hat{i}-i-\delta)$ seats can be provided for one group type $(\hat{i}-i-\delta)$ with $\delta$ seats of social distancing. Let $D_{\hat{i}}^{t}$ be the random variable that indicates the number of group type $\hat{i}$ in $t$ periods. The expected number of accepted people is $i + (\hat{i}-i-\delta)P(D_{\hat{i}-i-\delta}^{T-t} \geq X_{\hat{i}-i-\delta}+1)$, where $P(D_i^{T-t} \geq X_i)$ is the probability that the demand of group type $i$ in $(T-t)$ periods is no less than $X_i$, the remaining supply of group type $i$. Thus, the term, $P(D_{\hat{i}-i-\delta}^{T-t} \geq X_{\hat{i}-i-\delta}+1)$, indicates the probability that the demand of group type $(\hat{i}-i-\delta)$ in $(T-t)$ periods is no less than its current remaining supply plus 1. 

Similarly, when we retain the supply of group type $\hat{i}$ by rejecting the group type $i$, the expected number of accepted people is $\hat{i} P(D_{\hat{i}}^{T-t} \geq X_{\hat{i}})$. The term, $P(D_{\hat{i}}^{T-t} \geq X_{\hat{i}})$, indicates the probability that the demand of group type $\hat{i}$ in $(T-t)$ periods is no less than its current remaining supply.

Let $d^{t}(i,\hat{i})$ be the difference of expected number of accepted people between acceptance and rejection on group type $i$ occupying $(\hat{i}+\delta)$-size seats at period $t$. Then we have
\begin{equation*}
	d^{t}(i,\hat{i}) = \begin{cases}
    i + (\hat{i}-i-\delta)P(D_{\hat{i}-i-\delta}^{T-t} \geq X_{\hat{i}-i-\delta}+1) - \hat{i} P(D_{\hat{i}}^{T-t} \geq X_{\hat{i}}), &\text{if}~ \hat{i} = i+\delta+1, \ldots, M \\
    i - \hat{i} P(D_{\hat{i}}^{T-t} \geq X_{\hat{i}}), &\text{if}~ \hat{i} = i+1, \ldots, i+\delta.
		\end{cases}
\end{equation*}


One intuitive decision is to choose $\hat{i}$ with the largest difference. For all $\hat{i} = i+1, \ldots, M$, find the largest $d^{t}(i,\hat{i})$, denoted as $d^{t}(i,\hat{i}^{*})$. If $d^{t}(i,\hat{i}^{*}) >0$, we will plan to assign the group type $i$ in $(\hat{i}^{*}+\delta)$-size seats. Otherwise, reject the group.

Group-type control policy can only tell us which group type's seats are planned to provide for the smaller group based on the current planning, we still need to further compare the values of the stochastic programming problem when accepting or rejecting a group on the specific row. 

\subsubsection{Decision on Assigning The Group to A Specific Row}
To make the final decision, first, we determine a specific row by the tie-breaking rule, then we assign the group based on the values of relaxed stochastic programming. To determine the appropriate row for seat assignment, we can apply a tie-breaking rule among the possible options obtained by the group-type control. This rule helps us decide on a particular row when there are multiple choices available. 

\subsubsection*{Break Tie for Determining A Specific Row}
A tie occurs when there are serveral rows to accommodate the group. Suppose one group type $i$ arrives, the current seat planning is $\bm{H} = \{\bm{h}_{1}; \ldots, \bm{h}_{N}\}$, the corresponding supply is $\bm{X}$. Let $\beta_{j} = L_j - \sum_{i} (i+\delta) H_{ji}$ represent the remaining number of seats in row $j$ after considering the seat allocation for other groups. When $X_{i} > 0$, we assign the group to row $k \in \arg \min_{j} \{\beta_{j}\}$. That allows us to fill in the row as much as possible. When $X_{i} = 0$ and we plan to assign the group to seats designated for group type $\hat{i}, \hat{i}>i$, we assign the group to a row $k \in \arg \max_{j} \{\beta_{j}| H_{j \hat{i}}>0\}$. That helps to reconstruct the pattern with less unused seats. When there are multiple rows available, we can choose randomly. This rule in both scenarios prioritizes filling rows and leads to better capacity management.

As an example to illustrate group-type control and the tie-breaking rule, consider a situation where $L_1 =3, L_2 = 4, L_3 =5, L_4 =6$, $M =4$, $\delta =1$. The corresponding patterns for each row are $(0,1,0,0)$, $(0,0,1,0)$, $(0,0,0,1)$ and $(0,0,0,1)$, respectively. Thus, $\beta_1 = \beta_2 = \beta_3 =0$, $\beta_4 =1$. Now, a group of one arrives, and the group-type control indicates the possible rows where the group can be assigned. We assume this group can be assigned to the seats of the largest group according to the group-type control, then we have two choices: row 3 or row 4. To determine which row to select, we can apply the breaking tie rule. The $\beta$ value of the rows will be used as the criterion, we would choose row 4 because $\beta_4$ is larger. Because when we assign it in row 4, there will be two seats reserved for future group of one, but when we assign it in row 3, there will be one seat remaining unused.

In the above example, the group of one can be assigned to any row with the available seats. The group-type control can help us find the larger group type that can be used to place the arriving group while maximizing the expected values. Maybe there are multiple rows containing the larger group type. Then we can choose the row containing the larger group type according to the breaking tie rule. 
Finally, we compare the values of stochastic programming when accepting or rejecting the group, then make the corresponding decision.


% Notice that the relaxed stochastic programming will have the same value.

\subsubsection*{Compare The Values of Relaxed SSP}
Then, we compare the values of the relaxed stochastic programming when accepting the group at the chosen row versus rejecting it. This evaluation allows us to assess the potential revenues and make the final decision. Simultaneously, after this calculation, we can generate a new seat planning according to Algorithm \ref{seat_construction}. For the situation where the supply is enough in the first step, we can skip the final step because we already accept the group. Specifically, after we plan to assign the arriving group in a specific row, we determine whether to place the arriving group in the row based on the values of the stochastic programming problem. For the objective values of the relaxed stochastic programming, we consider the potential revenues that could result from accepting the current arrival, i.e., the Value of Acceptance (VoA), as well as the potential outcomes that could result from rejecting it, i.e., the Value of Rejection (VoR).

Suppose a group type $i$ arrives at period $t$. The set of scenarios at period $t$ is denoted as $\Omega^{t}$. The available supply at period $t$ before making the decision is $\mathbf{L}_{r}^{t} = (L_1^{t}, \ldots, L_N^{t})$. The VoR is the value of RSSP with the scenario set $\Omega^{t}$ and the capacity $\mathbf{L}_{r}^{t}$ when we reject group type $i$ at period $t$, denoted as RSSP$(\mathbf{L}_{r}^{t}, \Omega^{t})$. If we plan to accept group type $i$ in row $j$, we need to assign seats from row $j$ to group type $i$. Let $\mathbf{L}_{a}^{t}= (L_1^{t}, \ldots, L_j^{t}-n_{i}, \ldots, L_N^{t})$, then the VoA is calculated by RSSP$(\mathbf{L}_{a}^{t}, \Omega^{t})$.

In each period, we can calculate the relaxed stochastic programming values only twice: once for the acceptance option (VoA) and once for the rejection option (VoR). By comparing the values of VoA and VoR, we can determine whether to accept or reject the group arrival. The decision will be based on selecting the option with the higher expected value, i.e., if the VoA is larger than the VoR, we accept the arrival; if the VoA is less than the VoR, we will reject the incoming group.


% In such cases, we refer to the corresponding planning group row in the group-type control, where we determine which group to break in order to accommodate the incoming group. 

By combining the group-type control strategy with the evaluation of relaxed stochastic programming values, we obtain a comprehensive decision-making process within a single period. This integrated approach enables us to make informed decisions regarding the acceptance or rejection of incoming groups, as well as determine the appropriate row for the assignment when acceptance is made. By considering both computation time savings and potential revenues, we can optimize the overall performance of the seat assignment process.

% In essence, this decision-making approach weighs the potential benefits associated with accepting or rejecting an arrival, allowing us to select the option that maximizes the objective value and aligns with our overall goals.


\subsubsection{Regenerate The Seat Planning}
To optimize computational efficiency, it is not necessary to regenerate the seat planning for every period. Instead, we can employ a more streamlined approach. Considering that largest group type can meet the needs of all smaller group types, thus, if the supply for the largest group type diminishes from one to zero, it becomes necessary to regenerate the seat planning. This avoids rejecting the largest group due to infrequent regenerations. Another situation that requires seat planning regeneration is when we determine whether to assign the arriving group to a larger group. In such case, we can obtain the corresponding seat planning after solving the relaxed stochastic programmings. By regenerating the seat planning in such situations, we ensure that we have an accurate supply and can give the allocation of seats based on the group-type control and the comparisons of VoA and VoR.


The decision-making algorithm is shown below.

\begin{algorithm}[H]
  \caption{Dynamic Seat Assignment}
  \For{$t =1, \ldots, T$}
  {Initialize $\bm{X} = \bm{0}$\;
    Observe group type $i$\;
  \If{$i = M$, $X_{M} =0$ or $t =1$}
   {Generate seat planning $\bm{H}$ from Algorithm \ref{seat_construction}\; Update the corresponding $\bm{X}$\Comment*[r]{Regenerate the seat planning when the supply of the largest group type is 0}}
    \eIf{$X_{i} > 0$}
    {Find row $k$ such that $H_{ki} >0$ according to tie-breaking rule\; Accept group type $i$ in row $k$, $L_{k} \gets L_{k} -n_{i}$\; $H_{ki} \gets H_{ki} -1$, $X_{i} \gets X_{i} -1$\Comment*[r]{Accept group type $i$ when the supply is sufficient}}
    {Calculate $d^{t}(i, \hat{i}^{*})$\;
    \eIf{$d^{t}(i, \hat{i}^{*}) \geq 0 $}
    {Find row $k$ such that $H_{k \hat{i}^{*}} > 0$ according to the tie-breaking rule\; 
    Calculate the VoA under scenario $\Omega^{t}_{A}$ and the VoR under scenario $\Omega^{t}_{R}$\;
    \eIf{VoA $\geq$ VoR}
    {Accept group type $i$, $L_{k} \gets L_{k} - n_{i}$\; Regenerate $\bm{H}$ from Algorithm \ref{seat_construction}\; Update the corresponding $\bm{X}$\;}
    {Reject group type $i$\; Regenerate $\bm{H}$ from Algorithm \ref{seat_construction}\; Update the corresponding $\bm{X}$\;}}
    {Reject group type $i$\;}
    }}
\end{algorithm}
