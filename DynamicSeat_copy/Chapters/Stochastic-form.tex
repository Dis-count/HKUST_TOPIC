% !TEX root = ./sum1.tex

% random arrival -> seat planning -> suitable to dynamic case

\section{Seat Planning with Stochastic Demand}\label{sec_seat_planning}
In this section, we aim to obtain a seat planning which is suitable to the dynamic seat assignment.Specially, we develop the Scenario-based Stochastic Programming (SSP) to obtain the seat planning with available capacity. Due to the well-structured nature of SSP, we implement Benders decomposition to solve it efficiently. However, in some cases, solving the integer programming with Benders decomposition remains still computationally prohibitive. Thus, we can consider the LP relaxation first, then obtain a feasible seat planning by deterministic model. Based on that, we construct a seat planning composed of full or largest patterns to fully utilize all seats.

% Consider the seller who give the seat planning based on the scenarios then assign the groups to seats according to the realized true demand.

\subsection{Scenario-based Stochastic Programming Formulation}
Now suppose the demand of groups is stochastic, the stochastic information can be obtained from scenarios through historical data. Use $\omega$ to index the different scenarios, each scenario $\omega \in \Omega$. Regarding the nature of the obtained information, we assume that there are $|\Omega|$ possible scenarios. A particular realization of the demand vector can be represented as $\mathbf{d}_\omega = (d_{1\omega},d_{2\omega},\ldots,d_{M,\omega})^{\intercal}$. Let $p_{\omega}$ denote the probability of any scenario $\omega$, which we assume to be positive. To maximize the expected number of people accommodated over all the scenarios, we propose a scenario-based stochastic programming to obtain a seat planning.

The seat planning can be represented by decision variables $\mathbf{x} \in \mathbb{N}^{M \times N}$. Here, $x_{ij}$ represents the number of group type $i$ assigned to row $j$ in the seat planning. As mentioned earlier, we calculate the supply for group type $i$ as the sum of $x_{ij}$ over all rows $j$, denoted as $\sum_{j=1}^N x_{ij}$. However, considering the variability across different scenarios, it is necessary to model the potential excess or shortage of supply. To capture this characteristic, we introduce a scenario-dependent decision variable, denoted as $\mathbf{y}$. 
It includes two vectors of decisions, $\mathbf{y}^{+} \in \mathbb{N}^{M \times |\Omega|}$ and $\mathbf{y}^{-} \in \mathbb{N}^{M \times |\Omega|}$. Each component of $\mathbf{y}^{+}$, denoted as $y_{i\omega}^{+}$, represents the excess supply for group type $i$ for each scenario $\omega$. On the other hand, $y_{i\omega}^{-}$ represents the shortage of supply for group type $i$ for each scenario $\omega$.

Taking into account the possibility of groups occupying seats planned for larger group types when the corresponding supply is insufficient, we make the assumption that surplus seats for group type $i$ can be occupied by smaller group types $j<i$ in descending order of group size. This means that if there are excess supply available after assigning groups of type $i$ to rows, we can provide the supply to groups of type $j<i$ in a hierarchical manner based on their sizes. That is, for any $\omega$, $i \leq M-1$, 

$$y_{i \omega}^{+}=\left(\sum_{j=1}^N x_{ij}- d_{i \omega} + y_{i+1, \omega}^{+}\right)^{+}, ~y_{i \omega}^{-}=\left(d_{i \omega}- \sum_{j=1}^N x_{ij} - y_{i+1, \omega}^{+} \right)^{+},$$
where $(x)^{+}$ equals $x$ if $x>0$, $0$ otherwise. Specially, for the largest group type $M$, we have $y_{M \omega}^{+} = (\sum_{j=1}^N x_{Mj} - d_{M \omega})^{+}$, $y_{M \omega}^{-} = (d_{M \omega}- \sum_{j=1}^N x_{Mj})^{+}$. Based on the above mentioned considerations, the total supply of group type $i$ under scenario $\omega$ can be expressed as $\sum_{j= 1}^{N} x_{ij} + y_{i+1,\omega}^{+} - y_{i \omega}^{+}, i = 1, \ldots, M-1$. For the special case of group type $M$, the total supply under scenario $\omega$ is $\sum_{j= 1}^{N} x_{Mj} - y_{M \omega}^{+}$.


% Because the demand is unknown when the seat assignment is planned, there is no way to expect that the supply in the first stage can meet the demand exactly. Fortunately, we can find some remedies in practice, for example, seats of 5 can be assigned to a group of 4 with one empty seat as social distancing. However, the decision maker will confront seats shortage or excess without these measures. Therefore, to deal with possible demands, the wait-and-see measures (called recourses) should be considered in planning seat assignment.

% which is positive when the supply is larger than the actual demand, zero otherwise.
% which is positive when the supply is less than the actual demand and zero otherwise.

% which include the number of holding groups, $y_{i \omega}^{+}$, positive when the supply overestimates the actual demand and the number of short groups, $y_{i \omega}^{-}$, positive when the supply understimates the actual demand for group type $i$ in scenario $\omega$.

% The assignment will be determined before the realization of the random demand, here-and-now policy.

Then we have the following formulation:
  \begin{align}
  \quad \max \quad & E_{\omega}\left[(n_{M}-\delta) (\sum_{j= 1}^{N} x_{Mj} - y_{M \omega}^{+}) + \sum_{i=1}^{M-1} (n_i-\delta) (\sum_{j= 1}^{N} x_{ij} + y_{i+1,\omega}^{+} - y_{i \omega}^{+})\right] \\
  \text {s.t.} \quad & \sum_{j= 1}^{N} x_{ij}-y_{i \omega}^{+}+
  y_{i+1, \omega}^{+} + y_{i \omega}^{-}=d_{i \omega}, \quad i = 1,\ldots, M-1, \omega \in \Omega \label{DEF_constr1} \\
  & \sum_{j= 1}^{N} x_{ij} -y_{i \omega}^{+}+y_{i \omega}^{-}=d_{i \omega}, \quad i = M, \omega \in \Omega \label{DEF_constr2}\\
  & \sum_{i=1}^{M} n_{i} x_{ij} \leq L_j, j \in \mathcal{N}  \label{DEF_constr3} \\
  & y_{i \omega}^{+}, y_{i \omega}^{-} \in \mathbb{N}, \quad i \in \mathcal{M}, \omega \in \Omega \notag \\
  & x_{ij} \in \mathbb{N}, \quad i \in \mathcal{M}, j \in \mathcal{N} \notag.
  \end{align}

We use SSP$(\mathbf{L}, \Omega)$ to represent the above stochastic programming, and similarly, we use RSSP$(\mathbf{L}, \Omega)$ to represent the LP relaxation of SSP. 

% Notice that the value of RSSP$(\mathbf{L}, \Omega)$ is related with the total capacity $\sum_{j} L_{j}$ rather than $\mathbf{L}$. 

The objective function consists of two parts. The first part represents the number of people in
the group type $M$ that can be accommodated, given by $(n_{M}-\delta) (\sum_{j=1}^{N} x_{Mj} - y_{M\omega}^{+})$. The second part represents the number of people in group type $i$, excluding $M$, that can be accommodated, given by $(n_i-\delta) (\sum_{j=1}^{N} x_{ij} + y_{i+1,\omega}^{+} - y_{i\omega}^{+}), i = 1, \ldots, M-1$. The overall objective function is subject to an expectation operator denoted by $E_{\omega}$, which represents the expectation with respect to the scenario set. This implies that the objective function is evaluated by considering the average values of the decision variables and constraints over the different scenarios.

By reformulating the objective function, we have
\begin{align*}
  & E_{\omega}\left[\sum_{i=1}^{M-1} (n_i-\delta) (\sum_{j= 1}^{N} x_{ij} + y_{i+1,\omega}^{+} - y_{i \omega}^{+}) + (n_M-\delta) (\sum_{j= 1}^{N} x_{Mj} - y_{M \omega}^{+})\right] \\
  =& \sum_{j =1}^{N} \sum_{i=1}^M (n_i- \delta) x_{ij} - \sum_{\omega =1}^{|\Omega|} p_{\omega} \left(\sum_{i=1}^{M}(n_i- \delta)y_{i \omega}^{+} - \sum_{i=1}^{M-1}(n_i-\delta)y_{i+1, \omega}^{+}\right) \\
  =& \sum_{j =1}^{N} \sum_{i=1}^M i \cdot x_{ij} - \sum_{\omega =1}^{|\Omega|} p_{\omega} \sum_{i = 1}^{M} y_{i \omega}^{+}
\end{align*}

Here, $\sum_{j =1}^{N} \sum_{i=1}^M i \cdot x_{ij}$ indicates the maximum number of people that can be accommodated in the seat planning $\{x_{ij}\}$. The second part, $\sum_{\omega =1}^{|\Omega|} p_{\omega} \sum_{i = 1}^{M} y_{i \omega}^{+}$ indicates the expected excess supply for group type $i$ over scenarios.

% Plug in $s_i = i+1$, the objective function is $\sum_{j =1}^{N} \sum_{i=1}^m i x_{ij} - \sum_{\omega =1}^{S} p_{\omega} \sum_{i=1}^{m} y_{i \omega}^{+}$.
% The last equality holds because of $n_i- \delta = i, i \in \mathcal{M}$. 

In the optimal solution, at most one of $y_{i \omega}^{+}$ and $y_{i \omega}^{-}$ can be positive for any $i, \omega$. Suppose there exist $i_0$ and $\omega_0$ such that $y_{i_0 \omega_0}^{+}$ and $y_{i_0 \omega_0}^{-}$ are positive. Substracting $\min\{y_{i_0, \omega_0}^{+}, y_{i_0, \omega_0}^{-}\}$ from these two values will still satisfy constraints \eqref{DEF_constr1} and \eqref{DEF_constr2} but increase the objective value when $p_{\omega_0}$ is positive. Thus, in the optimal solution, at most one of $y_{i \omega}^{+}$ and $y_{i \omega}^{-}$ can be positive.

% \begin{prop}\label{prop_onescenario}
%   The deterministic problem \eqref{deter_upper} is a special case of stochastic programming when the number of scenarios $|\Omega|$ is equal to 1.
% \end{prop}
% That is to say, the optimal solution to the deterministic problem is one of the optimal solutions to the stochastic programming problem, while the optimal values remain unchanged.

Considering the analysis provided earlier, we find it advantageous to obtain a seat planning that only consists of full or largest patterns. However, the seat planning associated with the optimal solution obtained by solver to SSP may not consist of the largest or full patterns. We can convert the optimal solution to another optimal solution which is composed of the largest or full patterns.

\begin{prop}\label{prop_solution}
There exists an optimal solution to the stochastic programming problem such that the patterns associated with this optimal solution are composed of the full or largest patterns under any given scenarios.
\end{prop}

Proposition \ref{prop_solution} is different from proposition \ref{prop_construction}. Proposition \ref{prop_construction} can make sure that we can obtain a seat planning composed of full or largest patterns. Here, proposition \ref{prop_solution} states that there always exists an optimal solution where the corresponding patterns are either full or largest. However, we may not be able to directly find such an optimal solution as described in Proposition \ref{prop_solution}. Instead, we try to obtain the seat planning composed of full or largest patterns, as stated in Section \ref{seat_assignment}.

Then, we reformulate SSP$(\mathbf{L}, \Omega)$ in a matrix form to apply the Benders decomposition technique.
Let $\mathbf{n} = (n_1, \ldots, n_M)^{\intercal}$ represent the vector of seat sizes for each group type, where $n_i$ denotes the size of seats taken by group type $i$. Let $\mathbf{L} = (L_1, \ldots, L_N)^{\intercal}$ represent the vector of row sizes, where $L_j$ denotes the size of row $j$ as defined previously.
The constraint \eqref{DEF_constr3} can be expressed as $\mathbf{x}^{\intercal} \mathbf{n} \leq \mathbf{L}$. This constraint ensures that the total size of seats occupied by each group type, represented by $\mathbf{x}^{\intercal} \mathbf{n}$, does not exceed the available row sizes $\mathbf{L}$. We can use the product $\mathbf{x} \mathbf{1}$ to indicate the supply of group types, where $\mathbf{1}$ is a column vector of size $N$ with all elements equal to 1. 

The linear constraints associated with scenarios, denoted by constraints \eqref{DEF_constr1} and \eqref{DEF_constr2}, can be expressed in matrix form as:
\[\mathbf{x} \mathbf{1} + \mathbf{V} \mathbf{y}_\omega = \mathbf{d}_\omega, \omega\in \Omega,\]
where $\mathbf{V} = [\mathbf{W}, ~\mathbf{I}]$.

$$
\mathbf{W}=\left[\begin{array}{cccccc}
-1 & 1 & 0 & \ldots & \ldots & 0 \\
0 & -1 & 1 &    0   & \ldots & 0 \\
\vdots & \ddots & \ddots & \ddots & \ddots & \vdots \\
0  & \ldots   &  0  & -1 & 1 & 0 \\
0  & \ldots   &  \ldots  &  0 &  -1 & 1 \\
0 & \ldots & \ldots & \ldots & 0 & -1
\end{array}\right]_{M \times M}
$$

and $\mathbf{I}$ is the identity matrix with the dimension of $M$. For each scenario $\omega \in \Omega$,
$$
\mathbf{y}_{\omega}=\left[\begin{array}{l}
\mathbf{y}_{\omega}^{+} \\
\mathbf{y}_{\omega}^{-}
\end{array}\right], \mathbf{y}_{\omega}^{+}=\left[\begin{array}{lllll}y_{1 \omega}^{+} & y_{2 \omega}^{+} & \cdots & y_{M \omega}^{+}\end{array}\right]^{\intercal}, \mathbf{y}_{\omega}^{-}=\left[\begin{array}{llll}y_{1 \omega}^{-} & y_{2 \omega}^{-} & \cdots & y_{M \omega}^{-}\end{array}\right]^{\intercal}.
$$

As we can find, this deterministic equivalent form is a large-scale problem even if the number of possible scenarios $\Omega$ is moderate. However, the structured constraints allow us to simplify the problem by applying Benders decomposition approach. Before using this approach, we could reformulate this problem as the following form. Let $\mathbf{c}^{\intercal}\mathbf{x} = \sum_{j =1}^{N} \sum_{i=1}^M i \cdot x_{ij}$, $\mathbf{f}^{\intercal}\mathbf{y}_{\omega} = -\sum_{i=1}^{M} y_{i \omega}^{+}$. Then the SSP formulation can be expressed as below,

\begin{equation}\label{BD_master}
\begin{aligned}
\max \quad & \mathbf{c}^{\intercal} \mathbf{x}+ z(\mathbf{x}) \\
\text {s.t.} \quad & \mathbf{x}^{\intercal} \mathbf{n}  \leq \mathbf{L} \\
& \mathbf{x} \in \mathbb{N}^{M \times N},
\end{aligned}
\end{equation}
where $z(\mathbf{x})$ is defined as

$$z(\mathbf{x}) := E(z_{\omega}(\mathbf{x})) = \sum_{\omega \in \Omega} p_{\omega} z_{\omega}(\mathbf{x}),$$ and for each scenario $\omega \in \Omega$, 

\begin{equation}\label{BD_sub}
  \begin{aligned}
    z_{\omega}(\mathbf{x}) := \max \quad & \mathbf{f}^{\intercal} \mathbf{y}_{\omega} \\
    \text {s.t.} \quad & \mathbf{V} \mathbf{y}_{\omega} = \mathbf{d}_{\omega} - \mathbf{x} \mathbf{1} \\
     & \mathbf{y}_{\omega} \geq 0.
  \end{aligned}
\end{equation}

% The objective function of problem \eqref{sto_form} can be expressed as $c{'}\mathbf{x} + \sum_{\omega} p_{\omega}f{'}y_{\omega}$. 

We can solve problem \eqref{BD_master} quickly if we can efficiently solve problem \eqref{BD_sub}. Next, we will mention how to solve problem \eqref{BD_sub}.

\subsection{Solve SSP by Benders Decomposition}\label{solve_by_benders}
We reformulate problem \eqref{BD_master} into a master problem and a subproblem \eqref{BD_sub}. The iterative process of solving the master problem and subproblem is known as Benders decomposition. 
The solution obtained from the master problem provides inputs for the subproblem, and the subproblem solutions help update the master problem by adding constraints, iteratively improving the overall solution until convergence is achieved. Firstly, we generate a closed-form solution to problem \eqref{BD_sub}, then we obtain the solution to the LP relaxation of problem \eqref{BD_master} by the constraint generation.

\subsubsection{Solve The Subproblem}\label{second_stage}
% Consider a $\mathbf{x}$ such that $\mathbf{n x} \leq \mathbf{L}$ and $\mathbf{x} \geq 0$ and suppose that this represents the seat planning. 

Notice that the feasible region of the dual of problem \eqref{BD_sub} remains unaffected by $\mathbf{x}$. This observation provides insight into the properties of this problem. Let $\bm{\alpha}$ denote the vector of dual variables. For each $\omega$, we can form its dual problem, which is 

\begin{equation}\label{BD_sub_dual}
  \begin{aligned}
    \min \quad & \bm{\alpha}_{\omega}^{\intercal} (\mathbf{d}_{\omega}- \mathbf{x} \mathbf{1}) \\
    \text {s.t.} \quad & \bm{\alpha}_{\omega}^{\intercal} \mathbf{V} \geq \mathbf{f}^{\intercal}
  \end{aligned}
\end{equation}

% Let $\mathbb{P} = \{\bm{\alpha} \in \mathbb{R}^{M}|\bm{\alpha}^{\intercal} \mathbf{V} \geq \mathbf{f}^{\intercal}\}$.

\begin{lem}\label{feasible_region}
 The feasible region of problem \eqref{BD_sub_dual} is nonempty and bounded. Furthermore, all the extreme points of the feasible region are integral.
\end{lem}

Let $\mathbb{P}$ indicate the feasible region of problem $\eqref{BD_sub_dual}$. According to Lemma \ref{feasible_region}, the optimal value of the problem \eqref{BD_sub}, $z_{\omega}(\mathbf{x})$, is finite and can be achieved at extreme points of $\mathbb{P}$. Let $\mathcal{O}$ be the set of all extreme points of $\mathbb{P}$. Then, we have $z_{\omega}(\mathbf{x}) = \min_{\bm{\alpha}_{\omega} \in \mathcal{O}} \bm{\alpha}_{\omega}^{\intercal}(\mathbf{d}_{\omega}- \mathbf{x} \mathbf{1})$.

% We assume that $\mathbb{P}$ is nonempty and has at least one extreme point.  Then, either the dual problem \eqref{BD_sub_dual} has an optimal solution and $z_{\omega}(\mathbf{x})$ is finite, or the primal problem \eqref{BD_sub} is infeasible and $z_{\omega}(\mathbf{x}) = \infty$.  


Alternatively, $z_{\omega}(\mathbf{x})$ is the largest number $z_{\omega}$ such that $\bm{\alpha}_{\omega}^{\intercal}(\mathbf{d}_{\omega}- \mathbf{x} \mathbf{1}) \geq z_w, \forall \bm{\alpha}_{\omega} \in \mathcal{O}$. We use this characterization of $z_w(\mathbf{x})$ in problem \eqref{BD_master} and conclude that problem \eqref{BD_master} can thus be put in the form by setting $z_w$ as the variable:

\begin{equation}\label{BD_master2}
  \begin{aligned}
    \max \quad & \mathbf{c}^{\intercal} \mathbf{x} + \sum_{\omega \in \Omega} p_{\omega} z_{\omega} \\
    \text {s.t.} \quad & \mathbf{x}^{\intercal} \mathbf{n}  \leq \mathbf{L} \\
    & \bm{\alpha}_{\omega}^{\intercal}(\mathbf{d}_{\omega}- \mathbf{x} \mathbf{1}) \geq z_{\omega}, \forall \bm{\alpha}_{\omega} \in \mathcal{O}, \forall \omega \\
     & \mathbf{x} \in \mathbb{N}^{M \times N}
  \end{aligned}
\end{equation}

% Because the feasible region is bounded, then feasibility cuts are not needed. Let $z_{\omega}$ be the lower bound of $z_{\omega}(x)$ such that $\bm{\alpha}^{\intercal}(\mathbf{d}_{\omega}- \mathbf{x} \mathbf{1}) \geq z_{\omega}, \bm{\alpha} \in \mathcal{O}$, which is the optimality cut.

% \begin{corollary}\label{coro_1}
%   Only the optimality cuts, $\alpha^{\intercal}(\mathbf{d}_{\omega}- \mathbf{x} \mathbf{1}) \geq z_{\omega}$, will be included in the decomposition approach.
% \end{corollary}


% When $\mathbf{x}$ is fixed, we can give the optimal solution to problem \eqref{BD_sub_dual} rather than solving this linear programming. 


Before applying Benders decomposition to solve problem \eqref{BD_master2}, it is important to address the efficient computation of the optimal solution to problem \eqref{BD_sub_dual}.
When we are given $\mathbf{x}^{*}$, the demand that can be satisfied by the seat planning is $\mathbf{x}^{*} \mathbf{1} = \mathbf{d}_0 = (d_{1,0},\ldots,d_{M,0})^{\intercal}$.
By plugging them in the subproblem \eqref{BD_sub}, we can obtain the value of $y_{i, \omega}$ recursively:
\begin{equation}\label{y_recursively}
\begin{aligned}
  & y_{M \omega}^{-}=\left(d_{M \omega}-d_{M 0}\right)^{+} \\
  & y_{M \omega}^{+}=\left(d_{M 0}-d_{M \omega}\right)^{+} \\
  & y_{i \omega}^{-}=\left(d_{i \omega}-d_{i 0} - y_{i+1, \omega}^{+} \right)^{+}, i =1,\ldots, M-1 \\
  & y_{i \omega}^{+}=\left(d_{i 0}- d_{i \omega} + y_{i+1, \omega}^{+}\right)^{+}, i =1,\ldots, M-1
\end{aligned}
\end{equation}

The optimal solutions to problem \eqref{BD_sub_dual} can be obtained according to the value of $\mathbf{y}_{\omega}$.

\begin{prop}\label{optimal_sol_sub_dual}
  The optimal solutions to problem \eqref{BD_sub_dual} are given by 
\begin{equation}\label{BD_sub_simplified}
  \begin{aligned}
    \alpha_{i} = 0 \quad & \text{if}~  y_{i \omega}^{-} > 0,  i =1,\ldots, M~\text{or}~ y_{i \omega}^{-} = y_{i \omega}^{+} = 0, y_{i+1, \omega}^{+}> 0, i = 1,\ldots, M-1 \\
    \alpha_{i} = \alpha_{i-1}+1 \quad & \text{if}~ y_{i \omega}^{+} > 0, i =1,\ldots, M \\
    0 \leq \alpha_{i} \leq \alpha_{i-1}+1 \quad & \text{if}~ y_{i \omega}^{-} = y_{i \omega}^{+} = 0, i = M~\text{or}~ y_{i \omega}^{-} = y_{i \omega}^{+} = 0, y_{i+1, \omega}^{+}= 0, i = 1,\ldots, M-1
  \end{aligned}
\end{equation}
\end{prop}


Instead of solving this linear programming directly, we can compute the values of $\alpha_{\omega}$ by performing a forward calculation from $\alpha_{1\omega}$ to $\alpha_{M\omega}$.

\subsubsection{Constraint Generation}\label{bender_stage}
% Benders decomposition works with only a subset of those exponentially many constraints and adds more constraints iteratively until the optimal solution of Benders Master Problem(BMP) is attained. This procedure is known as delayed constraint generation.

Due to the computational infeasibility of solving problem \eqref{BD_master2} with an exponentially large number of constraints, it is a common practice to use a subset, denoted as $\mathcal{O}^t$, to replace $\mathcal{O}$ in problem \eqref{BD_master2}. This results in a modified problem known as the Restricted Benders Master Problem (RBMP). To find the optimal solution to problem \eqref{BD_master2}, we employ the technique of constraint generation. It involves iteratively solving the RBMP and incrementally adding more constraints until the optimal solution to problem \eqref{BD_master2} is obtained.


We can conclude that the RBMP will have the form:

\begin{equation}\label{BD_master3}
  \begin{aligned}
    \max \quad & \mathbf{c}^{\intercal} \mathbf{x} + \sum_{\omega \in \Omega} p_{\omega} z_{\omega} \\
    \text {s.t.} \quad & \mathbf{x}^{\intercal} \mathbf{n}  \leq \mathbf{L} \\
    & \bm{\alpha}_{\omega}^{\intercal}(\mathbf{d}_{\omega}- \mathbf{x} \mathbf{1}) \geq z_{\omega}, \bm{\alpha}_{\omega} \in \mathcal{O}^{t}, \forall \omega \\
     & \mathbf{x} \in \mathbb{N}^{M \times N}
  \end{aligned}
\end{equation}

Given the initial $\mathcal{O}^{t}$, we can have the solution $\mathbf{x}^{*}$ and $\mathbf{z}^{*} =(z^{*}_1,\ldots, z^{*}_{|\Omega|})$. Then $c^{\intercal} \mathbf{x}^{*} + \sum_{\omega \in \Omega} p_{\omega} z_{\omega}^{*}$ is an upper bound of problem \eqref{BD_master3}. When $\mathbf{x}^{*}$ is given, the optimal solution, $\bm{\tilde{\alpha}}_{\omega}$, to problem \eqref{BD_sub_dual} can be obtained according to Proposition \ref{optimal_sol_sub_dual}. Let $\tilde{z}_{\omega} = \bm{\tilde{\alpha}}_{\omega}(d_{\omega} - \mathbf{x}^{*} \mathbf{1})$, then $(\mathbf{x}^{*}, \mathbf{\tilde{z}})$ is a feasible solution to problem \eqref{BD_master3} because it satisfies all the constraints. Thus, $\mathbf{c}^{\intercal} \mathbf{x}^{*} + \sum_{\omega \in \Omega} p_{\omega} \tilde{z}_{\omega}$ is a lower bound of problem \eqref{BD_master2}.

If for every scenario $\omega$, the optimal value of the corresponding problem \eqref{BD_sub_dual} is larger than or equal to $z_{\omega}^{*}$, which means all contraints are satisfied, then we have an optimal solution, $(\mathbf{x}^{*}, \mathbf{z}^{*})$, to problem \eqref{BD_master2}. However, if there exists at least one scenario $\omega$ for which the optimal value of problem \eqref{BD_sub_dual} is less than $z_{\omega}^{*}$, indicating that the constraints are not fully satisfied, we need to add a new constraint $(\bm{\tilde{\alpha}}_{\omega})^{\intercal}(\mathbf{d}_{\omega} - \mathbf{x} \mathbf{1}) \geq z_{\omega}$ to RBMP.

% $z_{\omega}^{(0)} = \alpha_{\omega}^{*}(d_{\omega} - \mathbf{x}_0 \mathbf{1})$ will give a minimal upper bound of $z_{\omega}$, thus all the left constraints associated with other extreme points are redundant.when the extreme points are $\alpha_{\omega}$.

% The problem \eqref{lemma_eq} associated with $\alpha_{\omega}$ will give an optimal solution $(x_1, z_{\omega}^{1})$. (Upper bound)

To determine the initial $\mathcal{O}^{t}$, we have the following proposition.

\begin{prop}\label{one_ep_feasible}
RBMP is always bounded with at least one constraint for each scenario.
\end{prop}

From Proposition \ref{one_ep_feasible}, we can set $\bm{\alpha}_{\omega} = \mathbf{0}$ initially. Notice that only contraints are added in each iteration, thus $UB$ is decreasing monotone over iterations. Then we can use $UB - LB < \epsilon$ to terminate the algorithm.

\begin{algorithm}[h]
  \caption{Benders Decomposition}\label{cut_algo}
  \KwIn{Initial problem \eqref{BD_master3} with $\bm{\alpha}_{\omega} = 0, \forall \omega$, $LB = 0$, $UB = \infty$, $\epsilon$.}
  \KwOut{$\mathbf{x}^{*}$}
  \While{$UB - LB > \epsilon$}
    {Obtain $(\mathbf{x}^{*}, \mathbf{z}^{*})$ from problem \eqref{BD_master3}\;
    $UB \gets c^{\intercal} \mathbf{x}^{*} + \sum_{\omega \in \Omega} p_{\omega} z_{\omega}^{*}$\;
    \For{$\omega= 1, \ldots, |\Omega|$}
    {Obtain $\bm{\tilde{\alpha}}_{\omega}$ from Proposition \ref{optimal_sol_sub_dual}\; $\tilde{z}_{\omega}= (\bm{\tilde{\alpha}}_{\omega})^{\intercal}(\mathbf{d}_{\omega}- \mathbf{x}^{*} \mathbf{1})$\;
    \If{$\tilde{z}_{\omega} < z_{\omega}^{*}$}
    {Add one new constraint, $(\bm{\tilde{\alpha}}_{\omega})^{\intercal}(\mathbf{d}_{\omega}- \mathbf{x} \mathbf{1}) \geq z_{\omega}$, to problem \eqref{BD_master3}\;}
    }
    {$LB \gets c^{\intercal} \mathbf{x}^{*} + \sum_{\omega \in \Omega} p_{\omega} \tilde{z}_{\omega} $\;}
    }
\end{algorithm}


% After the algorithm terminates, we obtain the optimal $\mathbf{x}^{*}$. The demand that can be satisfied by the arrangement is $\mathbf{x}^{*} \mathbf{1} = \mathbf{d}_0 = (d_{1,0},\ldots,d_{M,0})^{\intercal}$. 

However, solving problem \eqref{BD_master3} even with the simplified constraints directly can be computationally challenging in some cases, so practically we first obtain the optimal solution to the LP relaxation of problem \eqref{BD_master}. Then, we generate an integral seat planning from this solution.

\subsection{Obtain The Feasible Seat Planning}\label{seat_assignment}
We may obtain a fractional optimal solution when we solve the LP relaxation of problem \eqref{BD_master}. This solution represents the optimal allocations of groups to seats but may involve fractional values, indicating partial assignments. Based on the fractional solution obtained, we use the deterministic model to generate a feasible seat planning. The objective of this model is to allocate groups to seats in a way that satisfies the supply requirements for each group without exceeding the corresponding supply values obtained from the fractional solution. To accommodate more groups and optimize seat utilization, we aim to construct a seat planning composed of full or largest patterns based on the feasible seat planning obtained in the last step. 


% This involves adjusting the allocation of seats within the seat planning to maximize the utilization of available seats. The goal is to create patterns where each row is either completely filled or consists of the largest possible group.

% these constraints while potentially resulting in patterns that are not necessarily full or largest.

% The decomposition method only gives a fractional solution and the stochastic model does not provide an appropriate seat planning when the number of people in scenario demands is way smaller than the number of the seats.
% The objective is to obtain the maximal number of people placed according to the demand scenarios. It will not provide an appropriate seat assignment when the number of people associated with scenario demands is way less than the number of available seats because there are multiple optimal solutions and the solution given by solver probably does not utilize all the empty seats.

% Suppose that each row has the same length. Then the optimal integrated solution 
% $(0,\ldots, x,d_{h+1}, \ldots, d_{m})$ has the same objective value with an integer solution. Deciding if these items can fit into a specified number of rows is the decision problem of the bin-packing problem. If the items associated with the integer solution can be put in the given number of rows, this solution is optimal; otherwise, it is not optimal. Since the bin-packing problem is NP-hard, problem \eqref{deter_upper} is also NP-hard.

% its objective is to obtain the maximal number of people served, not the optimal seat assignment. It will not provide an appropriate solution when the number of arriving people in the scenarios is way less than the number of total seats because it does not utilize all the empty seats.

Let the optimal solution to the LP relaxation of problem \eqref{BD_master3} be $\mathbf{x}^{*}$. Aggregate $\mathbf{x}^{*}$ to the number of each group type, $\tilde{X}_{i} =\sum_{j} x^{*}_{ij}, \forall i \in \mathbf{M}$. Solve the SPP$(\{\tilde{X}_{i}\})$ to obtain the optimal solution, $\mathbf{\tilde{x}}$, and the corresponding pattern, $\bm{H}$, then generate the seat planning by problem \eqref{improve_seat} with $\bm{H}$.


% To maximize the utilization of seats, we should assign full or largest patterns to each row. 
% We will check if every row is full. When row $j$ is not full, i.e., $\sum_{i} n_{i} x_{ij} < L_{j}$, let $\beta = L_{j} - \sum_{i} n_{i} x_{ij}$. If $\beta$ is no less than $n_M$, then let $\beta = \beta - n_M$, $s_M = s_M +1$ until $\beta$ is less than $n_M$. If row $j$ is still not full, then find the smallest group size in row $j$ and mark it as $i^0$. If the smallest group is exactly the largest, then the row represents a largest pattern and check next row. Otherwise, reduce the number of group type $i^0$ by one and increase the number of group type $\min \{(i^0+\beta), M\}$ by one. Continue this procedure until this row is full.


\begin{algorithm}
  \caption{Seat Planning Construction}\label{seat_construction}
  % \KwIn{Scenarios set $\Omega$, Seat layout}
  % \KwOut{$\mathbf{x}, \bm{H}$}
    {Obtain the optimal solution, $\mathbf{x}^{*}$, from the LP relaxation of problem \eqref{BD_master3}\;}
    {Aggregate $\mathbf{x}^{*}$ to the number of each group type, $\tilde{X}_{i} = \sum_{j} x^{*}_{ij}, i \in \mathbf{M}$\;}
    {Obtain the optimal solution, $\tilde{\mathbf{x}}$, and the corresponding pattern, $\bm{H}$, from SPP$(\{\tilde{X}_{i}\})$\;}
    {Construct the seat planning by problem \eqref{improve_seat} with $\bm{H}$\;}
\end{algorithm}
