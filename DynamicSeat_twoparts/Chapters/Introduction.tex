% !TEX root = sum1.tex
\section{Introduction}
% Since the onset of the pandemic, governments have implemented a variety of measures to contain the spread of the virus. Many of these measures have involved extensive testing and verification efforts to assess the effectiveness of different interventions. In this context, we analyze optimization techniques that can be used to assess the impact of social distancing requirements on seating arrangements and venue operations. The goal is to explore the practical implementation and optimization of social distancing measures to balance public health considerations with operational efficiency.

% 疫情过后,是时候总结经验教训,对一些疫情中的政策进行分析和评估,比如社交距离的需求,核酸检测,隔离等措施。我们的工作集中在社交距离的分析上,通过对相关政策的研究和具体的实施,可以得到更有价值的insight,帮助政府提供工具以及更好实行决策以减少由社交距离的政策所带来的负面影响。

% 总结经验,进一步分析;/检验;比如:social distancing, 检测试剂, 隔离等措施,我们的工作集中在SD, 系统的研究,具体的实施,有着更广阔的影响。疫情过后,是时候研究措施,对于政府来讲,传染性风险多大;另一方面,考虑对社会的影响;我们的工作focus on 在这类影响上。 我们为政府提供工具来分析实行social distancing 


After the pandemic, it is crucial to sum up the lessons learned and evaluate the policies implemented during this time, including social distancing, nucleic acid testing, and quarantine measures. Our focus will be on analyzing the requirements for social distancing. By examining relevant policies and their specific implementations, we aim to gain valuable insights that can assist the government in providing tools and making informed decisions to mitigate the negative impacts of social distancing policies. Our work aims to provide the government with targeted decision-making support so that they can adopt the optimal social distancing measures based on actual circumstances. This approach will help protect public health while minimizing damage to the socioeconomic system. Furthermore, it will enhance the overall effectiveness of pandemic control and better prepare us for future public health crises.

Social distancing is a proven concept to contain the spread of an infectious disease. As a general principle, social distancing measures can be implemented in various forms. The basic requirement of social distancing is the specification of a minimum physical distance between people in public areas. For example, the World Health Organization (WHO) suggests social distancing as to ``keep physical distance of at least 1 meter from others'' \cite{AdviceforPublic}. In the US, the Center for Disease and Control (CDC) refers to social distancing as ``keeping a safe space between yourself and other people who are not from your household'' \cite{CDC}. 

Note that under such a requirement, social distancing is applied concerning groups of people instead of each individual. In Hong Kong, the government has adopted social distancing measures, in the Covid 19 pandemic, by limiting the size of groups in public gathering to two, four, and six people per group over time. Moreover, the Hong Kong government has also adopted an upper limit of the total number of people in a venue; for example, restaurants can operate at 50\% or 75\% of their normal seating capacity.

% Note that under such a requirement, social distancing is actually applied with respect to groups of people. In Hong Kong, the government has adopted social distancing measures, in the recent Covid 19 pandemic, by limiting the size of groups in public gathering to two, four, and six people per group over time. Moreover, the Hong Kong government has also adopted an upper limit of the total number of people in a venue; for example, restaurants can operate at 50\% or 75\% of their normal seating capacity.

While the above practice of social distancing measures has been recognized for containing the spread of an infectious disease,  it is not clear how the entire economy will be affected in terms of the change of operations. This is an important issue in the service sector where social distancing implies fewer clients and lower revenue. The situation is especially complicated under multiple social distancing measures, such as physical distance between groups, limit on the size of groups, and the occupancy rate of the venue. This naturally raises questions regarding the relationship of these measures, e.g., which is relatively more effective under different conditions, whether they complement or contradict each other, and more importantly, is it possible to align these measures so that they can be implemented coherently? 


% The social distancing requirement.
% The measures implemented by the Hong Kong government primarily concentrate on restricting social distancing, group sizes and occupancy rates. However, implementing these policies in practice can pose challenges, particularly for fixed seating layouts with dynamic arrivals of people. In the original commercial case without social distancing requirements, customers did not need to sit together, so the focus was solely on total capacity. Under social distancing constraints, placing groups in row runs the risk of being unable to find matching demand, potentially leaving empty seats.

The distancing measures discussed can also be applied to other domains, such as the arrangement of animal cages and the placement of GPUs. During customs quarantine, different types of animals from various regions need to maintain a certain distance to meet epidemic prevention requirements. Implementing a reasonable distance-based placement can ensure safety while not occupying an excessive amount of space. Similarly, with the rapid development of big data and artificial intelligence, a large number of GPUs are required as computational tools. GPU clustering can lead to overheating, which is detrimental to heat dissipation and affects operational efficiency. Arranging the placement of GPUs with appropriate distances can help address this issue and improve the overall system performance.

We will address the above issues of social distancing in the context of seating arrangement in a venue, such as a cinema or a conference hall. The venue is equipped with seats of multiple rows.  People come in groups where each group of people will sit consecutively in one row. 
To avoid confusion, we discuss two related terms `seat planning' and `seat assignment' which will be used in the following parts. In our context, the seat planning is to determine the seat partition with the known customer distribution. The planning can be altered later when the planned seats don't match the size of a coming group or when the seat planning is disrupted after assigning a coming group. In the seat assignment, for the coming group, when accepting it, we assign the seats to the group, and the seats will not be used by others in the future.

In order to adhere to social distancing guidelines, it is important to understand the process of generating seat planning based on known groups and how to assign seats to incoming groups. Additionally, it is of interest to explore how the social distancing constraints impact the sellers and the specific policies formulated by the government to address social distancing concerns.

We intend to shed light on the problem described and propose a practical dynamic seat assignment policy. In particular, we investigate the following questions. 

1. How can we model the seat planning problem given the social distancing restrictions? What kind of property does this problem have? How can we give a seat planning to accommodate the maximum people with stochastic demand?

2. How to use the property of seat planning problem to design the dynamic seat assignment policy? How good is the performance of this policy compared with other policies?

3. What kind of insights regarding the social distancing and occupancy rates can we obtain when implementing the dynamic seat assignment policy?


To answer these questions, we construct the seat planning problem with deterministic and stochastic demand under social distancing requirement. For the deterministic situation, we have complete and accurate information about the demand for seating. We aim to provide a seat planning that maximizes the number of people accommodated. This situation is applicable in venues like churches or company meetings, where fixed seat layouts are available, and the goal is to assign seats to accommodate as many people as possible within the given layout. The seat planning obtained shows the utilization of all seats as many as possible. Thus, we introduce the concept of full or largest pattern to indicate the seat partition of each row. For the seat planning that does not utilize all available seats, we propose to improve the seat planning by incorporating full or largest patterns.

For the stochastic situation, we know that the demand distribution before the actual demand is realized. We aim to generate a seat planning that maximizes the expected number of people accommodated. This approach is suitable for venues where seats have been pre-allocated to ensure compliance with social distancing rules. With the given demand scenarios, we develop the scenario-based stochastic programming to obtain the seat planning. To solve this problem efficiently, we apply the Benders decomposition technique. However, in some cases, solving the integer programming with Benders decomposition remains still computationally prohibitive. Thus, we can consider the LP relaxation then obtain a feasible seat planning by deterministic model. Based on that, we construct a seat planning composed of full or largest patterns to utilize all seats fully.

We mainly focus on addressing the dynamic seat assignment problem with a given set of seats in the context of social distancing. Solving the problem by dynamic programming can be prohibitive due to the curse of dimensionality, which arises when the problem involves a large number of variables or states. To mitigate this complexity, we begin by generating a seat planning in the stochastic situation. This seat planning acts as a foundation for the seat assignment. Then, we develop the dynamic seat assignment policy which guides the allocation of seats to the incoming groups sequentially. In the numerical result, our policy performs well compared with other policies. 
We use $\tilde{T}$ to denote the gap point, which refers to the first period at which, on average, the number of people accepted without social distancing is not less than that accepted with social distancing plus one. By sampling many probability combinations, the results show that $\tilde{T}$ and the corresponding occupancy rate, $\beta(\tilde{T})$, can be estimated with $\gamma$, the expected number of people per period. Different $\gamma$ corresponds to different $\tilde{T}$. When the total number of periods, $T$, is less than $\tilde{T}$, we tend to accept all incoming groups. In this case, whether to implement the social distancing restriction is no difference. When $T$ is larger, there will be more groups rejected when implementing the social distancing requirement. The government can consider the potential losses when making policies regarding group size and occupancy rate. Similarly, the seller can implement corresponding measures to adhere to these requirements.


Our main contributions in this paper are summarized as follows. First, this study presents the first attempt to consider the arrangement of seat assignments with social distancing under dynamic arrivals. Our study provides a new perspective to help the government adopt a mechanism for setting seat assignments to implement social distancing during pandemic.

% While many studies in the literature highlight the importance of social distancing in controlling the spread of the virus, they often focus too much on the model and do not provide much insight into the operational significance behind social distancing \cite{barry2021optimal, fischetti2023safe}. Recent studies have explored the effects of social distancing on health and economics, mainly in the context of aircraft \cite{salari2020social, ghorbani2020model, salari2022social}. 

Second, we establish a deterministic model to analyze the effects of social distancing when the demand is known. Due to the medium size of the problem, we can solve the IP model directly. We then develop the scenario-based stochastic programming by considering the stochastic demands of different group types. By using Benders decomposition methods, we can obtain the seat planning quickly. 

Third, to address the problem in the dynamic situation, we first obtain a feasible seat planning from scenario-based stochastic programming. We then decide for each incoming group based on our dynamic seat assignment policy, either accepting or rejecting the group. Our results demonstrate a significant improvement over the traditional control policies and provide insights on the implementation of social distancing.


The rest of this paper is structured as follows. The following section reviews relevant literature. We describe the motivating problem in Section 3. In Section 4, we establish the stochastic model,analyze its properties and obtain the seat planning. Section 5 demonstrates the dynamic seat assignment policy to assign the seats for incoming groups. Section 6 gives the numerical results and the insights of implementing social distancing. The conclusions are shown in Section 7.
\newpage
