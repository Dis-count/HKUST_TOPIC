% !TEX root = sum1.tex
\section{Model}
% There are some rules: time Distancing:  The time gap of each used room. Space Distancing:  Enlarge the space distancing of each room as much as possible. 

% That is, if we have a room contains $q_k$ seat numbers, and distance ratio is r, then the number of customers who can be served $p_i$ is less than $q_k\cdot r$. This condition can be set as a constraint.

Virables:

Room numbers $k \in K = \{1,\ldots,|K|\}$. 

Room $k$ contains seat number $q_k$. 

Number of customers in session $i$ is $p_i$ for each $i \in N = \{1,\ldots,n\}$. $w_{ik}$ is the session i's start time in the room $k$. $s_i$ is the service time for session $i$.(Given, in our case, we set all the service times are 2 hours.)

Feasibility:

% Time window constraints:
% Time window $[a_{i},b_{i}]$ for each group, but it satisfies the time constraints during opening time $[E, L]$ for the room.

Capacity constraint:
The number of customers in a session $p_i$ cannot exceed the product of the largest room capacity and ratio r, that is $p_i \leq q_k\cdot r$.

Solution:

Define the time interval $t_{i}$ for session $i$. It can be variable or the constant. In our case, we set the time interval as the variable and it should be larger than half an hour.

Define a binary variable $x_{ijk}$ for each room. If the room k is used by $(i,j)$ and $i$ followed by $j$, then $x_{ijk} = 1$, else $x_{ijk} = 0$.

Define
$w_{ik}$ is the session i's start time in the room $k$.

$s_i$ is the service time for each session.(Given)

% Set it as a time window VRP problem and add the distance constraints.

Analysis:

Add two virtual nodes (0,n+1) for each room. One is the start node, its time window can be a time point E meaning the room is open; the other is the end node, its time window is also a time point L meaning the room is closed.

Expected result:
Show the specific scheduling for the sessions.

Give the corresponding service start time.

% First in First out.

Benchmark: Manual work.

Question: How to determine the objective function?

        %   How to determine the distance for the only one group in a room?

          How to compare the result with the benchmark?


MODEL:

\begin{align}
\min_{i,j,k} \quad & \sum_{(i,j) \in A} \sum_{k \in K} c_{ij} x_{ijk} \\
s.t. \quad  & \sum_{k \in K} \sum_{j \in \delta^+ (i)} x_{ijk} =1 & \forall i \in N  \\
& \sum_{j \in \delta^+ (0)} x_{0jk} =1 & \forall k \in K \\
& \sum_{i \in \delta^- (n+1)} x_{i,n+1,k} =1 & \forall k \in K \\
& \sum_{i \in \delta^- (j)} x_{ijk} - \sum_{i \in \delta^+ (j)} x_{ijk} = 0  & \forall k \in K, j \in N \\
& w_{ik} + s_i + t_{i} - w_{jk} \leq (1-x_{ijk}) M_{ij} & \forall k \in K, (i,j) \in A \\
% & a_i \sum_{j \in \delta^+ (i)} x_{ijk} \leq w_{ik} \leq b_i \sum_{j \in \delta^+ (i)} x_{ijk} & \forall k \in K, i \in N \\
& w_{0k}=E, w_{n+1,k}=L  & \forall k \in K \\
& t_{i} \geq 0.5 \sum_{j \in \delta^+(i)} x_{ijk}  & \forall k \in K, i \in N  \\
& p_i \sum_{j \in \delta^+ (i)} x_{ijk} \leq 0.3 q_k & \forall k \in K, i \in N \\
& x_{ijk} \in \{0,1\} & \forall k \in K, (i,j) \in A
\end{align}

The constraint (1) is to minimize the cost resulted by opening sessions.

The constraint (2) Every session i which is followed by session j is only served once by one room k.

The constraint (3) For every room k, start from session 0.

The constraint (4) For every room k, end at session (n+1).

The constraint (5) For every room k, session j will leave when it is served.

The constraint (6) session i start time + service time + interval(required) less than next session j start time. M for linearization.

% The constraint (7) Time window constraints for every session.?

The constraint (7) Add two node indicate the start node and end node.

The constraint (8) Time interval constraint.

The constraint (9) Space distance constraint.


\subsection{M0}
Maximize the distance.

Input: Service time $s_i$ for each session instead of the time window [a,b].

Add $p_0 = 0, s_0 = 0, E = 0/8, L=24.$

MODEL:


\begin{align}
max_{i,j,k} \quad & \sum_{(i,j) \in A} \sum_{k \in K} \frac{p_i}{q_k} x_{ijk} + \frac{1}{24} T_{ijk} \\
s.t. \quad  & \sum_{k \in K} \sum_{j \in \delta^+ (i)} x_{ijk} =1 & \forall i \in N  \\
& \sum_{j \in \delta^+ (0)} x_{0jk} =1 & \forall k \in K \\
& \sum_{i \in \delta^- (n+1)} x_{i,n+1,k} =1 & \forall k \in K \\
& \sum_{i \in \delta^- (j)} x_{ijk} - \sum_{i \in \delta^+ (j)} x_{ijk} = 0  & \forall k \in K, j \in N \\
& y_{ijk} \geq (x_{ijk}-1) M_{ij} & \forall k \in K, (i,j) \in A \\
& w_{0k}=E, w_{n+1,k}=L  & \forall k \in K \\
& x_{ijk} \in \{0,1\} & \forall k \in K, (i,j) \in A
\end{align}

How to change the quadratic terms to the linear terms(linearization)

Note that the
$y =x_1 x_2$  where $x_1 \in \{0,1\}, x_2 \in [l,u] \to$
$$
\begin{aligned}
& y \leq x_2 \\
& y \geq x_2 - u(1-x_1)    \\
& l x_1 \leq y \leq u x_1
\end{aligned}$$

Let $(w_{jk}-w_{ik}-s_i) = y_{ijk}$ and $T_{ijk} = x_{ijk} y_{ijk}$

% 多加三个 i*j*k 数量的约束. Let $l = 0$.

$$
\begin{aligned}
& T_{ijk} \leq y_{ijk} \\
& T_{ijk} \geq y_{ijk} - u(1-x_{ijk})    \\
& T_{ijk} \leq u x_{ijk}
\end{aligned}$$

The constraint (1) Maximize the distance.

The constraint (2) Every session i which is followed by session j is only served once by one room k.

The constraint (3) For every room k, start from group 0.

The constraint (4) For every room k, end at group (n+1).

The constraint (5) For every room k, group j will leave when it is served.

The constraint (6) i start time + service time + interval(required) < next j start time. M for linearization.

The constraint (7) Time window constraints for every group. 

The constraint (8) Add two node which indicate the start node and end node.


---------------------------------------------------------
可行性:

影院有自主排片权

排片的重要性:

排片决定观影人次数;

决定收入:、票房、食品酒水、衍生品、广告场租;

排片是影院的核心竞争力;

Factors to consider:

Sequence/Assignment/Time/Price

1. 黄金时间段是上座率最高的时间段,两小时内每厅至少排一场, 黄金时间段会随季节以及地域变化

2. 将票房最高的影片,在最黄金时间,排入最优最大厅

3. 把黄金时间段、黄金厅给黄金影片,从黄金向两边排起

4. 如果给一部电影相邻两个场次,最好两个影厅一大一小

5. 在同一个厅里,尽量避免插排两部不同的影片 A/B/A/B/A/B,只需安排同部电影 AAA,或 BBB 即可

6. 避免同时开场、同时散场。最短场间隔10分钟

7. 中文版、英文版都有的进口片,要根据观众的偏好排映

8. 进口动画影片,白天应更多安排中文版场次

9. 周六周日节假日应多排大片,票价可以考虑适当提高

10. 每日开场时间:节假日和有大片时可以适当提前,六一儿童节应提前早场开映时间

11. 影片每日结场时间:周五、周六、及各种节假日可以考虑推迟

12. 考虑成本控制。普通厅放映成本: 4张票. 巨幕厅:9张票.

Existing problems

1. 排片时缺少预估

2. 人流情况预估不够细化

3. 场次安排很少有影院个性化估算

4. 除最大厅和最小厅外,其他厅的安排随意

--------------------------------------------------

self-study room (Reusable resources)

有social distance, 不考虑 group 比较简单,考虑 group 不符合实际。 