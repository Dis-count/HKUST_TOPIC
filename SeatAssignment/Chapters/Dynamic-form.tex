% !TEX root = sum1.tex
\section{Dynamic demand situation}\label{dynamic_demand}

We also study the dynamic seating plan problem, which is more suitable for commercial use. In this situation, customers come dynamically, and the seating plan needs to be made without knowing the number and composition of future customers. 

It becomes a sequential stochastic optimization problem where conventional methods fall into the curse of dimensionality due to many seating plan combinations. To avoid this complexity, we develop an approach that aims directly at the final seating plans. Specifically, we define the concept of target seating plans deemed satisfactory. In making the dynamic seating plan, we will try to maintain the possibility of achieving one of the target seating plans as much as possible.

\subsection{Seat assignment scenarios and methods}
In this section, we will present several methods to assign seats under different scenarios.

There are three senarios:

1. The seat assignment will be arranged in advance, the groups only need to find the corresponding-size seats. This scenario applies for the reservation without seat selection. For example, some theaters and concerts only provide the seat reservation for the audiences. The movie hall can assign the seats without changing to save costs in one day because the same film genre will attract the same feature of different group types.

2. The groups can only select which row they sit. This scenario appears in the on-site seminar. 


3. Online booking. (Can select arbitrarily but with some constraints.)

When the customers book the tickets, they will be asked how many seats they are going to book at first. Then we give the possible row numbers for their selection. Finally we give the seat number for their choice.

The second step is based on the other choices of reserved groups, just need to check which rows in the seat assignment include the corresponding-size seats.

The third step need to check whether the group destroies the assignment. Use subset sum problem to check every position in the row.

There are six methods basically:

M1:
The intuitive but trivial method will be first-come-first-serve. Each request will be assigned row by row. When the capacity of one row is not enough for the request, we arrange it in the next row. If the following request can take up the remaining capacity of some row exactly, we place it in the row immediately. We check each request until the capacity is used up. We set this result as the baseline.

It can be used in scenario 1,2,3. For scenario 3, the performane will be worse without restriction.

% Set the maximal people from the offline sequence.

M2: The seat assignment(supply) is obtained from stochastic model. Then make the decision according to the nested policy  

It can be used in scenario 1,2,3. The seats can be placed in the cinema hall in advance.
% unsettle the useless seats.

M3: DP-based. We relax several rows to one row with the same capacity. Suppose there always exists one assignment under the capacity. Then we can use DP to make the decision in each period.

$$V(S,T) = \sum_{i \in N} p_i \max\{ {[V((S-s_i-1),T-1)+ s_i]}, {V(S,T-1)}\}$$

After obtaining the accpected sequence, we still need to check whether this sequence is feasible for the seat assignment. For most cases, it is feasible. That is the reason why we choose DP. When it is not feasible, we should delete the group one by one from the last arrival of the sequence until it is feasible.

% In practice, we can use a buffer to contain the accpected groups, when the groups can be assigned to a full row, then remove them from the buffer. When 

It can only be applied in scenario 1.

M4: Based on first-come-first-serve. For the arrival sequence, calculate the sum 

find the last arrival when the sum of the preceding arrivals does not exceed the capacity.

but use the deterministic model to update. 

It can be used in scenario 1,2. 


M5: 

Set the mean demand as the initial supply, update the supply from deterministic model by setting accpected demand as the lower bound. / can be used in scenario 1,2.

M6: 

The inital supply is obtained from stochastic model, update the supply from deterministic model by setting accpected demand as the lower bound. / Similar to the above.




If we need to assign seats to the group each period.

1.1 Every group can only choose which row to sit.
In each period, every group can choose to sit in some row with the corresponding capacity. 
After certain periods, we update the remaining seats in each row, then solve a sub-problem. 

1.2 Every group can choose where to sit according to the assignment.
In each period, every group can choose to sit everywhere as long as the assignment allows. If one group choose to sit in the middle of some row, then this row is divided into two rows. 
After certain periods, we update the remaining seats and rows, then solve a similar problem.

% Different types of movies will have different probabilities, consider the preference for policy when demand = supply.


Once: Obtain the supply from the stochastic model by benders decomposition. Use the deterministic model to obtain a heuristi supply. Then use the multi-class rule to decide whether to accept the group at each period.

Several: Initially, set the mean demand for all periods as the upper bound of demand. Then obtain the supply from the deterministic model. Set the accepted demand as the lower bound of demand, the upper bound of demand will be the sum of accepted demand and mean demand for the remaining periods. Update the lower bound and upper bound when some supply runs out.

% One counterexample: [15,21,13,3] /[15,21,10,3]  reject 4

\subsection{DP}


In fact, direct DP will have a gap, we should use a buffer to improve this method.

For example, the number of seats for 10 rows is 21. The demand is $[1,2,41,16]$. The optimal assignment is $[0, 0, 40, 10]$. But DP will give $[0, 0, 35, 14]$.

When a full pattern is reached, then delete the related groups and the row. Update the remaining demand.

We know $(0,0,4,1)$ is a largest and full pattern, thus an assignment constructed with these 10 patterns is an optimal assignment.

The results are shown below.

% (M1,M2,M3,M4,M5,M6) in the `performance compared to the optimal': 

\begin{table}[ht]
  \begin{tabular}{l|l|l|l|l}
  \hline
  \# samples & T & probabilities & \# rows & performance(\%) compared to the optimal \\
  1000  & 50  & [0.4,0.4,0.1,0.1] & 8 & (99.72, 100.00, 100.00, 100.00, 98.11, 100.00) \\
  1000  & 55  & [0.4,0.4,0.1,0.1] & 8 & (97.75, 99.83, 99.76, 99.76, 93.15, 99.76) \\ % slow
  1000  & 60  & [0.4,0.4,0.1,0.1] & 8 & (95.78, 99.20, 97.80, 97.80, 89.35, 97.65) \\
  1000  & 65  & [0.4,0.4,0.1,0.1] & 8 & (95.61, 99.10, 96.23, 96.23, 87.80, 96.12) \\
  \hline
  1000  & 40  & [0.25,0.25,0.25,0.25] & 8 & (99.94, 100.00, 100.00, 100.00, 98.22, 100.00) \\
  1000  & 45  & [0.25,0.25,0.25,0.25] & 8 & (97.19, 99.51, 99.09, 99.09, 91.31, 99.29) \\
  1000  & 50  & [0.25,0.25,0.25,0.25] & 8 & (95.23, 98.98, 97.21, 97.21, 87.73, 96.88) \\
  1000  & 55  & [0.25,0.25,0.25,0.25] & 8 & (94.84, 99.05, 95.70, 95.70, 85.49, 95.13) \\
  \end{tabular}
\end{table}

\newpage
