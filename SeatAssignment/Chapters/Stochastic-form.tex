% !TEX root = ./sum1.tex
% \section{Stochastic Demands Situation}

\subsection{Two-stage Scenario-based Stochastic Programming}
% We use the random demand(scenarios) to estimate the seat assignment. Then y indicate the recourse making decision.

Now suppose the demand of groups is stochastic, the stochastic information can be obtained from scenarios through historical data. Use $\omega$ to index the different scenarios, each scenario $\omega \in \Omega, \Omega=\{1,2,\ldots,S\}$ corresponds to a particular realization of the demand vector, $\mathbf{D}_\omega = (d_{1\omega},d_{2\omega},\ldots,d_{m,\omega})$. Let $p_{\omega}$ denote the probability of any scenario $\omega$, which we assume to be positive. To maximize the expected value of people over all the scenarios, we propose a two-stage scenario-based stochastic programming.
% As mentioned above, the objective is to give a seat assignment that can place as many groups as possible. 

Consider the decision makers who assign the groups to seats in two consecutive stages. The seat assignment and demand take place in the first and second stages, respectively. The first stage involves the initial seat assignment denoted by decision variables $\mathbf{x}\in \mathbb{Z}_{+}^{m \times N}$. Similar to the deterministic situation, let $x_{i,j}$ stand for the number of group type $i$ in row $j$. The supply for group type $i$ can be represented by $\sum_{j=1}^N x_{ij}$, which will be determined after the first stage.

% $\mathbf{s}_{i}^{0} =$
% These variables are scenario-independent and 

Because the demand is unknown when the seat assignment is planned, there is no way to expect that the supply in the first stage can meet the demand exactly. Fortunately, we can find some remedies in practice, for example, seats of 5 can be assigned to a group of 4 with one empty seat as social distance. However, the decision maker will confront seats shortage or excess without these measures. Therefore, to deal with possible demands, the wait-and-see measures (called recourses) should be considered in planning seat assignment.

% If no group of 1 comes in the future, this wait-and-see measure may leave two empty seats. 

Regarding the nature of the obtained information, we assume that there are $S$ possible scenarios and that the true demand is only revealed after $\mathbf{x}$ is chosen. In the second stage, the scenario-dependent decision variable, $\mathbf{y}$, is to be chosen. It includes two vectors of decisions, $\mathbf{y}^{+} \in \mathbb{Z}_{+}^{m \times S}$ and $\mathbf{y}^{-} \in \mathbb{Z}_{+}^{m \times S}$. Each component of $\mathbf{y}^{+}$, $y_{i \omega}^{+}$, represents the number of surplus seats for group type $i$. Similarly, $y_{i \omega}^{-}$ represents the number of inadequate seats for group type $i$.
Considering that the group can take the seats assigned to the larger group type, we assume that the surplus group type $i$ can be occupied by smaller group type $j<i$ in the descending order of the group size. That is, for any $\omega$, $i \leq m-1$, $y_{i \omega}^{+}=\left(\sum_{j=1}^N x_{ij}- d_{i \omega} + y_{i+1, \omega}^{+}\right)^{+}$ and $y_{i \omega}^{-}=\left(d_{i \omega}- \sum_{j=1}^N x_{ij} - y_{i+1, \omega}^{+} \right)^{+}$, where $(x)^{+}$ equals $x$ if $x>0$, $0$ otherwise. Specially, for the largest group type $m$, we have $y_{m \omega}^{+} = (\sum_{j=1}^N x_{ij} - d_{i \omega})^{+}$, $y_{m \omega}^{-} = (d_{i \omega}- \sum_{j=1}^N x_{ij})^{+}$.


% which is positive when the supply is larger than the actual demand, zero otherwise.
% which is positive when the supply is less than the actual demand and zero otherwise.


% which include the number of holding groups, $y_{i \omega}^{+}$, positive when the supply overestimates the actual demand and the number of short groups, $y_{i \omega}^{-}$, positive when the supply understimates the actual demand for group type $i$ in scenario $\omega$.


% The assignment will be determined before the realization of the random demand, here-and-now policy.

Then we have the deterministic equivalent form of the scenario-based stochastic programming:

\begin{equation}\label{sto_form}
    \begin{aligned}
    \max \quad & E_{\omega}\left[\sum_{i=1}^{m-1} (s_i-1) (\sum_{j= 1}^{N} x_{ij} + y_{i+1,\omega}^{+} - y_{i \omega}^{+}) + (s_m-1) (\sum_{j= 1}^{N} x_{mj} - y_{m \omega}^{+})\right] \\
    \text {s.t.} \quad & \sum_{j= 1}^{N} x_{ij}-y_{i \omega}^{+}+
    y_{i+1, \omega}^{+} + y_{i \omega}^{-}=d_{i \omega}, \quad i =1,\ldots,m-1, \omega \in \Omega \\
    & \sum_{j= 1}^{N} x_{ij} -y_{i \omega}^{+}+y_{i \omega}^{-}=d_{i \omega}, \quad i = m, \omega \in \Omega \\
    & \sum_{i=1}^{m} s_{i} x_{ij} \leq L_j, j =1,\ldots, N\\
    & y_{i \omega}^{+}, y_{i \omega}^{-} \in \mathbb{Z}_{+}, \quad i \in I, \omega \in \Omega \\
    & x_{ij} \in \mathbb{Z}_{+}, \quad i=1,\ldots,m, j =1,\ldots,N.
    \end{aligned}
  \end{equation}

The objective function contains two parts, the number of the largest group type that can be accommodated is $\sum_{j= 1}^{N} x_{mj} - y_{m \omega}^{+}$. The number of group type $i$ that can be accommodated is $\sum_{j= 1}^{N} x_{ij} + y_{i+1,\omega}^{+} - y_{i \omega}^{+}$.


Reformulate the objective function,

\begin{align*}
  & E_{\omega}\left[\sum_{i=1}^{m-1} (s_i-1) (\sum_{j= 1}^{N} x_{ij} + y_{i+1,\omega}^{+} - y_{i \omega}^{+}) + (s_m-1) (\sum_{j= 1}^{N} x_{mj} - y_{m \omega}^{+})\right] \\
  =& \sum_{j =1}^{N} \sum_{i=1}^m (s_i-1) x_{ij} - \sum_{\omega =1}^{S} p_{\omega} \left(\sum_{i=1}^{m}(s_i-1)y_{i \omega}^{+} - \sum_{i=1}^{m-1}(s_i-1)y_{i+1, \omega}^{+}\right) \\
  =& \sum_{j =1}^{N} \sum_{i=1}^m (s_i-1) x_{ij} - \sum_{\omega =1}^{S} p_{\omega} \left((s_1-1)y_{1 \omega}^{+} + \sum_{i=2}^{m}(s_{i}-s_{i-1}) y_{i \omega}^{+} \right)
\end{align*}

The objective is to obtain the maximal number of people placed, not the optimal seat assignment. It will not provide an appropriate seat assignment when the scenario demands are way less than the seats because it does not utilize all the empty seats. We will address this problem in Section 


% Plug in $s_i = i+1$, the objective function is $\sum_{j =1}^{N} \sum_{i=1}^m i x_{ij} - \sum_{\omega =1}^{S} p_{\omega} \sum_{i=1}^{m} y_{i \omega}^{+}$.

Let $\mathbf{s} = (s_1, \ldots, s_m)$, $\mathbf{L} = (L_1, \ldots, L_N)$ where $s_i$ is the size of seats taken by group type $i$ and $L_j$ is the length of row $j$ as we defined above. Then the row length constraint can be expressed as $\mathbf{s} \mathbf{x} \leq \mathbf{L}$.

The linear constraints associated with scenarios can be written in a matrix form as
\[\mathbf{x} \mathbf{1} + \mathbf{V} \mathbf{y}_\omega = \mathbf{d}_\omega, \omega\in \Omega,\]

where $\mathbf{1}$ is the 1-vector of size $N$, $\mathbf{V} = [\mathbf{W}, ~\mathbf{I}]$.  

$$
\mathbf{W}=\left[\begin{array}{ccccc}
-1 & 1 & \ldots & & 0 \\
& \ddots & \ddots & & \vdots \\
& & & & 1 \\
0 & & & & -1
\end{array}\right]_{m \times m}
$$

and $\mathbf{I}$ is the identity matrix. For each scenario $\omega \in \Omega$,
$$
\mathbf{y}_{\omega}=\left[\begin{array}{l}
\mathbf{y}_{\omega}^{+} \\
\mathbf{y}_{\omega}^{-}
\end{array}\right], \mathbf{y}_{\omega}^{+}=\left[\begin{array}{lllll}y_{1 \omega}^{+} & y_{2 \omega}^{+} & \cdots & y_{m \omega}^{+}\end{array}\right]^{T}, \mathbf{y}_{\omega}^{-}=\left[\begin{array}{llll}y_{1 \omega}^{-} & y_{2 \omega}^{-} & \cdots & y_{m \omega}^{-}\end{array}\right]^{T}.
$$

As we can find, this deterministic equivalent form is a large-scale problem even if the number of possible scenarios $\Omega$ is moderate. However, the structured constraints allow us to simplify the problem by applying Benders decomposition approach. Before using this approach, let us write this problem in the form of the two-stage stochastic programming.

Let $\mathbf{c}{'}\mathbf{x} = \sum_{j =1}^{N} \sum_{i=1}^m i x_{ij}$, $\mathbf{f}{'}\mathbf{y}_{\omega} = -\sum_{i=1}^{m} y_{i \omega}^{+}$. Then the formulation \eqref{sto_form} can be expressed as below,

\begin{equation}\label{BD_master}
  \begin{aligned}
\max \quad & c{'} \mathbf{x}+ z(\mathbf{x}) \\
\text {s.t.} \quad & \mathbf{s} \mathbf{x} \leq \mathbf{L} \\
& \mathbf{x} \in \mathbb{Z}_{+}^{m \times N},
\end{aligned}
\end{equation}

where $z(\mathbf{x})$ is the recourse function defined as 

$$z(\mathbf{x}) := E(z_{\omega}(\mathbf{x})) = \sum_{\omega \in \Omega} p_{\omega} z_{\omega}(\mathbf{x}),$$ and for each scenario $\omega \in \Omega$, 

\begin{equation}\label{BD_sub}
  \begin{aligned}
    z_{\omega}(\mathbf{x}) := \max \quad & \mathbf{f}{'} \mathbf{y}_{\omega} \\
    \text {s.t.} \quad & \mathbf{x} \mathbf{1} + \mathbf{V} \mathbf{y}_{\omega} = \mathbf{d}_{\omega} \\
     & \mathbf{y}_{\omega} \geq 0.
  \end{aligned}
  \end{equation}

% (Do I need to mention the convergence of the scenario-based problem?)

% The objective function of problem \eqref{sto_form} can be expressed as $c{'}\mathbf{x} + \sum_{\omega} p_{\omega}f{'}y_{\omega}$. 
Here $E$ is the expectation with respect to the scenario set. Problem \eqref{BD_sub} stands for the second-stage problem and $z_{\omega}(\mathbf{x})$ is the optimal value of problem \eqref{BD_sub}, together with the convention $z_{\omega}(x) = \infty$ if the problem is infeasible.

It is difficult to solve the above problem directly, we can relax problem \eqref{BD_master} to stochastic linear programming firstly. In section \ref{solve_by_benders}, we obtain an optimal linear solution by decomposition approach and generate a near-optimal seat assignment.

\section{Solve The Scenario-based Two-stage Problem}\label{solve_by_benders}

At first, we generate a closed-form solution to the second-stage problem in section \ref{second_stage}. Then we obtain the solution to the linear relaxation of problem \eqref{BD_master} by the delayed constraint generation. Finally, we obtain a near-optimal seat assignment from the linear solution.


\subsection{Solve The Second Stage Problem}\label{second_stage}

Consider a $\mathbf{x}$ such that $\mathbf{s x} \leq \mathbf{L}$ and $\mathbf{x} \geq 0$ and suppose that this represents our seat assignment for the first stage decisions. Once $\mathbf{x}$ is fixed, the optimal second stage decisions $\mathbf{y}_{\omega}$ can be determined by solving problem \eqref{BD_sub} for each $\omega$.

To solve this problem, we should only consider that $\mathbf{x}$ for which $z_{\omega}(\mathbf{x})$ are all finite. Notice that the feasible region of the dual of problem \eqref{BD_sub} does not depend on $x$. We can form its dual problem, which is 

\begin{equation}\label{BD_sub_dual}
  \begin{aligned}
    \min \quad & \alpha{'}_{\omega} (\mathbf{d}_{\omega}- \mathbf{x} \mathbf{1}) \\
    \text {s.t.} \quad & \alpha{'}_{\omega} \mathbf{V} \geq f{'}
  \end{aligned}
  \end{equation}

Let $P = \{\alpha|\alpha{'}V \geq \mathbf{f}{'}\}$. 
We assume that $P$ is nonempty and has at least one extreme point. Then, either the dual problem \eqref{BD_sub_dual} has an optimal solution and $z_{\omega}(x)$ is finite, or the primal problem \eqref{BD_sub} is infeasible and $z_{\omega}(x) = \infty$.  

Let $\mathcal{O}$ be the set of all extreme points of $P$ and $\mathcal{F}$ be the set of all extreme rays of $P$. Then $z_{\omega} > -\infty$ if and only if $(\alpha^{k}){'}(\mathbf{d}_{\omega}- \mathbf{x} \mathbf{1}) \geq 0, \alpha^{k} \in \mathcal{F}$, which stands for the feasibility cut. 

\begin{lem}\label{feasible_region}
  The feasible region of problem \eqref{BD_sub_dual}, $P$, is bounded. In addition, all the extreme points of $P$ are integral.
\end{lem}

\begin{pf}[Proof of lemma \ref{feasible_region}]
Notice that $V =[W,~I]$, $W$ is a totally unimodular matrix. Then, we have $\alpha{'}W \geq -\bar{s}, \alpha{'}I \geq 0$. Thus, the feasible region is bounded.
Further more, $\bar{s}_i = s_i - s_{i-1}, s_0 =1$ are integral, so the extreme points are all integral.
\qed
\end{pf}

Because the feasible region is bounded, then feasibility cuts are not needed. Let $z_{\omega}$ be the lower bound of $z_{\omega}(x)$ such that $(\alpha^{k}){'}(\mathbf{d}_{\omega}- \mathbf{x} \mathbf{1}) \geq z_{\omega}, \alpha^k \in \mathcal{O}$, which is the optimality cut.

\begin{corollary}
  Only the optimality cuts, $\alpha{'}(\mathbf{d}_{\omega}- \mathbf{x} \mathbf{1}) \geq z_{\omega}$, will be included in the decomposition approach.
\end{corollary}


\begin{corollary}
  When $s_i = i+1$, $f{'} = [-\mathbf{1},~\mathbf{0}], V =[W,~I]$, we have $\alpha{'}W \geq -1, \alpha{'}I \geq 0$. Thus, it is easy to find that the feasible region is bounded, i.e., $P$ does not contain any extreme rays. Furthermore, let $\alpha_0 = 0$, then we have $0 \leq \alpha_i \leq \alpha_{i-1} +1$, $i = 1, \ldots, m$.

  % \begin{align*}
  %   & 0 \leq \alpha_1 \leq 1,\\ 
  %   & 0 \leq \alpha_2 \leq \alpha_1 + 1, \\
  %   & \cdots, \\
  %   & 0 \leq \alpha_m \leq \alpha_{m-1} + 1
  % \end{align*}
\end{corollary}

\begin{corollary}
The optimal value of the problem \eqref{BD_sub}, $z_{\omega}(x)$, is finite and will be attained at extreme points of the set $P$. Thus, we have $z_{\omega}(x) = \min_{\alpha^j \in \mathcal{O}} (\alpha^{k}){'}(\mathbf{d}_{\omega}- \mathbf{x} \mathbf{1})$. 
\end{corollary}

% When $s_i$ is integral, we can still use the same way to obtain the dual optimal solution.

When we are given $x^{*}$, the demand that can be satisfied by the assignment is $\mathbf{x}^{*} \mathbf{1} = \mathbf{d}_0 = (d_{1,0},\ldots,d_{m,0})$.
Then plug them in the subproblem \eqref{BD_sub}, we can obtain the value of $y_{i \omega}$ recursively:
\begin{equation}\label{y_recursively}
\begin{aligned}
  & y_{m \omega}^{-}=\left(d_{m \omega}-d_{m 0}\right)^{+} \\
  & y_{m \omega}^{+}=\left(d_{m 0}-d_{m \omega}\right)^{+} \\
  & y_{i \omega}^{-}=\left(d_{i \omega}-d_{i 0} - y_{i+1, \omega}^{+} \right)^{+}, i =1,\ldots,m-1 \\
  & y_{i \omega}^{+}=\left(d_{i 0}- d_{i \omega} + y_{i+1, \omega}^{+}\right)^{+}, i =1,\ldots,m-1
\end{aligned}
\end{equation}

The optimal value for scenario $\omega$ can be obtained by $f{'} y_{\omega}$, then we need to find the dual optimal solution.


\begin{thm}\label{optimal_sol_sub_dual}
  The optimal solutions to problem \eqref{BD_sub_dual} are given by 
\begin{equation}\label{BD_sub_simplified}
  \begin{aligned}
    & \alpha_{i \omega} =0, i =1,\ldots,m \quad \text{if}~  y_{i \omega}^{-} > 0   \\
    & \alpha_{i \omega} = \alpha_{i-1, \omega}+1, i =1,\ldots,m \quad \text{if}~ y_{i \omega}^{+} > 0
  \end{aligned}
\end{equation}
\end{thm}

For some $i$, when $y_{i \omega}^{+} = 0$ and $y_{i \omega}^{-} = 0$, $\left(d_{i 0}- d_{i \omega} + y_{i+1, \omega}^{+}\right) = 0$, $d_{i \omega}- d_{i 0} = y_{i+1, \omega}^{+} \geq 0$.
If $y_{i+1, \omega}^{+} > 0$, $\alpha_{i \omega} = 0$;
if $y_{i+1, \omega}^{+} = 0$, $0 \leq \alpha_{i \omega} \leq \alpha_{i-1, \omega} +1$.

\begin{pf}[Proof of Theorem \ref{optimal_sol_sub_dual}]
  According to the complementary relaxation property, when
$d_{i \omega} > d_{i 0} \Rightarrow y_{i \omega}^{-} >0$, then $\alpha_{i \omega} =0$ for all $i$; when $d_{i \omega} < d_{i 0} \Rightarrow y_{i \omega}^{+} >0$, then $\alpha_{i \omega} = \alpha_{i-1,\omega} +1, i =1,\ldots,m$. 

When $d_{i \omega} = d_{i 0}$,  we can find that $\alpha_{i \omega} = \alpha_{i-1, \omega} + 1$ will minimize the objective function.

Let $\Delta d = d_{\omega} - d_0$, then the elements in $\Delta d$ will be a negative integer, positive integer and zero. 
% The value of $\alpha$ associated with zero does not directly affect the objective function. 
Only the negative element will affect the objective function.
The larger the value of $\alpha$ associated with a negative integer is, the smaller the objective function will be. Thus, let $\alpha_{i \omega} = \alpha_{i-1, \omega} + 1$ when $d_{i \omega} = d_{i 0}$ can obtain the minimized objective function.
\qed
\end{pf}

We can use the forward method, calculating from $\alpha_{1 \omega}$ to $\alpha_{m \omega}$, to obtain the value of $\alpha_{\omega}$ instead of solving linear programming.

\subsection{Delayed Constraint Generation}\label{bender_stage}
Benders decomposition works with only a subset of those exponentially many constraints and adds more constraints iteratively until the optimal solution of Benders Master Problem(BMP) is attained. This procedure is known as delayed constraint generation.

% Restricted Benders master problem:
Use the characterization of $z_{\omega}(x)$ in the problem \eqref{BD_master} and take into account the optimality cut, we can conclude the BMP will have the form:

\begin{equation}\label{BD_master2}
  \begin{aligned}
    \max \quad & c{'} x + \sum_{\omega \in \Omega} p_{\omega} z_{\omega} \\
    \text {s.t.} \quad & \sum_{i=1}^{m} s_{i} x_{ij} \leq L_j, j =1,\ldots, N \\
    & (\alpha^{k}){'}(\mathbf{d}_{\omega}- \mathbf{x} \mathbf{1}) \geq z_{\omega}, \alpha^k \in \mathcal{O}, \forall \omega \\
     & \mathbf{x} \geq 0
  \end{aligned}
\end{equation}

When substituting $\mathcal{O}$ with its subset, $\mathcal{O}^{t}$, the problem \eqref{BD_master2} becomes the Restricted Benders Master Problem(RBMP). 


% Then we have the restricted Benders master problem \eqref{BD_master2} by substituting $\mathcal{O}$ with its subset, $\mathcal{O}^t$.

% \begin{equation}\label{BD_master1}
%   \begin{aligned}
%     \max \quad & c{'} x + \sum_{\omega \in \Omega} p_{\omega} z_{\omega} \\
%     \text {s.t.} \quad & \sum_{i=1}^{m} s_{i} x_{ij} \leq L_j, j =1,\ldots, N \\
%     & (\alpha^{k}){'}(\mathbf{d}_{\omega}- \mathbf{x} \mathbf{1}) \geq z_{\omega}, \alpha^k \in \mathcal{O}, \forall \omega \\
%      & \mathbf{x} \geq 0
%   \end{aligned}
% \end{equation}


To determine the initial $\mathcal{O}^{t}$, we have the following lemma.

\begin{lem}\label{one_ep_feasible}
RBMP is always bounded with at least any one optimality cut for each scenario.
\end{lem}

\begin{pf}[Proof of lemma \ref{one_ep_feasible}]
  Suppose we have one extreme point $\alpha^{\omega}$ for each scenario. Then we have the following problem.
  \begin{equation}\label{lemma_eq}
    \begin{aligned}
      \max \quad & c{'} x + \sum_{\omega \in \Omega} p_{\omega} z_{\omega} \\
      \text {s.t.} \quad & \sum_{i=1}^{m} s_{i} x_{ij} \leq L_j, j =1,\ldots, N \\
      & (\alpha^{\omega}){'}\mathbf{d}_{\omega} \geq (\alpha^{\omega}){'} \mathbf{x} \mathbf{1} + z_{\omega}, \forall \omega \\
       & \mathbf{x} \geq 0
    \end{aligned}
  \end{equation}
  Problem \eqref{lemma_eq} reaches its maximum when $(\alpha^{\omega}){'}\mathbf{d}_{\omega} = (\alpha^{\omega}){'} \mathbf{x} \mathbf{1} + z_{\omega}, \forall \omega$. Substitute $z_{\omega}$ with these equations, we have 
  \begin{equation}\label{lemma_eq2}
    \begin{aligned}
      \max \quad & c{'} x - \sum_{\omega}p_{\omega}(\alpha^{\omega}){'} \mathbf{x} \mathbf{1} + \sum_{\omega} p_{\omega} (\alpha^{\omega}){'} \mathbf{d}_{\omega} \\
      \text {s.t.} \quad & \sum_{i=1}^{m} s_{i} x_{ij} \leq L_j, j =1,\ldots, N \\
      & \mathbf{x} \geq 0
    \end{aligned}
  \end{equation}
  Notice that $\mathbf{x}$ is bounded by $\mathbf{L}$, then the problem \eqref{lemma_eq} is bounded. Adding more optimality cuts will not make the optimal value larger. Thus, RBMP is bounded. 
  \qed
\end{pf}

Given the initial $\mathcal{O}^{t}$, we can have the solution $\mathbf{x}_{0}$ and $\mathbf{z}^{0} =(z^{0}_1,\ldots, z^{0}_S)$. Then $c{'} \mathbf{x}_0 + \sum_{\omega \in \Omega} p_{\omega} z_{\omega}^{0}$ is an upper bound of problem \eqref{BD_master2}. 


When $\mathbf{x}_0$ is given, the optimal solution, $\alpha_{\omega}^{1}$, to problem \eqref{BD_sub_dual} can be obtained according to Theorem \ref{optimal_sol_sub_dual}. $z_{\omega}^{(0)} = \alpha_{\omega}^{1}(d_{\omega} - \mathbf{x}_0 \mathbf{1})$ and $(\mathbf{x}_0, \mathbf{z}_{\omega}^{(0)})$ is a feasible solution to problem \eqref{BD_master2} because it satisfies all the constraints. Thus, $c{'} \mathbf{x}_0 + \sum_{\omega \in \Omega} p_{\omega} \mathbf{z}_{\omega}^{(0)}$ is a lower bound of problem \eqref{BD_master2}.

If for every scenario, the optimal value of the corresponding problem \eqref{BD_sub_dual} is larger than or equal to $z_{\omega}^{0}$, all contraints are satisfied, we have an optimal solution, $(x_0, z_{\omega}^{0})$, to the BMP. Otherwise, add one new constraint, $(\alpha_{\omega}^{1}){'}(\mathbf{d}_{\omega}- \mathbf{x} \mathbf{1}) \geq z_{\omega}$, to RBMP.

% $z_{\omega}^{(0)} = \alpha_{\omega}^{*}(d_{\omega} - \mathbf{x}_0 \mathbf{1})$ will give a minimal upper bound of $z_{\omega}$, thus all the left constraints associated with other extreme points are redundant.when the extreme points are $\alpha_{\omega}$.

 
% The problem \eqref{lemma_eq} associated with $\alpha_{\omega}$ will give an optimal solution $(x_1, z_{\omega}^{1})$. (Upper bound)


The steps of the algorithm are described as below,

\begin{algorithm}[H]\label{cut_algo}
  \caption{The benders decomposition algorithm}
    \begin{description}
    \item[Step 1.] Solve LP \eqref{lemma_eq} with all $\alpha_{\omega}^0 = \mathbf{0}$ for each scenario.
    Then, obtain the solution $(\mathbf{x}_0, \mathbf{z}^{0})$.

    \item[Step 2.] Set the upper bound $UB = c{'} \mathbf{x}_0 + \sum_{\omega \in \Omega} p_{\omega} z_{\omega}^{0}$.
    \item[Step 3.] 
    For $x_0$, we can obtain $\alpha_{\omega}^{1}$ and $z_{\omega}^{(0)}$ for each scenario, set the lower bound $LB = c{'} x_0 + \sum_{\omega \in \Omega} p_{\omega} z_{\omega}^{(0)}$
    \item[Step 4.]
    For each $\omega$, if $(\alpha_{\omega}^{1}){'}(\mathbf{d}_{\omega}- \mathbf{x}_0 \mathbf{1}) < z_{\omega}^{0}$, add one new constraint, $(\alpha_{\omega}^{1}){'}(\mathbf{d}_{\omega}- \mathbf{x} \mathbf{1}) \geq z_{\omega}$, to RBMP.
    \item[Step 5.] Solve the updated RBMP, obtain a new solution $(x_1, z^{1})$ and update UB.
    \item[Step 6.] Repeat step 3 until $UB - LB < \epsilon$.(In our case, UB converges.)
   \end{description}
  \end{algorithm}

\begin{remark}
From the lemma \ref{one_ep_feasible}, we can set $\alpha_{\omega}^0 = \mathbf{0}$ initially in Step 1. 
\end{remark}

\begin{remark}
  Notice that only contraints are added in each iteration, thus $LB$ and $UB$ are both monotone. Then we can use $UB - LB < \epsilon$ to terminate the algorithm in Step 6.
\end{remark}


After the algorithm terminates, we obtain the optimal $\mathbf{x}^{*}$. The demand that can be satisfied by the arrangement is $\mathbf{x}^{*} \mathbf{1} = d_0 = (d_{1,0},\ldots,d_{m,0})$.
Then we can obtain the value of $y_{i \omega}$ from equation \eqref{y_recursively}.

We show the results of Benders and IP in the section \ref{Bender_IP}.

\subsection{Obtain The Near-optimal Seat Assignment}\label{seat_assignment}

The stochastic model only gives a fractional solution and not provides an expected seat assignment when scenario demands are smaller compared with the seats. 
Thus, we make some changes to obtain the near-optimal seat assignment.

% its objective is to obtain the maximal number of people served, not the optimal seat assignment. It will not provide an appropriate solution when the number of arriving people in the scenarios is way less than the number of total seats because it does not utilize all the empty seats.

Suppose we obtain the optimal linear solution $x^{*}_{ij}$ from the stochastic model, set the supply $\mathbf{s}^{0} = \sum_{j} x^{*}_{ij}$ as the upper bound of demand in problem \eqref{deter_upper}. We can get a feasible integer solution by solving this problem, denote by $\mathbf{s}^{1}$ the corresponding supply. As we mentioned above, this solution does not utilize the empty seats when the demand is small. Thus, we should set the supply $\mathbf{s}^{1}$ as the lower bound of demand, then re-solve a seat assignment problem. We substitute the constraint $\sum_{j =1}^{N} x_{ij} \leq d_{i}, i=1,\ldots,m$ with the new constraint $\sum_{j =1}^{N} x_{ij} \geq s_{i}^{1}, i=1,\ldots,m$ in problem \eqref{deter_upper}, $s_{i}^{1}$ represents the number of group type $i$ we must allocate seats. The optimal solution to this problem with the lower bound will give an near-optimal seat assignment. We cannot ensure this solution is optimal for the original stochastic programming, but the numerical results show that this seat assignment has a close result to the optimal one under stochastic demands, and even has a better result when dealing with the dynamic demands.  


% Because the ratio of value to capacity is monotone for the size of groups, the solver can quickly solve this deterministic formulation without many branching operations.

% $d_{i}^{l}$ is the lower bound of the demand. (It can be the number of group type $i$ we have accepted.)
% $d_{i}^{u}$ is the upper bound of the demand. (deterministic demand)

% Firstly, we obtain the solution from stochatic programming. This solution corresponds to a supply for each group type. Then solve the deterministic model by setting the supply as the upper bound of demand to obtain another feasible supply. Finally, solve the deterministic model by setting this supply as the lower bound to obtain the seat assignment.


\begin{algorithm}[H]\label{near-optimal}
  \caption{Near-optimal seat assignment algorithm}
    \begin{description}
    \item[Step 1.] Obtain the solution from stochatic linear programming by benders decomposition.

    \item[Step 2.] Aggregate the solution to the supply for each group type.

    \item[Step 3.] Set the supply as the upper bound of demand in problem \eqref{deter_upper}. The lower bound is not included.
    
    \item[Step 4.] Aggregate the solution from Step 3 to a supply for each group type.

    \item[Step 5.] Obtain the near-optimal assignment from \eqref{deter_upper} by setting the supply in Step 4 as the lower bound.
   \end{description}
  \end{algorithm}

\begin{remark}
  Step 3 can give a feasible integer supply. In Step 5, problem \eqref{deter_upper} with this supply as the lower bound can always give an integer solution.  
  Thus, we can obtain the near-optimal seat assignment by solving stochastic programming once and deterministic programming twice.
  
  % setting the upper and lower bound alternately.
\end{remark}



% 1. Why should we don't use the subset sum problem to decompose the whole problem, it will destroy global optimality. But notice that when we arrange row by row, it may also affect the optimality.

% Many symmetry structure/ Every step we need to solve a multiple knapsack problem(difficult).

