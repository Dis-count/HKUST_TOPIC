% !TEX root = ./sum1.tex
\section{Stochastic demands situation}

\subsection{Deterministic equivalent form}

Consider the decision maker who has to act in two consecutive stages. The first stage involves the choice of cutting patterns denoted by decision vector $\mathbf{x}$. At the second stage, some new information about demands is obtained, then a new vector $\mathbf{y}$ of decisions is to be chosen.

$\mathbf{x} \in \mathbb{Z}_{+}^{n}$ is the vector of first-stage scenario-independent decision variables, each component $x_k$ stands for the pattern $k$.

Use $\omega$ to index the different scenarios, each scenario $\omega \in \Omega, \Omega=\{1,2,\ldots,S\}$ corresponds to a particular realization of the demand vector, $\mathbf{D}_s = (d_{1s},d_{2s},\ldots,d_{ms})$.

$\mathbf{y} \in \mathbb{Z}_{+}^{m}$ is the vector of second-stage scenario-dependent decision variables, which include the number of holding groups, $y_{i \omega}^{+}$, when the supply overestimates the actual demand and the number of short groups, $y_{i \omega}^{-}$, when the supply understimates the actual demand for group type $i$ in scenario $\omega$.


Regarding the nature of the obtained information, we assume that there are $S$ possible scenarios, and that the true scenario is only revealed after $\mathbf{x}$ is chosen. 

Let $p_{\omega}$ denote the probability of any scenario $\omega$, which we assume to be positive.

We use the same definition as above, the size of group type $i$, $s_i$.

The assignment will be determined before the realization of the random demand, here-and-now policy.


Considering that the seats assigned to some group type can be taken by the smaller group type, we assume that the holding group type $i$ can be utilized by the smaller group type $j = i-1$. Then we have the following scenario-based optimization problem:

\begin{equation}\label{sto_form}
    \begin{aligned}
    \max \quad & E_{\omega}\left[\sum_{i=1}^{m-1} (s_i-1) (\sum_{j= 1}^{N} x_{ij} + y_{i+1,\omega}^{+} - y_{i \omega}^{+}) + (s_m-1) (\sum_{j= 1}^{N} x_{mj} - y_{m \omega}^{+})\right] \\
    \text {s.t.} \quad & \sum_{j= 1}^{N} x_{ij}-y_{i \omega}^{+}+
    y_{i+1, \omega}^{+} + y_{i \omega}^{-}=d_{i \omega}, \quad i =1,\ldots,m-1, \omega \in \Omega \\
    & \sum_{j= 1}^{N} x_{ij} -y_{i \omega}^{+}+y_{i \omega}^{-}=d_{i \omega}, \quad i = m, \omega \in \Omega \\
    & \sum_{i=1}^{m} s_{i} x_{ij} \leq L_j, j =1,\ldots, N\\
    & y_{i \omega}^{+}, y_{i \omega}^{-} \in \mathbb{Z}_{+}, \quad i \in I, \omega \in \Omega \\
    & x_{ij} \in \mathbb{Z}_{+}, \quad i=1,\ldots,m, j =1,\ldots,N.
    \end{aligned}
  \end{equation}

The objective function contains two parts, the number of the largest group size that can be accommodated is $\sum_{j= 1}^{N} x_{mj} - y_{m \omega}^{+}$. The number of group size $i$ that can be accommodated is $\sum_{j= 1}^{N} x_{ij} + y_{i+1,\omega}^{+} - y_{i \omega}^{+}$.

Reformulate the objective function,

\begin{align*}
  & E_{\omega}\left[\sum_{i=1}^{m-1} (s_i-1) (\sum_{j= 1}^{N} x_{ij} + y_{i+1,\omega}^{+} - y_{i \omega}^{+}) + (s_m-1) (\sum_{j= 1}^{N} x_{mj} - y_{m \omega}^{+})\right] \\
  =& \sum_{j =1}^{N} \sum_{i=1}^m (s_i-1) x_{ij} - \sum_{\omega =1}^{S} p_{\omega} \left(\sum_{i=1}^{m}(s_i-1)y_{i \omega}^{+} - \sum_{i=1}^{m-1}(s_i-1)y_{i+1, \omega}^{+}\right) \\
  =& \sum_{j =1}^{N} \sum_{i=1}^m (s_i-1) x_{ij} - \sum_{\omega =1}^{S} p_{\omega} \left((s_1-1)y_{1 \omega}^{+} + \sum_{i=2}^{m}(s_{i}-s_{i-1}) y_{i \omega}^{+} \right)
\end{align*}

When $s_i = i+1$, the objective function is $\sum_{j =1}^{N} \sum_{i=1}^m i x_{ij} - \sum_{\omega =1}^{S} p_{\omega} \sum_{i=1}^{m} y_{i \omega}^{+}$.


This programming is in a deterministic equivalent form. 

Let $\mathbf{s} = (s_1, \ldots, s_m)$, $\mathbf{L} = (L_1, \ldots, L_N)$ where $s_i$ is the size of seats taken by group type $i$ and $L_j$ is the number of seats for row $j$. Then the row length constraint can be expressed as $\mathbf{s} \mathbf{x} \leq \mathbf{L}$.

The linear constraints associated with scenarios can be written in a matrix form as
\[\mathbf{x} \mathbf{1} + \mathbf{V}_\omega \mathbf{y}_\omega = \mathbf{d}_\omega, \omega\in \Omega\]

$\mathbf{1}$ is the 1-vector of size $N$. $\mathbf{V}_s = [\mathbf{W} ~ \mathbf{I}]$.  

$$
\mathbf{W}=\left[\begin{array}{ccccc}
-1 & 1 & \ldots & & 0 \\
& \ddots & \ddots & & \vdots \\
& & & & 1 \\
0 & & & & -1
\end{array}\right]_{m \times m}
$$

and $\mathbf{I}$ is the identity matrix.

$$
\mathbf{y}_{s}=\left[\begin{array}{l}
\mathbf{y}_{s}^{+} \\
\mathbf{y}_{s}^{-}
\end{array}\right], \quad s \in \Omega
$$

$\mathbf{y}_{s}^{+}=\left[\begin{array}{lllll}y_{1 s}^{+} & y_{2 s}^{+} & \cdots & y_{m s}^{+}\end{array}\right]^{T}, \mathbf{y}_{s}^{-}=\left[\begin{array}{llll}y_{1 s}^{-} & y_{2 s}^{-} & \cdots & y_{m s}^{-}\end{array}\right]^{T}$.

As we can find, this formulation is a large-scale problem even if the number of possible scenarios $\Omega$ is moderate. Thus, we need to reformulate the problem and use a decomposition method.

\subsection{Two-stage stochastic programming}

Let $c{'}\mathbf{x} = \sum_{j =1}^{N} \sum_{i=1}^m (s_i-1) x_{ij}$, $f{'}y_{\omega} = -\left((s_1-1)y_{1 \omega}^{+} + \sum_{i=2}^{m}(s_{i}-s_{i-1}) y_{i \omega}^{+}\right)$. The objective function of problem \eqref{sto_form} can be expressed as $c{'}\mathbf{x} + \sum_{\omega} p_{\omega}f{'}y_{\omega}$.

Once $x$ is fixed, the optimal second stage decision $y_{\omega}$ can be determined by solving for each $\omega$ the following problem:

\begin{equation}\label{BD_sub}
\begin{aligned}
  \max \quad & f{'} y_{\omega} \\
  \text {s.t.} \quad & \mathbf{x} \mathbf{1} + \mathbf{V} \mathbf{y}_\omega = \mathbf{d}_\omega \\
   & y_{\omega} \geq 0
\end{aligned}
\end{equation}


Let $z_{\omega}(x)$ be the optimal value of the problem \eqref{BD_sub}, together with the convention $z_{\omega}(x) = \infty$ if the problem is infeasible. 

Now go back to consider the optimization of $x$, we are faced with the problem:

\begin{equation}\label{BD_master}
  \begin{aligned}
    \max \quad & c{'} \mathbf{x} + \sum_{\omega \in \Omega} p_{\omega} z_{\omega}(\mathbf{x}) \\
    \text {s.t.} \quad & \sum_{i=1}^{m} s_{i} x_{ij} \leq L_j, j =1,\ldots, N \\
     & \mathbf{x} \geq 0
  \end{aligned}
\end{equation}

In order to solve this problem, we should only consider those $x$ for which $z_{\omega}(x)$ are all finite. Notice that the feasible region of the dual of problem \eqref{BD_sub} does not depend on $x$. We can form its dual, which is 

\begin{equation}\label{BD_sub_dual}
  \begin{aligned}
    \min \quad & \alpha{'}_{\omega} (\mathbf{d}_{\omega}- \mathbf{x} ) \\
    \text {s.t.} \quad & \alpha{'}_{\omega} \mathbf{V} \geq f{'}
  \end{aligned}
  \end{equation}

Let $P = \{\alpha|\alpha{'}V \geq f{'}\}$. 
We assume that $P$ is nonempty and has at least one extreme point. Then, either the dual problem \eqref{BD_sub_dual} has an optimal solution and $z_{\omega}(x)$ is finite, or the primal problem \eqref{BD_sub} is infeasible and $z_{\omega}(x) = \infty$.  

Let $\mathcal{O}$ be the set of all extreme points of $P$ and $\mathcal{F}$ be the set of all extreme rays of $P$. Then $z_{\omega} > -\infty$ if and only if $(\alpha^{k}){'}(\mathbf{d}_{\omega}- \mathbf{x} \mathbf{1}) \geq 0, \alpha^{k} \in \mathcal{F}$, which stands for the feasibility cut. 

\begin{thm}\label{feasible_region}
  The feasible region, P, is bounded. And the extreme points are all integral.
\end{thm}

\begin{pf}[Proof of theorem \ref{feasible_region}]
Notice that $V =(W,~I)$, $W$ is a totally unimodular matrix. Then, we have $\alpha{'}W \geq -\bar{s}, \alpha{'}I \geq 0$. Thus, the feasible region is bounded.
And $\bar{s}_i = s_i - s_{i-1}, s_0 =1$ are integral, so the extreme points are all integral.
\qed
\end{pf}

Remark: we only need to consider the optimality cut in the following decomposition method.

When $s_i = i+1$, $f{'} = [-\mathbf{1},~\mathbf{0}], V =(W,~I)$, we have $\alpha{'}W \geq -1, \alpha{'}I \geq 0$. Thus, it is easy to find that the feasible region is bounded, i.e., $P$ does not contain any extreme rays. Furthermore, let $\alpha_0 = 0$, then we have 
\begin{align*}
  & 0 \leq \alpha_1 \leq 1,\\ 
  & 0 \leq \alpha_2 \leq \alpha_1 + 1, \\
  & \cdots, \\
  & 0 \leq \alpha_m \leq \alpha_{m-1} + 1
\end{align*}

When $s_i$ is integral, we can still use the same way to obtain the dual optimal solution.

When we are given $x^{*}$, the demand that can be satisfied by the arrangement is $\mathbf{x}^{*} \mathbf{1} = \mathbf{d}_0 = (d_{1,0},\ldots,d_{m,0})$.
Then plug them in the subproblem \eqref{BD_sub}, we can obtain the value of $y_{i \omega}$ recursively:
\begin{equation}\label{y_recursively}
\begin{aligned}
  & y_{m \omega}^{-}=\left(d_{m \omega}-d_{m 0}\right)^{+} \\
  & y_{m \omega}^{+}=\left(d_{m 0}-d_{m \omega}\right)^{+} \\
  & y_{i \omega}^{-}=\left(d_{i \omega}-d_{i 0} - y_{i+1, \omega}^{+} \right)^{+}, i =1,\ldots,m-1 \\
  & y_{i \omega}^{+}=\left(d_{i 0}- d_{i \omega} + y_{i+1, \omega}^{+}\right)^{+}, i =1,\ldots,m-1
\end{aligned}
\end{equation}

The optimal value for scenario $\omega$ can be obtained by $f{'} y_{\omega}$, then we need to find the dual optimal solution.


\begin{thm}\label{optimal_sol_sub_dual}
  An optimal solution to problem \eqref{BD_sub_dual} is given by 
\begin{equation}\label{BD_sub_simplified}
  \begin{aligned}
    & \alpha_{i \omega} =0, i =1,\ldots,m \quad \text{if}~  y_{i \omega}^{-} > 0   \\
    & \alpha_{i \omega} = \alpha_{i-1, \omega}+1, i =1,\ldots,m \quad \text{if}~ y_{i \omega}^{+} > 0
  \end{aligned}
\end{equation}
\end{thm}

When $y_{i \omega}^{+} = 0$ and $y_{i \omega}^{-} = 0$, $\left(d_{i 0}- d_{i \omega} + y_{i+1, \omega}^{+}\right) = 0$, $d_{i \omega}- d_{i 0} = y_{i+1, \omega}^{+} \geq 0$.

If $y_{i+1, \omega}^{+} > 0$, $\alpha_{i \omega} = 0$,
if $y_{i+1, \omega}^{+} = 0$, $0 \leq \alpha_{i \omega} \leq \alpha_{i-1, \omega} +1$.

\begin{pf}[Proof of theorem \ref{optimal_sol_sub_dual}]
  According to the complementary relaxation property, when
$d_{i \omega} > d_{i 0} \Rightarrow y_{i \omega}^{-} >0$, then $\alpha_{i \omega} =0$ for all $i$; when $d_{i \omega} < d_{i 0} \Rightarrow y_{i \omega}^{+} >0$, then $\alpha_{i \omega} = \alpha_{i-1,\omega} +1, i =1,\ldots,m$. 

When $d_{i \omega} = d_{i 0}$,  we can find that $\alpha_{i \omega} = \alpha_{i-1, \omega} + 1$ will minimize the objective function.

Let $\Delta d = d_{\omega} - d_0$, then the elements in $\Delta d$ will be negative integer, positive integer and zero. The value of $\alpha$ associated with zero will not affect objective function directly, only affect the value of $\alpha$ associated with negative integer. The larger the value of $\alpha$ associated with negative integer is, the smaller the objective function will be. Thus, let $\alpha_{i \omega} = \alpha_{i-1, \omega} + 1$ when $d_{i \omega} = d_{i 0}$ can obtain the minimized objective function.
\qed
\end{pf}

We can use the forward method, starting from $\alpha_{1 \omega}$ to $\alpha_{m \omega}$, to obtain the value of $\alpha_{\omega}$ rather than solving a linear programming.


We know that $z_{\omega}(x)$ is finite and it is the optimal value of the problem \eqref{BD_sub} and this value will be attained at extreme points of the set $P$. Thus, we have $z_{\omega}(x) = \min_{\alpha^j \in \mathcal{O}} (\alpha^{k}){'}(\mathbf{d}_{\omega}- \mathbf{x} \mathbf{1})$. 

\subsection{Benders decomposition}

Let $z_{\omega}(x)$ be the upper bound of $z_{\omega}$ such that $(\alpha^{k}){'}(\mathbf{d}_{\omega}- \mathbf{x} \mathbf{1}) \geq z_{\omega}, \alpha^k \in \mathcal{O}$, which is the optimality cut.

Use the characterization of $z_{\omega}(x)$ in the problem 
\eqref{BD_master}, and take into account the optimality condition, we can conclude the master problem \eqref{BD_master} will have the form:

\begin{equation}\label{BD_master1}
  \begin{aligned}
    \max \quad & c{'} x + \sum_{\omega \in \Omega} p_{\omega} z_{\omega} \\
    \text {s.t.} \quad & \sum_{i=1}^{m} s_{i} x_{ij} \leq L_j, j =1,\ldots, N \\
    & (\alpha^{k}){'}(\mathbf{d}_{\omega}- \mathbf{x} \mathbf{1}) \geq z_{\omega}, \alpha^k \in \mathcal{O}, \forall \omega \\
     & \mathbf{x} \geq 0
  \end{aligned}
\end{equation}

Now use a small subset of $\mathcal{O}$, $\mathcal{O}^t$, to substitute the original one, then we have a relaxation of the master problem \eqref{BD_master1}:

\begin{equation}\label{BD_master2}
  \begin{aligned}
    \max \quad & c{'} x + \sum_{\omega \in \Omega} p_{\omega} z_{\omega} \\
    \text {s.t.} \quad & \sum_{i=1}^{m} s_{i} x_{ij} \leq L_j, j =1,\ldots, N \\
    & (\alpha^{k}){'}(\mathbf{d}_{\omega}- \mathbf{x} \mathbf{1}) \geq z_{\omega}, \alpha^k \in \mathcal{O}^{t}, \forall \omega \\
     & \mathbf{x} \geq 0
  \end{aligned}
\end{equation}


Given the initial $\mathcal{O}^{t}$, we can obtain the optimal $\mathbf{x}^{*}$ and $\mathbf{z}^{*}=(z^{*}_1,\ldots, z^{*}_S)$. We need to check whether $(\mathbf{x}^{*}, \mathbf{z}^{*})$ is also an optimal solution to the full master problem. 

\begin{lem}\label{one_ep_feasible}
When each scenario has at least any one optimality cut, the problem \eqref{BD_master2} is always bounded.
\end{lem}

\begin{pf}[Proof of lemma \ref{one_ep_feasible}]
  Suppose we have one extreme point $\alpha^{\omega}$ for each scenario. Then we have the following problem.
  \begin{equation}\label{lemma_eq}
    \begin{aligned}
      \max \quad & c{'} x + \sum_{\omega \in \Omega} p_{\omega} z_{\omega} \\
      \text {s.t.} \quad & \sum_{i=1}^{m} s_{i} x_{ij} \leq L_j, j =1,\ldots, N \\
      & (\alpha^{\omega}){'}\mathbf{d}_{\omega} \geq (\alpha^{\omega}){'} \mathbf{x} \mathbf{1} + z_{\omega}, \forall \omega \\
       & \mathbf{x} \geq 0
    \end{aligned}
  \end{equation}
  Problem \eqref{lemma_eq} reaches its maximum when $(\alpha^{\omega}){'}\mathbf{d}_{\omega} = (\alpha^{\omega}){'} \mathbf{x} \mathbf{1} + z_{\omega}, \forall \omega$. Substitute $z_{\omega}$ with these equations, we have 
  \begin{equation}\label{lemma_eq2}
    \begin{aligned}
      \max \quad & c{'} x - \sum_{\omega}p_{\omega}(\alpha^{\omega}){'} \mathbf{x} \mathbf{1} + \sum_{\omega} p_{\omega} (\alpha^{\omega}){'} \mathbf{d}_{\omega} \\
      \text {s.t.} \quad & \sum_{i=1}^{m} s_{i} x_{ij} \leq L_j, j =1,\ldots, N \\
      & \mathbf{x} \geq 0
    \end{aligned}
  \end{equation}
  Notice that $\mathbf{x}$ is bounded, then the problem \eqref{lemma_eq} is bounded. Adding more optimality cuts will not make the optimal value larger. Thus, the problem \eqref{BD_master2} is bounded. 
  \qed
\end{pf}

When some scenario only includes one optimality cut, we can substitute $z_{\omega}$ into the objective function. 

According to theorem \ref{optimal_sol_sub_dual}, we can obtain the optimal solution, $\alpha_{\omega}^{*}$, to problem \eqref{BD_sub_dual}. When $\mathbf{x}_0$ is given, $z_{\omega}^{0} = \alpha_{\omega}^{*}(d_{\omega} - \mathbf{x}_0 \mathbf{1})$ will give a minimal upper bound of $z_{\omega}$, thus all the left constraints associated with other extreme points are redundant.

Notice that $(x_0, z_{\omega}^{0})$ is a feasible solution ($c{'} x_0 + \sum_{\omega \in \Omega} p_{\omega} z_{\omega}^{0}$ is the lower bound) when the extreme points are $\alpha_{\omega}$. The problem \eqref{lemma_eq} associated with $\alpha_{\omega}$ will give an optimal solution $(x_1, z_{\omega}^{1})$. (Upper bound)


The steps of the algorithm can be described as below,

\begin{algorithm}[H]\label{cut_algo}
  \caption{The benders decomposition algorithm}
    \begin{description}
    \item[Step 1.] Solve LP \eqref{lemma_eq} with all $\alpha_{\omega}^0 = \mathbf{0}$ for each scenario.
    Then, obtain the solution $(x_0, z_{\omega}^{0})$ and dual solution $(\pi, \lambda)$.

    \item[Step 2.] Set the upper bound $UB = c{'} x_0 + \sum_{\omega \in \Omega} p_{\omega} z_{\omega}^{0}$ 
    \item[Step 3.] 
    For $x_0$, we can obtain $\alpha_{\omega}^{1}$ and $z_{\omega}^{(0)}$ for each scenario, set the lower bound $LB = c{'} x_0 + \sum_{\omega \in \Omega} p_{\omega} z_{\omega}^{(0)}$

    \item[Step 4.]
    If $(\alpha_{\omega}^{1}){'}(\mathbf{d}_{\omega}- \mathbf{x}_0 \mathbf{1}) < z_{\omega}^{(0)}$, add one new constraint, $(\alpha_{\omega}^{1}){'}(\mathbf{d}_{\omega}- \mathbf{x} \mathbf{1}) \geq z_{\omega}$, to LP \eqref{BD_master2}.

    \item[Step 5.] Solve the updated LP \eqref{BD_master2}, obtain a new solution $(x_1, z_{\omega}^{1})$ and update UB.
    \item[Step 6.] Repeat step 3 until $UB - LB < \epsilon$.(Or in our case, UB converges.)
   \end{description}
  \end{algorithm}

Remark: Step 1 results from the lemma \ref{one_ep_feasible}. Step 6, notice that only contraints are added, thus $LB$ and $UB$ are both monotone, then we can use $UB - LB < \epsilon$ to terminate the algorithm.

After the algorithm terminates, we obtain the optimal $x^{*}$. The demand that can be satisfied by the arrangement is $\mathbf{x}^{*} \mathbf{1} = d_0 = (d_{1,0},\ldots,d_{m,0})$.
Then we can obtain the value of $y_{i \omega}$ from equation \eqref{y_recursively}.

\subsection{Benders decomposition versus IP}

In view of the fact that IP of the deterministic model can be solved quickly, the column generation method does not show an obvious advantage. We can revise the stochastic model as follows:

The running times of solving IP directly and using Benders decomposition are shown in the table below. 

\begin{table}[ht]
  \begin{tabular}{l|l|l|l|l}
  \hline
  \# of Scenarios & running time of IP(s) & Benders (s) & \# of rows & \# of groups \\
  1000  & 5.1  & 0.13 & 30 & 8 \\
  5000  & 28.73 & 0.47 & 30 & 8 \\
  10000 & 66.81  & 0.91 & 30 & 8 \\
  50000 & 925.17 & 4.3 & 30 & 8 \\
  \hline
  1000  & 5.88 & 0.29 & 200 & 8 \\
  5000  & 30.0 & 0.62 & 200 & 8 \\
  10000 & 64.41 & 1.09 & 200 & 8 \\
  50000 & 365.57 & 4.56 & 200 & 8 \\
  \hline
  1000  & 17.15  & 0.18 & 30 & 16  \\
  5000  & 105.2  & 0.67 & 30 & 16  \\
  10000 & 260.88 & 1.28 & 30 & 16  \\
  50000 & 3873.16 & 6.18 & 30 & 16  \\
  \hline
  \end{tabular}
\end{table}

The parameters of the experiments are listed below.

The first one:
The number of rows is 30.
The number of groups is 8.
The number of seats for each row L is generated from (21, 50) randomly, about 1000 seats.
The scenarios of demands are generated from (150, 350) randomly.

The second one:
The number of rows is 200.
The number of groups is 8.
The number of seats for each row L is generated from (21, 50) randomly, about 7000 seats.
The scenarios of demands are generated from (1000, 2000) randomly.

The third one:
The number of rows is 30.
The number of groups is 16.
The number of seats for each row L is generated from 41-60 randomly, about 1500 seats.
The scenarios of demands are generated from (150, 250) randomly.

% problems:
% 3. When we add more constraints, the Benders shows something wrong.

\subsection{Scenario-based method}

The stochastic model only gives the maximal people that can be served, not the maximal seat assignment. It will not give an appropriate solution when the number of arriving people of the scenarios is way less than the number of total seats because it does not utilize all the empty seats.

We make the following changes to obtain the near-optimal seat assignment in the once-decision method.

Before that, we give the deterministic model with a lower bound and upper bound of demand as below,
\begin{equation}\label{deter_lower_upper}
    \begin{aligned}
      \max \quad & \sum_{j =1}^{N} \sum_{i = 1}^{m} (s_i -1) x_{ij} \\
      \text {s.t.} \quad & \sum_{i = 1}^{m} s_i x_{ij} \leq L_{j}, j=1,\ldots,N \\
      & \sum_{j =1}^{N} x_{ij} \geq d_{i}^{l}, i=1,\ldots,m \\
      & \sum_{j =1}^{N} x_{ij} \leq d_{i}^{u}, i=1,\ldots,m \\
      & x_{ij} \geq 0, i=1,\ldots,m, j=1,\ldots,N.
    \end{aligned}
  \end{equation}

$d_{i}^{l}$ is the lower bound of the demand. (It can be the number of group type $i$ we have accepted.)
$d_{i}^{u}$ is the upper bound of the demand.  

Firstly, we obtain the solution from stochatic programming. This solution corresponds to a supply for each group type. Then solve the deterministic model by setting the supply as the upper bound of demand to obtain another feasible supply. Finally, solve the deterministic model by setting this supply as the lower bound to obtain the seat assignment.

\begin{algorithm}[H]\label{scenario_based}
  \caption{The scenario-based method to deal with the dynamic situation}
  \begin{description}
    \item[Step 1.] Obtain a linear solution from stochastic programming \eqref{BD_master1}.
    \item[Step 2.] Obtain the near-optimal integral solution from deterministic model with lower and upper bounds.
    \item[Step 3.] Accept or reject group arrivals according to the nested policy in section \ref{nested_policy}.
    \item[Step 4.] According to different scenarios, we can fix the supply or re-calculate to give a new supply. 
    % update the scenario and the probability, add constraints when re-calculating stochastic programming.
    % we can update the supply whenever some demand exceeds the supply.
  \end{description}
\end{algorithm}

In step 4, the scenarios include the ticket reservation without seat selection, ticket reservation with only row selection, ticket reservation with limited seat selection. 

In part \ref{dynamic_demand}, different methods are derived to deal with these situations. 


% 1. Why should we don't use the subset sum problem to decompose the whole problem, it will destroy global optimality. But notice that when we arrange row by row, it may also affect the optimality.

% Many symmetry structure/ Every step we need to solve a multiple knapsack problem(difficult).

\subsection{How to generate scenarios}\label{MappingSeq}
It is challenging to consider all the possible realizations; thus, it is practicable to use discrete distributions with a finite number of scenarios to approximate the random demands. This procedure is often called scenario generation.

Some papers consider obtaining a set of scenarios that realistically represents the distributions of the random parameters but is not too large. \cite{feng2013scenario} \cite{casey2005scenario}
\cite{henrion2018problem}

Another process to reduce the calculation is called scenario reduction. It tries to approximate the original scenario set with a smaller subset that retains essential features.

Solving the deterministic formulation with a large set of scenarios is not tricky in our case.

For the stochastic situation, we assume the group sizes are discretized from independent random variables following some distribution.(non-negative)

Every time we can regenerate the scenario based on the realized demands. (Use the conditional distribution or the truncated distribution)


Suppose that the groups arrive from small to large according to their size. Once a larger group comes, the smaller one will never appear again.

When a new group arrives (suppose we have accepted $n$ groups with the same size), we accept or reject it according to the supply (when $n+1 < \text{supply}$, we accept it), then update the scenario set according to the truncated distribution. We can obtain a new supply with the new probability and scenario set.

With the conclusion of section , we know how to reject a request. Once we reject one group, we will reject all groups of the same size. 

% Fix the supply of this group size, and continue this procedure.  

% If groups arrive randomly, the procedure will be similar. 

% We don't care about the arrival sequence; only the number of groups matters. Because as long as the approximation about the number of groups is accurate, we can handle any sequence.


\subsubsection{Mapping sequences to scenarios by discrete periods}

Notice that this approach still works under the assumption that time is continuous.

Suppose there are $T$ independent periods, at most one group will arrive in each period.

There are $J$ different group types(including the group with no people). Let $\mathbf{y}$ be a discrete random variable indicating the number of people in the group. Let $\mathbf{p}$ be the vector probability, where $p(y = j) = p_j$, $j = 0,1,\ldots,J-1$ and $\sum_{j} p_{j} =1$. Then a sequence can be expressed as $\{y_{1}, y_{2}, \ldots, y_{T}\}$. (It can be modeled as a multinomial distribution, $p(\mathbf{Y} \mid \mathbf{p})=\prod_{j=0}^{J-1} p_j N_j$).

Let $N_{j} = \sum_{t} I(y_t = j)$, i.e., the count number of times group type $j$ arrives during $T$ periods. Then the set of counts $N_{j}$ (scenarios) follows a multinomial distribution, $$p\left(N_0, \ldots, N_{J-1} \mid \mathbf{p}\right)=\frac{T !}{N_{0}!, \ldots, N_{J-1}!} \prod_{j=0}^{J-1} p_{j}^{N_j}, T = \sum_{j=0}^{J-1} N_{j}$$

% scenario:

% sequence will affect the result.
% Then, if we fix the seat assignment as $[3,3]$. For the sequence, $2,2,2,2,2,3,3,3,3,3$, we will place one $2$ in a three-seat. 
% But for the sequence, $3,3,3,3,3,2,2,2,2,2$, we will accept three $3$ and three $2$.

Show the complexity: the number of different sequences $J^{T}$, and the number of scenarios is $O(T^{J-1})$.

Obtained by DP:
Use $D(T,J) $ to denote the number of scenarios, which equals to the number of different solutions to $x_{1}+\ldots + x_{J} = T, \mathbf{x} \geq 0$.
Then we know the recursion relation $D(T, J) = \sum_{i= 0}^{T} D(i, J-1)$ and $D(i,2) = i+1, D(i,1) = 1$.
$D(T,3) = \frac{(T+2)(T+1)}{2}, D(T,J) = O(T^{J-1})$.

The number of scenarios is too large to enumerate all possible cases.
Thus, we choose to sample some sequences from the multinomial distribution.


% Example:
% The group types are $[2,3,4,5]$. The number of periods is $20$. The number of given rows is 4 and the number of seats is 22.
% Each group arrives with the same probability.
% The number of sequences generating from multinomial distribution is $1000$. Then, we can obtain $[0,3,6,11]$ from stochastic programming. When the number of sequences is 5000, we still obtain $[0,3,6,11]$. It shows that sampling is practical.

\subsubsection{Mapping sequences to scenarios by continuous time}

When the time is continuous, we only need to count the number of different group types, the method will be the same.

For the nonhomogeneous Poisson process, $N_{i}(T) \sim Poisson (\int_0^{T} p_i(s)\lambda(s) \mathrm{d} s)$. If the Poisson process is homogeneous, $N_{i} \sim Poisson(\lambda p_i T)$ during the interval $[0, T]$. Then for a realization of the Poisson process, we can still apply the method in Section \ref{MappingSeq}.

\subsection{Nested policy when given supply}\label{nested_policy}
% Recall that the stochastic programming only considers the situation that small-size groups can use the surplus large-size seats.

Once we obtain a solution from the stochastic programming, we need to follow some basic rules to assign the seats.
\begin{itemize}
    \item When the supply of one arriving group is enough, we will accept the group directly.
    \item When the supply of one arriving group is 0, the demand can be satisfied by only one larger-size supply.
    \item When one group is accpected to occupy the larger-size seats, the rest empty seat(s) can be reserved for the future demand.
\end{itemize}

we can assign the seats to the corresponding-size group. But when a group comes while the corresponding supply is 0, should we assign this group to the larger-size seats? Now we demonstate the nested policy for this problem.

If we accept a group of $i$ to take over $j$-size seats, then the expected served people is $i + (j-i-1)P(D_{j-i-1} \geq x_{j-i-1}+1)$, where $i < j$, $P(D_i \geq x_i)$ is the probability of that the expected demand of group type $i$ in the following periods is no less than $x_i$, the remaining supply of group type $i$.

When a group of $i$ occupies $j$-size seats, $(j-i-1)$ seats can be provided for one group of $j-i-1$ with one seat of social distancing.
Thus, $P(D_{j-i-1} \geq x_{j-i-1}+1)$ indicates the probability of that the demand of group type $(j-i-1)$ in the future is no less than its current remaining supply plus 1. If $j -i-1 =0$, then this term equals 0.

Similarly, when the expected demand of group of $j$ in the future is no less than its remaining supply currently, we would reject a group of $i$, the expected served people is $j P(D_{j} \geq x_{j})$.

Let $d(i,j)$ be the difference of expected served people between acceptance and rejection on group $i$ occupying $j$-size seats. Then $d(i,j) = i + (j-i-1)P(D_{j-i-1} \geq x_{j-i-1}+1) - j P(D_{j} \geq x_{j}), j >i$.

One intuitive decision is to choose the largest difference.

We can obtain $d(i,j) = j P(D_{j} \leq x_{j} -1) - (j-i-1)P(D_{j-i-1} \leq x_{j-i-1}) -1$ after reformulating. 
Let $F_{j}(x;T)$ be the cumulative distribution function of the number of arrival groups $D_{j}$ in $T$ periods. Then $F_{j}(x; T_{r}) = P(D_{j} \leq x)$, and $D_{j}$ follows a binomial distribution $B(T_{r}, p_{j})$, where $T_{r}$ is the numebr of remaining periods.

Thus, $d(i,j) = j F_{j}(x_{j}-1; T) - (j-i-1) F_{j-i-1}(x_{j-i-1}; T) -1$. For all $j >i$, find the largest $d(i,j)$, denoted as $d(i,j^{*})$.

If $d(i,j^{*}) >0$, we will place the group $i$ in $j^{*}$-size seats. Otherwise, the group will be rejected.

The algorithm is shown below:

\begin{algorithm}[H]\label{several_class}
  \caption{Nested policy under given supply}
  \begin{description}
    \item[Step 1.] Obtain a supply, $\X^{0} = [x_1,\ldots,x_{J}]$, from the stochastic programming.
    \item[Step 2.] For the arrival group type $i$ at period $T{'}$, if $x_{i} > 0$, accept it. Let $x_{i} = x_{i} -1$. Go to step 4.
    \item[Step 3.] If $x_{i} = 0$, find $d(i,j^{*})$. If $d(i,j^{*})>0$, accpect group type $i$. Set $x_{j^{*}} = x_{j^{*}} -1$. Let $x_{j-i-1} = x_{j-i-1} + 1$ when $j-i-1>0$. If $d(i,j^{*}) \leq 0$, reject group type $i$.
    \item[Step 4.] If $T{'} \leq T$, move to next period, set $T{'} = T{'}+1$, go to step 2.
  \end{description}
\end{algorithm}

% Example:

% If the supply is $[0,3,6,11]$, then here comes a group of 1. There will be three choices.
% \begin{align*}
% 1 \geq 2 P(D_{2} & \geq x_{2}) \\
% 1 + 1\cdot P(D_{1}\geq 1) & \geq 3 P(D_{3}\geq x_{3}) \\
% 1 + 2\cdot P(D_{2}\geq (1+ x_{2})) & \geq 4 P(D_{4}\geq x_{4})
% \end{align*}
% $\mathbf{x}$ is the remaining supply right now.

\begin{table}[ht]
  \caption{Decomposion and IP with nested policy}
  \begin{tabular}{l|l|l|l|l|l}
  \hline
  \# samples & T & probabilities & \# rows & people served by decomposition & people served by IP \\
  1000  & 45  & [0.4,0.4,0.1,0.1] & 8 & 0.12 & 0.12 \\
  1000  & 50  & [0.4,0.4,0.1,0.1] & 8 & 1.86 & 1.86 \\
  1000  & 55  & [0.4,0.4,0.1,0.1] & 8 & 5.39 & NA  \\ % slow
  1000  & 60  & [0.4,0.4,0.1,0.1] & 8 & 7.41 & NA  \\
  1000  & 65  & [0.4,0.4,0.1,0.1] & 8 & 9.03 & 9.07 \\
  \hline
  1000  & 37  & [0.25,0.25,0.25,0.25] & 8 & 0.10 & 0.10 \\
  1000  & 40  & [0.25,0.25,0.25,0.25] & 8 & 1.54 & 1.54 \\
  1000  & 45  & [0.25,0.25,0.25,0.25] & 8 & 7.17 & 7.17 \\
  1000  & 50  & [0.25,0.25,0.25,0.25] & 8 & 8.83 & 8.95 \\
  1000  & 55  & [0.25,0.25,0.25,0.25] & 8 & 11.10 & 11.10 \\
  \hline
  5000  & 300  & [0.25,0.25,0.25,0.25] & 30 & 0.00 & 0.00 \\
  5000  & 350  & [0.25,0.25,0.25,0.25] & 30 & 5.28 & 5.28 \\
  5000  & 400  & [0.25,0.25,0.25,0.25] & 30 & 8.16 & 8.20 \\
  5000  & 450  & [0.25,0.25,0.25,0.25] & 30 & 11.00 & 11.04 \\
  \hline
  \end{tabular}
\end{table}

Need to update the table to show benders decomposition and IP.

Fix the number of seats for each row: 20, 40.
Each entry of people served is the average of 50 instances.

IP will spend more than 2 hours in some instances, as `NA' showed in the table.
