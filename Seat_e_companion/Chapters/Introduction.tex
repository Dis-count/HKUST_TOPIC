% !TEX root = sum1.tex


%{\bf Terminologies to use}

%We use {\em seating management} to refer to the general problem which includes {\em seat planning with deterministic requests}, 
%{\em seat planning with stochastic requests}, and {\em seat assignment}.

%Each problem is defined for an {\em event} which has multiple {\em seating requests}, where each request has a {\em group} of people to be seated.

\section{Introduction}
Social distancing has proven effective in containing the spread of infectious diseases. For instance, during the recent COVID-19 pandemic, the fundamental requirement of social distancing involved establishing a minimum physical distance between individuals in public spaces. However, the principles of social distancing can also be applied to various industries beyond health prevention.
For example, restaurants may adopt social distancing practices to enhance guest experience and satisfaction while fostering a sense of privacy. In event management, particularly for large gatherings, social distancing can improve comfort and safety, even in non-health-related contexts, by providing attendees with more personal space.


As a general principle, social distancing measures can be defined from different dimensions. The basic requirement of social distancing is the specification of a minimum physical distance between individuals in public areas. For example, the World Health Organization (WHO) suggests to``keep physical distance of at least 1 meter from others'' \cite{AdviceforPublic}. In the US, the Centers for Disease Control and Prevention (CDC) describes social distancing as ``keeping a safe space between yourself and other people who are not from your household'' \cite{CDC}. 
It's important to note that this requirement is typically applied with respect to groups of people. For instance, in Hong Kong, the government has implemented social distancing measures during the Covid-19 pandemic by limiting the size of groups in public gatherings to two, four, and six people per group over time. Moreover, the Hong Kong government has also established an upper limit on the total number of people in a venue; for example, restaurants were allowed to operate at 50\% or 75\% of their normal seating capacity. 

From a company's perspective, social distancing can disrupt normal operations in certain sectors. For example, a restaurant needs to change or redesign the layout of its tables to comply with social distancing requirements. Such change often results in reduced capacity, fewer customers, and consequently, less revenue. In this context, affected firms face the challenge of optimizing its operational flow when adhering to social distancing policies.
From a government perspective, the impact of enforcing social distancing measures on economic activities is a critical consideration in decision-making. Facing an outbreak of an infectious disease, a government must implement social distancing policy based on a holistic analysis. This analysis should take into account not only the severity of the outbreak but also the potential impact on all stakeholders. What is particularly important is the evaluation of business losses suffered by the industries that are directly affected.  

%However, social distancing requirements are not universally applicable across all venues. Strict physical distancing measures can significantly reduce the seating capacity of cinemas with limited rows and seat spacing, making such policies impractical for some businesses. Therefore, we aim to develop a policy that is feasible for all venues to implement, balancing the needs of both government regulations and business operations. By evaluating the impact of these strategies, we seek to provide actionable insights for effective policy implementation and sustainable business practices.

We will address the above issues of social distancing in the context of seating management. Consider a venue, such as a cinema or a conference hall, which is used to host an event. The venue is equipped with seats of multiple rows. During the event, requests for seats arrive in groups, each containing a limited number of individuals. Each group can be either accepted or rejected, and those that are accepted will be seated consecutively in one row. Each row can accommodate multiple groups as long as any two adjacent groups in the same row are separated by one or multiple empty seats to comply with social distancing requirements. The objective is to maximize the number of individuals accepted for seating. 

Seat management is critically dependent on demand patterns. We will consider three distinct problems related to seat management: the Seat Planning with Deterministic Requests (SPDR) problem, the Seat Planning with Stochastic Requests (SPSR) problem, and the Seat Assignment with Dynamic Requests (SADR) problem. In the first problem, SPDR problem, complete information about seating requests in groups is known. This applies to scenarios where the participants and their groups are identified, such as family members attending a church gathering or staff from the same office at a company meeting. In the second problem, SPSR problem, the requests are unknown but follow a probabilistic distribution. This problem is relevant in situations where a new seating layout must accommodate multiple events with varying seating requests. For example, during the COVID-19 outbreak, some theaters physically removed some seats and used the remaining ones to create a seating plan that accommodates stochastic requests. In the third problem, SADR problem, groups of seating requests arrive dynamically. The problem is to decide, upon the arrival of each group of request, whether to accept or reject the group, and assign seats for each accepted group. Such seat assignment is applicable in commercial settings where requests arrive as a stochastic process, such as ticket sales in movie theaters.


%We will consider three problems for managing the seats, referred to as seat planning with deterministic requests problem (SPDRP), seat planning with stochastic requests problem (SPSRP), and seat assignment with dynamic demand problem (SADRP), respectively. As we elaborate below, each of these models defines a standalone problem with suitable situations. Together, they are inherently connected to each other, jointly forming a suite of solution schemes for seating management under the social distancing constraints.

%In the first problem, SPDRP, we are given the complete information about seating requests in groups, and the problem is to find a seat plan which specifies a partition of the layout into small segments to match the seating requests. Such a problem is applicable for cases of which participants and their groups are known, such as people from the same family in a church gathering, and staff from the same office in a company meeting. We formulate the problem by Integer Programming and discuss some characteristics of the optimal plan.
 
%In the second problem, SPSRP, we need to find a seat plan facing the requests in terms of a probabilistic distribution. This problem may find its applications in situations where a new layout needs to be made for serving multiple events with different seating requests. For example, there are theaters \cite{Berlin_theater} physically removing some seats during the Covid-19 outbreak, where the remaining seats essentially form a seat plan with stochastic requests. We formulate the problem by scenario-based optimization and develop solution approaches by Benders decomposition.

%In the third problem, SADDP, groups of seating requests arrive dynamically. The problem is to decide, upon the arrival of each group of request, whether to accept or reject the group, and assign seats for each accepted groups. Seat assignment is applicable in commercial settings where requests arrive as a stochastic process, such as ticket sales in movie theaters. A DP-based heuristic approach is employed to determine whether to accept or reject each request, followed by a group-type control policy deciding whether to assign seats to the accepted groups.


We develop models and derive optimal solutions for each of these three problems. Specifically, we formulate SPDR problem using Integer Programming and discuss the key characteristics of the optimal seating plan. For SPSR problem, we utilize scenario-based optimization and develop solution approaches based on Benders decomposition. Regarding SADR problem, we implement a two-stage seat-plan-based assignment approach. In the initial decision phase, a relaxed dynamic programming evaluates each incoming request to determine acceptance. The accepted requests then proceed to the assignment phase, where the group-type control allocation is performed. This seat-plan-based assignment policy outperforms traditional bid-price and booking-limit policies. Although each of these models represents a standalone problem tailored to specific situations, they are closely interconnected in terms of problem-solving methods and managerial insights. In the seat planning with deterministic requests (SPDR) problem, we identify important concepts such as the full pattern and the largest pattern, which play a crucial role in developing solutions for the other two problems. Additionally, SPDR problem serves as a useful offline benchmark for evaluating the performance of policies in SADR problem. Furthermore, the solution to SPSR problem can serve as a reference seat plan for dynamic seat assignment in SADR problem.

% The duality analysis in the SPDR problem also aids in developing the bid-price policy for the SADR problem.

%Besides developing models and solution schemes for operational solutions satisfying social distancing requirements, we are also interested in understanding the impact of social distancing realized over particular events. Note that although the seating capacity is reduced by social distancing, this does not necessarily mean the same reduction of the number of people to be held for an event, especially when the event needs a small number of seats. For example, consider a seat plan with 70 seats available in a venue of 100 seats, i.e., a 30\% reduction of the seating capacity. If an event held in the venue needs less than 70 seats, then it is possible that there will be a small number of people to be rejected, which implies that the loss caused by the social distancing is much less than 30\%. It is important for a government to include such an effect in policy making.


We investigate the impact of social distancing on revenue loss. To facilitate this analysis, we introduce the concept of the threshold of request-volume, which represents the upper limit on the number of requests an event can accommodate without being affected by social distancing measures. Specifically, if an event receives fewer requests than the threshold of request-volume, it will experience virtually no revenue loss due to social distancing. Our computational experiments demonstrate that the threshold of request-volume primarily depends on the mean of group size and is relatively insensitive to the specific distribution of group sizes. This finding provides a straightforward method for estimating the threshold of request-volume and evaluating the impact of social distancing.
In some instances, the government imposes a maximum allowable occupancy rate to enforce stricter social distancing requirements. To assess this effect, we introduce the concept of the threshold of occupancy rate, defined as the occupancy rate at the threshold of request-volume. The maximum allowable occupancy rate is effective for an event only if it is lower than the event's threshold of occupancy rate. Moreover, it becomes redundant if it exceeds the maximum achievable occupancy rate for all events.



%2. Our models and analysis are developed for the social distancing requirement on the physical distance and group size, where we can determine a threshold occupancy rate for any given event in a venue, and a maximum achievable occupancy rate for all events. Sometimes the government also imposes a maximum allowable occupancy rate to tighten the social distancing requirement. This maximum allowable rate is effective for an event if it is lower than the threshold occupancy rate of the event. Furthermore, the maximum allowable rate will be redundant if it is higher than the maximum achievable rate for all events.

These qualitative insights are stable with respect to the government policy's strictness and the specific characteristics of various venues, such as minimum physical distance, allowable largest group size, and venue layout. When the minimum physical distance increases, both the threshold of occupation rate and maximum achievable occupation rate decrease accordingly. Conversely, when the allowable largest group size decreases, the number of accepted requests may increase; however, both the threshold of occupation rate and maximum achievable occupation rate decline. Although venue layouts may vary in shapes (rectangular or otherwise) and row lengths (long or short), the threshold of occupancy rate and maximum achievable occupancy rate do not exhibit significant variation.


The rest of this paper is structured as follows. We review the relevant literature in Section \ref{literature}. Then we introduce the major issues brought by social distancing and define the seating planning with deterministic requests in Section \ref{problem_description}. In Section \ref{sec_seat_planning}, we establish the stochastic model, analyze its properties and obtain the seat planning. Section \ref{sec_dynamic_seat} demonstrates the seat-plan-based assignment policy to assign seats for incoming requests. Section \ref{sec_result} gives the numerical results and insights of implementing social distancing. Conclusions are shown in Section \ref{sec_conclusion}.
% \newpage
