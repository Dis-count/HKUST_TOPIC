\begin{table}[h]
    \centering
    \caption{Performances of Different Policies}\label{tab_perf}
    \begin{tabular}{cccccc}
    \hline
    Distribution & T & SPBA (\%) & RDPH (\%) & BPC (\%) & BLC (\%) \\
    % \Xcline{1-1}{0.4pt}\Xcline{3-3}{0.4pt}\Xcline{4-4}{0.4pt}
    % \cmidrule(r){0-1} \cmidrule(lr){3-3} \cmidrule(lr){4-4} \cmidrule(lr){5-5} \cmidrule(lr){6-6} \cmidrule(l){7-7}
    \hline
    \multirow{5}{*}{$D_1$} & 60 & 100.00 & 100.00 & 100.00 & 88.56 \\
    & 70    & 99.53 & 99.01 & 98.98 & 92.69  \\
    & 80    & 99.38 & 98.91 & 98.84 & 97.06  \\
    & 90    & 99.52 & 99.23 & 99.10 & 98.24  \\
    & 100   & 99.58 & 99.27 & 98.95 & 98.46 \\
    \hline
    \multirow{5}{*}{$D_2$} & 60  & 100.00 & 100.00 & 100.00 & 93.68  \\
       & 70  & 100.00 & 100.00 & 100.00 & 92.88  \\
       & 80  & 99.54 & 97.89 & 97.21 & 98.98  \\
       & 90  & 99.90 & 99.73 & 99.44 & 99.61  \\
       & 100 & 100.00 & 100.00 & 100.00 & 99.89  \\
    \hline
    \multirow{5}{*}{$D_3$} & 60  & 100.00 & 100.00 & 100.00 & 91.07  \\
    & 70  & 99.85 & 99.76 & 99.73 & 90.15 \\
    & 80  & 99.22 & 98.92 & 98.40 & 96.98  \\
    & 90  & 99.39 & 99.12 & 98.36 & 96.93  \\
    & 100  & 99.32 & 99.18 & 98.88 & 97.63  \\
      \hline
      \multirow{5}{*}{$D_4$} & 60  & 99.25 & 99.18 & 99.13 & 93.45  \\
       & 70  & 99.20 & 98.65 & 98.54 & 97.79  \\
       & 80  & 99.25 & 98.69 & 98.40 & 98.22 \\
       & 90  & 99.29 & 98.65 & 98.02 & 98.42  \\
       & 100 & 99.60 & 99.14 & 98.32 & 98.68 \\
    \hline
    \end{tabular}
  \end{table}


  \begin{table}[h]
   \centering
   \caption{Performances of Different Policies}\label{tab_perf}
   \begin{tabular}{cccccccccc}
   \hline
    & T & SPBA (\%) & RDPH (\%) & BPC (\%) & BLC (\%) & W1 (\%) & W2 (\%) & W3 (\%) & W4 (\%) \\
   % \Xcline{1-1}{0.4pt}\Xcline{3-3}{0.4pt}\Xcline{4-4}{0.4pt}
   \cmidrule(r){0-1} \cmidrule(lr){3-6} \cmidrule(lr){7-10} 
   % \hline
   \multirow{5}{*}{$D_1$} & 60 & 100.00 & 100.00 & 100.00 & 88.56 & 100.00 & 100.00 & 100.00 & 56.14  \\
   & 70    & 99.53 & 99.01 & 98.98 & 92.69  & 97.12 & 96.45 & 95.83 & 78.79 \\
   & 80    & 99.38 & 98.91 & 98.84 & 97.06  & 97.20 & 95.83 & 95.89 & 89.93 \\
   & 90    & 99.52 & 99.23 & 99.10 & 98.24  & 97.90 & 96.55 & 95.86 & 94.48 \\
   & 100   & 99.58 & 99.27 & 98.95 & 98.46  & 97.26 & 97.24 & 96.55 & 95.86 \\
   \hline
   \multirow{5}{*}{$D_2$} & 60  & 100.00 & 100.00 & 100.00 & 93.68 & 100.00 & 100.00 & 100.00 & 63.73  \\
      & 70  & 100.00 & 100.00 & 100.00 & 92.88 & 100.00 & 100.00 & 100.00 & 72.27 \\
      & 80  & 99.54 & 97.89 & 97.21 & 98.98 & 97.83 & 94.07 & 94.81 & 95.56 \\
      & 90  & 99.90 & 99.73 & 99.44 & 99.61 & 98.54 & 96.40 & 95.68 & 94.89 \\
      & 100 & 100.00 & 100.00 & 100.00 & 99.89 & 100.00 & 100.00 & 100.00 & 100.00 \\ 
   \hline
   \multirow{5}{*}{$D_3$} & 60  & 100.00 & 100.00 & 100.00 & 91.07 & 100.00 & 100.00 & 100.00 & 71.43  \\
   & 70  & 99.85 & 99.76 & 99.73 & 90.15 & 97.84 & 94.20 & 94.20 & 74.63 \\
   & 80  & 99.22 & 98.92 & 98.40 & 96.98 & 96.48 & 94.41 & 95.07 & 88.73 \\
   & 90  & 99.39 & 99.12 & 98.36 & 96.93 &  \\
   & 100  & 99.32 & 99.18 & 98.88 & 97.63  \\
     \hline
     \multirow{5}{*}{$D_4$} & 60  & 99.25 & 99.18 & 99.13 & 93.45  \\
      & 70  & 99.20 & 98.65 & 98.54 & 97.79  \\
      & 80  & 99.25 & 98.69 & 98.40 & 98.22 \\
      & 90  & 99.29 & 98.65 & 98.02 & 98.42  \\
      & 100 & 99.60 & 99.14 & 98.32 & 98.68 \\
   \hline
   \end{tabular}
 \end{table}


 \subsubsection*{Tie-Breaking for Row Selection}\label{tie-break}
 A tie occurs when there are several rows to assign the request. To determine the appropriate row for seat assignment, we can apply the following tie-breaking rules among the possible options. Suppose one request of type ${i}$ arrives, the current seat plan is $\bm{H} = [\bm{h}_{1}^{\intercal}, \ldots, \bm{h}_{N}^{\intercal}]$, the corresponding supply is $\bm{X}$. Let $\beta_{j} = L_j - \sum_{i} n_{i} H_{ij}$ represent the remaining number of seats in row $j$ after considering the seat allocation for the assigned requests. 
 
 Given the condition $X_{i} > 0$, we first identify the candidate set $\mathcal{A} =\arg \min_{j \in \mathcal{N}} \{\beta_{j}| H_{ij} > 0\}$, by minimizing the $\beta$ values among the feasible set. From this set, we then select $\mathcal{B} = \arg \min_{j \in \mathcal{A}} \{L_{j}\}$ to determine the rows with minimum size. The request is ultimately assigned to an arbitrary row $k \in \mathcal{B}$ that maximizes seat utilization. 
 
 When $X_{i} = 0$ and the request is accepted to take the seats planned for type $\hat{i}, \hat{i}>i$, 
 the candidate set $\mathcal{A} =\arg \max_{j} \{\beta_{j}| H_{\hat{i} j}>0 \}$
 
 $k \in \arg \max_{j} \{\beta_{j}| H_{j \hat{i}}>0\}$. 
 
 That can help reduce the number of rows that are not full. 
 
 we assign the request to a row 
 
 When there are multiple $k$s available, we can choose one arbitrarily. 
 
 This rule in both scenarios prioritizes filling rows and leads to better seat management.