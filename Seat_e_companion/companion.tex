\documentclass{article}
\usepackage{hyperref} % 可选:添加超链接支持
\usepackage{graphicx}

\usepackage{xr}
\externaldocument{sum1}

\usepackage{indentfirst}
\setlength{\parindent}{2em}  % 用于首行缩进

\usepackage{amsmath}
\usepackage{enumerate}
\usepackage{mathtools}
\usepackage{amsthm} % 使用定理环境
\usepackage{amssymb}

\usepackage{bm}
\usepackage{pdfpages}
\usepackage{tabu}
\usepackage{multirow}
\usepackage{multicol}
\usepackage{float}
\usepackage{makecell}
\usepackage{booktabs}
\usepackage{url}

\usepackage[style=ieee]{biblatex} % 或 \usepackage{natbib}
\addbibresource{ec.bib} % 指定参考文献数据库


% 自定义 e-companion 的编号格式(例如 EC Section 1)
\renewcommand{\thesection}{EC.\arabic{section}} % 节编号格式:EC.1, EC.2
\renewcommand{\thefigure}{EC.\arabic{figure}}   % 图编号格式:EC.1, EC.2
\renewcommand{\thetable}{EC.\arabic{table}}    % 表编号格式:EC.1, EC.2

\newenvironment{pf}[1]{\noindent\textbf{#1}\par\medskip}{\hfill $\blacksquare$\par\medskip}
\newtheorem{corollary}{\hspace{2em}Corollary}
\newtheorem{prop}{\hspace{2em}Proposition}

% 自定义定理样式:编号格式为 EC.X,并保留缩进
\newtheoremstyle{ecompanion} % 样式名
  {2em}                     % 上方间距
  {2em}                     % 下方间距
  {\normalfont}             % 正文字体
  {}                        % 缩进(由\hspace控制)
  {\bfseries}               % 标题字体
  {.}                       % 标题后标点
  {5pt plus 1pt minus 1pt}  % 标题后间距
  {\thmname{#1}\thmnumber{ EC.#2}\thmnote{ (#3)}} % 编号格式

% 应用样式并定义 lemma 环境
\theoremstyle{ecompanion}
\newtheorem{lem}{Lemma} % 自动添加 EC. 前缀

\usepackage[linesnumbered,tworuled]{algorithm2e}
\SetKwComment{Comment}{/* }{ */}
% ===== 关键修改:重定义算法编号格式 =====
\makeatletter
\renewcommand{\fnum@algocf}{Algorithm EC.\thealgocf} % 添加 EC. 前缀
\makeatother
\RestyleAlgo{ruled}

\begin{document}

% 标题注明这是 e-companion
\title{Electronic Companion to \\ \textit{Seating Management under Social Distancing}}
\author{}
\date{}
\maketitle

% 补充内容

% !TEX root = sum1.tex
\newpage

\appendix
\section{Policies for Dynamic Situations}\label{policies}

\subsubsection*{Relaxed Dynamic Programming Heuristic (RDPH)}

According to the RDP formulation in \eqref{DP_relaxed}, we can determine whether to accept or reject each request. For accepted requests, we must then decide on specific seat assignments. However, without a predefined seat plan, this assignment process lacks clear guidelines. To resolve this, we implement a modified Best Fit rule \citep{johnson1974fast}, assigning each group to the row with the minimal remaining capacity that can still accommodate it. An important prerequisite for this assignment is verifying seat availability. Specifically, if the group size exceeds the maximum remaining capacity across all rows, the request must be rejected.

This policy is stated in the following algorithm.

\begin{algorithm}[H]
  \caption{RDP Heuristic}\label{algo_dp_heuris}
  Calculate $V^{t}(l)$ by \eqref{DP_relaxed}, $\forall t =2, \ldots, T; \forall l = 1, 2, \ldots, l^{1}=\tilde{L}$\;
  \For{$t =1, \ldots, T$}{
    {Observe a request of group type $i$\;}
    \eIf{$\max_{j \in \mathcal{N}} L_{j}^{t} \geq n_i$ and $V^{t+1}(l^{t}) \leq V^{t+1}(l^{t}-n_i) + i$}
    {Set $k = \arg \min_{j \in \mathcal{N}}\{L_j^{t}|L_j^{t} \geq n_i\} $ and break ties\;
    Assign the group to row $k$, let $L_{k}^{t+1} \gets L_{k}^{t} - n_{i}$, $l^{t+1} \gets l^{t}- n_{i}$\;}
    {Reject the group and let $L_{k}^{t+1} \gets L_{k}^{t}$, $l^{t+1} \gets l^{t}$\;}}
\end{algorithm}

\subsubsection*{Bid-Price Control (BPC) Policy}
Bid-price control is a classical approach discussed extensively in the literature on network revenue management. It involves setting bid prices for different group types, which determine the eligibility of groups to take the seats. Bid-prices refer to the opportunity costs of taking one seat. As usual, we estimate the bid price of a seat by the shadow price of the capacity constraint corresponding to some row. In this section, we will demonstrate the implementation of the bid-price control policy. 

The dual of LP relaxation of the SPDR problem is:

\begin{equation}\label{bid-price_dual}
  \begin{aligned}
  \min \quad & \sum_{i=1}^{M} d_i z_i + \sum_{j= 1}^{N} L_j \beta_{j} \\
  \text {s.t.} \quad & z_{i} + \beta_j n_i \geq (n_i-\delta), \quad i \in \mathcal{M}, j \in \mathcal{N} \\
  & z_{i} \geq 0, i \in \mathcal{M}, \beta_{j} \geq 0, j \in \mathcal{N}.
  \end{aligned}
\end{equation}

In \eqref{bid-price_dual}, $\beta_{j}$ can be interpreted as the bid-price for a seat in row $j$. A request is only accepted if the revenue it generates is no less than the sum of the bid prices of the seats it uses. Thus, if $i -\beta_{j} n_i \geq 0$, we will accept the group type $i$. And choose $j^{*} = \arg \max_{j} \{i -\beta_{j} n_i\}$ as the row to allocate that group.


\begin{lem}\label{bid-price}
 The optimal solution to problem \eqref{bid-price_dual} is given by $z_1 ,\ldots, z_{\tilde{i}} =0$, $z_{i} = \frac{\delta(n_i-n_{\tilde{i}})}{n_{\tilde{i}}}$ for $i = \tilde{i}+1, \ldots, M$ and $\beta_j = \frac{n_{\tilde{i}} - \delta}{n_{\tilde{i}}}$ for all $j$.
\end{lem}

The bid-price decision can be expressed as $i - \beta_j n_i = i - \frac{n_{\tilde{i}} - \delta}{n_{\tilde{i}}} n_i = \frac{\delta (i - \tilde{i})}{n_{\tilde{i}}}$. When $i < \tilde{i}$, $i - \beta_j n_i < 0$. When $i \geq \tilde{i}$, $i - \beta_j n_i \geq 0$. This implies that group type $i$ greater than or equal to $\tilde{i}$ will be accepted if the capacity allows. However, it should be noted that $\beta_j$ does not vary with $j$, which means the bid-price control cannot determine the specific row to assign the group to. We maintain the same tie-breaking rule as in RDPH, assigning each group to the row with the minimal residual capacity while still satisfying the accommodation requirement.

% In practice, groups are often assigned arbitrarily based on availability when the capacity allows, which may result in the empty seats.

The bid-price control policy based on the static model is stated below.

\begin{algorithm}[H]
  \caption{Bid-Price Control}\label{algo_bid}
  \For{$t =1, \ldots, T$}{
    {Observe a request of group type $i$\;}
    {Solve the LP relaxation of the SPDR problem with $\bm{d}^{t} = (T-t) \cdot \bm{p}$ and $\mathbf{L}^{t}$\;
    Obtain $\tilde{i}$ such that the aggregate optimal solution is $x e_{\tilde{i}} + \sum_{i=\tilde{i}+1} ^{M} d_{i}^{t} e_{i}$\;}
    \eIf{$i \geq \tilde{i}$ and $\max_{j \in \mathcal{N}}{L_j^{t}} \geq n_i$}
    {Set $k = \arg \min_{j \in \mathcal{N}}\{L_j^{t}|L_j^{t} \geq n_i\} $ and break ties\;
    Assign the group to row $k$, let $L_{k}^{t+1} \gets L_{k}^{t} - n_{i}$ \;}
    {Reject the group\;}}
\end{algorithm}


\subsubsection*{Booking-Limit Control (BLC) Policy}
The booking-limit control policy involves setting a maximum number of reservations that can be accepted for each request. By controlling the booking-limits, revenue managers can effectively manage demand and allocate inventory to maximize revenue. In this policy, we solve the SPDR problem with the expected demand. Then for every type of requests, we only allocate a fixed amount according to the static solution and reject all other exceeding requests. 

% When we solve the LP relaxation of SPDRP, the aggregate optimal solution is the limits for each group type. Interestingly, the bid-price control policy is found to be equivalent to the booking limit control policy.

% we can develop the booking limit control policy.

\begin{algorithm}[H]
  \caption{Booking-Limit Control}\label{algo_booking}
  \For{$t =1, \ldots, T$}{
    {Observe a request of group type $i$\;}
    {Solve the SPDR problem with $\bm{d}^{t} = (T-t) \cdot \bm{p}$ and $\mathbf{L}^{t}$\;
    Obtain the seat plan, $\bm{H}^{t}$\;}
    \eIf{$X_i > 0$}
    {Set $k = \arg \min_{j \in \mathcal{N}} \{L_j^{t} - \sum_{i}n_i H_{ij}^{t}|H_{ij}^{t} >0\}$\;
    Break ties arbitrarily\;
    Assign the group to row $k$, let $L_{k}^{t+1} \gets L_{k}^{t} - n_{i}$, $H_{ik}^{t+1} \gets H_{ik}^{t}-1$\;}
    {Reject the group\;}}
\end{algorithm}


% \subsubsection*{First-Come-First-Served (FCFS) Policy}
% In the seat assignment for each group arrival, the intuitive but trivial method will be on a first-come-first-served basis. Each accepted request will be assigned seats row by row. If the capacity of a row is insufficient to accommodate a request, we will allocate it to the next available row. If a subsequent request can fit exactly into the remaining capacity of some row, we will assign it to that row immediately. Then continue to process requests in this manner until all rows cannot accommodate any groups.

% \begin{algorithm}[H]
%   \caption{FCFS Policy Algorithm}\label{algo_fcfs}
%   \For{$t =1, \ldots, T$}{
%     {Observe a request of group type $i$\;}
%     \eIf{$\max_{j \in \mathcal{N}}{L_j^{t}} \geq n_i$}
%     {Set $k = \arg \min_{j \in \mathcal{N}}\{L_j^{t}|L_j^{t} \geq n_i\}$\;
%     Break ties arbitrarily\;
%     Assign the group to row $k$, let $L_{k}^{t+1} \gets L_{k}^{t} - n_{i}$\;}
%     {Reject the group and let $L_{k}^{t+1} \gets L_{k}^{t}$\;}}
% \end{algorithm}

\newpage


% !TEX root = sum1.tex
\clearpage
\section{Proofs}

\begin{pf}{Proof of Proposition \ref{sol_relax_deter}}
  We model the problem as a special case of the multiple knapsack problem, then we consider the LP relaxation of this problem. In the model, groups are categorized into $M$ distinct types. Each type $i$ is characterized by a fixed group size $n_i$, which serves as the weight, and an associated profit equal to $i$. For every type $i$, there are $d_i$ groups. Altogether, the total number of groups is given by $K = \sum_{i=1}^{M} d_i$. Each individual group $k$ inherits its profit and weight from its type; specifically, if group $k$ belongs to type $i$, then its profit $p_k$ is $i$, and its weight $w_k$ is $n_i$. To apply the greedy approach for the LP relaxation of \eqref{deter_upper}, sort these groups in non-increasing order of their profit-to-weight ratios: $\frac{p_1}{w_1} \geq \frac{p_2}{w_2} \geq \ldots \geq \frac{p_K}{w_K}$. The break group $b$ is the smallest index such that the cumulative weight of group 1 to group $b$ meets or exceeds the total capacity $\tilde{L}$: $b=\min \{j: \sum_{k=1}^j w_k \geq \tilde{L}\}$, where $\tilde{L} = \sum_{j=1}^{N} L_j$ is the total size of all rows. For the LP relaxation of \eqref{deter_upper}, the Dantzig upper bound \citep{dantzig1957discrete} is given by $u_{\mathrm{MKP}}=\sum_{j=1}^{b-1} p_j+\left(\tilde{L}-\sum_{j=1}^{b-1} w_j\right) \frac{p_b}{w_b}$. The corresponding optimal solution is to accept the whole groups from $1$ to $b-1$ and fractional $(\tilde{L}-\sum_{j=1}^{b-1} w_j)$ group $b$. Suppose the group $b$ belong to type $\tilde{i}$, then for $i < \tilde{i}$, $x_{ij}^{*} = 0$; for $i > \tilde{i}$, $x_{ij}^{*} = d_{i}$; for $i = \tilde{i}$, $\sum_{j=1}^{N} x_{ij}^{*} = (\tilde{L} - \sum_{i = \tilde{i}+1}^{M} {d_i n_i})/ n_{\tilde{i}}$.
\end{pf}


\begin{pf}{Proof of Lemma \ref{bid-price}}
According to the Proposition \ref{sol_relax_deter}, the aggregate optimal solution to LP relaxation of problem \eqref{deter_upper} takes the form $x e_{\tilde{i}} + \sum_{i=\tilde{i}+1} ^{M} d_{i} e_{i}$, then according to the complementary slackness property, we know that $z_1= \ldots= z_{\tilde{i}} = 0$. This implies that $\beta_j \geq \frac{n_i - \delta}{n_i}$ for $i = 1,\ldots, \tilde{i}$. Since $\frac{n_i - \delta}{n_i}$ increases with $i$, we have $\beta_j \geq \frac{n_{\tilde{i}} - \delta}{n_{\tilde{i}}}$. Consequently, we obtain $z_{i} \geq n_i - \delta - n_i \frac{n_{\tilde{i}} - \delta}{n_{\tilde{i}}} = \frac{\delta(n_i-n_{\tilde{i}})}{n_{\tilde{i}}}$ for $i = h+1, \ldots, M$.

Given that $\mathbf{d}$ and $\mathbf{L}$ are both no less than zero, the minimum value will be attained when $\beta_j = \frac{n_{\tilde{i}} - \delta}{n_{\tilde{i}}}$ for all $j$, and $z_i = \frac{\delta(n_i-n_{\tilde{i}})}{n_{\tilde{i}}}$ for $i = \tilde{i}+1, \ldots, M$.  
\end{pf}

\newpage

\section{Empirical Probabilities and Realistic Seat Layouts}\label{appen_3}
\subsection{Empirical Probabilities}
We select Movie A (representing the suspense genre) and Movie B (representing the family fun genre) as target movies to analyze group information and their corresponding probability distributions, denoted as $D3$ and $D4$, respectively.

% The seat plans for the tickets were obtained from a Hong Kong cinema website. We focused on scattered seat plans and excluded cases where the number of consecutive seats exceeded four. By counting the occurrences of different group types, we obtained these distributions. 

% $\hat{p}_i \pm z_{\alpha / 2} \sqrt{\frac{\hat{p}_i\left(1-\hat{p}_i\right)}{N}} = $
% 0.122 $\pm$ 0.011
% 0.501 $\pm$ 0.016
% 0.132 $\pm$ 0.011
% 0.246 $\pm$ 0.014

% Similarly, 

% 0.336 $\pm$ 0.025
% 0.516 $\pm$ 0.027
% 0.067 $\pm$ 0.013
% 0.081 $\pm$ 0.015


We make the screenshots about the ticket seat plans from a Hong Kong cinema website at different time intervals. When tickets were sold in advance of the movie screening, the seats were typically scattered. Therefore, we treated consecutive seats as belonging to the same group, while excluding cases where the number of consecutive seats exceeds four. 

We counted the frequency of different group types in the seat plans to derive their probability distributions. For Movie A, the frequencies for the four group types are 112, 460, 121, and 226, with a total of 919 observations. For Movie B, the frequencies are 116, 178, 23, and 28, with a total of 345 observations. We keep two decimal places, then obtain the probability:

$p_1^{A} =  0.12$, $p_2^{A} =  0.50$, $p_3^{A} = 0.13$, $p_4^{A} = 0.25$ and $p_1^{B} =  0.34$, $p_2^{B} =  0.52$, $p_3^{B} = 0.07$, $p_4^{B} = 0.08$.

% ${p}_i \pm z_{\alpha / 2} \sqrt{\frac{{p}_i\left(1-{p}_i\right)}{N}}, z_{\alpha / 2} = 1.96$

Using the normal distribution approximation method (with a 95\% confidence interval), the confidence intervals for the probabilities of each group type for Movie A is presented as follows:
$CI_1^{A} =  0.122 \pm 0.011$, $CI_2^{A} =  0.501 \pm 0.016$, $CI_3^{A} = 0.132 \pm 0.011$, $CI_4^{A} = 0.246 \pm 0.014$

Similarly, the confidence intervals for the probabilities of each group type for Movie B are:

$CI_1^{B} =  0.336 \pm 0.025$, $CI_2^{B} =  0.516 \pm 0.027$, $CI_3^{B} = 0.067 \pm 0.013$, $CI_4^{B} = 0.081 \pm 0.015$


% HKFAC, KTTTS, SWHCC, SWCC, NCWCC

% https://www.lcsd.gov.hk/en/ticket/seat.html

\subsection{Realistic Seat Layouts}

\begin{figure}[ht]
  \caption{Layout A}
    \centering
      \includegraphics[width=0.6\textwidth]{./Figures/Layouts/Layout_A.png}
\end{figure}
  
\begin{figure}[ht]
  \caption{Layout B}
    \centering
      \includegraphics[width=0.6\textwidth]{./Figures/Layouts/Layout_B.png}
\end{figure}

\begin{figure}[ht]
  \caption{Layout C}
    \centering
      \includegraphics[width=0.6\textwidth]{./Figures/Layouts/Layout_C1.png}
\end{figure}

\begin{figure}[ht]
  \caption{Layout D}
    \centering
      \includegraphics[width=0.6\textwidth]{./Figures/Layouts/Layout_D.png}
\end{figure}

\begin{figure}[ht]
  \caption{Layout E}
    \centering
      \includegraphics[width=0.6\textwidth]{./Figures/Layouts/Layout_E.png}
\end{figure}



% \section{Proof of Theorem 1} \label{sec:ec-proof}
% Here we provide the full proof omitted in the main text...

% \section{Additional Experiments} \label{sec:ec-experiments}
% \subsection{Dataset Details} \label{sec:ec-data}
% The extended dataset is described in Table~\ref{tab:ec-data}.

% % 示例表格
% \begin{table}[ht]
% \centering
% \caption{Extended Dataset Summary} \label{tab:ec-data}
% \begin{tabular}{lc}
% \textbf{Feature} & \textbf{Value} \\
% Size & 10,000 samples \\
% Dimensions & 20 \\
% \end{tabular}
% \end{table}

\printbibliography[title={References in E-Companion}]

\end{document}