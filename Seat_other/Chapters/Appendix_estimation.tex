% \section{Estimates of $q^{\textup{th}}$ and $\rho^{\textup{th}}$}

To estimate the threshold of request-volume $q^{\textup{th}}$, we aim to find the maximum period such that all requests can be assigned into the seats during these periods, i.e., for each group type $i$, we have $\bm{X}_{i} = \sum_{j =1}^{N} x_{ij} \geq d_i$. Meanwhile, we have the capacity constraint $\sum_{i =1}^{M} n_{i} x_{ij} \leq L_j$, thus, $\sum_{i =1}^{M} n_i d_i \leq \sum_{i =1}^{M} \sum_{j =1}^{N} n_i x_{ij} \leq \sum_{j =1}^{N} L_{j}$. Notice that $E(d_i) = p_i T$, we have $\sum_{i =1}^{M} n_i p_i T \leq \sum_{j =1}^{N} L_{j}$ by taking the expectation. Recall that $\tilde{L} = \sum_{j =1}^{N} L_{j} = \sum_{j =1}^{N} L_{j}^{0} + N \delta$ denotes the total size of the layout. From this, we derive the inequality $T \leq \frac{\tilde{L}}{\gamma + \delta}$. 


Under the ideal assumption that all requests fully occupy the available capacity, the ideal threshold of request-volume can be estimated as $\hat{q}^{\textup{th}} = (1-p_0) \frac{\tilde{L}}{\gamma + \delta}$. Correspondingly, the threshold of occupancy rate under this ideal scenario is: 
$\hat{\rho}^{\textup{th}}= \frac{\gamma \hat{q}^{\textup{th}}}{\tilde{L} - N \delta} = \frac{\gamma}{\gamma +\delta} \frac{(1-p_0) \tilde{L}}{\tilde{L}- N \delta}$.

However, in practice, the actual expected maximum request-volume will be smaller than $\hat{q}^{\textup{th}}$ since perfectly filling all available capacity is nearly impossible. To account for this, we introduce discount factors $c_1$ and $c_2$ for both thresholds, respectively. For the threshold of request-volume:
$\tilde{q}^{\textup{th}} =  \frac{c_1 (1-p_0) \tilde{L}}{\gamma + \delta}$, where $c_1$ adjusts for deviations from the ideal assumption.
For the threshold of occupancy rate:
$\tilde{\rho}^{\textup{th}} = \frac{c_2 \gamma}{\gamma +\delta} \frac{(1-p_0) \tilde{L}}{\tilde{L}-N \delta}$, where $c_2$ similarly reflects practical constraints.


% Under the condition of fixed other parameters, the estimate of the threshold of request-volume $\tilde{q}^{\textup{th}}$ decreases with higher mean group size $\gamma$ while the estimate of threshold of occupancy rate $\tilde{\rho}^{\textup{th}}$ increases with higher mean group size $\gamma$, which follows the pattern as the figure shows.

We analyze how the estimates of the threshold of request-volume $\tilde{q}^{\textup{th}}$ and the threshold of occupancy rate $\tilde{q}^{\textup{th}}$ vary with respect to these parameters: mean group size $\gamma$, maximum group size $M$, physical distance $\delta$, and total size of the layout $\tilde{L}$.

1. Estimate of threshold of request-volume ($\tilde{q}^{\textup{th}} = \frac{c_1 (1-p_0) \tilde{L}}{\gamma + \delta}$):
\begin{itemize}
    \item Decreases with higher mean group size $\gamma$.
    \item Indirectly decreases with larger $M$ (when $\gamma = \sum_{i=1}^{M} i p_i$ grows with $M$).
    \item Increases linearly with total layout size $\tilde{L}$.
\end{itemize}

2. Estimate of threshold of occupancy rate ($\tilde{\rho}^{\textup{th}} = \frac{c_2 \gamma}{\gamma +\delta} \frac{(1-p_0) \tilde{L}}{\tilde{L}-N \delta}$):
\begin{itemize}
    \item Increases with larger mean group size $\gamma$.
    \item Indirectly increases with larger $M$ (when $\gamma = \sum_{i=1}^{M} i p_i$ rises with $M$).
    \item Decreases with larger total layout size $\tilde{L}$.
\end{itemize}

3. Both thresholds exhibit non-trivial dependencies on $\delta$ due to the coupling in $\tilde{L} = \sum_{j =1}^{N} L_{j}^{0} + N \delta$. 

Assuming the discount factors $c_1$ and $c_2$ remain constant as $\delta$ increases, both $\tilde{q}^{\textup{th}}$ and $\tilde{\rho}^{\textup{th}}$ decrease monotonically-- provided the total capacity satisfies $\sum_{j=1}^{N} L_{j}^{0} \geq N \gamma$, a condition typically met in practical layouts.

The derivation is as follows. Recall that $\tilde{L} = \sum_{j =1}^{N} L_{j}^{0} + N \delta$. To ensure the monotonic decrease in thresholds, the following inequality must hold: 

$$\frac{(1-p_0) (\sum_{j =1}^{N} L_{j}^{0} + N \delta)}{\gamma + \delta} \geq \frac{(1-p_0) (\sum_{j =1}^{N} L_{j}^{0} + N (\delta +1))}{\gamma + (\delta +1)},$$ which simplifies to the condition $\sum_{j=1}^{N} L_{j}^{0} \geq N \gamma$.


% Under the condition of fixed other parameters, the estimate of the threshold of request-volume $\tilde{q}^{\textup{th}}$ decreases with higher mean group size $\gamma$ and increases linearly with the total size of the layout $\tilde{L}$. Generally, $\gamma$ increases with the maximum group size $M$ (since $\gamma = \sum_{i=1}^{M} i p_i$), leading to an indirect reduction in $\tilde{q}^{\textup{th}}$. When other parameters are held constant, the estimate of threshold of occupancy rate $\tilde{\rho}^{\textup{th}}$ increases with higher mean group size $\gamma$ (and thus with larger maximum group size $M$, if $\gamma$ raises) and decreases with larger $\tilde{L}$. The estimated $\tilde{q}^{\textup{th}}$ and $\tilde{\rho}^{\textup{th}}$ exhibit non-trivial dependencies on the physical distance parameter $\delta$ since $\tilde{L} = \sum_{j =1}^{N} L_{j}^{0} + N \delta$ couples their dynamics. 

% Assuming the discount factors $c_1$ and $c_2$ remain constant as $\delta$ increases, both $\tilde{q}^{\textup{th}}$ and $\tilde{\rho}^{\textup{th}}$ decrease monotonically, provided the total capacity satisfies $\sum_{j=1}^{N} L_{j}^{0} > N \delta$. This condition is typically met in common practical layouts.