% !TEX root = sum1.tex

\subsection{Timeline}

19.01.2023
The Hong Kong Government announced the end of isolation requirements for COVID-19 patients

18.10.2022
Starting from this day, Hong Kong loosened the restrictions further, increasing the number of diners per table from 8 to 12. The circuit breaker mechanism for cruises has also been revoked

19.05.2022
Starting from today, the second stage of relaxation in Hong Kong social distancing measures will be implemented as scheduled. That includes extension of the dinnertime dine-in service hours in catering premises, allowing bars/pubs as well as other scheduled premises regulated under the Prevention and Control of Disease (Requirements and Directions) (Business and Premises) Regulation (Cap. 599F) which are currently closed, and allowing eating and drinking in cinemas etc

07.01.2022
Starting from this day, Hong Kong tightened their social distancing measures, including the closure of pubs \& bars, amusement game centres, bathhouses, fitness centres, places of amusement and party rooms. All events and dine-in services at catering businesses are instructed to end at 6 pm.

The Hong Kong government issued a statement, stating that the banquet on 3 January with several officials attended, may involve another preliminary positive case who attended the event from 6 pm to 8 pm. If the case is confirmed eventually, all participants will be required to undergo quarantine. Some reports revealed there were more than 200 participants in the banquet and some of them didn't wear masks

05.01.2022
HK new guidelines and rules:

The Hong Kong Government announced new guidelines starting from January 7th, 2022: 
Dine-in services are banned from 6 PM to 5 AM. 
Fifteen types of venues such as bars, gyms, theme parks and swimming pools will be shut.
Starting from February 15th, Italy also tightened its workplace vaccine rules that anyone aged over 50 who are working must present a health pass proving they have been vaccinated or currently recovering from COVID-19

28.09.2021

The Hong Kong government announced most of the social distancing measures will remain unchanged for two weeks. Meanwhile, the maximum number of people allowed per banquet in catering businesses adopting the Type D mode of operation will be relaxed to 240

18.02.2021 Hong Kong government begins to ease its social distancing measures:

Dine-in services at restaurant can be extended by four hours until 10 pm
The public group gatherings limit rises from two people to four people
Gyms and theme parks can now reopen, but bars, clubs, and karaoke lounges will remain closed

10.12.2020 The Hong Kong Government continues to tighten social distancing restrictions to cope with the fourth wave of COVID-19 infections:
Social gathering that includes more than a group of two people is banned.
No dine-in services are allowed after 6:00 pm.
Gym, beauty salons, amusement parks, swimming pools and other premises are also closed.

23.11.2020 Hong Kong entered a fourth wave of COVID-19 outbreak with over 600 cases linked to the dance club cluster:
Hong Kong closed all kindergartens and primary schools as a response to COVID-19 cases increasing.
Plans for the Hong Kong-Singapore travel bubble has been put on hold.
Hong Kong government launched a contact tracing initiative in a mobile app format to track untraceable infections.

30.10.2020 Hong Kong government continues to ease social distance rules as COVID-19 cases decline:
Restaurants can now offer dine-in services to 2 AM.
Pubs are allowed to serve up to 4 people per table and restaurants can serve up to 6 people per table.
The maximum of public gathering remains unchanged at 4 people.
France imposed a second national lockdown in the country due to the surge of COVID-19 cases. People are required to stay at home, no social gathering is allowed

18.09.2020 Hong Kong government continues to ease social distancing restrictions as cases decline:
Dine-in services at catering premises are extended to 12 pm.
Some indoor premises can reopen, includes bars, karaoke, theme parks.

11.09.2020 Hong Kong government further relaxes social distancing restrictions:
Public group gathering limit rises from two to four people.
Wearing a mask is no longer mandatory for outdoor exercise.

04.09.2020 Hong Kong government further relaxes social distancing restrictions:
Restaurants can now offer dine-in services until 10 pm.
Some indoor premises are allowed to reopen, including gyms, and massage parlours.

28.08.2020 Hong Kong government lifts the social distancing restrictions gradually:
Restaurants can now offer dine-in services to 9 pm.
Beauty salons, cinemas, and some outdoor sports venues can now reopen.

27.07.2020 As a result, the Hong Kong government announces a series of new measures that will come into effect from July 29th:
Gathering in public will be limited to only two people per group. Members of the same family are exempted.
Restaurants are unable to offer dine-in services for the whole day. Certain public establishments are exempted, such as eateries in public hospitals.
Masks are now required outdoors as well. There are no exemptions for exercising or smoking.

19.07.2020 The Hong Kong government announced a series of new measures, including:
Restaurants are unable to offer eat-in services from 6 pm to 5 am. Only takeaway will be available during this time. Mask-wearing is now mandatory in all indoor public places.
Gyms, beauty salons and 15 other venues are also temporarily closed.

27.03.2020 The Hong Kong government announces further restrictions, banning any indoor or outdoor gathering of more than four people. Restaurants were also required to operate at half their capacity and to set their tables at least 1.5 metres apart.

25.03.2020 Hong Kong announces the closure of its border to all incoming non-residents arriving from overseas. Transiting through Hong Kong was also no longer allowed. In addition, all returning residents were subject to compulsory quarantine for 14 days. Returning residents from high-risk countries were required to go through enhanced screening procedures and submit a saliva sample for testing.

link: https://www.otandp.com/covid-19-timeline

% only provide the seat sections that can be selected at the time of booking or simply do 
% (Row/second selection for extention)


% For example, some theaters only provide seat reservations for the audience.


% (Online for extention)
% the movie halls can seal the empty seats to prevent people from taking seats.

% For reservation with row selection, 
% the concerts will allow people to choose the section and row when making the reservation.
% 根據客戶交易的日期及時間以先後次序分配,不設自行選擇座位
% 预订票指的是,第一类,预订门票并付完款之后,无需选座,大麦网会按照付款先后顺序配送门票;第二类,预订门票并付完款之后,需要选座的场次,大麦也会提前短信通知您所在分组的选座时间,准时参加即可

% https://zhuanlan.zhihu.com/p/39125507
% 优先购票  公开发售
% 由于位置必须相连,相连座位门票一般会比单张门票位置稍差,如果是单人购票,建议把相连座位的勾去掉,这样系统会按从近到远的顺序自动为你筛选出较近的门票。

%  为了公平 分阶段 

% We don't care about the arrival sequence; only the number of groups matters. Because as long as the approximation about the number of groups is accurate, we can handle any sequence.

