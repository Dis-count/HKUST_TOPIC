\section{Seat Assignment with Dynamic Requests}\label{sec_dynamic_seat}

In many commercial situations, requests arrive sequentially over time, and the seller must immediately decide whether to accept or reject each request upon arrival while ensuring compliance with the required spacing constraints. If a request is accepted, the seller must also determine the specific seats to assign. Importantly, each request must be either fully accepted or entirely rejected; once seats are assigned to a group, they cannot be altered or reassigned to other requests.

To model this problem, we formulate it using dynamic programming approach in a discrete-time framework. Time is divided into $T$ periods, indexed forward from $1$ to $T$. We assume that in each period, at most one request arrives and the probability of an arrival for a group type $i$ is denoted as $p_i$, where $i \in \mathcal{M}$. The probabilities satisfy the constraint $\sum_{i=1}^M p_i \leq 1$, indicating that the total probability of any group arriving in a single period does not exceed one. We introduce the probability $p_0 = 1 - \sum_{i=1}^{M} p_i$ to represent the probability of no arrival in each period. To simplify the analysis, we assume that the arrivals of different group types are independent and the arrival probabilities remain constant over time. This assumption can be extended to consider dependent arrival probabilities over time if necessary.

The remaining capacity in each row is represented by a vector $\mathbf{L} = (l_1, l_2, \ldots, l_N)$, where $l_j$ denotes the number of remaining seats in row $j$. Upon the arrival of a group type $i$ at time $t$, the seller needs to make a decision denoted by $u_{i,j}^{t}$, where $u_{i,j}^{t} = 1$ indicates acceptance of group type $i$ in row $j$ during period $t$, while $u_{i,j}^{t} = 0$ signifies rejection of that group type in row $j$. The feasible decision set is defined as $$U^{t}(\mathbf{L}) = \left\{u_{i,j}^{t} \in \{0,1\}, \forall i \in \mathcal{M}, \forall j \in \mathcal{N} \bigg| \sum_{j=1}^{N} u_{i,j}^{t} \leq 1, \forall i \in \mathcal{M}; n_{i}u_{i,j}^{t}\mathbf{e}_j \leq \mathbf{L}, \forall i \in \mathcal{M}, \forall j \in \mathcal{N}\right\}.$$
Here, $\mathbf{e}_j$ represents an N-dimensional unit column vector with the $j$-th element being 1, i.e., $\mathbf{e}_j = (\underbrace{0, \cdots, 0}_{j-1}, 1, \underbrace{0, \cdots, 0}_{N-j})$. The decision set $U^{t}(\mathbf{L})$ consists of all possible combinations of acceptance and rejection decisions for each group type in each row, subject to the constraints that at most one group of each type can be accepted in any row, and the number of seats occupied by each accepted group must not exceed the remaining capacity of the row.

Let $V^{t}(\mathbf{L})$ denote the maximum expected revenue earned by the optimal decision regarding group seat assignments at the beginning of period $t$, given the remaining capacity $\mathbf{L}$. Then, the dynamic programming formulation for this problem can be expressed as:

\begin{equation}\label{DP}
V^{t}(\mathbf{L}) = \max_{u_{i,j}^{t} \in U^{t}(\mathbf{L})}\left\{\sum_{i=1}^{M} p_i \bigl( \sum_{j=1}^{N} i u_{i,j}^{t} + V^{t+1}(\mathbf{L} - \sum_{j=1}^{N} n_i u_{i,j}^{t}\mathbf{e}_j)\bigr) + p_0 V^{t+1}(\mathbf{L})\right\}
\end{equation}
with the boundary conditions $V^{T+1}(\mathbf{L}) = 0, \forall \mathbf{L}$, which implies that the revenue at the last period is 0 under any capacity. The initial capacity is denoted as $\mathbf{L}_{0} = (L_1, L_2, \ldots, L_N)$. Our objective is to determine group assignments that maximize the total expected revenue during the horizon from period 1 to $T$, represented by $V^{1}(\mathbf{L}_{0})$.


Solving the dynamic programming problem in equation \eqref{DP} presents computational challenges due to the curse of dimensionality that arises from the large state space.

We propose our policy for assigning arriving requests in a dynamic context. First, we employ the traditional bid-price control policy. Then, we improve the bid-price control policy based on the seat plan (patterns).


% \subsection{Relaxed Dynamic Programming}
% To simplify the complexity of the dynamic programming formulation in \eqref{DP}, we employ a relaxed dynamic programming (RDP) approach by aggregating all rows into a single row with the total capacity $\tilde{L} = \sum_{j=1}^{N} L_j$. This relaxation yields preliminary seat assignment decisions for each group arrival, where the rejection by the RDP is final (no further evaluation is needed), the acceptance by the RDP is tentative and must be validated according to the current seat plan in the subsequent group-type control.

% Let $u_{i}^{t} \in $ denote the RDP's decision variable for accepting ($u_{i}^{t} = 1$) or rejecting ($u_{i}^{t} = 0$) a type $i$ request in period $t$. The value function of the relaxed DP with the total capacity $l$ in period $t$, denoted by $V^{t}(l)$, is the following:

% \begin{equation}\label{DP_relaxed}
% V^{t}(l) =  \max_{u_{i}^{t} \in \{0,1\}} \left\{ \sum_{i=1}^{M} p_i \left[V^{t+1}(l-n_i u_{i}^{t})+ i u_{i}^{t}\right] + p_0 V^{t+1}(l)\right\}
% \end{equation}
% with the boundary conditions $V^{T+1}(l) =0, \forall l \geq 0$ and $V^{t}(0) =0, \forall t$.


% To make the initial decision, we compute the value function $V^{t}(l)$ and compare the values of accepting versus rejecting the request. Preliminarily accepted requests are then verified and assigned in the subsequent group-type control stage.

\subsection{Traditional BPC Policy}
Bid-price control is a classical approach discussed extensively in the literature on network revenue management. It involves setting bid prices for different group types, which determine the eligibility of groups to take the seats. Bid-prices refer to the opportunity costs of taking one seat. As usual, we estimate the bid price of a seat by the shadow price of the capacity constraint corresponding to some row. In this section, we will demonstrate the implementation of the bid-price control policy. 

The dual of LP relaxation of the SPDR problem is:

\begin{equation}\label{bid-price_dual}
    \begin{aligned}
    \min \quad & \sum_{i=1}^{M} d_i z_i + \sum_{j= 1}^{N} L_j \beta_{j} \\
    \text {s.t.} \quad & z_{i} + \beta_j n_i \geq (n_i-\delta), \quad i \in \mathcal{M}, j \in \mathcal{N} \\
    & z_{i} \geq 0, i \in \mathcal{M}, \beta_{j} \geq 0, j \in \mathcal{N}.
    \end{aligned}
  \end{equation}

In \eqref{bid-price_dual}, $\beta_{j}$ can be interpreted as the bid-price for a seat in row $j$. A request is only accepted if the revenue it generates is no less than the sum of the bid prices of the seats it uses. Thus, if $i -\beta_{j} n_i \geq 0$, meanwhile, the capacity allows, we will accept the group type $i$. And choose $j^{*} = \arg \max_{j} \{i -\beta_{j} n_i\}$ as the row to allocate that group.
  
\begin{lem}\label{bid-price}
The optimal solution to problem \eqref{bid-price_dual} is given by $z_1 = \ldots = z_{\tilde{i}} =0$, $z_{i} = \frac{\delta(n_i-n_{\tilde{i}})}{n_{\tilde{i}}}$ for $i = \tilde{i}+1, \ldots, M$ and $\beta_j = \frac{n_{\tilde{i}} - \delta}{n_{\tilde{i}}}$ for all $j$.
\end{lem}


\begin{algorithm}[H]
    \caption{Bid-Price Control}\label{algo_bid}
    \For{$t =1, \ldots, T$}{
      {Observe a request of group type $i$\;}
      {Solve problem \eqref{bid-price_dual} with $\bm{d}^{t} = (T-t) \cdot \bm{p}$ and $\mathbf{L}^{t}$\;
      Obtain $\tilde{i}$ such that the aggregate optimal solution is $x e_{\tilde{i}} + \sum_{i=\tilde{i}+1} ^{M} d_{i}^{t} e_{i}$\;}
      \eIf{$i \geq \tilde{i}$ and $\max_{j \in \mathcal{N}}{L_j^{t}} \geq n_i$}
      {Set $k = \arg \min_{j \in \mathcal{N}}\{L_j^{t}|L_j^{t} \geq n_i\} $ and break ties\;
      Assign the group to row $k$, let $L_{k}^{t+1} \gets L_{k}^{t} - n_{i}$ \;}
      {Reject the group\;}}
\end{algorithm}



% Let $val_{\theta}(I; \{d_{i}\})$ denote the optimal objective value of \eqref{theta_deter}.

% \begin{align}
%     \quad \max \quad & \sum_{i=1}^{M}  \sum_{j= 1}^{N} (n_i - \delta) x_{ij} \label{theta_deter} \\
%     \text {s.t.} \quad & \sum_{j= 1}^{N} x_{ij} \leq d_{i}, \quad i \in \mathcal{M}, \notag \\ 
%     & \sum_{i=1}^{M} n_{i} x_{ij} \leq \theta L_j, j \in \mathcal{N}, \notag 
% \end{align}

% Let $V_{\theta}^{OPT}(I)$ denote the expected value under optimal policy (relaxed) during $\theta T$ periods for instance $I$ (probability distribution).

% $V_{\theta}^{OPT}(I) = E_{\{d_{i}\}} [val_{\theta}(I; \{d_{i}\})] \leq val_{\theta} (I; \{E[d_{i}]\}) = val_{\theta}(I; \{\theta T p_{i}\})$

Let $V(\text{T-BPC})$, $V(\text{DLP})$ denote the optimal value of \eqref{bid-price_dual} and the LP relaxation of the \textup{SPDR} problem with expected demand, respectively. Then we have $V(\text{T-BPC}) = V(\text{DLP})$.

However, traditional bid-price control (BPC) has two drawbacks. First, when capacity permits, the policy treats all rows as equally preferable, making no distinction between them. Second, the decision to accept a request may not be feasible if certain rows lack sufficient capacity.

% Only consider single seat, not the whole group type.

To address these limitations, we propose an improved BPC policy. The key idea is to represent row structures using patterns and then leverage these patterns for decision-making.


\subsection{Improved BPC Policy}
Suppose that $S(L_{j})$ is the set of all patterns for row $j$. Let $v^t(\bm{L})$ indicate the maximal expected value to go at time $t$, given the capacity $\bm{L}$.

The dynamic programming can be expressed as follow:
\begin{equation}
    \begin{array}{lr}
    v^t(\bm{L})=\mathbb{E}_{i \sim p^t}\Bigg[\max\Big\{\max_{j :\bm{h} \in S(L_{j}), 
    h_{i} \geqslant 1}\left\{v^{t+1}(\bm{L}- e_{j}^{T} \cdot n_{i})+ {i}\right\}, v^{t+1}(\bm{L})\Big\}\Bigg], & \forall t \in[T], A \geqslant 0, \\
    v^{T+1}(S(\bm{L}))=0, & \forall A \geqslant 0.
    \end{array}
\end{equation}

The corresponding bid-price problem:

\begin{equation}\label{improve_bid}
    \begin{aligned}
    \min \quad & \sum_{i=1}^M \alpha_i d_i+ \sum_{j=1}^N \gamma_j \\
    \mathrm{s.t.} \quad & \alpha_i+\beta_{i j} \geq n_i-\delta, \quad \forall i, j, \\
    & \sum_{i=1}^M \beta_{i j} h_i \leq \gamma_j, \quad \forall j, \bm{h} \in S(L_j), \\
    & \alpha_i \geq 0, \quad \forall i, \\
    & \gamma_j \geq 0, \quad \forall j .
    \end{aligned}
\end{equation}

$\alpha_{i}$ represents marginal revenue for group $i$. $\beta_{ij}$ represents the cost for group $i$ assigned in row $j$. $\gamma_{j}$ represents the capacity cost associated with row $j$.

Let $V(\text{I-BPC})$ denote the expected optimal value of \eqref{improve_bid} under expected demand.

\begin{lem}
For any optimal bid-prices, $\beta_{j}$, in \eqref{bid-price_dual}, there exist optimal bid-prices, $\beta_{ij}$, in \eqref{improve_bid} such that we have the relation:
$\beta_{ij} \leq n_{i} \beta_{j}, \forall i$. Furthermore, $V(\text{T-BPC}) \geq V(\text{I-BPC})$.
\end{lem}

% $\beta_{ij} = n_{i} \beta_{j}$ is feasible for \eqref{improve_bid}, then the optimal bid-price 

This lemma shows that the improved bid-price control (I-BPC) does not uniformly reject all small groups. Instead, it selectively accepts them based on pattern-based allocation, enabling more flexible decision-making. Meanwhile, the lemma confirms that the I-BPC policy provides a tighter approximation to the optimal value than the T-BPC policy.

\begin{lem}
$V(\text{DLP}) \geq V(\text{HO})$ results from the concave property. 
\end{lem}

Let $val(\{d_{i}\})$ indicate the value of the relaxation of the SPDR problem.

$V(\text{HO}) = E_{\{d_{i}\}} [val(\{d_{i}\})] \leq val(\{E[d_{i}]\}) = V(\text{DLP})$.

Then, the following hierarchy holds: $V(\text{DLP}) = V(\text{T-BPC}) \geq V(\text{I-BPC}) \geq V(\text{HO}) \geq V(\text{OPT})$.

According to the definition, we have $V(\text{HO}) \geq V(\text{OPT})$.

Unlike traditional BPC, the improved policy accounts for row-specific characteristics in allocation decisions. Meanwhile, it guarantees feasible placement. Once a request is accepted, the policy ensures it can be assigned to a suitable row without additional feasibility checks.


\begin{algorithm}[H]
    \caption{Improved Bid-Price Control}\label{algo_improve_bid}
    \For{$t =1, \ldots, T$}{
      {Observe a request of group type $i$\;}
      {Solve problem \eqref{bid-price_dual} with $\bm{d}^{t} = (T-t) \cdot \bm{p}$ and $\mathbf{L}^{t}$\;}
      \eIf{$i - \beta_{ij} \geq 0$}
      {Set $k = \arg \max_{j \in \mathcal{N}}\{i - \beta_{ij}\} $ and break ties\;
      Assign the group to row $k$, let $L_{k}^{t+1} \gets L_{k}^{t} - n_{i}$\;}
      {Reject the group type\;}}
\end{algorithm}

The I-BPC policy better captures the structure of the capacity and features greater
flexibility than the T-BPC policy does. However, the bid-price policies are
established via a ``dual'' formulation, which might lose some information contained in the primal
problem.

Let $y_{j \bm{h}}$ denote the proportion of pattern $\bm{h}$ used in row $j$. The primal problem can be formulated as:

\begin{equation}\label{improve_primal}
    \begin{aligned}
    \max \quad & \sum_{i=1}^M \sum_{j=1}^N\left(n_i-\delta\right) x_{i j} \\
    \text {s.t.} \quad & \sum_{j=1}^N x_{i j} + x_{i0} = d_i, \quad i \in \mathcal{M}, \\
    & x_{i j}=\sum_{\bm{h} \in S(L_{j})} h_i y_{j \bm{h}}, \quad i \in \mathcal{M}, j \in \mathcal{N}, \\
    & \sum_{\bm{h} \in S(L_{j})} y_{j \bm{h}} \leq 1, \quad j \in \mathcal{N}.
    % & y_{j \bm{h}} \in \{0,1\}, \quad j \in \mathcal{N}, \bm{h} \in S(L_{j}).
    \end{aligned}
\end{equation}

Here, $x_{i0}$ is the number of unassigned groups of type $i$. The first set of constraints demonstrate that for each group type $i$, the sum of assigned groups and unassigned groups equals the total demand. The second set of constraints shows that the number of groups of type $i$ assigned to row $j$ equals the sum of $h_{i}$ (the count of type $i$ groups in pattern $\bm{h}$) weighted by the pattern proportions $y_{i \bm{h}}$. The total proportion of patterns uesd in each row $j$ cannot exceed 1.

\begin{algorithm}[H]
    \caption{Dynamic Primal}\label{algo_improve_primal}
    \For{$t =1, \ldots, T$}{
      {Observe a request of group type $i$\;}
      {Solve problem \eqref{improve_primal} with $\bm{d}^{t} = (T-t) \cdot \bm{p}$ \;
      Obtain an optimal solution $x_{ij}$ \;}
      {Set $k = \arg \max_{j}\{x_{ij}\}$ and break ties\;
      Assign the group to row $k$ ($k = 0$ means that the request is rejected), let $L_{k}^{t+1} \gets L_{k}^{t} - n_{i}$\;}}
\end{algorithm}


\subsection{Solve the dynamic primal}
% A pattern $\bm{h}$ is said to be inefficient if a mixture of other patterns can be used to generate more revenue for the same (or lower) consumption rate.

The pattern $\bm{h}$ is efficient for row $j$ if and only if, for some $(\alpha_{1}, \ldots, \alpha_{M}, \gamma_{j})$, $\bm{h}$ is the optimal solution to $$\max_{\bm{h}} \sum_{i=1}^{M} [n_i - \delta - \alpha_{i}] h_{i} - \gamma_{j}$$

To generate all efficient patterns, we need to solve the subproblem for each row $j$:

\begin{align}\label{subproblem}
    \max \quad & \sum_{i=1}^{M} [n_i - \delta - \alpha_{i}] h_{i} - \gamma_{j} \\
    \text {s.t.} \quad & \sum_{i= 1}^{M} n_{i} h_{i} \leq L_j, \notag \\
    & h_{i} \in \mathbb{N}, \quad i \in \mathcal{M}. \notag
\end{align} 

If the optimal value of \eqref{subproblem} is larger than $0$, the primal $\ref{improve_primal}$ reaches the optimal. Otherwise, a new pattern can be generated.

\begin{lem}
    If $y_{j \bm{h}}^{*} > 0$ is the optimal solution to \eqref{improve_primal}, then $\bm{h}$ is an efficient pattern.
\end{lem}


A pattern $\bm{h}$ is dominant if there is no distinct pattern $\bm{h}{'}$ where every component of $\bm{h}{'}$ is greater than or equal to the corresponding component of $\bm{h}$. The efficient pattern must be a dominating pattern. 

It is obvious that the dominating patterns include the full and largest patterns.

The dominating pattern can be regarded as the feasible patterns when the number of occupied seats is in $[L-\delta, L]$.

Let $DP(L, M)$ denote the number of dominating patterns. Let $DP_{t} (L, M)$ denote the number of all feasible patterns with length $L$.

$DP(L, M) = DP_{t} (L, M) - DP_{t} (L-\delta-1, M)$.

\begin{equation}
DP_{\text {t}}(L, M)=\sum_{k=0}^{\lfloor L /(M+\delta)\rfloor} D P_{\text {t}}(L-k(M+\delta), M-1)
\end{equation}

polynomial

\newpage

\subsection{Static BLC Policy}
Booking limit control policy:

\begin{align}
    \quad \max \quad & \sum_{i=1}^{M}  \sum_{j= 1}^{N} (n_i - \delta) x_{ij} \label{theta_deter} \\
    \text {s.t.} \quad & \sum_{j= 1}^{N} x_{ij} \leq d_{i}, \quad i \in \mathcal{M}, \notag \\ 
    & \sum_{i=1}^{M} n_{i} x_{ij} \leq L_j, j \in \mathcal{N}, \notag 
\end{align}

Let $d_{i}^{*} = \sum_{j} x_{ij}^{*}$, $x_{ij}^{*}$ is an integral optimal solution to \eqref{theta_deter} with $d_{i} = \sum_{t} p_{i}^{t}$ (Expected demand).

Let $d_{i}$ indicate the number of group type $i$ during time $T$. $d_{i} = \sum_{t} \bm{1}_{i_{t} = i}$.
Let $val(I; \{d_{i}\})$ denote the optimal objective value of \eqref{theta_deter}.

$V^{BL}(I) = E_{\{d_{i}\}}[\sum_{i} (n_{i}-\delta) \min\{d_{i}^{*}, d_{i}\}]$, $V^{OPT}(I) = E_{\{d_{i}\}} [val(I; \{d_{i}\})] \leq val(I; \{E[d_{i}]\})$.

$val(I; \{d_{i}\})$ is concave in $d_{i}$.

\begin{align*}
   & V^{OPT}(I) - V^{BL}(I) \\
\leq & val(I; \{E[d_{i}]\}) - V^{BL}(I) \\
= & val(I; \{E[d_{i}]\}) - val(I; \{\lfloor E[d_{i}]\rfloor\}) + val(I; \{\lfloor E[d_{i}]\rfloor\}) - E_{\{d_{i}\}}[\sum_{i} (n_{i}-\delta) \min\{d_{i}^{*}, d_{i}\}] \\
\leq & \sum_{i} (n_{i} - \delta) + N \sum_{i} i + E_{\{d_{i}\}}[\sum_{i} (n_{i}-\delta) (d_{i}^{*} - \min\{d_{i}^{*}, d_{i}\})] \\
= & \sum_{i} (n_{i} - \delta) + N \sum_{i} i + E_{\{d_{i}\}}[\sum_{i} \frac{1}{2}(n_{i}-\delta) (d_{i}^{*} - d_{i} + |d_{i}^{*} - d_{i}|)] \\
\overset{\text{(a)}}{\leq} & \sum_{i} (n_{i} - \delta) + N \sum_{i} i + \frac{1}{2} \sum_{i} (n_{i}-\delta)(d_{i}^{*} - E[d_{i}] + |d_{i}^{*} - E[d_{i}]| + \sqrt{\Var[d_{i}]}) \\
\leq & \sum_{i} (n_{i} - \delta) + N \sum_{i} i + \frac{1}{2} \sum_{i} (n_{i}-\delta) \sqrt{\Var[d_{i}]} \\
\leq & \sum_{i} (n_{i} - \delta) + N \sum_{i} i + \frac{1}{2} \sum_{i} (n_{i}-\delta) \sqrt{T p_{i} (1- p_{i})} = O(\sqrt{T})
\end{align*}

Thus, $\lim_{T \to \infty} (V^{OPT}(I) - V^{BL}(I))/T \to 0$.

$val(I; \{E[d_{i}]\}) - val(I; \{\lfloor E[d_{i}]\rfloor\}) \leq val(I; \{\lceil E[d_{i}]\rceil\}) - val(I; \{\lfloor E[d_{i}]\rfloor\}) = \sum_{i} (n_{i} - \delta)$


$LP -IP \leq \sum_{i} \sum_{j} (n_{i} - \delta) (x_{ij}^{*} - \lfloor x_{ij}^{*} \rfloor) \leq N \sum_{i} i$ $\Rightarrow$ $val(I; \{\lfloor E[d_{i}]\rfloor\}) \leq IP + N \sum_{i} i$.

$IP = \sum_{i} \sum_{j} (n_{i} - \delta) x_{ij}^{*} = \sum_{i} (n_{i} - \delta) d_{i}^{*}$

$(a)$ results from the following inequalities: $|d_{i}^{*} -d_{i}| = |(d_{i}^{*}-E[d_{i}]) + (E[d_{i}] -d_{i})| \leq |d_{i}^{*}-E[d_{i}]| + |d_{i} - E[d_{i}]|$. Take the expectation, we have $E[|d_{i}^{*} -d_{i}|]\leq |d_{i}^{*}-E[d_{i}]| + E[|d_{i} - E[d_{i}]|]$. $E[|d_{i} - E[d_{i}]|] \leq \sqrt{\Var[d_{i}]}$(Since $E[|X|] \leq \sqrt{E[X^{2}]}$). $d_{i}^{*} \leq E[d_{i}]$.

% 0-1 multiple

% \begin{align}
%     \quad \max \quad & \sum_{i=1}^{M}  \sum_{j= 1}^{N} p_i x_{ij} \\
%     \text {s.t.} \quad & \sum_{i= 1}^{M} w_{i} x_{ij} \leq L_{j}, \quad j \in \mathcal{N} \\ 
%     & \sum_{j=1}^{N} x_{ij} \leq 1, i \in \mathcal{M}  \\
%     & x_{ij} \in \{0,1\}, \quad i \in \mathcal{M}, j \in \mathcal{N}. 
% \end{align}

% Here, $M = \sum_{i=1}^{m} d_{i}$ represents the number of groups. $p_{k} = (n_{i} - \delta), w_{k} = n_{i}$ if group $k$ belongs to type $i$.

Surrogate relaxation (0-1 single):

\begin{align}\label{one_row}
    \quad \max \quad & \sum_{i = 1}^{M} (n_i - \delta) x_{i} \\
    \text {s.t.} \quad & x_{i} \leq d_{i}, \quad i \in \mathcal{M},  \\ 
    & \sum_{i=1}^{M} n_{i} x_{i} \leq L.
\end{align}

LP optimal solution: $[0, \ldots, 0, X_{\tilde{i}}, d_{\tilde{i}+1}, \ldots, d_{M}]$, $X_{\tilde{i}} = \frac{L - \sum_{i = \tilde{i}+1}^{M} {d_i n_i}}{n_{\tilde{i}}}.$

One feasible IP optimal solution: $[0, \ldots, 0, \lfloor X_{\tilde{i}} \rfloor, d_{\tilde{i}+1}, \ldots, d_{M}]$.

$LP - IP \leq \tilde{i} (X_{\tilde{i}} - \lfloor X_{\tilde{i}} \rfloor)$

% single-leg RM: bid-price and booking limit expected revenue loss of $O(\sqrt{k})$ even with re-solving.

\begin{align*}
    & V^{OPT}(I) - V^{BL}(I) \\
 \leq & val(I; \{E[d_{i}]\}) - V^{BL}(I) \\
 = & val(I; \{E[d_{i}]\}) - val(I; \{\lfloor E[d_{i}]\rfloor\}) + val(I; \{\lfloor E[d_{i}]\rfloor\}) - E_{\{d_{i}\}}[\sum_{i} (n_{i}-\delta) \min\{d_{i}^{*}, d_{i}\}] \\
 \leq & \sum_{i} (n_{i} - \delta) + \tilde{i} (X_{\tilde{i}} - \lfloor X_{\tilde{i}} \rfloor) + E_{\{d_{i}\}}[\sum_{i} (n_{i}-\delta) (d_{i}^{*} - \min\{d_{i}^{*}, d_{i}\})] \\
 \leq & \sum_{i} (n_{i} - \delta) + \tilde{i} (X_{\tilde{i}} - \lfloor X_{\tilde{i}} \rfloor) + \frac{1}{2} \sum_{i} (n_{i}-\delta) \sqrt{T p_{i} (1- p_{i})}
 \end{align*}
 

\begin{equation}
    V^{t}(l) =  \max_{u_{i}^{t} \in \{0,1\}} \left\{ \sum_{i=1}^{M} p_i \left[V^{t+1}(l-n_i u_{i}^{t})+ i u_{i}^{t}\right] + p_0 V^{t+1}(l)\right\}
\end{equation}

Always accept the largest group unless the capacity is insufficient.

% \begin{lem}\label{bid-price}
%  The optimal solution to problem is given by $z_1 = \ldots = z_{\tilde{i}} =0$, $z_{i} = \frac{\delta(n_i-n_{\tilde{i}})}{n_{\tilde{i}}}$ for $i = \tilde{i}+1, \ldots, M$ and $\beta_j = \frac{n_{\tilde{i}} - \delta}{n_{\tilde{i}}}$ for all $j$.
% \end{lem}

We consider the problem with one row (stochastic knapsack problem).

The DP (optimal online) policy: $V_{t}(l- n_{i}) - V_{t}(l) + i \geq 0$.

$E[\text{loss}] = V^{\text{off}} - V_{\pi}^{on} \geq V^{\text{opt}} - V_{\pi}^{on}$

One sample path. $d^{r}$ realization of $M$ types. 

$V_{t}(l) = \sum_{i = \hat{i}+1}^{M} r_{i} d_{i}^{r} + r_{\hat{i}}(l- \sum_{i= \hat{i}+1}^{M} d_{i}^{r})$

Static deterministic heuristic policy: accept $i \geq \hat{i}$ if $\bar{d}_{\hat{i}+1}+ \ldots + \bar{d}_{M} < l \leq \bar{d}_{\hat{i}}+ \ldots + \bar{d}_{M}$.


Let $V^{\text{OPT}}(I)$ denote the expected value under offline optimal policy (relaxed) during $T$ periods for instance $I$ (capacity, probability distribution).

The revenue loss between the static deterministic heuristic and the optimal is bounded by $C \sqrt{T}$.

Let $\gamma_{i}$, $\gamma_{i}^{0}$ denote the number of type $i$ accepted and rejected by some heuristic policy, respectively.


% Re-solving (each stage) bid-price (DLP) is equivalent to the optimal policy.

\begin{align*}
    \text{OPT}(L, \hat{d}, \gamma): \quad \max \quad & \sum_{i = 1}^{M} (n_i - \delta) x_{i} \\
    \text {s.t.} \quad & x_{i}^{0} + x_{i} = \hat{d}_{i}, \quad i \in \mathcal{M},  \\ 
    & x_{i} \geq \gamma_{i}, \quad i \in \mathcal{M}, \\
    & x_{i}^{0} \geq \gamma_{i}^{0}, \quad i \in \mathcal{M}, \\
    & \sum_{i=1}^{M} n_{i} x_{i} \leq L.
\end{align*}

Heuristic policy: At time t, solve problem \eqref{one_row} with $d_{i} = d_{i}^{t} = (T-t) * p_{i}$, $L = L^{t}$. When $x_{i}\geq 1$ for the request of type $i$, accept the request.



$d^{[1, T]}$ is the demand realization during $[1, T]$. $\gamma^{[1, t)}$ represents the number of requests rejected and accpeted by some heuristic policy during $[1, t)$. 

$OPT(L, d^{[1, T]}, \gamma^{[1,t+1)})$ can be interpreted as the total reward obtained under a virtual policy where we first follow the heuristic policy during $[1, t+1)$ and then from time $t+1$ we follow the optimal solution assuming that we know the future demands.

For one sample path of the requests, the revenue loss can be decomposed into $T$ increments.

\begin{align*}
    & OPT(L, d^{[1, T]}, 0) - OPT(L, d^{[1, T]}, \gamma^{[1, T]}) \\
 = & \sum_{t=1}^{T} [OPT(L, d^{[1,T]}, \gamma^{[1,t)}) - OPT(L, d^{[1,T]}, \gamma^{[1,t+1)})] \\
 \leq & \sum_{t=1}^{T} (n_{M} - \delta)
\end{align*}

Let $L^{t} = L-\sum_{i}n_{i} \gamma_{i}^{[1,t)}$.

The expected revenue loss can be upper bounded:

\begin{align*}
    & E[OPT(L, d^{[1, T]}, 0) - OPT(L, d^{[1, T]}, \gamma^{[1, T]})] \\
 \leq & (n_{M} - \delta) \sum_{t=1}^{T} P(OPT(L, d^{[1, T]}, \gamma^{[1,t)}) - OPT(L, d^{[1, T]}, \gamma^{[1,t+1)}) > 0) \\
 = & (n_{M} - \delta) \sum_{t=1}^{T} P(OPT(L^{t}, d^{[t, T]}, 0) - OPT(L^{t}, d^{[t, T]}, \gamma^{[t,t+1)}) > 0) \\
 \leq & (n_{M} - \delta) \sum_{t=1}^{T} P(x_{i^{t}}^{*,t} <1) \\
 = & (n_{M} - \delta) \sum_{t=T_{0}}^{T} P(x_{i^{t}}^{*,t} <1) \\
 \leq & (n_{M} - \delta) \max_{i}\{\frac{1}{p_{i}}\} 
\end{align*}


\begin{lem}
$OPT(L^{1}, \hat{d} + d^{[1, t_2)} , \gamma^{[1, t_2)}) = \sum_{i} (n_{i} - \delta) \gamma_{i}^{[1, t_1)} + OPT(L^{t}, \hat{d}+d^{[t_1, t_2)}, \gamma^{[t_1, t_2)})$
\end{lem}

For any optimal solution $x^{*}$ of $OPT(L^{t}, \hat{d}+d^{[t_1, t_2)}, \gamma^{[t_1, t_2)})$, $x^{*} + \gamma^{[1, t_1)}$ is a feasible solution of $OPT(L^{1}, \hat{d}+d^{[1, t_2)}, \gamma^{[1, t_2)})$. For any optimal solution $x^{*}$ of $OPT(L^{1}, \hat{d}+d^{[1, t_2)}, \gamma^{[1, t_2)})$, $x^{*}- \gamma^{[1, t_1)}$ is a feasible solution of $OPT(L^{t}, \hat{d}+d^{[t_1, t_2)}, \gamma^{[t_1, t_2)})$ because $x^{*}- \gamma^{[1, t_1)} \geq \gamma^{[1, t_{2})}- \gamma^{[1, t_1)} = \gamma^{[t_1, t_2)}$.


The first inequality results from $E[A] \leq r_{M} E[\bm{1}_{A>0}] = r_{M} P(A>0)$.

The first equation follows from Lemma. (Let $t_1 = t_2 = t$, $\hat{d} = d^{[t, T]}$; let $t_1 = t, t_2 = t+1$, $\hat{d} = d^{[t+1, T]}$).

The second equation is as follows. If $x_{i^{t}}^{*,t} \geq 1$, then $x^{*,t}$ is still feasible for $OPT(L^{t}, d^{[t, T]}, \gamma^{[t,t+1)})$. (Because the optimal policy)

$x_{i^{t}}^{*,t}$ is the optimal solution for $\text{OPT}(L^{t}, d^{[t, T]}, 0)$ at time $t$.

Let $T- T_{0} = \max_{i}\{\frac{1}{p_{i}}\}$

% The loss can be divided with capacity loss and decision loss.

For N rows,

\begin{align*}
    OPT(\bm{L}, \hat{d}, \gamma): \quad \max \quad & \sum_{i = 1}^{M} \sum_{j = 1}^{N} (n_i - \delta) x_{ij} \\
    \text {s.t.} \quad & \sum_{j=1}^{N} x_{ij} + x_{i0} = \hat{d}_{i}, \quad i \in \mathcal{M},  \\ 
    & \sum_{j=1}^{N} x_{ij} \geq \gamma_{i}, \quad i \in \mathcal{M}, \\
    & x_{i0} \geq \gamma_{i}^{0}, \quad i \in \mathcal{M}, \\
    & \sum_{i=1}^{M} n_{i} x_{ij} \leq L_{j}, \quad j \in \mathcal{N}.
\end{align*}

