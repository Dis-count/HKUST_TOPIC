% !TEX root = sum1.tex
\section{Literature Review}\label{literature}

Seating management is a practical problem that presents unique challenges in various applications, each with its own complexities, particularly when accommodating group-based seating requests. For instance, in passenger rail services, groups differ not only in size but also in their departure and arrival destinations, requiring them to be assigned consecutive seats \citep{clausen2010off, deplano2019offline}. In social gatherings such as weddings or dinner galas, individuals often prefer to sit together at the same table while maintaining distance from other groups they may dislike \citep{lewis2016creating}. In parliamentary seating assignments, members of the same party are typically grouped in clusters to facilitate intra-party communication as much as possible \citep{vangerven2022parliament}. In e-sports gaming centers, customers arrive to play games in groups and require seating arrangements that allow them to sit together \citep{kwag2022optimal}.

Incorporating social distancing into seating management has introduced an additional layer of complexity, sparking a new stream of research. Some works focus on the layout design and determine seating positions to maximize physical distance between individuals, such as students in classrooms \citep{bortolete2022support} or customers in restaurants and beach umbrella arrangements \citep{fischetti2023safe}. Other works assume the seating layout is fixed, and assign seats to individuals while adhering to social distancing guidelines. For example, \citet{salari2020social} and \citet{pavlik2021airplane} consider the seat assignment in the airplanes. The above studies consider the seating management with social distancing for individual requests.


Our work relates to seating management with social distancing for group-based requests, which has found its applications in various areas, including single-destination public transits \citep{moore2021seat}, airplanes \citep{ghorbani2020model, salari2022social}, trains \citep{haque2022optimization, haque2023social}, and theaters \citep{blom2022filling}. Due to the diversity of these applications, there are different issues to be addressed. For example, \citet{salari2022social} consider the distance between different groups and develop a seating assignment strategy that outperforms the simplistic airline policy of blocking all middle seats. In \citet{haque2023social}, the design of seat allocation for groups with social distancing takes into account the transmission risk within the train and between different stops. Our work is closely related to \citet{blom2022filling}, who address the group-based seating problem in theaters. While they primarily focuses on scenarios with known groups (referred to as seat planning with deterministic requests in our work), we investigate a broader range of demand forms. In addtion to the deterministic requests, we also study group-based seat planning with stochastic requests and explore dynamic seat assignment, assuming that groups arrive sequentially according to a stochastic process.


From a technical perspective, when all requests are known, the seating planning with deterministic requests (SPDR) problem can be formulated as a multiple knapsack problem \citep{martello1990knapsack}. 
While existing literature has primarily focused on deriving bounds or competitive ratios for general
multiple knapsack problems \citep{khuri1994zero, ferreira1996solving, pisinger1999exact, chekuri2005polynomial}, our work distinguishes itself by analyzing the specific structure and properties of solutions to the SPDR problem. This approach offers valuable insights for our investigation into situations involving dynamic demand.


While the dynamic stochastic knapsack problem (e.g., \citet{kleywegt1998dynamic, kleywegt2001dynamic}, \citet{papastavrou1996dynamic}) has been extensively studied in the literature, these works primarily consider a single knapsack scenario where requests arrive sequentially and their resource requirements and rewards are unknown until they arrive. In contrast, the seat assignment with dynamic requests (SADR) problem extends this framework by incorporating multiple knapsacks, adding another layer of complexity to the decision-making process.
Research on the dynamic or stochastic multiple knapsack problem is limited. \citet{perry2009approximate} employs multiple knapsacks to model multiple time periods for solving a multiperiod, single-resource capacity reservation problem. This essentially remains a dynamic knapsack problem but involves time-varying capacity. \citet{tonissen2017column} considers a two-stage stochastic multiple knapsack problem with a set of scenarios, wherein the capacity of the knapsacks may be subject to disturbances. This problem is similar to the SPSR problem in our work, where the number of items is stochastic.


Generally speaking, the SADR problem relates to the revenue management (RM) problem, which has been extensively studied in industries such as airlines, hotels, and car rentals, where perishable inventory must be allocated dynamically to maximize revenue \citep{van2005introduction}. Network revenue management (NRM) extends traditional RM by considering multiple resources (e.g., flight legs, hotel nights) and interdependent demand \citep{williamson1992airline}. The standard NRM problem is typically formulated as a dynamic programming (DP) model, where decisions involve accepting or rejecting requests based on their revenue contribution and remaining capacity \citep{talluri1998analysis}. However, a significant challenge arises because the number of states grows exponentially with the problem size, rendering direct solutions computationally infeasible. To address this, various control policies have been proposed, such as bid-price \citep{adelman2007dynamic, bertsimas2003revenue}, booking limits \citep{gallego1997multiproduct}, and dynamic programming decomposition \citep{talluri2006theory, liu2008choice}. These methods typically assume that demand arrives individually (e.g., one seat per booking). However, in our problem, customers often request multiple units simultaneously, requiring decisions that must be made on an all-or-none basis for each request. This requirement introduces significant complexity in managing group arrivals \citep{talluri2006theory}.


A notable study addressing group-like arrivals in revenue management examines hotel multi-day stays \citep{bitran1995application, goldman2002models, aydin2018decomposition}. While these works focus on customer classification and room-type allocation, they do not prioritize real-time assignment. The work of \cite{zhu2023assign}, which addresses the high-speed train ticket allocation, is related to our SADR problem in terms of real-time seat assignment. While \cite{zhu2023assign} processes individual seat requests and implicitly accommodates group-like traits through multi-leg journeys (e.g., passengers retaining the same seat across connected segments), our SADR context explicitly treats groups as booking entities involving simultaneous multi-seat reservations (e.g., a family booking four seats in a single transaction). 

% \cite{zhu2023assign} is a related work to our SADR problem with respect to the real-time seat assignment, which addresses the high-speed train ticket allocation. 



% The introduction of group-based characteristics further complicates seat management.
% A notable study that addresses group arrivals in revenue management is hotel revenue management \citep{bitran1995application, goldman2002models}, where customers request multi-day stays but room assignments can be deferred until check-in.


% in high-speed train ticket sales \citep{zhu2023assign}, where each booking must be assigned to a specific seat for the entire multi-leg journey in real time. Both scenarios involve assignment challenges, but with key differences: hotels allow flexible room allocation, while train seating requires immediate assignment. Our SADR problem aligns more closely with the latter - requiring real-time group assignment - but differs in that requests specify only group size, without fixed start dates (for hotels) or boarding stops (for trains).
 



% Second, regarding resource reusability, \cite{zhu2023assign} allows seats to be reused across different journey legs, whereas in SADR, seats are non-reusable once assigned and remain occupied for the entire duration.

% \cite{zhu2023assign} is the most closely related work to our SADR problem, which considers real-time seat assignment for the high-speed train ticket allocation. There are two critical differences between us. First, \cite{zhu2023assign} process individual seat requests and consider the connected segments where the group shows through multi-leg journeys (e.g., passengers occupying the same seat across connected segments); whereas in our SADR context, groups are explicit booking entities comprising multiple seats reserved simultaneously (e.g., a family reserving four seats as a single transaction). The second difference concerns resource reusability. In \cite{zhu2023assign}, seats are reusable because they can be allocated to different passengers across multiple journey legs. In contrast, SADR's seats are non-reusable since once assigned, they remain occupied for the entire duration. 

% as both involve  with group-like constraints.

% Zhu consider the connected segments where the group shows

% only processes individual seat requests, with 'groups' emerging abstractly through multi-leg journeys (e.g., passengers occupying the same seat across connected segments); 



% These distinctions necessitate novel optimization approaches for SADR, particularly in handling bulk assignments and irreversible resource allocation.

% First, in our context, a 'group' is explicitly defined by its size (e.g., a family booking four seats together), whereas in train ticket allocation, the 'group' is an abstract concept derived from multi-leg journeys (e.g., passengers occupying the same seat across connected segments). That is, the train ticket allocation processes individual seat requests, whereas SADR handles group requests (multiple seats in one booking).
 
