% !TEX root = sum1.tex

\section*{Abstract}
We consider the critical challenge of seat planning and assignment in the context of social distancing measures, which have become increasingly vital in today's environment. Ensuring that individuals maintain the required social distances while optimizing seating management requires careful consideration of several factors, including group sizes, venue layouts, and fluctuating demand patterns. We introduce critical concepts and analyze seat planning with deterministic requests. Subsequently, we introduce a scenario-based stochastic programming approach to address seat planning with stochastic requests. The seat planning can serve as the foundation for the seat assignment. Furthermore, we explore a dynamic situation where groups arrive sequentially. We propose a seat-plan-based assignment policy for either accommodating or rejecting incoming groups. This policy outperforms traditional bid-price and booking-limit strategies. Our study evaluates social distancing measures through the lens of revenue optimization, delivering actionable insights for both venue operators and policymakers. For venue managers, we develop an operational framework to optimize seat allocations under distancing constraints, and quantitative tools to assess occupancy rate tradeoffs (considering the layout configurations). For policymakers, we provide analytical methods to evaluate the economic impact of distancing mandates, and guidelines for setting policy parameters (including achievable occupancy rate, maximum group size, physical distance requirements).

% two aspects: venue manager: operational solution -> framework implement

% policy maker: provide tools to analyse the impact


% (considering venue-specific characteristics including maximum group size, physical distance requirements, and layout configurations).


% We investigate the impact of social distancing from the perspective of revenue loss. The findings furnish valuable insights for policymakers and venue managers regarding seat occupancy rates. We provide an operational solution for the venue manager, provide a practical framework for implementing social distancing protocols while optimizing seat allocations. For the policy maker, we provide tools to analyse the impact of implementing the social distancing measures.


Keywords: seating management, social distancing, scenario-based stochastic programming, dynamic seat assignment.

% We also explore a relaxed setting where seat assignments can be made after the booking period.


% This study tackles the challenge of seat planning and assignment with social distancing measures. Initially, we analyze seat planning with deterministic requests. Subsequently, we introduce a scenario-based stochastic programming approach to formulate seat planning with stochastic requests. We also investigate the dynamic situation where groups enter a venue and need to sit together while adhering to physical distancing criteria. The seat plan can serve as the basis for the assignment. Combined with relaxed dynamic programming, we propose a dynamic seat assignment policy for either accommodating or rejecting incoming groups. Our method outperforms traditional bid-price and booking-limit control policies. The findings furnish valuable insights for policymakers and venue managers regarding seat occupancy rates and provide a practical framework for implementing social distancing protocols while optimizing seat allocations.
