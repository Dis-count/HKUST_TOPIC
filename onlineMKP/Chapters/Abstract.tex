% !TEX root = sum1.tex

\section*{Abstract}
We examine the dynamic stochastic multi-type multiple knapsack problem. Its dynamic programming formulation is computationally prohibitive to solve directly. To address this, we first develop a canonical bid-price control policy, a threshold policy that accepts items above a certain type-value and rejects those below. However, since this policy provides a uniform bid-price across all knapsacks, it cannot distinguish between them and requires a feasibility check for each item. We then develop a pattern-based bid-price control, which assigns a bid-price to an entire item configuration rather than to individual size. This policy offers a superior approximation of the dynamic program's value function. Finally, we propose a dynamic primal policy to capture information potentially absent from the dual formulation. The numerical results show that the performances of the pattern-based bid-price control and dynamic primal policy are better than the traditional bid-price control.


% (considering venue-specific characteristics including maximum group size, physical distance requirements, and layout configurations).

% We investigate the impact of social distancing from the perspective of revenue loss. The findings furnish valuable insights for policymakers and venue managers regarding seat occupancy rates. We provide an operational solution for the venue manager, provide a practical framework for implementing social distancing protocols while optimizing seat allocations. For the policy maker, we provide tools to analyse the impact of implementing the social distancing measures.


Keywords: multi-type multiple knapsack problem, revenue management, bid-price control.

% We also explore a relaxed setting where seat assignments can be made after the booking period.


% This study tackles the challenge of seat planning and assignment with social distancing measures. Initially, we analyze seat planning with deterministic requests. Subsequently, we introduce a scenario-based stochastic programming approach to formulate seat planning with stochastic requests. We also investigate the dynamic situation where groups enter a venue and need to sit together while adhering to physical distancing criteria. The seat plan can serve as the basis for the assignment. Combined with relaxed dynamic programming, we propose a dynamic seat assignment policy for either accommodating or rejecting incoming groups. Our method outperforms traditional bid-price and booking-limit control policies. The findings furnish valuable insights for policymakers and venue managers regarding seat occupancy rates and provide a practical framework for implementing social distancing protocols while optimizing seat allocations.