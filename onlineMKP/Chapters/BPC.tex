\section{Online MMKP}\label{sec_dynamic_seat}
Consider a set $\mathcal{N} = \{1, 2, \ldots, N\}$ of knapsacks, where each knapsack $j$ has a capacity $c_j \in \mathbb{Z}^{+}$. There is also a set $\mathcal{M} = \{1, 2, \ldots, M\}$ of distinct item types. Each item of type $i$ has a size $w_i \in \mathbb{Z}^{+}$ and yields a profit $r_i \in \mathbb{Z}^{+}$ when placed entirely into a knapsack. The item types are ordered such that their profit-to-weight ratios, $r_i / w_i$, are monotonically increasing in $i$.

Requests for these items arrive sequentially. Upon the arrival of a request (which specifies its type), the seller must immediately decide whether to accept or reject it. If accepted, the seller must also assign it to a specific knapsack with sufficient remaining capacity. Each item must be placed whole into a single knapsack; partial assignments or reassignments are not permitted.


To model this problem, we adopt a dynamic programming framework based on discrete time periods $t = 1, 2, \ldots, T$. In each period, at most one request arrives. Let $\lambda_i^t$ denote the probability that a request for an item of type $i \in \mathcal{M}$ arrives at time $t$. These probabilities satisfy $\sum_{i=1}^M \lambda_i^t \leq 1$ for all $t$, and we define $\lambda_0^t = 1 - \sum_{i=1}^M \lambda_i^t$ as the probability of no arrival in period $t$. Arrival events are assumed to be independent across time periods.


The system state is the vector of remaining capacities $\mathbf{C} = (c_1, \ldots, c_N)$. When an item of type $i$ arrives, the seller chooses a decision variable $u_{i,j}^t \in \{0,1\}$ for each knapsack $j$. The feasible set $U^t(\mathbf{C})$ is defined by:

$$
U^t(\mathbf{C})=\left\{u_{i, j}^t \in\{0,1\} \left\lvert\, \begin{array}{ll}
\text { (a) } & \sum_{j=1}^N u_{i, j}^t \leq 1 \quad \forall i \in \mathcal{M} \\
\text { (b) } & w_i u_{i, j}^t \leq c_j \quad \forall i \in \mathcal{M}, \forall j \in \mathcal{N}
\end{array}\right.\right\} .
$$

Constraint (a) ensures the item is assigned to at most one knapsack. Constraint (b) ensures that if the item is assigned to knapsack $j$ ($u_{i,j}^t=1$), its weight $w_i$ does not exceed $c_j$. The original vector form of this constraint, $w_{i}u_{i,j}^{t}\mathbf{e}_j \leq \mathbf{C}$ (where $\mathbf{e}_j$ is the $j$-th standard basis vector), is mathematically equivalent to (b).



Let $v^{t}(\mathbf{C})$ denote the value function, representing the maximum expected revenue obtainable from period $t$ onward, given the current remaining capacity vector $\mathbf{C}$. The Bellman equation is given by:

\begin{equation}\label{DP}
v^{t}(\mathbf{C}) = \max_{u_{i,j}^{t} \in U^{t}(\mathbf{C})}\left\{\sum_{i=1}^{M} \lambda_i^{t} \bigl( \sum_{j=1}^{N} r_i u_{i,j}^{t} + v^{t+1}(\mathbf{C} - w_i u_{i,j}^{t}\mathbf{e}_j)\bigr) + \lambda_0^{t} v^{t+1}(\mathbf{C})\right\}
\end{equation}

The boundary condition is $v^{T+1}(\mathbf{C}) = 0$ for all $\mathbf{C} \geq \mathbf{0}$, indicating that no more revenue can be earned after the final period $T$.

Let $\mathbf{C}_0 = (C_1, C_2, \ldots, C_N)$ be the initial capacity vector. The objective is to compute $v^{1}(\mathbf{C}_0)$, the maximum total expected revenue over the entire horizon from $t=1$ to $t=T$, and to find the policy of item assignments that achieves this value.


Solving the dynamic programming problem presented in Equation \eqref{DP} is computationally intractable for realistic problem sizes due to the curse of dimensionality inherent in the large state space.

To overcome this challenge, we propose a heuristic assignment policy. We first outline a traditional bid-price control policy. We then enhance this approach by introducing a novel bid-price control policy that leverages patterns.



% For any policy $\pi$, let $V^{\pi}$ denote the expected revenue collected under $\pi$. Among all policies, a special one is the dynamic programming (DP) policy as it achieves the maximal expected revenue. Apart from DP, another special policy is the hindsight optimum (HO), where the decision maker has full information on the demand realization for the entire time horizon and optimizes over the allocation schemes. Here, it is impossible for the HO to be attained because we can never know the full demand realization ahead. For ease of analysis, the hindsight optimum for the sample path is computed by solving a relaxed static problem and $V^{\text{HO}}$ is the expected value over all sample paths. Then for any online policy $\pi$, we have $$V^{\text{HO}} \geq V^{\text{DP}} \geq V^{\pi}.$$ It is well known that DP, despite its optimality, is computationally complex owing to the ``curse of dimensionality.'' Thus we use HO as our benchmark to evaluate the performance of any policy $\pi$.


\subsection{BPC Policy}
Bid-price control is a classical and widely studied methodology in network revenue management. The core idea is to set thresholds, known as bid prices, that represent the opportunity cost of consuming one unit of capacity. An item is accepted only if its revenue exceeds the estimated opportunity cost of the capacity it requires.

Typically, these bid prices are derived from the shadow prices of the capacity constraints in a deterministic approximation of the underlying stochastic problem. In this section, we detail the implementation of a bid-price control policy for our model.

We begin by formulating a deterministic linear programming (LP) approximation, specifically the LP relaxation of a multi-type multiple knapsack problem. This model uses expected demand over the horizon. Let $x_{ij}$ denote the number of type $i$ items assigned to knapsack $j$, and let $d_i = \sum_{t=1}^{T} \lambda^{t}_{i}$ represent the expected number of requests for type $i$. The formulation is as follows:


\begin{align}
\quad \max \quad & \sum_{i=1}^{M}  \sum_{j= 1}^{N} r_{i} x_{ij} \label{e0} \\
\text {s.t.} \quad & \sum_{j= 1}^{N} x_{ij} \leq d_{i}, \quad i \in \mathcal{M}, \label{deter_upper}\\ 
& \sum_{i=1}^{M} w_{i} x_{ij} \leq c_j, j \in \mathcal{N}, \label{capa_con} \\
& x_{ij} \geq 0, \quad i \in \mathcal{M}, j \in \mathcal{N}. \notag
\end{align}

The objective \eqref{e0} is to maximize total expected revenue. Constraint \eqref{deter_upper} ensures the total number of accepted type $i$ items does not exceed its expected demand. Constraint \eqref{capa_con} ensures the total weight in each knapsack $j$ does not exceed its initial capacity $C_j$.

The monotonic increase of the profit-to-weight ratio $r_i/w_i$ with type index $i$ implies that items with a higher index are more profitable per unit of capacity. Consequently, the optimal solution to the LP relaxation exhibits a greedy structure, preferentially utilizing higher-indexed item types. This structural property is formalized in Proposition \ref{sol_relax_deter}.

\begin{lem}\label{sol_relax_deter}
For the LP relaxation of the \textup{MMKP} problem, there exists an index $\tilde{i}$ such that the optimal solutions satisfy the following conditions: $x_{ij}^{*} = 0$ for all $j$, $i = 1,\ldots, \tilde{i}-1$; $\sum_{j=1}^{N} x_{ij}^{*} = d_{i}$ for $i = \tilde{i}+1,\ldots, M$; $\sum_{j=1}^{N} x_{ij}^{*} = \frac{L - \sum_{i = \tilde{i}+1}^{M} {d_i w_i}}{w_{\tilde{i}}}$ for $i = \tilde{i}$.
\end{lem}

The dual of LP relaxation of the MMKP problem is:

\begin{equation}\label{bid-price_dual}
    \begin{aligned}
    \min \quad & \sum_{i=1}^{M} d_i z_i + \sum_{j= 1}^{N} c_j \beta_{j} \\
    \text {s.t.} \quad & z_{i} + \beta_j w_i \geq r_i, \quad i \in \mathcal{M}, j \in \mathcal{N} \\
    & z_{i} \geq 0, i \in \mathcal{M}, \beta_{j} \geq 0, j \in \mathcal{N}.
    \end{aligned}
  \end{equation}

In \eqref{bid-price_dual}, $\beta_{j}$ can be interpreted as the bid-price for one size in knapsack $j$. A request is only accepted if the revenue it generates is no less than the sum of the bid prices of the sizes it uses. Thus, if $r_i -\beta_{j} w_i \geq 0$, meanwhile, the capacity allows, we will accept the item type $i$. And choose knapsack $j^{*} = \arg \max_{j} \{r_i -\beta_{j} w_i\}$ to allocate that item.

\begin{prop}\label{bid-price}
The optimal solution to problem \eqref{bid-price_dual} is given by $z_1 = \ldots = z_{\tilde{i}} =0$, $z_{i} = \frac{r_{i} w_{\tilde{i}} - r_{\tilde{i}} w_{i}}{w_{\tilde{i}}}$ for $i = \tilde{i}+1, \ldots, M$ and $\beta_j = \frac{r_{\tilde{i}}}{w_{\tilde{i}}}$ for all $j$.
\end{prop}

According to Proposition \ref{bid-price}, the decision inequality becomes $r_i -\beta_{j} w_i = r_{i} - \frac{r_{\tilde{i}}}{w_{\tilde{i}}} w_{i} \geq 0$. This establishes the threshold policy: reject item type $i, i < \tilde{i}$ and accept item type $i, i \geq \tilde{i}$.

\begin{algorithm}[H]
    \caption{Bid-Price Control}\label{algo_bid}
    \For{$t = 1, \ldots, T$}{
      {Observe a request of item type $i$\;}
      {Solve problem \eqref{bid-price_dual} with $\bm{d}^{[t,T]}$ and $\mathbf{C}^{t}$\;
      Obtain $\tilde{i}$ such that the aggregate optimal solution is $x e_{\tilde{i}} + \sum_{i=\tilde{i}+1} ^{M} d_{i}^{t} e_{i}$\;}
      \eIf{$i \geq \tilde{i}$ and $\max_{j \in \mathcal{N}}{c_j^{t}} \geq w_i$}
      {Set $k = \arg \min_{j \in \mathcal{N}}\{c_j^{t}|c_j^{t} \geq w_i\} $ and break ties\;
      Assign the item to knapsack $k$, let $c_{k}^{t+1} \gets c_{k}^{t} - w_{i}$ \;}
      {Reject the request\;}}
\end{algorithm}

% Let $val_{\theta}(I; \{d_{i}\})$ denote the optimal objective value of \eqref{theta_deter}.

% \begin{align}
%     \quad \max \quad & \sum_{i=1}^{M}  \sum_{j= 1}^{N} (n_i - \delta) x_{ij} \label{theta_deter} \\
%     \text {s.t.} \quad & \sum_{j= 1}^{N} x_{ij} \leq d_{i}, \quad i \in \mathcal{M}, \notag \\ 
%     & \sum_{i=1}^{M} n_{i} x_{ij} \leq \theta L_j, j \in \mathcal{N}, \notag 
% \end{align}

% Let $V_{\theta}^{OPT}(I)$ denote the expected value under optimal policy (relaxed) during $\theta T$ periods for instance $I$ (probability distribution).

% $V_{\theta}^{OPT}(I) = E_{\{d_{i}\}} [val_{\theta}(I; \{d_{i}\})] \leq val_{\theta} (I; \{E[d_{i}]\}) = val_{\theta}(I; \{\theta T p_{i}\})$

Let $z(\text{BPC})$, $z(\text{DLP})$ denote the optimal value of \eqref{bid-price_dual} and the LP relaxation of the \textup{MMKP} problem with expected demand, respectively. Then we have $z(\text{BPC}) = z(\text{DLP})$.

However, the BPC policy has two drawbacks. First, the capacity feasibility need to be checked when to accept a request. Second, when capacities permit, the policy treats all knapsacks as equally preferable, making no distinction among them.

\begin{example}
Consider an instance with $M =3$, $N =4$, $w_{1} = 3, w_{2} = 4, w_{3} = 5$, $r_{1} = 4, r_{2} = 6, r_{3} = 8$, $\bm{C} = [7, 8, 8, 4]$ and $T = 8$. The stationary arrival probability is $\lambda_{1} = \lambda_{3} = \frac{1}{4}$, $\lambda_{2} = \frac{1}{2}$, giving an expected demand vector $\bm{d} = (2, 4, 2)$.

Under traditional bid-price control, the bid-price for each knapsack $j$ is $\beta_{j}^{*} = \frac{4}{3}$, yielding a total expected reward $z(\text{BPC}) = \frac{124}{3}$. This solution exhibits two major drawbacks: 

It fails to differentiate between knapsacks, assigning them a uniform bid-price. It permits potentially infeasible assignments. For instance, even though $r_{3} - \beta_{4}^{*} * w_{3} > 0$, assigning a type 3 item (with weight 5) to knapsack 4 (with capacity 4) is infeasible.
\end{example}

\subsection{BPC Policy Based on Patterns (BPP)}
To better account for the differences between knapsacks when placing items, we propose an enhanced dynamic programming (DP) formulation. The key idea is to represent the state of each knapsack using feasible assignment patterns, which encode the combinations of items it can hold, rather than merely tracking its residual capacity. This approach more fully captures the knapsack's combinatorial structure.

Formally, a pattern for a knapsack is a vector $\bm{h} = [h_{1}, \ldots, h_{M}]$ where each $h_i$ denotes the number of type-$i$ items assigned. A pattern $\bm{h}$ is feasible for a knapsack with capacity $c_{j}$ if it satisfies $\sum_{i=1}^{M} w_{i} h_{i} \leq c_{j}$. We denote the set of all feasible patterns for knapsack $j$ by $S(c_{j})$. 

Let $v^t(\bm{C})$ denote the maximal expected value-to-go at time $t$, given the remaining capacity $\bm{C}$. The enhanced dynamic programming formulation is as follows:

\begin{equation}
v^t(\bm{C}) \geq \mathbb{E}_{i \sim \lambda^t}\left[\left\{
\begin{array}{ll}
\max \left\{\max\limits_{j:\bm{h} \in S(c_j), h_i \geq 1}\left\{v^{t+1}\left(\bm{C}-e_j^T \cdot w_i\right)+r_i\right\}, v^{t+1}(\bm{C})\right\},&\exists j, \text{satisfying} \bm{h} \in S(c_j), h_i \geq 1, \\
v^{t+1}(\bm{C}) & \text{otherwise}.
\end{array}\right]\right.
\end{equation}


The DP formulation involves two layers of maximization when there exists at least one knapsack $j$ satisfying $\bm{h} \in S(c_j), h_{i} \geq 1$. The inner maximization evaluates the optimal placement of item type $i$ across all feasible knapsacks $j$ where the pattern $\bm{h}$ is feasible for knapsack $j$ (i.e., $\bm{h} \in S(c_{j})$) and the item type $i$ can be accommodated (i.e., $h_{i} \geq 1$). The outer maximization compares the value of accepting $i$ (via the inner maximization) and rejecting $i$ (retaining $v^{t+1}(\bm{C})$). If no such $j$ exists, the request $i$ is rejected.


For notational convenience, we define $q_i$ as follows:
\begin{equation*}
    q_i= \begin{cases}\max _{j: \bm{h} \in S\left(c_j\right), h_i \geq 1}\left\{v^{t+1}\left(C-e_j^T w_i\right)+r_i\right\} & \text { if } \exists j \text{ satisfying } \bm{h} \in S\left(c_j\right), h_i \geq 1 \\ 0 & \text { otherwise }\end{cases}
\end{equation*}

We can solve the following program to compute $v^1(\bm{C})$ for any given capacity $\bm{C}$:

\begin{equation}\label{dp_bid}
    \begin{aligned}
    \min \quad & v^{1}(\bm{C}) \\
    \mathrm{s.t.} \quad & v^{t}(\bm{C}) \geq \mathbb{E}_{i \sim \lambda^t}\Bigg[\max\Big\{ q_{i}, v^{t+1}(\bm{C})\Big\}\Bigg], \\
    & v^{T+1}(\bm{C}) \geq 0.
    \end{aligned}
\end{equation}


Solving \eqref{dp_bid} remains computationally intractable. We therefore adopt an Approximate Dynamic Programming (ADP) approach \citep{adelman2007dynamic} to approximate the value function $v^{t}(\bm{C})$. Our proposed approximation is:

\begin{equation}\label{appro_dp}
    \hat{v}^{t}(\bm{C}) = \theta^{t} + \sum_{j=1}^{N} \max_{\bm{h} \in S(c_{j})} \{\sum_{i=1}^{M} \beta_{ij}^{\dag} h_{i}\}.
\end{equation}


The term $\beta_{ij}^{\dag}$ can be regarded as the approximated value for each type $i$ in knapsack $j$. Unlike traditional linear approximations, our approach retains the linear term $\theta^{t}$ but introduces a nonlinear component for each knapsack $j$. Specifically, we maximize the linear combination $\sum_{i=1}^{M} \beta_{ij}^{\dag} h_{i}$ over the feasible set $S(c_{j})$.

% Our approximation extends classical linear ADP by incorporating resource-specific nonlinear terms through constrained maximization over feasible allocations. While similar separable corrections appear in resource allocation ADP, our explicit use of $\max_{\bm{h} \in S(c_j)}$ captures local constraints more directly.


% Our approximation adopts a separable nonlinear structure, but unlike classical approximations, the nonlinearity is implicitly defined by constrained maximization over feasible allocations. This captures problem-specific constraints.

Substituting \eqref{appro_dp} into \eqref{dp_bid}, we have:

\begin{align}
    \theta^{t} - \theta^{t+1} = \hat{v}^{t}(\bm{C}) - \hat{v}^{t+1}(\bm{C}) \geq \sum_{i} \lambda_{i}^{t} \max\left\{q_{i} - v^{t+1}(\bm{C}), 0\right\}
\end{align}

For the case where there exists a knapsack $j$ satisfying both conditions: $\bm{h}^{*} \in \arg\max_{\bm{h} \in S(c_j)} \sum_{i} \beta_{ij}^{\dag} h_{i}$ and $h_{i}^{*} \geq 1$, we establish the value difference: 

\begin{align*}
    & v^{t+1}(\bm{C} - e_{j}^{T} w_{i}) - v^{t+1}(\bm{C}) \\ 
  = & \max_{\bm{h} \in S(c_{j}- w_{i})} \{\sum_{i} \beta_{ij}^{\dag} h_{i}\} - \max_{\bm{h} \in S(c_{j})} \{\sum_{i} \beta_{ij}^{\dag} h_{i}\} \\
  = & -\beta_{ij}^{\dag} \leq 0
\end{align*}

The acceptance threshold is then defined as:
$$\alpha_{i} = \max\left\{\max_{j}\left\{r_i - \beta_{ij}^{\dag} \right\}, 0\right\},$$ 
with $\alpha_{i} = 0$ when no qualifying knapsack $j$ exists.

Let $\gamma_{j} = \max_{\bm{h} \in S(c_{j})} \{\sum_{i} \beta_{ij}^{\dag} h_{i}\}$. This yields:

\begin{align*}
    \theta^{1} & = \sum_{t=1}^{T} (\theta^{t} - \theta^{t+1}) \geq \sum_{t} \sum_{i} \alpha_{i} \lambda_{i}^{t} = \sum_{i} d_{i} \alpha_{i} \\
    \hat{v}^{1}(\bm{C}) & = \sum_{i} d_{i} \alpha_{i} + \sum_{j} \gamma_{j}
\end{align*}

Since $\hat{v}^{1}(\bm{C})$ constitutes a feasible solution to \eqref{dp_bid}, we have $\hat{v}^{1}(\bm{C}) \geq v^{1}(\bm{C}) =  V^{DP}$.

If all $j$ exist for $\bm{h}^{*} \in \arg\max_{\bm{h} \in S(c_j)} \sum_{i} \beta_{ij}^{\dag} h_{i}, h_{i}^{*} \geq 1$, the corresponding bid-price problem can be expressed as:

\begin{equation}\label{improve_bid}
    \begin{aligned}
    \min \quad & \sum_{i=1}^M \alpha_i d_i+ \sum_{j=1}^N \gamma_j \\
    \mathrm{s.t.} \quad & \alpha_i+ \beta_{ij}^{\dag} \geq r_i, \quad \forall i, j, \\
    & \sum_{i=1}^M \beta_{ij}^{\dag} h_i \leq \gamma_j, \quad \forall j, \bm{h} \in S(c_j), \\
    & \alpha_i \geq 0, \forall i, \quad \beta_{ij}^{\dag} \geq 0, \forall i, j \\
    & \gamma_j \geq 0, \quad \forall j.
    \end{aligned}
\end{equation}

$\alpha_{i}$ represents marginal revenue for type $i$. $\beta_{ij}^{\dag}$ represents the cost for type $i$ assigned in knapsack $j$. $\gamma_{j}$ represents the capacity cost associated with knapsack $j$.

If for a given item type $i$, no knapsack $j$ contains an optimal allocation $\bm{h}^{*} \in \arg\max_{\bm{h} \in S(c_j)} \sum_{i} \beta_{ij}^{\dag} h_{i}$ with $h_{i}^{*} \geq 1$, then the variable $\alpha_{i}$ must be zero. This indicates that the constraint associated with item type $i$ and knapsack $j$ is redundant. Although these constraints appear difficult to enforce in advance, the following lemma reveals that in practice we can avoid imposing additional restrictions. Simply solving problem \eqref{improve_bid} will inherently satisfy all required conditions.

% \begin{lem}
% For the optimal solution, $r_{i} \geq \beta_{ij}^{\dag}$.
% \end{lem}

\begin{lem}\label{BPP}
Define $\mathcal{J}_{i} = \{j \in \mathcal{N} \mid r_i - \beta_{ij}^{\dag *} \geq r_i - \beta_{ik}^{\dag *},\forall k \in \mathcal{N}, r_i - \beta_{ij}^{\dag *} >0 \}$.
If $\mathcal{J}_{i} = \varnothing$, the first set of constraints for $i$ is redundant.
If $\mathcal{J}_{i} \neq \varnothing$, then there exists a $j{'} \in \mathcal{J}_{i}$ such that:

$$\bm{h}^{*} \in \arg\max_{\bm{h} \in S(c_{j{'}})} \sum_{i} \beta_{ij{'}}^{\dag *} h_{i}, h_{i}^{*} \geq 1.$$
\end{lem}

This lemma guarantees that when such a $j{'}$ exists, the first set of constraints for $i$ becomes active; otherwise, it remains inactive. Consequently, we have $z(\text{BPP}) = \hat{v}^{1}(\bm{C})$.

The control policy is then defined as follows. For the arriving type-$i$ item, if $\alpha_{i} > 0$, accept the type $i$. Assign the type $i$ to a knapsack $j \in \mathcal{J}_{i}$ satisfying: $$\bm{h}^{*} \in \arg\max_{\bm{h} \in S(c_j)} \sum_{i} \beta_{ij}^{\dag *} h_{i}, h_{i}^{*} \geq 1.$$ If $\alpha_{i} = 0$ and there exists a knapsack $j$ such that: $\bm{h}^{*} \in \arg\max_{\bm{h} \in S(c_j)} \sum_{i} \beta_{ij}^{\dag *} h_{i}, h_{i}^{*} \geq 1$, accept the type $i$ and assign it to knapsack $j$; otherwise, reject the type $i$.

Let $z(\text{BPC})$ and $z(\text{BPP})$ denote the expected optimal value of \eqref{bid-price_dual} and \eqref{improve_bid}, respectively.

\begin{prop}\label{BPC_relation}
    We have $z(\text{BPC}) \geq z(\text{BPP})$.
\end{prop}
    
% For the optimal $\beta_{j}^{*}$ in \eqref{bid-price_dual}, there exist optimal $\beta_{ij}^{\dag *}$ in \eqref{improve_bid} satisfying $\beta_{ij}^{\dag *} \geq w_{i} \beta_{j}^{*}$ for all $i$. 

% $\beta_{ij}^{\dag} = w_{i} \beta_{j}$ is feasible for \eqref{improve_bid}, then the optimal bid-price 
    
Both bid-price approaches give upper bounds on the value function at any state, meanwhile it follows that BPP provides a tighter approximation to the value function more accurately than BPC does.

Under the approximation \eqref{appro_dp}, the BPP policy operates as follows:

% accept type $i$ if $r_i - \beta_{ij}^{\dag} > 0$ and at least one knapsack can assign type $i$, check  item type $i$ otherwise. 

% $\max_{j \in \mathcal{N}}{c_j^{t}} \geq w_i$

\begin{algorithm}[H]
    \caption{Bid-Price Control Based on Patterns}\label{algo_improve_bid}
    \For{$t =1, \ldots, T$}{
      {Observe a request of item type $i$\;}
      {Solve problem \eqref{improve_bid} with $\bm{d}^{[t, T]}$ and $\mathbf{L}^{t}$\;}
      \eIf{$\alpha_i > 0$}
      {Assign the type $i$ to a knapsack $j \in \mathcal{J}_{i}$ satisfying: $$\bm{h}^{*} \in \arg\max_{\bm{h} \in S(c_j)} \sum_{i} \beta_{ij}^{\dag *} h_{i}, h_{i}^{*} \geq 1.$$
      Let $c_{j}^{t+1} \gets c_{j}^{t} - w_{i}$\;}
      {\eIf{There exists $j$ such that $\bm{h}^{*} \in \arg\max_{\bm{h} \in S(c_j)} \sum_{i} \beta_{ij}^{\dag} h_{i}, h_{i}^{*} \geq 1$}{Assign the item in knapsack $j$, let $c_{j}^{t+1} \gets c_{j}^{t} - w_{i}$\;}
      {Reject the request\;}}}
\end{algorithm}

% Furthermore, the BPP policy exhibits a threshold structure similar to the BPC policy. There exists a threshold type $\tilde{i}$ (or $\tilde{i}-1$, depending on the remaining capacity and expected future demand) such that all item types $i \leq \tilde{i}$ (or $i \leq \tilde{i}-1$) are rejected and larger types $i > \tilde{i}$ (or $i > \tilde{i}-1$) are accepted.

% Specifically, we have 
% \begin{lem}\label{BPP_decision}    
%     For the threshold index $\tilde{i}$ in Proposition \ref{sol_relax_deter},
%     \begin{itemize}
%         \item If $\sum_{i=\tilde{i}}^{M} d_{i} w_{i} > \sum_{j=1}^{N} c_{j}$, then $r_i - \beta_{ij}^{\dag} \leq 0$, for $i \leq \tilde{i}$,
%         \item If $\sum_{i=\tilde{i}}^{M} d_{i} w_{i} = \sum_{j=1}^{N} c_{j}$, then $r_i - \beta_{ij}^{\dag} \leq 0$, for $i \leq \tilde{i}-1$.
%     \end{itemize}
% \end{lem}

% Lemma \ref{BPP_decision} indicates how the relation between the remaining capacity and the expected demand affects the decision.

% when $\sum_{i=\tilde{i}}^{M} d_{i} w_{i} > \sum_{j=1}^{N} c_{j}$, item type $i, i \leq \tilde{i}$ is rejected and item type $i, i > \tilde{i}$ is accepted; when $\sum_{i=\tilde{i}}^{M} d_{i} w_{i} = \sum_{j=1}^{N} c_{j}$, item type $i, i \leq \tilde{i}-1$ is rejected and item type $i, i \geq \tilde{i}$ is accepted.

% Proof:
% $\sum_{j} x_{ij}^{*} < d_{i}$, then $\alpha_{i}^{*} = 0$. $i - \beta_{ij}^{\dag} \leq 0$.

% Lemma \ref{BPC_relation} shows that the I-BPC policy does not uniformly reject all small groups. Instead, it selectively accepts them based on pattern-based allocation, enabling more flexible decision-making. Unlike the traditional BPC, the I-BPC policy accounts for row-specific characteristics in allocation decisions. Therefore, the I-BPC policy better captures the structure of the capacity than the BPC policy does.


Unlike the traditional BPC, the BPP does not require explicit feasibility checks on capacity. However, it still needs to verify whether the optimal pattern contains the given request. This introduces a new challenge, as bid-price policies are inherently derived from a dual formulation, which may inherently omit key information preserved in the primal problem.

\begin{example}
Continue with the above example:

For the BPP, the optimal bid price is $\bm{\beta}_{1}^{*} = [4, 4, 4, 6]$, $\bm{\beta}_{2}^{*} = [6, 6, 6, 6]$, $\bm{\beta}_{3}^{*} = [10, 8, 8, 8]$. We can verify that  $z(BPP) = 40 < z(BPC) = \frac{124}{3}$. The reduced reward for each item type 3 is: $r_{3} - \bm{\beta}_{3}^{*} = [-2, 0, 0, 0]$, which indicates type 3 is excluded from knapsack 1. Also, the optimal patterns for knapsack 4, $[0, 1, 0]$ and $[1, 0, 0]$. Thus, type 3 can only be assigned to knapsack 2 or 3.

% The corresponding primal variables are $\bm{\alpha} = [0, 0, 0]$ and $\bm{\gamma} = [10, 12, 12, 6]$.

% A drawback is that one must check if an item type can be assigned to a knapsack $j_0$ by verifying the condition $\bm{h}^{*} \in \arg\max_{\bm{h} \in S(c_{j_0})} \sum_{i} \beta_{ij_0}^{\dag *} h_{i}$ with $h_{i}^{*} \geq 1$.


\end{example}

While this verification is cumbersome under pure bid-price frameworks, the primal problem offers a more direct way to guarantee the existence of such a request-pattern match. To address this gap, we propose the dynamic primal formulation.