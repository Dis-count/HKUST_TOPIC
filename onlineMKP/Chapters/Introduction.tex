% !TEX root = sum1.tex


%{\bf Terminologies to use}

%We use {\em seating management} to refer to the general problem which includes {\em seat planning with deterministic requests}, 
%{\em seat planning with stochastic requests}, and {\em seat assignment}.

%Each problem is defined for an {\em event} which has multiple {\em seating requests}, where each request has a {\em group} of people to be seated.

\section{Introduction}
We will address the online multi-type multiple knapsack problem. 

% Consider a venue, such as a cinema or a conference hall, which is used to host an event. The venue is equipped with seats of multiple rows. During the event, requests for seats arrive in groups, each containing a limited number of individuals. Each group can be either accepted or rejected, and those that are accepted will be seated consecutively in one row. Each row can accommodate multiple groups as long as any two adjacent groups in the same row are separated by one or multiple empty seats to comply with social distancing requirements. The objective is to maximize the number of individuals accepted for seating. 


We develop several policies for online MMKP. 

% Specifically, we formulate the SPDR problem using Integer Programming and discuss the key characteristics of the optimal seating plan. For the SPSR problem, we utilize scenario-based optimization and develop solution approaches based on Benders decomposition. In addressing the SADR problem, we implement a two-stage seat-plan-based assignment approach. In the first decision phase, a relaxed dynamic programming evaluates each incoming request to determine its acceptance. The accepted requests then proceed to the second assignment phase, where the group-type control allocation is performed. This seat-plan-based assignment policy outperforms traditional bid-price and booking-limit policies. Although each of these models represents a standalone problem tailored to specific situations, they are closely interconnected in terms of problem-solving methods and managerial insights. In the seat planning with deterministic requests (SPDR) problem, we identify important concepts such as the full pattern and the largest pattern, which play a crucial role in developing solutions for the other two problems. Additionally, the SPDR problem serves as a useful offline benchmark for evaluating the performance of policies in the SADR problem. Furthermore, the solution to the SPSR problem can serve as a reference seat plan for dynamic seat assignment in the SADR problem.


% We investigate the impact of social distancing from the perspective of revenue loss. To facilitate this analysis, we introduce the concept of the threshold of request-volume, which represents the upper limit on the number of requests an event can accommodate without being affected by social distancing measures. Specifically, if an event receives fewer requests than the threshold of request-volume, it will experience virtually no revenue loss due to social distancing. Our computational experiments demonstrate that the threshold of request-volume primarily depends on the mean group size and is relatively insensitive to the specific distribution of group sizes. This finding provides a straightforward method for estimating the threshold of request-volume and evaluating the impact of social distancing.

% In some instances, the government imposes a maximum allowable occupancy rate to enforce stricter social distancing requirements. To assess this effect, we introduce the concept of the threshold of occupancy rate, defined as the occupancy rate at the threshold of request-volume. The maximum allowable occupancy rate is effective for an event only if it is lower than the event's threshold of occupancy rate. Moreover, it becomes redundant if it exceeds the maximum achievable occupancy rate for all events.


% These qualitative insights are stable with respect to the government policy's strictness and the specific characteristics of various venues, such as minimum physical distance, allowable maximum group size, and venue layout. When the minimum physical distance increases, the threshold of request-volume, threshold of occupation rate and maximum achievable occupation rate decrease accordingly. Conversely, when the allowable maximum group size decreases, the number of accepted requests will increase; however, both the threshold of occupation rate and maximum achievable occupation rate decline. Although venue layouts may vary in shapes (rectangular or otherwise) and row lengths (long or short), the threshold of occupancy rate and maximum achievable occupancy rate do not exhibit significant variation.


The rest of this paper is structured as follows. We review the relevant literature in Section \ref{literature}. Then we introduce the key concepts of seat planning with social distancing and formulate the . Section \ref{sec_dynamic_seat} presents the bid-price and resolving dynamic primal policies to assign seats for incoming requests. Section \ref{sec_result} presents the experimental results and provide insights gained from implementing social distancing. Conclusions are shown in Section \ref{sec_conclusion}.

