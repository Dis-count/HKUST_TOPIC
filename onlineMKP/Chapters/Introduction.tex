% !TEX root = sum1.tex
%{\bf Terminologies to use}

%We use {\em seating management} to refer to the general problem which includes {\em seat planning with deterministic requests}, 
%{\em seat planning with stochastic requests}, and {\em seat assignment}.

%Each problem is defined for an {\em event} which has multiple {\em seating requests}, where each request has a {\em group} of people to be seated.

\section{Introduction}
% We will address the dynamic stochastic multi-type multiple knapsack problem. 
The dynamic stochastic multiple knapsack problem with multi-type items extends the classical knapsack problem to a dynamic and multi-dimensional setting, where items of distinct types arrive sequentially, and decisions to accept or reject them must be made immediately without knowledge of future arrivals. Each item is characterized by a type-dependent size and value, and must be placed into one of multiple knapsacks with non-identical capacities. This framework captures critical resource allocation challenges across various domains, including cloud computing, advertising systems, production planning, and energy management.

In online advertising, the problem provides a foundational framework for modeling a specific, yet relevant, aspect of real-time ad allocation. This problem can be abstracted as one where a platform must assign dynamically arriving ad campaigns from a finite set of types to heterogeneous slot inventories with varying capacities. In this formalization, campaigns of the same type share identical sizes and values, representing a standardized product. For example, the platform manages distinct advertising slot inventories, differentiated by their single-dimensional capacity, abstracted as "total available impressions." Their capacities are: 500,000 impressions and 1,000,000 impressions. Ad requests are from various advertisers but are grouped into a finite set of types. All requests of the same type have identical contractual requirements and value per campaign. Type-A (Luxury Auto Advertisers): Each campaign from this category requires \it{size} = 100,000 impressions and pays \it{value} = \$30,000. Type-B (Fast-Moving Consumer Goods Advertisers): Each campaign requires \it{size} = 200,000 impressions and pays \it{value} = \$45,000. Type-C (Local Service Advertisers): Each campaign requires \it{size} = 150,000 impressions and pays \it{value} = \$30,000. The DSMKP thus captures the core challenge of sequentially allocating these stochastic, typed requests to maximize revenue under capacity constraints.


% It is crucial to note that this model represents a significant simplification of real-world online advertising systems, which involve multifaceted objectives, complex user-targeting rules, and intricate auction dynamics.

The DSMKP also provides a simplified yet insightful model for cloud resource scheduling. Here, servers (knapsacks) differ only in their total compute capacity, and tasks (items) arrive online from a finite set of types with fixed sizes and values. This abstraction allows for a focused study on the core problem of allocating stochastic, typed workloads to maximize revenue. While real-world systems are far more complex, this model isolates the critical trade-off between accepting an incoming task and reserving capacity for future, potentially more profitable ones.


In modern manufacturing, the problem arises when heterogeneous production lines (knapsacks) with varying capacities must dynamically process incoming customer orders (items) of distinct types—such as urgent or standard—each with a specific processing time and profit margin. Similarly, in smart grid operations, the same structural problem emerges when multiple energy storage units (knapsacks) with distinct capacities must fulfill randomly arriving energy requests (items) from electric vehicles or industrial users, which are categorized by priority and have different energy demands and economic values. In both cases, despite vastly different domains, the core challenge remains the same: to make real-time, irrevocable decisions on accepting and assigning typed items to knapsacks to maximize total reward, without knowing future arrivals. The DSMKP provides a unified formalism to study this essential trade-off between utilizing current capacity and reserving it for future, potentially more valuable opportunities, making it a critical tool for optimizing online decision-making under uncertainty.


% In telecommunications networks, multiple communication channels (knapsacks)—such as fiber-optic links or wireless spectra—have different bandwidth capacities. Data flows (items) arrive dynamically and are categorized by service type (e.g., video streaming, VoIP, or bulk transfer), each requiring a certain bandwidth (size) and offering a quality-of-service value (e.g., priority or revenue). The OMKP formulation aids in designing online algorithms that assign flows to channels to maximize total utility or adherence to service-level agreements. The problem is exacerbated by the diversity of flow types and channel characteristics in multi-protocol label switching (MPLS) or software-defined networking (SDN) contexts.


These applications underscore the broad applicability of the DSMKP in real-world systems where heterogeneous resources must be allocated dynamically under uncertainty. Developing efficient algorithms for this problem--with guarantees on scalability and robustness--is therefore of significant theoretical and practical interest. This paper aims to address this gap by proposing novel strategies for the DSMKP and validating them in one or more of the above domains.


% Resource allocation
% Applications: Cloud computing, Ad placement, Production scheduling, Energy management, Network 
% bandwidth allocation

We develop several policies for DSMKP. The rest of this paper is structured as follows. We review the relevant literature in Section \ref{literature}. Section \ref{sec_dynamic_seat} presents the bid-price and resolving dynamic primal policies to assign the incoming requests. Section \ref{sec_result} presents the experimental results and provides insights gained from implementing dynamic primal. Conclusions are shown in Section \ref{sec_conclusion}.


% Knapsacks (Servers - Varying Single-Dimensional Capacity):
% A cloud provider manages multiple servers. These servers are functionally identical, and the only difference between them is their total available capacity, abstracted as "compute units."

% Their capacities are, for example: Server 1: 100 units, Server 2: 150 units, Server 3: 200 units.

% Key Point: Any task can be assigned to any server with sufficient remaining capacity.

% Items (Tasks - Multi-type with Fixed Properties):
% Incoming user tasks belong to a finite set of types. Each task type has a deterministic resource requirement (size) and an associated reward (value).

% Type-1 (High-Priority Task): Size = 20 units, Value = \$50.
% Type-2 (Medium-Priority Task): Size = 30 units, Value = \$60.
% Type-3 (Low-Priority Batch Task): Size = 40 units, Value = \$60.

% Core Problem: To assign randomly arriving ad requests from multiple advertisers to different advertising slot inventories, maximizing total platform revenue. Requests are categorized into a finite set of types based on their industry vertical and performance goals.

% Knapsacks (Ad Inventories - Varying Single-Dimensional Capacity):

% The platform manages distinct advertising slot inventories, differentiated by their single-dimensional capacity, abstracted as "total available impressions."

% Their capacities are, for example: Inventory : 500,000 impressions, Inventory : 1,000,000 impressions.

% Items (Ad Requests - Multi-type with Fixed Properties):
% Ad requests are from various advertisers but are grouped into a finite set of types. All requests of the same type have identical contractual requirements and value per campaign. 

% Type-A (Luxury Auto Advertisers): Each campaign from this category requires Size = 100,000 impressions and pays Value = \$30,000.
% Type-B (Fast-Moving Consumer Goods Advertisers): Each campaign requires Size = 200,000 impressions and pays Value = \$45,000.
% Type-C (Local Service Advertisers): Each campaign requires Size = 150,000 impressions and pays Value = \$30,000.

% Ad requests are standardized packages sold to advertisers. Multiple packages of the same type are available for purchase.



