% !TEX root = sum1.tex
\section{Literature Review}\label{literature}

The Multiple Knapsack Problem (MKP) \citep{martello1990knapsack} is a well-studied combinatorial optimization problem with broad practical implications. Existing literature has largely concentrated on establishing bounds or competitive ratios for general MKP instances \citet{khuri1994zero, ferreira1996solving, pisinger1999exact, chekuri2005polynomial}. In contrast, our work focuses on analyzing the specific solution characteristics of the multi-type multiple knapsack problem, providing a more tailored understanding of its dynamic form.

A closely related stream of research addresses the Dynamic Stochastic Knapsack Problem, as examined in \citet{kleywegt1998dynamic, kleywegt2001dynamic} and \citet{papastavrou1996dynamic}. These studies typically consider a single knapsack setting where requests arrive sequentially, with resource requirements and rewards revealed only upon arrival. However, they do not incorporate multiple knapsacks, which limits their applicability to more complex resource allocation settings.


In comparison, the DSMMKP generalizes this framework by introducing multiple knapsacks, significantly increasing the complexity of real-time decision-making. Moreover, in DSMMKP settings, additional information--such as the arrival probabilities of different item types--is often available, enabling more informed and potentially more effective allocation strategies. 

Research on dynamic or stochastic variants of the Multiple Knapsack Problem (MKP) remains relatively limited. Existing studies, such as those on the online multiple knapsack problem \citep{bienkowski2020optimal, sun2020competitive}, typically operate under the assumption that no prior information about item arrivals is available before they occur. Similarly, \citet{tonissen2017column} investigates a two-stage stochastic multiple knapsack problem in which knapsack capacities are subject to perturbations across a set of predefined scenarios. In contrast, our work considers a more informed setting where the arrival probability of each item type is known in advance, enabling more proactive and potentially more effective allocation strategies.

% Research on dynamic or stochastic variants of the MKP remains relatively limited. For instance, the online multiple knapsack problem is considered without knowing the arrival information of items before their arrivals \citep{bienkowski2020optimal, sun2020competitive}. Meanwhile, \citet{tonissen2017column} investigates a two-stage stochastic multiple knapsack problem where knapsack capacities may be perturbed according to a set of predefined scenarios. While our study assumes that the arrival probability is known before the arrival.




Generally speaking, the DSMMKP problem is related to the Revenue Management (RM) problem. RM has been extensively studied in industries such as airlines, hotels, and car rentals, where the core challenge is to dynamically allocate perishable inventory to maximize revenue \citep{van2005introduction}. Network revenue management (NRM) extends this framework by considering multiple interconnected resources (e.g., flight legs, hotel nights) and interdependent demands \citep{williamson1992airline}.


The standard NRM problem is typically formulated as a dynamic programming (DP) model, where decisions involve accepting or rejecting requests based on their revenue contribution and remaining capacity \citep{talluri1998analysis}. 

A fundamental challenge, however, is the ``curse of dimensionality'', as the state space grows exponentially with problem size, making exact DP solutions computationally intractable. Consequently, various approximate control policies have been developed, including bid-price controls \citep{adelman2007dynamic, bertsimas2003revenue}, booking limits \citep{gallego1997multiproduct}, and dynamic programming decomposition methods \citep{talluri2006theory, liu2008choice}.

However, a significant challenge arises because the number of states grows exponentially with the problem size, rendering direct solutions computationally infeasible. To address this, various control policies have been proposed, such as bid-price \citep{adelman2007dynamic, bertsimas2003revenue}, booking limits \citep{gallego1997multiproduct}, and dynamic programming decomposition \citep{talluri2006theory, liu2008choice}. 


In our problem, items often request multiple units simultaneously, requiring decisions that must be made on an all-or-none basis for each request. This requirement introduces significant complexity in managing group arrivals \citep{talluri2006theory}.

A notable study addressing group-like arrivals in revenue management examines hotel multi-day stays \citep{bitran1995application, goldman2002models, aydin2018decomposition}. While these works focus on customer classification and room-type allocation, they do not prioritize real-time assignment. The work of \cite{zhu2023assign}, which addresses the high-speed train ticket allocation and processes individual seat requests and implicitly accommodates group-like traits through multi-leg journeys (e.g., passengers retaining the same seat across connected segments).