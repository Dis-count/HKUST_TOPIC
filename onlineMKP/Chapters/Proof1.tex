% !TEX root = sum1.tex
\clearpage
\section{Proofs}

\begin{pf}{Proof of Proposition \ref{sol_relax_deter}}
  We model the problem as a special case of the multiple knapsack problem, then we consider the LP relaxation of this problem. In the model, groups are categorized into $M$ distinct types. Each type $i$ is characterized by a fixed group size $n_i$, which serves as the weight, and an associated profit equal to $i$. For every type $i$, there are $d_i$ groups. Altogether, the total number of groups is given by $K = \sum_{i=1}^{M} d_i$. Each individual group $k$ inherits its profit and weight from its type; specifically, if group $k$ belongs to type $i$, then its profit $p_k$ is $i$, and its weight $w_k$ is $n_i$. To apply the greedy approach for the LP relaxation of \eqref{deter_upper}, sort these groups in non-increasing order of their profit-to-weight ratios: $\frac{p_1}{w_1} \geq \frac{p_2}{w_2} \geq \ldots \geq \frac{p_K}{w_K}$. The break group $b$ is the smallest index such that the cumulative weight of group 1 to group $b$ meets or exceeds the total capacity $\tilde{L}$: $b=\min \{j: \sum_{k=1}^j w_k \geq \tilde{L}\}$, where $\tilde{L} = \sum_{j=1}^{N} L_j$ is the total size of all rows. For the LP relaxation of \eqref{deter_upper}, the Dantzig upper bound \citep{dantzig1957discrete} is given by $u_{\mathrm{MKP}}=\sum_{j=1}^{b-1} p_j+\left(\tilde{L}-\sum_{j=1}^{b-1} w_j\right) \frac{p_b}{w_b}$. The corresponding optimal solution is to accept the whole groups from $1$ to $b-1$ and fractional $(\tilde{L}-\sum_{j=1}^{b-1} w_j)$ group $b$. Suppose the group $b$ belong to type $\tilde{i}$, then for $i < \tilde{i}$, $x_{ij}^{*} = 0$; for $i > \tilde{i}$, $x_{ij}^{*} = d_{i}$; for $i = \tilde{i}$, $\sum_{j=1}^{N} x_{ij}^{*} = (\tilde{L} - \sum_{i = \tilde{i}+1}^{M} {d_i n_i})/ n_{\tilde{i}}$.
\end{pf}


\begin{pf}{Proof of Lemma \ref{bid-price}}
According to the Proposition \ref{sol_relax_deter}, the aggregate optimal solution to LP relaxation of problem \eqref{deter_upper} takes the form $x e_{\tilde{i}} + \sum_{i=\tilde{i}+1} ^{M} d_{i} e_{i}$, then according to the complementary slackness property, we know that $z_1= \ldots= z_{\tilde{i}} = 0$. This implies that $\beta_j \geq \frac{n_i - \delta}{n_i}$ for $i = 1,\ldots, \tilde{i}$. Since $\frac{n_i - \delta}{n_i}$ increases with $i$, we have $\beta_j \geq \frac{n_{\tilde{i}} - \delta}{n_{\tilde{i}}}$. Consequently, we obtain $z_{i} \geq n_i - \delta - n_i \frac{n_{\tilde{i}} - \delta}{n_{\tilde{i}}} = \frac{\delta(n_i-n_{\tilde{i}})}{n_{\tilde{i}}}$ for $i = h+1, \ldots, M$.

Given that $\mathbf{d}$ and $\mathbf{L}$ are both no less than zero, the minimum value will be attained when $\beta_j = \frac{n_{\tilde{i}} - \delta}{n_{\tilde{i}}}$ for all $j$, and $z_i = \frac{\delta(n_i-n_{\tilde{i}})}{n_{\tilde{i}}}$ for $i = \tilde{i}+1, \ldots, M$.  
\end{pf}

\newpage