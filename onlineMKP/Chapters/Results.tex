% !TEX root = sum1.tex
\section{Computational Experiments}\label{sec_result}
\subsection{Performances}

The fundamental system configuration consists of $N = 10$ knapsacks, each with an identical capacity of $L_j = 20$ for all $j$.

We define several experimental scenarios by varying the combination of item parameters--namely size $w$ and profit $r$ vectors--and the demand probability distribution $D$ over the item types. In the first set of experiments, the demand distribution is $D_1 = [0.2, 0.5, 0.3]$ for three item types. This distribution is examined under three distinct profit-to-size profiles. The first profile uses sizes $w = [3, 4, 5]$ and profits $r = [4, 6, 8]$, where the profit-to-weight ratio exceeds one for every item type. The second profile shares the same sizes, $w = [3, 4, 5]$, but with profits set equal to the sizes, $r = [3, 4, 5]$, resulting in a uniform profit-to-weight ratio of one. The third profile inverts the size order with $w = [8, 6, 4]$ and assigns profits $r = [5, 4, 3]$, leading to profit-to-weight ratios that are all less than one. For this experiment set, the total time horizon $T$ is tested at values of 60 and 80 periods.

A second set of experiments employs a uniform demand distribution $D_2 = [0.25, 0.25, 0.25, 0.25]$ across four item types. The corresponding item parameters are sizes $w = [3, 5, 7, 9]$ and profits $r = [2, 4, 6, 8]$. For this configuration, the total number of periods $T$ takes on the values 60, 80, and 100 to evaluate system performance under varying time horizons.

\begin{table}[h]
  \centering
  \caption{Performance of Policies}\label{tab_perf}
  \begin{tabular}{cccccc}
  \hline
  \hline
  Distribution & T & $r_i/w_i$ & Primal (\%) & BPC (\%) & BPP (\%)  \\
  % \Xcline{1-1}{0.4pt}\Xcline{3-3}{0.4pt}\Xcline{4-4}{0.4pt}
  % \cmidrule(r){0-1} \cmidrule(lr){3-3} \cmidrule(lr){4-4} \cmidrule(lr){5-5} \cmidrule(lr){6-6} \cmidrule(l){7-7}
  \hline
  \multirow{5}{*}{$D_1$} & 60  & $>$1 & 99.15 & 95.30 & 97.98  \\
                         & 80  & $>$1 & 99.37 & 95.70 & 98.18  \\
                         & 60  & =1   & 99.68 & 95.66 & 98.88  \\  % same w r
                         & 80  & =1   & 99.80 & 96.26 & 99.44  \\
                         & 60  & $<$1 & 98.56 & 95.86 & 97.66 \\
                         & 80  & $<$1 & 97.00 & 95.67 & 97.31  \\
    \hline
    \multirow{5}{*}{$D_2$} & 60  & $<$1 & 99.65 &  94.63 & 99.84  \\
     & 80  & $<$1 & 99.81      & 96.89       &  99.91     \\
     & 100 & $<$1 & 99.94 & 99.41  & 99.97 \\
  \hline
  \end{tabular}
\end{table}


\subsection{Loss}
For each policy $\pi$, the loss of $\pi$, denoted as $L_{\pi}$, is defined as the average of $r^{\text{OFF}}(\omega) - r^{\pi}(\omega)$ over all sample paths $\omega$. 

Here, we present the loss under policies BPC, BPP and DPP. The number of sample paths for each data point is 100.

\begin{figure}[h]
  \centering
  \includegraphics[width = 10cm]{./Figures/loss}
  \caption{Loss over $T$}  \label{fig:loss}
\end{figure}


% $\frac{r_{i}}{c_{i}} = C$, then the optimal policy exists?

% when there are two knapsacks 
% Compare one knapsack performance